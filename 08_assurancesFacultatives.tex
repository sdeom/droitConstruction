% !TEX root = ./droitConstruction.tex

\chapter{Les assurances facultatives}

	\DO et RCD sont obligatoires, mais il y a des assurances de choses et de responsabilité civiles facultatives.

	Elles couvrent les autres désordres qui ne sont pas de nature décennale.

	\paragraph{L'assurance << tout risque chantier >>}

		C’est une police d’assurance souscrite par le \MO et couvre les dommages causés aux choses, à l’ouvrage lui-même aux existants, au baraquement, aux matériaux, par tout sinistre survenu sur le chantier, quel qu’en soit l’auteur et la victime.

		Elle produit effet jusqu’à la réception des travaux.

		\medbreak Elle couvre les évènements les plus divers tels que l’incendie, vol, dégâts des eaux, accident de chantier, effondrement, les défauts de conception ou d’exécution s’ils provoquent un désordre brutal et accidentel. Elle ne couvre pas les défauts de conception ou d’exécution qui provoquent une simple malfaçon ou une non-conformité sauf stipulation particulière contraire.

		\medbreak Cette garantie a pour intérêt d’accorder avant toute recherche de responsabilité une couverture financière et de palier les défauts de solvabilité des intervenants.

		Les franchises des plafonds sont valables et opposables à la victime

	\paragraph{L’assurance des responsabilite civile par les constructeurs}

		De leur coté les constructeurs peuvent souscrire des assurance couvrant les dommages exclus du champ d’application de la RCD, et donc couvrir des dommages de nature autre que décennale.

		Comme les dommages qui n’atteignent pas la gravité décennale, les dommages intermédiaires, les dommages corporels, les dommages immatériels, les \TAV.

		Ces polices couvrent tous les faits dommageables produits au cours de la période de validité du contrat d’assurance.

		\medbreak Les franchises et plafond sont valables et opposables à la victime

		\medbreak Les exclusions de garantie sont valables si elles sont formelles et limitées. Elles doivent être claires et ne pas être sujettes à interprétation.


	\paragraph{Police Universelle de Chantier (PUC)}

		C'est une police qui regroupe la \do, la \rcd, et \cnr parfois la tout risque chantier. C'est une police assez couteuse de sorte qu'elle est rarement souscrite. En revanche elle assure une protection maximale pour le \Mo qui s'adresse à un interlocuteur unique.

		C’est cet assureur qui va payer sans rechercher qui est responsable, en revanche après indemnisation de la victime il va rechercher les responsables mais seulement pour aller récupérer les franchises auprès des constructeurs.
