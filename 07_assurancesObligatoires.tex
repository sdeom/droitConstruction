% !TEX root = ./droitConstruction.tex

\chapter{Les assurances obligatoires}

	Il existe deux types d'assurances obligatoires :
	\begin{itemize}
		\item l'assurance de responsabilité, assurance de responsabilité civile décennale qui couvre l’activité des constructeurs ;
		\item l'assurance de chose l'\ado qui couvre l’immeuble assuré.
	\end{itemize}

	\medbreak La dernière grande réforme est du \printdate{4/1/1978} complétée notamment par l'ordonnance du \printdate{8/6/2005}

	\medbreak Les principaux textes sont :
	\begin{itemize}
		\item les \articlesDuEtSuivants[L]{241-1}{\ca}, pour l'\arcd ;
		\item les \articlesDuEtSuivants[L]{242-1}{\ca}, pour l'\ado ;
		\item les \articlesDuEtSuivants[L]{241-1}{\ca}, en ce qui concerne les dispositions communes.
	\end{itemize}


\section{Le principe de l'assurance obligatoire}

	\subsection{Le champs commun des assurances obligatoires}


		\subsubsection{Les travaux couverts par l'obligation d'assurance}

			Avant l'ordonnance \printdate{8/1/2005} l'obligation d'assurance ne portait que sur les travaux de bâtiment.
			Seuls les locateurs d’ouvrage qui réalisaient des travaux de bâtiment devaient souscrire une assurance obligatoire, c’était très restrictif.

			De sorte que la \JP a progressivement et considérablement élargi le champ d’application de l’obligation d’assurance en soumettant tous les travaux réalisés en faisant appel aux techniques de travaux de bâtiment.
			Ainsi, les travaux portant sur un mur de soutènement, les stations de métro, les \vrd, les dalles de plafonds, \etc rentraient dans le champ d’application des assurances obligatoires car c’étaient des travaux réalisés en faisant appel au technique de travaux de bâtiment.

			Le législateur est intervenu conscient que la formule utilisée était restrictive, depuis ordonnance du \printdate{8/6/2005} l’obligation porte sur les travaux de construction, notion plus large que celle de bâtiment.
			Cette notion inclus tous les ouvrages immobiliers, \CAD tout ce qui est ancré dans le sol, donc tout ce qui est ouvrage est construction au sens décennale et doit être couvert.

			L'\articleDu[L]{243-1-1}{\ca}liste des exclusions :
			\begin{itemize}
				\item les ouvrages maritimes, lacustre, fluviaux, les ouvrages infrastructure routière, portuaire, aéroportuaire, héliportuaire, ferroviaire, \etc, bref l'ensemble des ouvrages de génie civil ;
				\item les voiries, les parcs de stationnement, les canalisations, les ouvrages sportifs non couverts, les ouvrages de production, de stockage, et de distribution d’énergie, les ouvrages de télécommunication, et les éléments d’équipement.
			\end{itemize}

			Si les ouvrages ou les éléments d’équipement visés dans la deuxième exception, constituent un accessoire à un ouvrage soumis à l’obligation d’assurance alors ils sont soumis à une obligation d’une assurance

		\subsubsection{Les dommages couverts par l'obligation d'assurance}

			\paragraph{Les dommages de nature décennale, provenant d'un sinistre, affectant l'ouvrage}

				Les assurances obligatoires ne couvrent que les dommages à l’ouvrage lui-même ce qui exclut les dommages consécutifs : les dommages annexes \CAD les dommages incorporels, les dommages aux tiers et les dommages aux mobiliers.

				Il faut que le dommage soit survenu à l'occasion d'un sinistre, ce qui exclut les non façons et les défauts de conformité.
				En revanche couvre les travaux non prévus à l'origine (\emph{i.e.} absence d'ouvrage) mais qui aurait été indispensable pour éviter le sinistre.
				\begin{exemple}
					Dans le cas d'une inondation au sous-sol à cause d’une absence de drainage qui aurait dû être prévu à l’origine, on a bien un sinistre accidentel avec l’inondation du sous-sol, dans ce cas-là les assurances obligatoires couvrent les travaux non prévus à l’origine mais qui auraient été indispensables pour éviter le sinistre
				\end{exemple}

				L'assurance obligatoire ne couvre que les dommages de nature décennale, dont le critère déterminant est la gravité décennale, de sorte que les dommages survenus pendant les travaux sont exclus. C’est le critère déterminant

				\medbreak Ces critères :
				\begin{enumerate}
					\item dommage affectant l'ouvage,
					\item provenant d'un sinistre,
					\item de gravité decennale ;
				\end{enumerate}
				sont cumulatifs. Il faut donc un dommage à l’ouvrage lui-même, survenu à l’occasion d’un sinistre, de nature décennale, sur un ouvrage reçu (les dommages intermédiaires sont exclus)

				\medbreak Le fondement de la responsabilité est indifférent, dans le cadre des actions récursoires entre constructeurs et de son assurance, les assurances de responsabilité civile décennale ne peuvent pas dénier leur garantie au motif que le fondement de la responsabilité de leur assuré est délictuel, ce qui compte c’est le dommage et la gravité du dommage, alors l’assurance peut être tenue à garantir son assuré\footnote{\jurisCourDeCas[17-13833]{\civTrois*}{8/11/2018} : le fondement juridique de la responsabilité est indifférent}.

				\subparagraph{Cas des ouvrages existants}
				Aux termes de l'\articleDu[L]{243-1-1 \II}{\ca} << {\itshape ces obligations d'assurance ne sont pas applicables aux ouvrages existants avant l'ouverture du chantier, à l'exception de ceux qui, totalement incorporés dans l'ouvrage neuf, en deviennent techniquement indivisibles}. >>

				L’interprétation traditionnellement admise de cette disposition a conduit à n'intégrer dans le champ de l’assurance obligatoire que les dommages sur les existants techniquement indivisibles des ouvrages neufs. Seuls les dommages affectant les existants techniquement indivisibles avec les ouvrages neufs étaient couverts par les assurances obligatoires.
				Néanmoins dans des arrêts \printdate{15/6/2017} et \printdate{14/9/2017}, la \CourDeCas a admis que la garantie décennale pouvait être mobilisée en cas d’atteinte à un élément d’équipement dissociable ou non d’origine ou installée sur existant dès lors qu’il rend l’ouvrage en son ensemble impropre à sa destination, alors relève de la responsabilité décennale

				La \civTrois a tiré les conséquences de la jurisprudence et a soumis les éléments d'équipement installés sur existant\footnote{\jurisCourDeCas{\civTrois*}{26/10/2017}}

				Le legislateur a tenté de contrer cette jurisprudence à l'occasion de la loi ELAN en réécrivant l'\articleDu[L]{243-1-1 \II}{\ca}, car elle très favorable aux \Mo et engage la responsabilité personnelle des artisans. Le Conseil constitutionnel a cependant déclaré le texte inconstitutionnel faute de lien avec le projet de loi initial.

				Aujourd’hui c’est toujours cette \JP qui s’applique, et les artisans restent soumis à cette obligation d’assurance.

			\paragraph{Les dommages affectant les travaux régulièrement déclarés}

				Seuls les dommages affectant des travaux exactement déclarés par l’assuré à l’assureur sont couverts par la police d’assurance. Cela s'explique par le fait que la prime est calculée sur le risque auquel s'expose l'assureur. Le \lo, ou le \MO, doit déclarer le chantier, la construction envisagée, pour que l’assureur puisse déterminer son risque.

				\medbreak En cas de déclaration inexacte ou en cas d’absence de déclaration, les juges peuvent soit :
				\begin{itemize}
					\item annuler la police d'assurance si l’assurance a été intentionnellement trompée  --- fraude visée par l'\articleDu[L]{113-8}{\ca} --- et dans cette hypothèse les primes payées restent acquises à l’assureur sans indemnité , la nullité du contrat étant opposable à la victime ;
					\item faire application du mécanisme de la réduction proportionnelle de l'\articleDu[L]{113-9}{\ca}\footnote{Mécanisme de reduction de l'indemnité en proportion du taux de prime payé par rapport au taux de prime qui aurait du être appliqué si les risques avaient été complètement et exactement déclarés. en pratique les juges entérinent les calculs proposés par l’assureur.}.
				\end{itemize}
				Le mécanisme de la réduction proportionnelle s'applique également à la victime, ce qui est trés pénalisant pour celle-ci. Lorsque le chantier n’est pas du tout déclaré la victime ne pourra rien obtenir.

	\subsection{La preuve de la souscription d'une assurance}

		La forme de la preuve est consacrée à l’\articleDu[L]{243-2}{\ca} : << {\itshape les personnes soumises aux obligations prévues par les articles L. 241-1 à L. 242-1 du présent code doivent être en mesure de justifier qu'elles ont satisfait auxdites obligations} >>. La preuve se fait par la production d'une attestation d'assurance. Elle doit être jointe aux devis et factures (sans sanction).

		Un arrêté du \printdate{05/01/2016} fixe le modèle d’attestation. Les mentions minimales concernent :
		\begin{itemize}
			\item le nom,
			\item l’adresse et les coordonnées complètes de l’assureur,
			\item l’identification de l’assuré,
			\item le \no de contrat,
			\item sa période de validité,
			\item la date d’établissement de l’attestation,
			\item les activités déclarées et assurées,
			\item l’adresse et le coût de l’opération déclarée,
			\item et la date d’ouverture de chantier
		\end{itemize}
		L’attestation doit donc fournir des informations exactes sur l’étendue des garanties offertes et ne pas égarer le \MO.
		Etant précisé que les exclusions et limitations de garantie doivent être mentionnées.

		L’auteur de l’attestation est l’assureur.

		\subparagraph{S’agissant du moment de la preuve}
		Toute personne physique ou morale soumise aux obligations d’assurance doivent être en mesure de justifier qu’elles ont souscrit un contrat d’assurance à l’ouverture de tout chantier.

		Le \MO peut demander à tout intervenant à l’acte de construire de justifier qu’il satisfait à ses obligations d’assurance à tout moment, après la période de validité pour l’année suivante.


	\subsection{Les sanctions du défaut d'assurance}

		Le défaut d'assurance est sanctionné.

		\paragraph{Sanction d'ordre pénal}
			L'\articleDu[L]{243-3}{\ca} prévoit 6 mois d'emprisonnent et \montant{75 000} d'amende.

			Les sanctions pénales ne s’appliquent pas aux personnes physiques qui construisent pour occuper le logement ou des membres de sa famille. de sorte que particulier qui construit pour lui même n'encourt pas de sanction pénale.

		\paragraph{Sanction d'ordre civil} Il existe également des sanctions d'ordre civil :
		\begin{itemize}
			\item Le \Mo engage sa responsabilité civile s’il n’a pas souscrit de \DO.

			\item L’architecte engage sa responsabilité civile s’il engage des entreprises non assurées ou s’il ne conseille pas le \MO de souscrire une \DO

			\item L’entrepreneur locateur d’ouvrage, engage certes sa responsabilité civile à l’égard du MO, mais en tant que gérant il engage également sa responsabilité personnelle s'il n'assure pas sa société \footnote{\jurisCourDeCas{\civTrois}{10/3/2016} : faute détachable de ses fonctions}.

			\item Le notaire engage sa responsabilité civile s'il ne vérifie pas l'existence ou la réalité des assurances obligatoires. Son devoir de conseil doit le conduire à attirer l’attention de l’acquéreur sur l’absence d’assurance, notamment sur les risques encourus en cas d’absence d’assurance. A défaut, le notaire engage sa responsabilité civile.

			Lorsqu’un acte translatif de la propriété ou de la jouissance intervient avant l’expiration du délai de 10 ans, l'\articleDu[L]{243-2}{\ca} impose au notaire de faire mention de l’existence ou de l’absence de l’assurance de responsabilité civile et d'\ado dans le corps de l’acte ou en annexe.

			\item Le vendeur engage sa responsabilité civile s’il n’a pas souscrit de \DO.

			\item Le syndic engage sa responsabilité civile s’il n’a pas proposé au syndicat de copropriétaire la souscription d’une \ado où s’il a proposé des locateurs d’ouvrage non assurés.
		\end{itemize}


	\subsection{L'obligation d'assurer}

		Ceux qui se voient refuser une police peuvent saisir le bureau central de tarification (BCT) qui va déterminer un taux de prime et imposer à une compagnie d'assurance d'assurance d'assurer le \lo.


\section{L'assurance dommages-ouvrage}

	il s'agit d'une assurance de chose, donc applicable en dehors de toute recherche de responsabilité

	L'\articleDu[L]{242-1}{\ca} stipule qu'elle doit être souscrite avant le début du chantier. Mais le législateur n’a prévu aucune sanction. Donc, en l'absence de sinistre, il est toujours possible au \Mo de chercher une police.

	Ici on paie une prime au début ce ne sont pas des échéances mensuelles.


	\subsection{Le domaine de l'assurance dommages-ouvrage}



		\subsubsection{Les personnes assujetties}

			Le principe est posé par l'\articlesDu[L]{242-1}{\ca} : << {\itshape toute personne physique ou morale qui, agissant en qualité de propriétaire de l'ouvrage, de vendeur ou de mandataire du propriétaire de l'ouvrage, fait réaliser des travaux de construction, doit souscrire avant l'ouverture du chantier, pour son compte ou pour celui des propriétaires successifs, une assurance garantissant, en dehors de toute recherche des responsabilités, le paiement de la totalité des travaux de réparation des dommages de la nature de ceux dont sont responsables les constructeurs au sens de l'article 1792-1, les fabricants et importateurs ou le contrôleur technique sur le fondement de l'article 1792 du code civil.} >>

			De sorte que toute personne qui construit ou fait construire un ouvrage --- y compris le castor, le vendeur d’immeuble à construire, le promoteur immobilier, le syndicat de copropriétaires, le copropriétaire --- doit souscrire une \DO\footnote{Le constructeur de maison individuelle ne souscrit pas une \DO, c’est le \MO qui souscrit une \DO. Le constructeur souscrit une assurance décennale.}

			L’assurance peut être souscrite par le \MO ou par son mandataire.

			\subparagraph{Exception} Ne sont pas soumis à l’obligation d’assurance  :
				\begin{itemize}
				\item l'État ;
				\item les personnes morales assurant la maitrise d’ouvrage dans le cadre d’un contrat de partenariat public privé ;
				\item les personnes morales de droit public ou privé remplissant les conditions cumulatives suivantes\footnote{\ArticleDu[L]{242-1}{\ca}} :
					\begin{itemize}
						\item l'activité du \Mo doit être importante, ce qui est le cas lorsque que sont remplies 2 des 3 conditions suivantes :
						\begin{enumerate}
							\item le chiffre d'affaire du dernier exercice doit être supérieur à 12,8 millions d'euros,
							\item le nombre moyen de salarié doit être supérieur à 250,
							\item le total du dernier bilan doit être supérieur à 6,2 millions d'euros ;
						\end{enumerate}
					\item les travaux doivent être réalisé pour le compte du \Mo (donc la \DO demeure si en vue de la vente) ;
					\item l'ouvrage réalisé doit être à usage autre que l'habitation.
					\end{itemize}
				\end{itemize}

		\subsubsection{Les bénéficiaires de la garantie}

			La DO est une assurace de chose. elle est souscrite pour le compte

			La police se transmet en même temps que la cnstruction aux acquéreurs successifs. elle bénéficie donc au propriétaire de l'ouvrage au moment ...

			en cas de vente régularisée entre la et le règlement de l'indemnité. c'est l'acquéreur même sii \jurisCourDeCas{\civTrois}{15/9/2016}. Sauf clause contraire. De surcroit le vendeur qui a supporté le coups des réparations après la vente... en qualité de subrogé ... \jurisCourDeCas{\civUn}{21/2/1995}.

			L'assurance \do ne bénéficie pas au locataire.

		\subsubsection{L'étendue de la garantie}



			\paragraph{Le quantum de la garantie}

				L'assureur doit prefinancer tous les travaux de nature à mettre fin au désordre 7/12/2005. L'\ado couvre donc les couts des travaux de reprise des domamges de nature decenalle affectant les ouvrages, les élement d'éqi techniqu

				elle ne couvre pas les dommages annexes.

				les franchises sont interdites depuis L 242-1 \ca prévoit que << totalité >>

				L 243-9 \ca : peuvent comporter des plafonds de garantie, de sorte qu'il est possible autre que l'habitation. Un décret R 243-3 \ca vient préciser le montant de plafond la garantie ne peut être inférieur au total de la construction déclaré par le \Mo ou si 150 M eris HT, le plafond est à 150 M.

				Seul 3 cas d'exclusion de garantie :
				\begin{enumerate}
					\item fait intentionnel\footnote{Caratérisée lorsque l'assuré à voulu par con comportement le dommage tel qu'il s'est réalisé} ou dol\footnote{Suppose la preuve que l'assuré est su que son comportement est ... (civ 2 1h00 16-23103), mais en pratique la civ 3 ne distingue pas le dol de la faute intentionnelle} L 113-1 du \ca\footnote{Car dans ce cas, il n'y a pas d'aléa} ;
					\item l'effet de l'usure ou de l'usage normal, et du défaut d'entretien ;
					\item la cause étrangère (fait de guerre\etc)
				\end{enumerate}
				Les exclusions de garante sont opposables à la victime. Ce qui peut ... si le constructeur est en faillite.

				\paragraph{Prise en compte de la TVA} Cela dépend \index{TVA} Même solution que décennale. Il appartient d'apporter la preuve qu'il ne récupère pas la TVA s'il demande une indemnité TTC.

			\paragraph{Le point de départ de la garantie}

				\subparagraph{Principe} L'\ado prend normalmeent effet qu'après la gpa, \cad un an après la réception. Elle prend fin à compter de l'expérition d'une priode de 10

				2 excepti 8 aliné L242-1 \ca :
				\begin{enumerate}
					\item avant la réception en cas de résiliation du contrat de la pour inexécution hypo de l'abandon de chantier.
					\begin{itemize}
						\item avant réception
						\item dommage de nature décennale
						\item mise en demeure de l'\E restée infructueuse\footnote{suaf si elle s'avère impossible ou inutile en raison de la cessation d'activité de l\E 23/6/1998, notamment à la suite d'une liquidation judiciaire --- mais reste nécessaire en cas de rdressement judiciaire avec poursuite d'activité} (l'assignation vaut mise en demeure)
						\item contrat doit être résilié (judiciaire ou unilatérale), étant précisé que la liquidation emporte 13/2/2020
					\end{itemize}
					\item après réception, lorsque l'\E n'a pas executé ses obligations malgré une mise en demeure Hypo de l'\E défaillant dans le cadre de la gpa
					\begin{itemize}
						\item une réception
						\item dommage de nature décennale réservé ou apparu dans le délai d'1 an
						\item mise en demeure de l'\E restée infructueuse
					\end{itemize}
				\end{enumerate}

a 243-1 annexe 2 A : obligation de l'assurée

	\subsection{La mise en œuvre de l'assurance dommages-ouvrage et la gestion des sinistres}

		Comme en Amour, tout commence par une déclartion...

		\subsubsection{La déclaration de sinistre}

			L'assuré doit par une déclaration de sinistre, et non pas pas par une assignation. Si avant ou dans les 60 jour, il perd

			2f formes reconnues : \lrar ou remise contre recepissé

			Elle doit impérativement être régularisée dans le délai biennal à compter du sisnistre L114-1 du \ca

			la date du sinstré avant récpetion = resiliation du marché 13/2/202
			après réception, date à laquelle le \Mo a eu connaissance du sinistre

			Le délai de 2 ans n'est pas enfermé dans le délai décennal Si il assigne l'assureur dans le délai mais pos cela ne constitue pas une faute 31/3/2004. En pratique déclarer immédiatement, car c'est un moyen d'éviter que l'assureur puisse reprocher de l'avoir privé du mécanisme de la subrogation (L121-12 du \ca)\footnote{... de sorte que pour illustration :  8/2/2018 17-10010}.

			délai biennale est un délai de prescription. Il peut être interrompu par une des causes ordinaires l'interruption de la prescription\footnote{= forclusion + reconnaissance} :
			\begin{enumerate}
				\item citation en justice
				\item reconventionnelle
				\item reconnaissance de responsabilité
			\end{enumerate}
			Mais
			+ un cas propre à l'assurance L114-2 : \lrar
			+ courrier electronique
			+ ???

			Donc c'est un nouveau délai de même durée qui recommence à courir. En cas d'assignation en référé expertise le délai est suspendu jusqu'à la remise du rapport définitif (L). Si le reliquat est inférieur à 6 mois.

			Le contenu est précisé dans les << clauses types de 2009 >> annexe 2 de A 243-1

			L'assureur dispose d'un délai de 10 jours pour lui notifier une demande de renseignement.

		\subsubsection{La prise de position de l'assureur sur le principe de sa garantie}

			L'assureur dispose d'un délai de 60 jours à compter de la déclaration de sinistre.

			L'assureur désigne un expert, sauf dans 2 cas :
			\begin{itemize}
				\item lorsque la ... est manifestement injustifié
				\item lorsque le cout des travaux de reprise est inférieur à \montantTtc{1 800}
			\end{itemize}
			Il dispose alors d'un délai de 15 jours pour notifier un refus de garantie ou une offre d'indemnité.
			L'assuré peut contester la position de l'assurer pour

			dans les autres cas le recours à l'expertise est obligatoire. L'expert peut faire l'objet d'une récusation ...

			Rapport préliminaire dans lequel il va relater les circonstances ... sur lequel l'assureur prendra position. Il peut le cas échénat préconsier des mesures conservatoires. Il doit être communiqué préalablement à la notification,, ou au plus trad lors de cette notification.

		\subsubsection{La proposition d'indemnisation de l'assureur}

			L'assureur qui a admis la mobilisation de sa garantie doit proposer une indemnité à compter de son acceptation.

			l'expert va rédiger un rapport définitif comportant des proposition ... et estimation concernant ...

			L'assureur doit notifier le rapport définitif ...

			En cas de difficultés exceptionnelles dues à la nature ou à l'importance du sinistre, l'assureur peut proposer à l'assuré la fixation d'un délai supplémentaire pour déterminer. Elle doit êre fondée exclusivement sur des considérations d'ordre technique et être motivée. Le délai sp est subordonné et ne peut excéder 135 jours (en plus des 30)

			la proposition de l'assureur peut revêtir un caractère provisionnel.

		\subsubsection{Le versement de l'indemnité}

			L'assureur est libre d'accepter ou de refuser. Sa réponse n'est pas enfermée dans un délai.

			Si l'as refuse, il peut lui demander de verser les 3/4 de la somme proposée. Dans cette hypo versement dans les 15 jours. L'assuré peut ensuite assigné son assureur pour obtenir un coplément d'indemnité.

			S'il accepte la proposition d'un indemnité. Il doit verser le montant convenu dans les 15 jours.

			... . A défaut d'accord l'assuré peut saisir le juge.

		\subsubsection{L'affectation de l'indemnité versée}

			Le \Mo est obligé d'affecter le montant de l'indemnité à la reprise des désordres civ 3 17/12/2003

			l'assuré s'engage à autoriser l'assureur à constater ... des travx ... A 243-1 annexe 2 du \ca

			L'assureur peut obtenir la restitution des sommes non affectées ou excédant le cout réel des travaux.

	\subsection{Les sanctions de l'assureur en cas de non respect de ses obligations}

		Ces sanctions sont cumulatives et limitatives, de sorte qu ela victime ne pourra pas demander de \di supplémentaires. L242-1

		L'ado doit garantir l'efficacité des trvx prefinancés. A defaut il engage sa responsabilité civile personnelle. C'est-à-dire que si les travaux . La victime pourra alors non pas grâce à la do, mais grâce à la responsabilité de l'ado. Bien évidemment. délais de droit commn (5 à compter de la manifetstaion)

		\subsubsection{Les sanctions en cas de dépassement des délais}

			Ces sanctions de trois ordres :
			\begin{itemize}
				\item \textbf{L'impossibilité de refuser la garantie}, de sorte que l'assureur ne pourra invoquer le défaut de caractère décennal, ni la nullité du contrat  ou le mécanisme de la réduction d'aléa, il ne pourra pas non plus invoquer la prescription biennale sauf si la déclaration intervient plus de deux ans après la fin décennale\footnote{\jurisCourDeCas{\civTrois*}{20/6/2012}} ou si \~25'\jurisCourDeCas{\civTrois*}{20/6/2012}

				L... les délais 60 30/90 15\jurisCourDeCas{\civTrois*}{30/6/2016}

				\item \textbf{La possibilité pour l'assuré d'engager les dépenses nécessaires} après l'avoir notifié à l'assureur \articleDu{L}{242-1 alinéa 5}{Code des assurances}. Viens en des déalis ... De sorte que le \Mo peut commander les travaux de reprise.

				L'assureur peut contester le chiffrage devant les tribunaux.

				\item \textbf{La majoration de l'indemnité versée} \articleDu{L}{242-1 alinéa 5}{Code des assurances}. L'indemnité versée est de plein droit CITE .... cette sanction s'apllique que le \Mo ai ou non les
			\end{itemize}

		\subsubsection{Les sanctions en cas d'offre manifestement insuffisante}

			L242-1 alinéa 5 cCA

			\begin{itemize}
				\item \textbf{La possibilité pour l'assuré d'engager les dépenses nécessaires}
				\item \textbf{La majoration de l'indemnité versée}
			\end{itemize}

		\subsubsection{La sanction en cas de défaut de notification du rapport préliminaire}

			Les textes prévoient que l'ado doit notifier .. le rapport préliminaire à la victime. La jurisprudence à dcouvert : l'impossibilité pour l'assureur de refuser sa garantie

			 \jurisCourDeCas{\civTrois*}{24/9/2013}

	\subsection{Les action récursoires de l'assureur dommages-ouvrage}

		L'ado qui indemnise la victime est subrogé dans ses droits et peut exercer une action récursoire contre les constructeurs responsables et leurs assureurs.

		Dans la mesure où il y a subrogation, il récupère tous les droits et actions, il pourra agir. Quand bien même .. il pourra récupérer auprès des constructeurs --- et peut être --- il pourra agir sur fondement des \garSpec ou sur le fondement contractuel \jurisCourDeCas{\civTrois*}{13/7./2016}

		Il pourra obtenir le remboursement de toutes les sommes versées sauf celles qui lui sont imputable : les pénalités.

		CONVENTION CRAC. elle prévoit également une absence de recours

\section{L'assurance de responsabilité civile décennale}

	Elle est visée au L 242-1 et al du CA

	\subsection{Le domaine de l'assurance de responsabilité civile décennale}



		\subsubsection{Les personnes assujetties}

			Doivent être couverte par une \rcd  toute personne physique ou morale dont la responsabilité
			archi
			entrepreneur
			techniicien
			vendeur apres acheveement (qu'il est contruit ou fait construire)
			promoteur 1931-1
			le vendeur à c (CNR)
			contructeur de
			fabricant d'epers

			sous traitant mais assurance facultative

		\subsubsection{Les activités et les procédés techniques d'exécution déclarés}

			L'assurance ne couvre que les travaux afférant aux secteurs d'activité déclarés.

			lorsqu'il va

			néanmoins la \CourDeCas a jugé que ... les trvx implique carrelage.

			l'assureur ne couvre que les procédés techiques d'exe décalrés. En l'espèce, \civTrois* a considéré que l'entre ... n'est pas couverte

		\subsubsection{L'étendu de la garantie}

			\paragraph{Le quantum de la garantie}

				... couvre le coût de nature décennale affectant l'ouvrage à la réalisation duquel l'assuré a contribué. cela exclu les préjudices annexes\index{préjudices annexes}

				\subparagraph{Franchise}\label{rcFranchise} Il est possible de prévoir des franchises, mais elles sont inopposables à la victime.
				De sorte que la victime sera indemnisé de la totalité de son préjudice, et l'assureur se retournera contre son assuré pour obtenir le remboursement de la franchise.

				\subparagraph{Plafonds} Il est possible de prévoir un plafonds sauf si ... (idem DO renoie), le montant etc. ou à 150M si > 150 M R243-3 CA

				\subparagraph{Exclusion de garantie} : clauses types A 241-1 et 1 243-1 du CA. ces clauses prévoient qu'il n'est pas possible d'exclure du champ de garantie certains type de trvx (contructeur de mi par exemple) ou certains types d'ouvrage (par exmeple les constructions ouvertes). Il n'est pas non plus possible d'exclure certaine techiques de construction. De même une clause qui aurait pour effet d'exclure certains types de sinistre serait réputée non écrite (exemple : limité aux seuls défauts de solidité). c'est également valable pour certains type de contrat, ainsi ... car elle aurait pour effet ...

				Le législateur a voulu assurer la garantie la plus large possible.

				Il n'est pas possible autre que celles L 113-1 CA ... Les exclusions de garanties sont opposables à la victime.

				\subparagraph{La déchéance de garantie}\label{rcDecheance} Elle est prévue par les clause types qui prévoient que << l'assuré est déchu en cas d'inobservation inexcusable des règles de l'art >>. La déchéance n'st pas opposable à la victime.

			\paragraph{L'application de la garantie dans le temps}



				\subparagraph{Le point de départ de la garantie}

					La garantie s'applique aux trvx aynat fait l'objet d'une ouverture de chnatier pendant.
					Il s'agit d'une date unique applicable à l'ensemble des travx de constru.
					Ce sont les clausetiers qui précisent à quoi correpondent.

					Une distinction doit être opéré selon PC ou non. Si PC = DOC, sinon soit date du prmeir OS soit à défaut de la date efective de commencement des trvx.

					en pratique certains locateurs d'ouvrage commencent avant la date d'ouverture, ou postérieurement. Dans ces cas là les clause type précisent que pour l'assuré post la date à prendre en compte est celle où il commence effectivement. Pour l'assuré qui réalise ses prestations antérieurement et qui a cessé sont activité à cette date, date à prendre ne compte est celle de la signature de son marché ou à défaut la date de tout acte pouvant être considérée comme le point de départ de ses prestations.

				\subparagraph{L'expiration de la garantie}

					dernier alinéa

					<< Tout contrat d'assurnce est réputée comporter une clause >>

					de sorte qu'il y a un maintient de la garantie ... sans prime subséquente. La prme est calculée en tenant compte de ce maintient de la garantie.

	\subsection{La mise en œuvre de l'assurance de responsabilité civile décennale}


		\subsubsection{La mise en œuvre de l'assurance de responsabilité civile décennale par le \Mo}

			C'est l'hyp de l'action directe. Action dircete de la victime à l'encontre. le \Mo peut actionner directement sans qu'il soit nécessaire de mettre en œuvre la do; Il peut également sans nécessairement avoir à mettre en cause l'assuré. \jurisCourDeCas{\civTrois*}{7/11/2013}

			\jurisCourDeCas{civ.}{28/3/1939} \jurisCourDeCas{civ. 2\ieme{}}{11/6/2009}

			la \CourDeCas a également admis que la victime puisse au-delà du délai de garantie, tant que l'assureur reste exposé au recours dès lors que valablement actionné. C'est-à-dire ...Le \Mo peut donc agir  2 ans à compter de la réclamation\footnote{\jurisCourDeCas{\civUn*}{11/03/1986} ; \jurisCourDeCas{\civDeux*}{13/9/2017} ; \jurisCourDeCas{\civTrois*}{26/11/2003}} Ainsi lorsque

			\textbf{Attention} : un arrêt sévère du \printdate{29/3/2018} de la \CourDeCas \civTrois* qui a jugé que l'assignation de l'assureur en sa qualité d'ado n'interromp pas ... en sa qualité dc

		\subsubsection{L'action récursoire de l'assureur dommages-ouvrage contre l'assureur de responsabilité civile décennale}

			L'ado est subrogé

			l'action doit impérativement exercé dans le mê délai. Les réglement entre assureur CRAC

		\subsubsection{La mise en œuvre de l'assurance de responsabilité civile décennale par un constructeur}

			\paragraph{La mise en œuvre de l'assurance par le constructeur assuré}

				... peut se retourner contre son propre assureur.

				Soit il attend d'être condamné soit, dans le cadre de la même de sorte que in solidum

				Il doit déclarer le sinsitre dans le délai prévu par les polices , sans que ce délai puisse être inférieur à 5 jours. Néanmoins un déclaration tardive ne peut pas entrainer la déchéance des droits de l'assuré sauf si l'assureur parvient à démontrer que le retard slui a causé un préjudice (ex: l'a privé de ses recours).

				le constructeur doit intenter L114-1 CA : 2 ans à compter (au fonds ou en expertise)

			\paragraph{La mise en œuvre de l'assurance par un constructeur autre que l'assuré}

				Le constructeur qui a indemnisé le \Mo, suite à une condamnation \emph{in solidum}, peut se retourner . Il pourra obtenir le remboursement de tout ou partie en fonction du partage de responsabilité arrêté par le juge.

				Le constructeur devra diviser ses recours

	\subsection{Les actions récursoires de l'assureur de responsabilité civile décennale}

		L'assureur de responsabilité civile peut se retourner contre son assuré :
		\begin{itemize}
			\item en cas de déchéance (cf. \ref{rcDecheance}) ;
			\item en cas de franchise (cf. \ref{rcFranchise}) ;
		\end{itemize}

	1:42 Il peut également ... notamment sur le fondement des vices cachés. il est subrogé dans les droits de son assuré.
