% !TEX root = ./droitConstruction.tex

\chapter{La responsabilité des fabricants}

Ne traite que ... autre que les fabricants d'EPERS\index{EPERS}

\section{La responsabilité contractuelle des fabricants}

	\subsection{La responsabilité des fabricants à l'égard du \Mo}

		\subsubsection{La garantie légale de conformité}

			 L ??? code consommation

			Elle ne s'applique que conommateur / pro.  Est assimilé à un veneur pro la personne qui conclut un c L217-1 ode consommation

			Elle gar

			Il faut que le défaut existe L217-7 présume existant au moment de la délivrance les défaut de conformité qui apparaissent dans un délai de 24 mois à partir de la délivrance  du bien.

			L'action doit être introduite dans les deux ans à compter de la délivrance L217-12. Le ... peut choisir le mode de réparation qui lui parait le plus approprié : ... ; sauf à c que le cout soit manifestement disproportionné. L217-10 si la répartion ou le remplaceemnt sont impossibles, l'acquéreur à le choix. Le tout sans préjudice d'éventuels dommages et intérêts L217-11.

			La garantie légale de conformité laisse subsitet la garantied es vices cacahs.

		\subsubsection{La garantie des vices cachés}

			1641 et suivants du cciv. Le \Mo à l'encontre du fabriquant auquel il est contractuellement lié, mais a jurisprudence\footnote{civ 3} a admis qu'il puisse agir sur le fondement à l'encontre du cocontractant du \lo.

			Le vice doit rendre la chose impropre à l'usage à laquelle on la destine, ou diminuer l'usage que l'acheteur ne l'aurait pas acquise ou en aurait donner un moindre pris.

			Se prescrit délai 2 ans à compter de la découverte du vice.

			l'acheteur à le choix entre garder la chose et se faire restituer une partie du prix ou redndre la  chose et se faire restituer le prix. Si le vendeur connaissit les vices d ela choses, il peut etre tenu, outre la restitution du prix,  des dommages et inyerets.

		\subsubsection{Le défaut de conformité}

			à l'encontre du contractuellement lié mais a jurisprudence\footnote{civ 3} a admis qu'il puisse agir sur le fondement à l'encontre du fabricant  cocontractant du \lo.

			S'entend d'une différence de qualité ou de nature des

			c'est au \Mo de prouver la non conformité

			5 ans à compter de la décourevet de la non conformité. Dans hypo cocontract, le point de départ est fixé à la livraison à l'\E civ 3 7/6/2018

			Attention, dès lors que le défaut rend impropre 3 civ garantie vice caché

		\subsubsection{Le manquement à l'obligation de conseil}

			Le \Mo .... le fabriquant est débiteur d'une obligation de conseil.

			c'est au débiteur d'apporter la preuve qu'il a respecter son obligation.

			Le fabriquant doit conseiller au \Mo l'emploi d'un produit adapté à ses possibilité et à ses besoins. Il doit fournir toutes les info. nécessaires sur les modalité de mise en œuvre.

	\subsection{La responsabilité des fabricants à l'égard des locateurs d'ouvrage}

		L'action des \lo à l'égard des locateurs d'ouvrage est nécessairement de nature contractuelle.

		elle peut etre fondée sur :
			\begin{itemize}
				\item la garantie des vices cachés,
				\item le défaut de conformité,
				\item le manquement à l'obligation de conseil.
			\end{itemize}

		Attention si le fabricant s'immisce et prend une part active au point que son action est assimilée à celle d'un \Moe dans ce cas 1792 c cas 3 22/2/2018

\section{La responsabilité du fabricant du fait des produits défectueux}

	1245 cciv et suivants. Les fabricants de produit mobilier, qu'il s'agisse ... dès lors que leur responsabilité ne peut être recherchée sur le fondement de l'article 1792-4 et qu'ils ne sont pas fabricant d'EPERS

	Cf. ce qui a déjà été dit...
