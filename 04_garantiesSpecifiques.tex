% !TEX root = ./droitConstruction.tex

\chapter{Les garanties spécifiques des constructeurs}

Les \garSpec sont : la décennale, la biennale et la \gpa.

\section{Le point de départ des garanties : la réception}

	\index{Reception@Réception!Definition@Définition}\label{Reception@Réception}\begin{citationArticleCciv}{1792-6}
		La réception est l'acte par lequel le maître de l'ouvrage déclare accepter l'ouvrage avec ou sans réserves. Elle intervient à la demande de la partie la plus diligente, soit à l'amiable, soit à défaut judiciairement. Elle est, en tout état de cause, prononcée contradictoirement.

		\lips
	\end{citationArticleCciv}

	Il ne faut pas confondre la réception et la livraison. La livraison correspond à la prise de possession de l'ouvrage, c'est une notion que l'on retrouve surtout en \VEFA ou \aCompleter.

	La réception intervient entre le \Mo et le locateur d’ouvrage. Le \Mo déclare accepte l’ouvrage avec ou sans réserve. A l’amiable ou à défaut judiciairement, et en tout état de cause contradictoirement.

	A l’absence de réception, pas possible de mobiliser la retenue légale de garantie qui est conditionnée avec la réception avec réserve.


	En l’absence de réception, seule la responsabilité contractuelle de droit commun peut être mobilisée, 5 ans à compter de la survenance du dommage. Il s'agit d'une responsabilité de nature contractuelle nécessitant en principe : faute, préjudice et lien de causalité. La \CourDeCas considère cependant que les entrepreneurs sont tenus d’une obligation de résultat.


	Le \Moe est tenu d’une obligation de moyen donc démonstration d’une faute.


	\jurisCourDeCas{\civTrois*}{24/05/2006} : prescription de 10 ans à compter de l’apparition des dommages avant réception, mais depuis réforme de la prescription donc à priori 5 Ans.


	\subsection{Les modalités de la réception}

		\subsubsection{Le moment de la réception}

			La réception intervient à l’achèvement des travaux mais il ne s’agit pas d’une condition de la réception, il est possible de réception des travaux achevés \jurisCourDeCas{\civTrois*}{30/01/2019}


			L’intérêt est de faire jouer les assurances, \do ou responsabilité civile décennale.


			Lorsqu’il y a succession d’entreprise sur le même lot, il est recommandé à l’entreprise de provoquer la réception pour ventiler les responsabilités par rapport au prédécesseur. S’il accepte l’ouvrage en l’état alors il engage sa responsabilité.


			\paragraph{La réception est unique.}

			La question se pose de savoir si la réception est unique pour l’ouvrage en son entier ou pour chacune des entreprises.


			Il s'agit d'une question délicate car le marché peut être conclu en lot séparé.


			Si le marché est << tout corps d’état >>, alors l'unicité de la réception ne fait pas de difficulté. Il convient cependant de préciser dans le contrat \ST* que la réception des travaux du \ST interviendra avec celle de l'/ep.


			\medbreak Si l’ouvrage est réalisé par tranche, la \JP admet la réception par tranche, mais unique par tranche.


			La difficulté apparait lorsque le contrat est conclu << corps d’état séparé. >> Au regard de l’esprit du texte, il n'est \emph{a priori} pas possible de réaliser une réception par locataire d’ouvrage alors réception distincte pour chaque type de travaux.


			Néanmoins, la norme AFNOR P0301 admet la réception par entrepreneur sauf en cas d’entrepreneur groupé.


			La réception partielle par lot n’est pas prohibée par la loi. En revanche, la \civTrois a jugé en 2017 que la réception partielle à l’intérieur d’un lot n’est pas possible.


			\medbreak La \CourDeCas a admis que si les travaux sont réalisés par palier successif, il est alors possible réceptionner par palier. \aValider



		\subsubsection{Le formes de la réception}

			L'\articleDu{1792-6}{\cciv} prévoit que la réception intervient à la demande de la partie la plus diligente soit à l’amiable soit judiciaire.


			Il y a donc deux type de réception prévue par la loi : expresse ou judiciaire.

			Par ailleurs, la jurisprudence qui a découvert une troisième forme de réception dite tacite

			\paragraph{Réception expresse}\index{Reception@Réception!Expresse@Réception expresse}\label{receptionExpresse}

			\subparagraph{Contenu}

			Les parties manifestent expressément dans un document écrit leur volonté non équivoque d’accepter l’ouvrage tel qu’il a été réalisé.

			La réception est prononcée entre locataire d’ouvrage et le \MO --- ou une personne mandatée spécifiquement par celui-ci.

			En pratique l’entrepreneur et le \MO (souvent assisté au maître d’œuvre) se rendent sur place et vont faire le tour de l’ouvrage et lister dans document écrit les réserves : non façon, malfaçon et vices de construction

			\subparagraph{Caractère contradictoire}

			L’exigence de la contradiction ne nécessite pas la signature formelle du \PV, dès lors que la présence est bien établie lors de la réception.

			La \CourDeCas a pu jugé que l’absence de l’entrepreneur ne saurait priver du caractère contradictoire le \PV pour le \MO, s’il a été valablement convoqué aux opérations de conception\footnote{\jurisCourDeCas{\civTrois*}{7/3/2019} : une \LRAR 4 jours avant la réception avec en plus télécopie suffit à considérer le caractère contradictoire}.

			\subparagraph{Date de la réception expresse}


			Elle figure sur le document établi, le procès-verbal.

			Donc la réception sera prononcée au jour apposé sur le \PV. On peut tout à fait rétroagir la réception, en indiquant une date antérieure à la date d'élaboration du procès-verbal.

			Des aménagements conventionnels de la réception expresse sont également envisageables.

			Ces aménagements ont lieu généralement au moment de la conclusion du contrat de louage d’ouvrage, mais ils aussi possible par avenant.

			Cela peut consister à une pré-réception \CAD une première visite --- les OPR opération préalable à la réception --- pour qu’il y ait au moment de la réception beaucoup moins de réserves.

			\medbreak Autre possibilité : aménager la réception expresse en cas de refus de l’une des parties. L’idée est d’éviter la situation de blocage. On a une partie qui veut réceptionner l’ouvrage, si le contrat prévoit qu’en cas de convocation de l’autre partie par voie d’huissier la réception pourra intervenir et en cas de procès-verbal de constat dressé par huissier notifié à l’autre partie alors point de départ de la réception est la notification du constat.

			\paragraph{Réception tacite}\index{Reception@Réception!Tacite@Réception tacite}\label{receptionTacite}

			Il s'agit d'une création prétorienne non visée par le texte. C’est une réception amiable, non contentieuse, mais les parties ne savaient pas qu’il fallait réceptionner.

			Dans cette hypothèse, la \CourDeCas va constater la réception tacite. Elle ne fait que constater que la réception a eu lieu, ce qui est différent de la réception judiciaire qui est prononcée.

			Le juge recherche si le \MO par son comportement a manifesté sa volonté non équivoque de recevoir l’ouvrage. La réception tacite n’est pas envisageable lorsque le contrat impose une réception expresse.

			\subparagraph{Conditions d'existences}


			le juge va chercher à caractériser la volonté non équivoque du \MO d’accepter l’ouvrage\footnote{\jurisCourDeCas{\civTrois*}{13/07/2017}}.

			L’appréciation du comportement du \MO relève de l’appréciation souveraine des juges du fonds avec un faisceau d’indice :
			\begin{itemize}
				\item La prise de possession est un élement important mais non déterminant. La simple prise de possession ne suffit pas à caractériser la réception tacite.

				\item Le paiement de l’intégralité du prix.

				\item La manifestation de la volonté du \MO de déterminer les travaux.

				\item L’envoi d’une lettre de réclamation.
			\end{itemize}

			La prise de possession des lieux et le paiement du prix présume la volonté non équivoque du \MO de réceptionner l’ouvrage\footnote{\jurisCourDeCas{\civTrois*}{20/04/2017} paiement de \pourcent{95} du prix (comme la retenue légale de garantie)}.

			A défaut de prise de possession et de paiement intégral du prix la réception tacite ne peut être caractérisée\footnote{\jurisCourDeCas{\civTrois*}{13/07/2017}}.

			La réception tacite d’un \MO d’un immeuble d’habitation n’est pas soumis à la constatation de l’habitabilité et l’achèvement de l’ouvrage. %25/01/2011

			\subparagraph{Date de réception}

			IL appartient au juge de fixer la date. Le juge doit fixer la date de la réception tacite, une fois constaté la volonté non équivoque de réceptionner l’ouvrage.

			Généralement, à la date de prise de possession de l’ouvrage, ou à la date du paiement intégral, date d’envoi des réserves. Les parties peuvent conventionnellement fixer la date de la réception tacite

			L'absence de contestation dans un délai de 2 mois  vaut réception tacite\footnote{\jurisCourDeCas{\civTrois*}{04/04/2019}}.


			\paragraph{Réception judiciaire}


			En cas de situation conflictuelle entre \MO et locataire d’ouvrage, entrainant, par exemple Le \MO refuse de se rendre à la réunion de réception.

			Dans ce cas là, la partie la plus diligente a la faculté de saisir le juge pour prononcer la réception. Sachant que le refus n’a pas besoin d’être abusif, il faut juste un refus

			Seule les parties au contrat ont qualité pour demander la réception judiciaire.

			Très souvent le juge ordonnera une expertise et demande à l’expert de déterminer si l’ouvrage est en état d’être reçu\footnote{\jurisCourDeCas{\civTrois*}{12/10/2017} : un ouvrage en état d’être reçu est ouvrage utilisable conformément à sa destination et qui n'est pas être affecté de défaut ou vice substantiel. Il s'agit d'une noton différente de celle de l'achèvement de l’ouvrage, \jurisCourDeCas{\civTrois*}{18/10/2018} : il suffit que l’ouvrage soit en état d’être habité} et à quel date.

			Elle peut être prononcée avec ou sans réserve\footnote{\jurisCourDeCas{\civTrois*}{17/10/2019}}.

			\textbf{Date de la réception} : date proposée par l’expert à laquelle l'ouvrage est en état d’être reçu, ou en état d’être habité, ou date d’abandon de chantier, ou date d’entrée dans les lieux, cela est apprécié par le juge.

			Les dispositions applicables au \CCMI n’empêchent pas une réception judiciaire\footnote{\jurisCourDeCas{\civTrois*}{21/11/2019}}.

	\subsection{Les effets de la réception à l'égard des désordres affectant l'ouvrage}



		\subsubsection{Les désordres apparents réservé}

		Lorsqu'il y a les désordres apparents réservés --- que la récetion soit express, tacite ou judiciaire, la formulation de réserve permet de mettre en oeuvre la retenue légale et d’agir sur le fondement de la garantie de parfait d’achèvement.

		A l’origine les dommages réservés n’étaient repris que sur la base de la garantie de parfait achèvement d’un an.
		Il n’était pas possible la biennale et décennale et RC.

		La \CourDeCas a admis de cumuler la responsabilité de droit commun concurremment avec la garantie de parfait achèvement. Pour les autres \garSpec c’est un principe de non cumul.

		Ce cumul est permis car le délai d’un an de la \gpa est trés court alors après le délai 10 an possible d’agir sur le \RC pour les désordres réservés à la réception.

		La garantie de parfait achèvement vise la reprise des seuls dommages. Les dommages annexes seront pris en charge par la \RC, avec la preuve d’une faute.
		La \civTrois ajugé que les locateur d’ouvrage sont tenus d’une obligation de résultat, pour les désordres réservés jusqu’à la levée des réserves, même pour la contractuelle.

		Pour le parfait achèvement il y a une présomption de responsabilité, donc même pour la RC.
		Deuxième évolution favorable au \MO

		Possibilité d’invoquer la responsabilité décennale pour les désordres réservés à la réception qui se sont révélés postérieurement dans leur ampleur et leur importance, Civ 3°12/10/1994

		\textbf{Attention} : il n'y pas d’assurance responsabilité décennale pour les désordres réservés.
		Assurance \DO peut couvrir en cas de \MED de l’entrepreneur.


		\subsubsection{Les désordres apparents non réservés}

			Ils sont alors purgés. Aucune action sur le fondement de la garantie parfait achèvement, ni sur le biennal, ni sur la décennale, si RC car la C.Cass considère que ce désordre est purgé (possible manquement à la mission d’assistance de l’architecte ou Maître d’œuvre, si mission spécifique d’assistance aux opérations de réception.)

		\subsubsection{Les désordres cachés}

		Lors de la réception, s’ils sont cachés ils ne peuvent être réservés.

		La réception constitue le point de départ des désordres intermédiaires, décennale, parfait achèvement, biennal, RC
		Si les désordres se manifestent dans l’année qui suit la réception, possible sur le parfait achèvement qui se cumule avec RC ou biennale ou décennale si la gravité le permet.
		Si le désordre se manifeste Réception, alors sur le fondement biennal, décennale et RC. Alors la réception marque le point de départ de la mobilisation de l’assurance.


\section{La garantie décennale}

	1792 et s du Code civil

	\subsection{Les conditions de mise en œuvre de la garantie décennale}

		\subsubsection{Les conditions communes aux trois garanties spécifiques}\index{GarantiesSpecifiques@\garSpec!ConditionsCommunes@Conditions communes}\label{garantiesSpecifiquesConditions}

			Il y a quatre conditions communes aux trois garanties :
			\begin{enumerate}
				\item un ouvrage de construction,
				\item un ouvrage recu,
				\item un dommage caché,
				\item un lien d'imputabilité.
			\end{enumerate}

			\paragraph{Un ouvrage de construction}

				1792 : tout constructeur d’un ouvrage est responsable

				\subparagraph{La définition de la notion d'ouvrage}\index{Ouvrage}

				La définition est jurisprudentielle.

				D’une construction sur le sol ou en sous-sol. Notion plus large que celle d’édifice ou de bâtiment. Il doit s’agir d’une construction ancrée dans le sol, nature immobilier

				Ex : mobil home, tout dépend s’il est ancré sur le sol, la chape sera un ouvrage
				Péniche : pas ouvrage, abri piscine repliable non plus

				L’ouvrage doit donc avoir une nature immobilière, et relever une technique de construction (différent de la technique industrielle notamment pour les éléments d’équipement

				Certains éléments d’équipement peuvent être considérés comme des ouvrages (chaudière, cheminée)

				\subparagraph{La problématique des travaux sur existant}

				PPE / de la responsabilité contractuelle de droit commun car si on est au-delà de 10 ans sur les existants pas de décennale
				EXCEPTION : JP a admis dans certaines hypothèses les dommages sur existant des garanties décennales
				1.	La garantie décennale peut trouver à s’appliquer quand les travaux sur existants constituent en eux même un ouvrage comme des travaux de rénovation lourde qui modifient la structure, redistribue les pièces, et donc constitue un ouvrage alors entraine décennale.
				3 hypothèses :
				i.	Travaux de rénovation lourdes
				ii.	Travaux avec apports d’éléments nouveaux : travaux qui nécessitent apport de matériaux ou éléments nouveaux, constituent en eux même des ouvrages entrainant la décennale
				iii.	Ravalement étanche constitue eux même un ouvrage entrainant responsabilité décennale ( il y a des décisions du juge du fond qui apprécie l’imperméabilisation comme un ouvrage ce qui n’est pas la position de la C.Cass)
				2.	Décennale couvre désordres causés aux existants s’ils sont indissociables des travaux neufs et si les désordres des existants sont la conséquence des désordres aux travaux neufs. Dès lors si on ne peut pas déterminer la cause exacte des dommages, alors la décennale joue.
				3.	Si on ne peut pas déterminer si les désordres proviennent de l’existant ou des travaux neuf
				4.	Désordre d’origine dès lors qu’il porte sur un élément d’équipement installé sur existant lorsqu’il porte atteinte à la destination de son ouvrage 14/06/2017 ; 14/09/2017 ; 26/10/2017
				Ex : cheminée : avant il fallait que la cheminée soit un ouvrage pour entrainer décennale sur l’existant


			\paragraph{Un dommage}

				\subparagraph{Un dommage à l'ouvrage}

				Il faut un dommage dont la cause et l’origine sont indifférentes, alors même qu’il provient d’un vice du sol.

				Etude de sol obligatoire aux actes de vente, mais pas de sanction

				Le constructeur engage alors sa responsabilité.

				Si cela vient d’un défaut de conformité ou d’une faute, c’est bien mais c’est indifférent.

				Il est indifférent que la cause des désordres n’ait pu être déterminée avec précision (différent de l’imputabilité)

				La non-conformité n’entre pas le champ de la garantie décennale alors responsabilité contractuelle de droit commun. Toutefois, dans un sens favorable au MO si la conformité a pour conséquence une atteinte à la destination ou à la solidité de l’immeuble alors action décennale possible.
				Ex:: menuiserie extérieure en bois au lieu de PVC et donc infiltrante.

				\subparagraph{Un dommage caché à la réception}

					Le dommage doit donc être caché, c’est à l’entrepreneur de rapporter la preuve du caractère apparent du dommage, en ce qu’il doit être apparent aux yeux d’un profane dans son ampleur et ses conséquences.

				\subparagraph{Un dommage apparu dans le délai de la garantie}

					Le dommage doit apparaître dans le délai de la garantie que la victime souhaite invoquer.

					\begin{description}
						\item[Les dommages futurs] Dommage qui est judiciairement dénoncé à l’intérieur du délai décennal, qui n’a pas la gravité décennale au jour où il est judiciairement dénoncé mais qu’il atteindra à coup sûr dans le délai décennal.

						C.Cass Civ 29/01/2003 : le dommage doit atteindre à coup sûr la gravité décennale dans le délai décennal.
						C.Cass Civ 3° 23/10/2013 ou 28/02/2018


						\item[Les désordres évolutifs] Celui qui est apparu après le délai de 10 ans, est la conséquence du désordre dénoncé dans le délai décennal.

						La C.Cass admet que l’assignation initiale interrompt pour les dommages dénoncés dans l’assignation mais aussi dans leur aggravation postérieure
						-	Désordre initial dénoncé dans le délai de garantie
						-	Désordre initial doit avoir atteint la gravité décennale dans le délai de la garantie
						-	Nouveau désordre doit constituer une aggravation, une suite ou une conséquence du désordre initial donc ce qui exclut les désordres nouveaux sans lien avec le précédent
						Civ 3° 2006 : il faut donc que le siège du dommage soit le même ouvrage
						En pratique ils sont peu nombreux

					\end{description}




			\paragraph{Un lien d'imputabilité entre le dommage et l'activité du constructeur}

			La mise en œuvre des responsabilités spécifiques du constructeur, il faut établir un lien d’imputabilité Civ 3°  20/05/2015

			Il faut une imputabilité du dommage à l’activité du constructeur. Différent du lien de causalité qui renvoie à la faute


		\subsubsection{La gravité décennale}\index{GraviteDecennale@Gravité décennale}\label{graviteDecennale}

		Le dommage doit soit affecté la solidité de l’ouvrage ou le rendre impropre à sa destination, ou atteindre la solidité d’un élément d’équipement indissociable

		L’origine est indifférente (vice de construction ou vice de sol), une de ces conditions suffit.


			\paragraph{Une atteinte à la solidité de l'ouvrage}

			Qui remet en cause la sécurité physique de l’ouvrage, fissure structurelle, les fissures esthétiques rentrent dans le champ de la RC de droit commun, ou possible atteinte à la destination de l’ouvrage alors décennale

			Généralement l’atteinte à la solidité ne fait pas débat


			\paragraph{Une atteinte à la destination de l'ouvrage}

			L’ouvrage doit être impropre à sa destination.
			-	L’atteinte doit être appréciée au regard de la destination première de l’ouvrage, CAD celle à laquelle on peut raisonnable s’attendre. Une maison d’habitation doit être habitable
			-	Au regard de la destination contractuelle, CAD décidée par les partie Civ 10/10/1992
			Les constructeurs ont l’obligation de se renseigner sur la destination exacte de l’ouvrage.
			L’impropriété à la destination ne suppose pas que le risque soit déjà réalisé, à cet égard la JP a considéré que l’impropriété peut être régularisé en cas de risque d’inondation rendant impropre à la destination
			En cas de mauvais fonctionnement de ballon d’eau chaude dans une prison / Le non respect des règles parasismique Civ 3° 11/05/2011
			Mais Civ 3° 05/07/2018, si les juges du fonds que le défaut des normes parasismique ne peut être engagée si le risque n’est pas déterminé
			-	Même si les désordres n’affectent une partie de l’ouvrage à sa destination
			-	Alors même que les normes réglementaires sont respectées
			C.Cass 27/10/2006 21/09/2011, le respect des normes légales et réglementaires n’empêchent pas la caractérisation du désordre décennale entrainant une impropriété
			-	Atteinte à la destination de l’élément d’équipement : Indifférent que l’élément soit dissociable ou non, et indifférent si existant ou non (à vérifier)
			17/06/2017 ; 14/09/2017 ; 26/10/2017

			En matière de performance énergétique, impropriété retenue que si dommage lié au produit, à la conception et mise en œuvre de l’ouvrage L111-13-1 du CCH, que si surconsommation est strictement encadré par ces destinations.
			Ex : erreur d’implantation, règles de sécurité, défaut de nivellement d’un terrain de tennis, dysfonctionnement d’une pompe à chaleur.


			\paragraph{Une atteinte à la solidité d'un éléments d'équipement indissociable de l'ouvrage}

			13/02/2020 : refus de qualification d’élément d’équipement à un enduit de façade dès lors qu’il n’est pas destiné à fonctionner (on avait cette condition sur le biennal, c’est la première fois en décennale)
			Donc désormais 1792-2 ne peut s’appliquer que si l’élément d’équipement en cause est destiné à fonctionner (mue par un dynamisme propre)
			Elément doit être indissociable de l’ouvrage 1792-2 al 2 : indissociabilité quand la dépose, démontage ou remplacement ne peut s’effectuer sans détérioration de l’ouvrage.
			Ex : un carrelage fixé avec mortier c’est indissociable, avec colle c’est dissociable
			Radiateur scellé dans le mur c’est indissociable, juste fixé alors dissociable.
			L’élément ne doit pas avoir une fonction professionnelle, car les éléments à vocation professionnelle soit exclue de la décennale.


	\subsection{Les redevables de la garantie décennale}

		\subsubsection{Les constructeurs}

			\paragraph{Les personnes réputées constructeurs}

			L'\articleDu{1792}{\cciv} répute constructeur trois type de personnes :
			\begin{enumerate}
				\item Les personnes liées au \MO par un contrat de louage d’ouvrage : architecte, entrepreneur, technicien, contrôleur technique, \etc

				\textbf{Attention} L'\articleDu[L]{111-24}{\cch} stipule que le contrôleur technique n'est soumis que dans les limites de sa mission \CAD qui ne peut être tenu que s’il est prouvé que le fait à l’origine du dommage entrait dans le champ de ses missions et il ne peut être tenu vis-à-vis des constructeurs qu’à concurrence de sa part de responsabilité donc en cas de condamnation \emph{in solidum}. Il ne supporte donc pas l’insolvabilité des autres.

				Les sous-traitants ne sont pas visés car ils ne concluent pas directement avec le \MO, et ne sont ni réputés ni assimilés constructeur.

				\item Les personnes vendant après achèvement un immeuble qu’elles ont construit ou fait construire (exclu VEFA \aValider) cela vise « le castor » ou faut construire et revend dans un délai de 10 ans.

				Pour le castor, le point de départ ne sera pas la réception, mais l’achèvement des travaux attesté par tout moyen comme la date à laquelle l’ouvrage était utilisable et propre à sa fonction, la DAT peut être un moyen.

				Le fait que les désordres étaient apparents ou connus au moment de la vente, car ce qui compte est la réception ou de l’achèvement, le vendeur ne peut pas se prévaloir d’une clause de garantie car l'\articleDu{1792}{\cciv} est d’ordre public, alors clause réputée non écrite

				La \JP considère que la garantie décennale n’est pas exclusive de la garantie des vices cachés.

				\item Personne bien qu’agissant en qualité de mandataire du propriétaire de l’ouvrage, accomplisse une mission assimilable à un locateur d’ouvrage. Ici pour palier la fraude de personne qui se réfugie sous un mandat, comme le promoteur de l'\articleDu{1831-1}{\cciv}, le gérant d’une société d’Attribution, \etc

				De même le \MO délégué peut voir sa responsabilité décennale engagée s’il s’est comporter en locateur d’ouvrage.

				Est réputé constructeur le constructeur de maison individuelle.
			\end{enumerate}



			\paragraph{Les personnes assimilées constructeurs}

			Le vendeur d’immeuble à construire : est assimilé constructeur, responsabilité biennale, décennale à l’égard de l’acquéreur

			Le vendeur d’immeuble à rénover : L262-3 al 3 CCH, débiteur de la biennale et décennale


		\subsubsection{Les fabricants d'éléments pouvant entrainer la responsabilité solidaire}

		La C.Cass appelle à la suppression de ces dispositions, renvoi aux ouvrages pour ce passage.

			\paragraph{La notion d'EPERS}

				\aCompleter

			\paragraph{Mise en œuvre et nature de la responsabilité}

				\aCompleter

	\subsection{Les bénéficiaires de la garantie décennale}

		\subsubsection{Le \Mo}

			C’est la personne physique ou morale pour le compte de laquelle les travaux sont réalisés

		\subsubsection{L'acquéreur de l'ouvrage}

		Il est expressément visé par 1792 du Code civil. Bien qu’il ne soit pas visé par 1792, les sous-acquéreurs sont également visés tant que le délai décennale n’est pas expiré, ce qui donne qualité et intérêt à agir.

		Sauf clause contraire, l’acquéreur peut agir pour les désordres apparus avant l’acquisition. L’intérêt est d’être propriétaire au jour de l’action. L’acquéreur de l’ouvrage aura alors intérêt et qualité.

		Sauf clause contraire où le vendeur se serait réservé le droit d’agir contre le constructeur, notamment si le MO pour vendre le bien a procédé aux réparations, s’il prouve avoir supporté les conséquences du dommage dans son patrimoine (trouble de jouissance, perte spécifique comme financière)


		\subsubsection{Les copropriétaires et le syndicat des copropriétaires}

		PPE :
		-	Syndicat est compétent pour solliciter la réparation des dommages affectant les parties communes, et non pas le syndic. Le syndic doit être habilité pour agir en justice, sauf urgence comme expertise. Depuis 2018, les tiers ne peuvent plus invoquer le défaut d’habilitation désormais seuls les copropriétaires peuvent invoquer ce défaut d’habilitation
		-	Chaque copropriétaire est habilité à intérêt et qualité, compétent, en cas de dommage affectant ses parties privatives

		TEMPERAMENT
		-	Le syndicat des copro peut solliciter la réparation des désordres des parties privatives si par la généralisation ils affectent la conservation de l’immeuble Civ 3° 12/05/1993

		Le syndicat peut aussi solliciter la réparation des dommages des parties communes et privatives de manière indivise. Dans cette hypothèse l’intérêt du délai par le syndicat profite aux copropriétaires pour la réparation des préjudices personnels Civ 3° 31/03/2004

		-	Chaque copropriétaire peut agir si l’atteinte au partie commune lui porte préjudice personnellement


		\subsubsection{Les locataires et les crédits-preneurs}

		LES LOCATAIRES

		Civ 3° reconnait la qualité de MO que le propriétaire de l’ouvrage et les acquéreurs successifs.
		Le locataire ne peut pas par principe agir sur le fondement de 1792 du Code civil

		Sauf dans le bail, une cession des droits sur le fondement des articles 1792 du Code civil

		A défaut de cession de droit, les seuls fondements possibles à l’encontre des constructeurs par le locataire, responsabilité contractuelle de droit commun si c’est le locataire qui a commandé et payé les travaux, délictuelle si c’est le bailleur qui a commandé et payé le constructeur.

		Le locataire bénéficie d’une action à l’encontre de son bailleur 1221 du Code civil : garantie des vices et défauts de la chose louée, le bailleur aura alors une action récursoire à l’encontre des constructeurs par garantie spécifique ou RC

		CREDIT PRENEUR : crédit bail

		Avant la levée de l’option d’achat, le crédit preneur est assimilé à un locataire donc ne bénéficie pas de 1792 et suivants, sauf cession de droit dans contrat de bail avec clause de subrogation Civ 3° 16/05/2001

		Après la levée d’option, il devient propriétaire, il peut donc agir sur le fondement de la garantie décennale tant que l’on est dans les délais.


	\subsection{Le régime de responsabilité}

	La garantie décennale est exclusive de toute responsabilité, il faut alors nécessairement agir sur ce fondement dès lors que les conditions sont remplies, et ce à peine d’irrecevabilité.

		\subsubsection{La présomption de responsabilité}

		La garantie décennale repose sur une présomption de responsabilité. Pas besoin de caractériser une faute, une absence de faute est indifférente, c’est une responsabilité de plein droit.

		Impossible de déroger à cette présomption 1792-5 : toute clause contraire est réputée non écrite.

		Les moyens de défense sont alors absence de réception, absence d’ouvrage, caractère apparent, lien d’imputabilité


		\subsubsection{Les causes d'exonération}

		Force majeur, fait d’un tiers, fait du MO
		1.	La force majeure
		Irrésistibilité imprévisibilité et extériorité. Rarement admise par la JP. Un arrêté de catastrophique naturelle ne suffit pas, sauf des calamités exceptionnelles ont été admises
		Nv critère : doit empêcher l’exécution par la force majeure On ne sait pas si cela sera plus admis ou pas par la JP
		Les effets : 1218 du Code civil si l’empêchement est temporaire, l’exécution de l’obligation est suspendue avec résolution possible. Si empêchement définitif, résolution de plein droit et les parties sont libérées de leurs obligations dans les conditions de 1351 du Code civil
		2.	Le faut d’un tiers (Tiers au chantier)
		Aucun intervenant réputé constructeur ne peut se prévaloir de la faute d’un autre intervenant pour exonérer sa responsabilité
		I ne peut pas se prévaloir des vices des matériaux mais peut agir du fournisseur en garantie
		Il ne peut pas se prévaloir de l’insuffisance des règles de l’art ou des DTU pour s’exonérer de sa responsabilité.
		Quelques exemple d’exonération : acte de vandalisme d’un tiers à l’ouvrage
		3.	Le fait du MO

		Les constructeurs invoquent souvent le fat du MO comme exonératoire mais rarement admis.
		a.	L’immixtion du MO notoirement compétent.
		Ici le MO est notoirement compétent s’immisce dans la construction de l’ouvrage.
		Il fait nécessairement un acte d’immixtion, il faut un acte positif
		Ex : le MMO choisi des matériaux et techniques de constructions. Mais s’il donne juste son accord ce n’est pas suffisant, il doit imposer les matériaux ou participation à la conception de l’ouvrage
		Le MO doit notoirement compétent CAD avoir des connaissances techniques comme un architecte
		Médecin, véto.. seront considérés comme compétents sur la partie concernant leur activité

		Conséquence de cette immixtion, il y a partage de responsabilité car l’entreprise est tenue de son obligation de conseil, dans certains cas exonération totale, soumis à l’appréciation du juge du fond.
		b.	L’acceptation délibéré des risques par le MO
		Le MO passe délibérément outre les conseils et avertissements du constructeur.
		Conditions :
		-	Le constructeur doit avoir présenté les risques dans leur ampleur et leur conséquence au MO
		-	Malgré cela le MO doit décider de poursuivre de l’opération sans tenir compte des conseils et avertissements du constructeur
		-	Pas nécessaire que le MO soit notoirement compétent, il suffit qu’il soit pleinement informé
		Ex : pose de carrelage inadéquate ou matériaux inadapté.
		Pas d’acceptation délibéré des risques si absences de recours à un maître d’œuvre, et la volonté de faire des économies n’est pas en soi une acceptation délibérée des risques
		Conséquences :
		-	Partage de responsabilité car il appartenait au constructeur de refuser les travaux qu’il savait inefficace Civ 3° 21/05/2014
		-	Exonération totale dans cas exceptionnel c’est plus rare

		c.	Mauvaise utilisation de l’ouvrage
		Elle se situe après réception, en ce cas exonération de la responsabilité du constructeur
		Ex : dépassement des charges d’un plancher
		Il y a également une absence d’entretien de l’ouvrage. A ce moment là il n’y a pas d’imputabilité


		\subsubsection{La durée de la garantie décennale}

			\paragraph{Le délai de garantie} Il est donné par l'\articleCciv{1792-4-1}.

			Cet article prévoit que << {\itshape toute personne physique ou morale dont la responsabilité peut être engagée en vertu des articles 1792 à 1792-4 du présent code est déchargée des responsabilités et garanties pesant sur elle, en application des articles 1792 à 1792-2, après dix ans à compter de la réception des travaux ou, en application de l'article 1792-3, à l'expiration du délai visé à cet article} >>. Le délai est donc de 10 ans à compter de la réception.

			\paragraph{L'interruption du délai de garantie}

				\subparagraph{Les modes d'interruption du délai de garantie}

				Depuis la réforme de la prescription du \printdate{17/06/2008}, il semblerait que le délai de garantie ne puisse être interrompu que par deux actes :
				\begin{itemize}
					\item la citation en justice,
					\item et la signification de conclusion reconventionnelle.
				\end{itemize}

				La reconnaissance de responsabilité qui était reconnue par les tribunaux comme mode interruptif du délai, antérieurement à la réforme de 2008, ne constituerai plus une cause valable d'interruption puisque l'\articleCciv{2240}\footnote{<<  {\itshape La reconnaissance par le débiteur du droit de celui contre lequel il prescrivait interrompt le délai de prescription.} >>} ne vise que les délais de prescription, or le délai de la décennale est un délai de \textbf{forclusion}.

				Il faut comprendre par << citation en justice >> ou assignation, tant les assignations au fond\footnote{
					L’article 750 du Code de procédure civile dispose : « La demande en justice est formée par assignation \lips ». L’assignation est donc un acte introductif d’instance, qui peut marquer le début d’un procès.

					Le demandeur assigne le défendeur à comparaître, c’est-à-dire à se présenter, devant une juridiction définie. L’assignation se distingue de la requête, qui est un acte introductif d’instance par lequel le demandeur saisit le tribunal pour qu’il convoque directement les parties.

					On parle d’assignation « au fond » lorsque le juge devant lequel les parties sont citées à comparaître va se prononcer sur tous les aspects de droit et de procédure de l’affaire. L’assignation « au fond » est également appelée « assignation à toutes fins ». On l’oppose à l’assignation en référé, qui ne concerne que certains points à trancher dans l’urgence.
				}
				que les assignations en référé.  Par exemple : la demande d'expertise en référé est suffisante.

				La citation doit préciser sans équivoques les désordres dont la réparation est demandée. Elle doit viser l'ensemble des responsables : \E, \archi, \Moe ; sachant qu'il est important de les mentionner parce que l'assignation de l'assureur n'a pas de caractère interruptif à l'égard du constructeur assuré --- mais uniquement à l'égard de l'assureur. Si le constructeur n'est pas assigné, il n'y aura pas d'action à l'encontre de celui-ci.

				En vertu de l'\articleCciv{2241}\footnote{<< {\itshape La demande en justice, même en référé, interrompt le délai de prescription ainsi que le délai de forclusion.

					Il en est de même lorsqu'elle est portée devant une juridiction incompétente ou lorsque l'acte de saisine de la juridiction est annulé par l'effet d'un vice de procédure.} >>}

					Particularité en matière de copro : délai interrompu par chaque copropriétaire lorsque le dommage affecte les parties privatives mais également pour les dommages qui affectent les parties communes dès lors que le dommage cause un préjudice personnel. Le délai peut également être interrompu par le syndicat des copro lorsque le dommage affecte les PC mais aussi compétent pour interrompre les délais lorsque le dommage affecte PP et engendre un trouble collectif. Les désordres doivent alors causer les mêmes préjudices à l’ensemble des copropriétaires et doit leur préjudicier de la même manière
					07/09/2011 27/09/2000 c’est donc le juge qui doit caractériser ces éléments pour apprécier l’intérêt à agir du syndicat
					Le syndicat peut interrompre si désordre affecte PC et PP de manière indivisible alors l’interruption du délai réalisé par le syndicat profite au copropriétaire personnellement pour son préjudice personnel. Attention à la problématique de l’habilitation du syndic pour agir au fond (pas le cas en référé) mais en cas de procédure au fond, le syndic doit être habilité au préalable ou a postériori pour agir au fond sinon irrecevable. Cela ne peut pas être soulevé d’office par le juge mais possible par le défenseur. Désormais cela ne peut être soulevé que par un copropriétaire (article 55 suite à la réforme) L’autorisation doit clairement et précisément énumérer un désordre visé, mais la JP dit que pas nécessaire (17 :01 :2019 décret précis) il faut indiquer les réserves, les désordres dans l’habilitation mais la JP dit que pas besoin de mentionner précisément les personnes visées (défendeur) dès lors qu’elles sont déterminables (civ 3° 23/01/2020)
					Néanmoins, conseil résolution visant toutes les personnes où on est sur

					-	Conclusions reconventionnelles : elles sont interruptives du délai de forclusion dans les mêmes conditions que l’assignation donc envers les personnes à l’encontre de qui on souhaite interrompre les délais, elles doivent viser les désordres. Le syndic peut être habilité pour réaliser des demandes reconventionnelles, sauf s’il s’agit d’un moyen de défense


				\subparagraph{Les effets de l'interruption du délai}

				C’est un délai de forclusion, cela a un impact important. Les citations en justice et signification des conclusions reconventionnelles, interrompent le délai de garantie (pas suspension) donc un nouveau délai de même durée qui recommence à courir à partir de l’issue de l’instance.

				L’ordonnance COVID est une suspension des délais mais pas interruptif, avec reprise à la fin de la crise. La formulation est prorogation du délai, si le délai de forclusion expire pendant la crise sanitaire. En réalité il est réputé avoir été fait dans les délais
				Si c’est une assignation 10 ans qui recommence à courir à compter de l’ordonnance désignant l’expert, si c’est au fond, à l’issu de la première instance ou de l’appel, donc jusqu’à que le litige trouve une issue.
				En assignation en référé provision, délai recommence le jour de l’ordonnance
				A compter du prononcé de la décision et non pas de la signification.
				Pour l’expertise, les mesures d’expertises suspendent les délais de prescription selon l’article 2239 du Code civil, mais Civ 3/06/2015 la suspension 2239 pas applicable au délai de forclusion donc pas applicable aux garanties spécifiques.


		\subsubsection{L'obligation \emph{in solidum} et les recours des constructeurs entre eux}

			En cas d'insolvabilité d'un \lo, elle se répartie entre tous les autres \lo solvables. L'insolvabilité ne pèse pas sur la victime mais sur les autres ...

			Les constructeurs dont l’activité est à l’origine de dommage peuvent être condamnés in solidum à indemniser le MO si le MO le sollicite. Toute clause visant à écarter cette solidarité serait réputée non écrite, en application de 1792-5 du Code civil (toute clause contraire aux garantie spécifiques est réputée non écrites)

			La clause qui a pour effet d’écarter cette solidarité serait valable que dans l’hypothèse où une responsabilité contractuelle est recherchée mais pas sur les garanties spécifiques\footnote{\jurisCourDeCas{\civTrois*}{14/02/2019} ; \jurisCourDeCas{\civTrois*}{17/10/2019}}

			L’obligation in solidum permet au MO de s’adresser pour le tout à n’importe quel constructeur. C’est un avantage pour le MO, car il a la possibilité de s’adresser à l’un quelconque des constructeurs pour obtenir le paiement de la totalité des paiements des \DI qui lui ont été accordés par le juge. Il s’adresse la plupart du temps au constructeur ou débiteur le plus solvable comme l’assurance. De même il est possible de demander une condamnation in solidum avec le ST, il pourrait s’adresser au ST mais pas généralement le plus solvable.
			Le constructeur qui a été actionné par la victime a une action récursoire à l’encontre des co-constructeurs coobligés, action fondée sur le droit commun Civ 3°08/02/2012
			Plusieurs hypothèses envisageables :
			-	Elle peut être de nature contractuelle s’il existe un contrat entre les parties (exemple : vendeur immeuble à construire et locateur d’ouvrage, mais ici le vendeur d’immeuble à construire bénéficie des garanties spécifiques à l’égard des locateurs d’ouvrage alors pas besoin de démontrer une faute)
			-	Action de nature aussi délictuelle dans les hypothèses où il n’existe pas de lien contractuel entre les coobligés comme co-locateur d’ouvrage entre eux, entre ST entre eux ou locateur d’ouvrage et ST, alors responsabilité délictuelle avec démonstration d’une faute, préjudice et lien de causalité, le rapport d’expertise est alors probant qui déterminera la faute de chacun et la part de responsabilité de chacun.
			Les délais d’action entre constructeurs en fonction des différentes hypothèses :
			-	Si action de nature contractuelle 1792 et suivants sont enfermés dans les délais de 2 et 10 ans à compter de la réception, en fonction de la garantie spécifique qui est invoquée

			Pour les autres actions contractuelles, quand on ne peut pas agir sur le fondement des garanties spécifiques, elles sont enfermées dans un délai de 10 ans à compter de la réception article 1792-4-3 du Code civil, dans la relation de MO et constructeur.

			Pour actions entre entrepreneur principal et ST, voir précédemment.

			-	Action de nature délictuelle : ici cas entre co-locateur d’ouvrage non lié par un contrat entre eux, ou entre ST ou entre ST et locateur d’ouvrage, prescription 5 ans à compter de la manifestation du dommage ou de son aggravation 2224 du Code civil. La Cour de cassation a refusé d’applique l’article 1792-4-3. Pur certains auteurs sa formulation était générale et s’appliquait aux actions entre constructeurs entre eux. D’autres auteurs pensaient que les actions récursoires ne devaient pas être enfermées dans un délai de 10 ans, ce qui a été retenu par la C.Cass, c’est le délai de droit commun qui s’applique et donc 5 ans à compter de la manifestation du dommage ou de son aggravation, l’article 1792-4-3 ne profite qu’au MO.
			18-25.915 Civ 3°16/01/2020
			Qu’est ce qui fixe la manifestation du dommage ou son aggravation ?
			Les tribunaux fixent couramment Le point de départ du délai de 5 ans à l’assignation du constructeur par le MO ou la victime, Civ 3° 16/01/2020 18-25.915
			Attention à l’arrêt de la Civ 3° 13/09/2006 : ici la C.Cass a jugé que le point de départ de l’action du constructeur à l’encontre d’un autre constructeur devait être fixé au jour du dommage s’est manifesté à l’égard du MO et non pas au jour où le constructeur a été assigné, donc 5 ans à compter du dommage à l’égard du MO. Décision non favorable au constructeur car elle enferme l’action récursoire du constructeur dans un délai dont il ne pouvait avoir connaissance. C’est une décision qui existe mais plus applicable
			Concernant le quantum de l’action récursoire :
			Le co-débiteur du MO n’agit à l’encontre des autres co-débiteurs que pour obtenir le remboursement de leur part de responsabilité personnelle, donc l’action est limitée dans ce que doit chacun des co-obligés au titre de la condamnation.
			Ex : 3 constructeurs co-obligés à hauteur d’1/3. Un constructeur a payé le tout, il devra se retourner sur chacun pour obtenir un tiers. Il ne peut pas se retourner à l’égard de l’un pour obtenir les 2/3.
			En cas d’insolvabilité d’un locateur d’ouvrage, l’insolvabilité se répartie entre tous les autres, la part de l’insolvable se répartie entre tous les autres solvables. L’insolvabilité pèse sur les personnes condamnées in solidum.

	\subsection{La réparation du dommage}

		\subsubsection{La preuve du dommage}

			Bien que le \lo soit tenu d'une présomption de responsabilité, ou encore sur une obligation de résultat, il incombe à la victime d'apporte la preuve du préjudice.

			Bien que le constructeur soit tenue d’une présomption de responsabilité, ou sur une obligation de résultat (pour responsabilité contractuelle de droit commun), il faut démontrer le préjudice, trouble de jouissance, perte financière, malfaçon… la victime doit en rapporter la preuve.

			La victime doit rapporter la preuve de la réalité du préjudice et de son étendue. Cette preuve est généralement rapportée grâce à l’expertise judiciaire, avec aussi nécessité de déterminer les solutions de reprises.
			Article 145 du Code de procédure civile : désignation expertise en référé, avec missions types
			Alors agir en référé devant le TJ du lieu de situation de l’immeuble avec désignation expert avec mission
			L’expertise doit être menée contradictoirement sous peine de nullité du rapport. C’est important car le contradictoire de l’expertise le rapport va s’imposer à tous et pourra servir de base essentielle, principale à la victime pour obtenir réparation. La Cour de cassation, Ch Mixte 28/09/2012, prévoit que l’irrégularité affectant le déroulement de l’expertise régi par les règles de nullité des règles des actes de procédure alors pas inopposabilité du rapport mais sa nullité, si on en démontre le grief. Si une partie n’a pas été convoquée, c’est une cause de nullité s’il en démontre le grief de la victime.
			Le MO victime doit attrait aux opérations d’expertise l’ensemble des constructeurs dont la responsabilité est susceptible d’être engagée et assurances.
			Le juge n’est pas lié par les conclusions de l’expert.
			Après dépôt du pré-rapport possible dire, mais il faut le faire de préférence avant le pré-rapport. Après le pré-rapport dires récapitulatifs.
			Deux précisions concernant l’opposabilité des rapports d’expertise :
			-	Concernant les rapports d’expertise amiable, le juge peut fonder décision sur rapport d’expertise amiable, si régulièrement versé au débat et à la discussion des parties et pas l’unique moyen de preuve. Donc pas de décision exclusivement sur ce rapport. C.Cass Ch Mixte 28/09/2012 11-18.710 Civ 2° 13/09/2018 ici expertise amiable non contradictoire.
			-	Concernant les rapports d’expertise judiciaire, les conclusions sont opposables aux parties à l’expertise même non convoquées dès lors que le contenu claire et précis a été débattu contradictoirement devant la juridiction 28/09/2012 Ch mixte 11-11.381

			Les juges peuvent fonder leur décision sur rapport d’expertise judiciaire même si défendeur n’a pas été partie aux opérations d’expertise à condition que régulièrement versé au débat et soumis à la discussion contradictoire et pas unique moyen des parties car devient comme un rapport amiable, alors même raisonnement que le précédent, CAD si versé aux débats pas unique élément de preuve et librement débattu et discuté Civ 2° 7/09/2017 16-15.531


		\subsubsection[L'étendue de la réparation]{L'étendue de la réparation : le principe de réparation intégrale}

			Le principe : le dommage, tout le dommage, rien que le dommage.

			\paragraph{La réparation de tout le dommage}

				\subparagraph{La réparation des dommages affectant l'ouvrage} Le principe est simple le \Mo doit être placé dans la situation qui aurait été la sienne si le dommage ne s'était pas produit. A cet égard, le \Mo peut demander la remise en l'état de l'ouvrage à l'identique. Il peut donc faire reprendre la fissure mais également la peinture.

				Un abattement pour vétusté est-il possible ? Le \Mo a droit \aCompleter même s'il bénéficie d'une plus-value de fait. La jurisprudence est le refus d'un abattement, probablement parce qu'il n'est pas possible de construire vieux. elle a refusée de \aCompleter\footnote{Contrairement à la doctrine administrative} au du principe de réparation intégral. Exemple du drainage périphérique \aCompleter.

				Le principe de réparation intégral \aCompleter. C'est l'assureur qui va payer.

				Le \Mo pouvait demander la démolition-reconstruction quel que soit le coups. Depuis la réforme ... 1280 cciv, qui vient entériner des jurisprudence dès lors qu'il 1221 cciv.

				la réparation doit être pérenne, s'ils s'avèrent inadapté, la victime peut intenter une nouvelle  sans que ceux 3eme 12/5/99

				Le PPE : le MO doit être placé dans la situation si le dommage ne s’était pas produit. Le MO victime peut demander la remise en l’état de l’ouvrage à l’identique de sorte que si le dommage consiste en une fissure, reprise de la fissure et de la peinture.
				Pas de prise en compte d’un coefficient de vétusté, la JP c’est le refus d’affecter un coefficient de vétusté, de prendre en compte l’état de l’ouvrage au moment du désordre, donc reconstruction à neuf.
				La JP a refusé de faire jouer la théorie de l’enrichissement injustifié car contraire au principe de réparation intégrale. En marché public cette théorie est acceptée.
				La réparation doit comprendre la réparation d’élément non prévu à l’origine si cela est nécessaire pour supprimer le désordre.
				Ex : suppression de drainage obligation pour réparation intégrale ou création d’un mur de soutènement
				Ce PPE s’applique pour toutes les garanties spécifiques
				Le MO pouvait demander une déconstruction et reconstruction totale de l’ouvrage si cela était nécessaire, sur le fondement de la responsabilité décennale avec le principe de réparation intégrale
				Mais depuis réforme droit des contrats, exécution en nature doit être proportionnelle 1221 du Code civil
				La réparation doit être pérenne, si la réparation est inadaptée, la victime a une nouvelle action à / des constructeurs, une nouvelle action, sans qu’on peut lui opposer le premier délai décennale Civ 3° 12/05/1999


				\subparagraph{La réparation des troubles annexes} Le principe est très clair, sur le fondement de ???. Il doit également ouvrir ???

				Il s'agit des troubles de jouissances, des manques à gagner, des pertes de loyer, des pertes d'exploitation, des frais médicaux en cas de dommages corporels, des préjudices moraux, \etc L'assurance ne couvrira pas les préjudices annexes, mais seulement les dommages à l'ouvrage.

				Sur le fondement de garantie décennale, la réparation intégrale permet la réparation des troubles annexes\index{TroublesAnnexes@Troubles annexes}
				Il s’agit des troubles de jouissance, des manques à gagner, les pertes de loyer, les troubles moraux
				La garantie décennale pour les préjudices annexes est obligatoire mais pas l’assurance qui ne porte que sur l’ouvrage.
				Dommages matériels consécutifs + dommages immatériels
				La responsabilité contractuelle couvre les dommages matériels et immatériels.


				\subparagraph{La prise en compte de la TVA} \label{tvaIndemnite}\index{Indemnite@Indemnité!TVA@prise en compte de la TVA} L'indemnité versée est-elle ttc ou ht ? Si la victime récupère la tva, il s'agit d'une indemnité ht, sinon il s'agit d'une indemnité ttc.

				La preuve de la non suggestion à la tva est à la charge de la victime\jurisCourDeCas{\civTrois*}{6/11/2007}.

				Le taux de tva est celui en vigueur au jour ou le juge statut.

			\paragraph{La réparation du seul dommage} Il n'est pas possible d'obtenir des améliorations déguisée. Ainsi, si le désordre concerne des tuiles, il n'est pas possible de demander le remplacement des tuiles par des ardoises, mais uniquement par des nouvelles tuiles. De même, en cas d'insuffisance de chauffage, il n'est pas possible de demander une climatisation réversible qui permet également le froid.

		\subsubsection{Les formes de la réparation}

			Elle peut intervenir en nature ou en équivalent ?

			En nature c'est la reprise physique du dommage. elle peut être fait par l'auteur du dommage ou par un tiers.

			Par équivalent : des dommages et intérêts a

			La victime peut imposer au constructeur la réparation en nature. Par contre, le constructeur ne peut pas imposer une réparation en nature \jurisCourDeCas{\civTrois*}{28/9/2005}.

			Il est possible de combiner les deux.

\section{La Garantie biennale}

	1792-3 garantie de bon fonctionnement des éléments d’équipement durant 2 ans à compter de la réception.

	Il est possible d’étendre le délai contractuellement. Très peu usé, même cette garantie car souvent rend impropre à la destination

	Il s'agit d'une garantie résiduelle qui est peu retenue.

	\subsection{Les conditions de mise en œuvre de la garantie}

		\subsubsection{Les conditions communes aux trois garanties spécifiques}

		Ainsi que vu en \vref{garantiesSpecifiquesConditions} que les quatres conditions communes sont :
		\begin{enumerate}
			\item un ouvrage de construction,
			\item un ouvrage recu,
			\item un dommage caché,
			\item un lien d'imputabilité.
		\end{enumerate}

		Il ne faut pas que les éléments d’équipement professionnels n’aient pas de fonction exclusivement professionnel, sinon sont exclus des garanties spécifiques

		Il ne faut pas que pro exclusivement -> exclu

		\subsubsection{Les conditions propres à la garantie biennale}\index{GarantiesBi@\bi!ConditionsCommunes@Conditions}\label{garantiesBiConditions}

		Il faut une atteinte un element d'equipement dissociable, \cad sans enlèvement de matière ni détérioration de l'ouvarge.

		Il ne faut pas que le desordre atteigne la gravité décennale

		Il faut que l'élément d'équipement aie vocation à fonctionner, il doit être mu par un mécanisme propre. \jurisCourDeCas[12-12016]{\civTrois*}{13/2/2013} dans ce cas on rentre dans la responsabilité contractuelle.

		Il faut que le désordre intervienne dans le délais et que l'action soit introduite dans le délais également.

		4 Conditions :
		-	Il faut une atteinte à un élément d’équipement dissociable de l’ouvrage
		Une atteinte à un élément d’équipement dissociable à l’ouvrage CAD qu’il peut être retiré de l’ouvrage sans atteinte à l’ouvrage et sans enlèvement de matière. Il doit donc être dissociable.
		-	Le désordre ne doit pas atteindre la gravité décennale
		-	Il faut que l’élément d’équipement ait vocation à fonctionner, il doit être mue par un dynamisme propre
		Pendant très longtemps, la JP a admis de faire jouer la garantie biennale pour des éléments inertes.
		Mais, depuis un arrêt du 13/02/2013 12-12.016, la Cour de cassa a exclu les éléments inertes du champ d’application de la garantie biennale, cela a été confirmé de nombreuses fois, appel 11/09/2013 (pour le carrelage) C.Cass 18/02/2016 (revêtement végétal d’étanchéité) alors la biennale n’est pas applicable alors responsabilité contractuelle
		-	Il faut que le désordre apparaisse dans le délai de la garantie dans le délai de 2 ans, et que l’action soit introduite également dans ce délai.



		\subsection{Le régime et les modalités de mise en œuvre de la garantie biennale}

			L'action doit être introduite par les même bénéficiaire : \Mo,

			Ceux sont les même redevables :

			Elle repose également sur le même régime de responsabilité. elle repose sur une présomption de responsabilité. On retrouve les même causes d'exonérations, la même nature de délai. De même, il est possible de demander une condamnation \emph{in solidum} des lors que le dommage peut être imputé à plusieurs constructeurs.

			De même, si les conditions de la biennale sont remplies il faut nécessairement pas \rcdc

			Enfin, le principe de réparation intégral va trouver à s'appliquer ??? en nature ou en équivalent.

			L’action doit être introduite par les mêmes bénéficiaire de la garantie décennale, le MO les acquéreurs successifs. Et les mêmes redevables de la garantie
			La garantie repose sur le même régime de responsabilité de la garantie décennale ainsi la responsabilité biennale repose sur une présomption de responsabilité avec les mêmes causes d’exonération, la même nature de délai, c’est un délai de forclusion qui ne peut être interrompu que par citation ou conclusions reconventionnelles, interruption pas de suspension.
			Possible de demander condamnation in solidum dès lors que plusieurs constructeurs sont responsables
			La responsabilité biennale est exclusive de tout autre régime de responsabilité. Si les conditions de la biennale sont remplies, il faut nécessairement agir sur la biennale. Donc si désordre 3 ans alors que les conditions sont remplies, il n’est pas possible de basculer sur une autre garantie, sauf si atteinte à la destination alors décennale.
			Le principe de réparation intégrale se trouve à s’appliquer, en nature ou en équivalent, sur les dommages matériels et annexes
			Ex : volet roulant qui ne fonctionne pas en position ouverte


\section{La garantie de parfait achèvement}

1792-6 cciv cite
Donc couvre que si les conditions communes au 3 catégories :
Exception : désordre apparents mais réservé.

	C’est une garantie annale fondée sur l’\articleDu{1792-6}{\cciv}. La garantie de parfait achèvement couvre les désordres réservés à la réception et apparus dans un délai d’un an, et ne peut être mise en œuvre que lorsque les conditions communes aux 3 garanties spécifiques sont remplies (un ouvrage de construction – réception – dommage caché (exception à la garantie de parfait achèvement pour les désordres réservés) – lien d’imputabilité)

		\subsection{Le débiteur de la garantie}

		Le texte par el de l'E. Seul l'entrepreneur concerné. Les autres constructeurs ou assimilé constructeur ne sont pas concernés par la gpa. pas de castor, pas d'archi.

		Il n'y a pas d'in solidum, sauf si plusieurs E ont concourus au désordre. Chaque E doit reprendre sa partie.

		Il dispose cependnt d'une action récursoire envers les autres E qui pourraient être à l'orgine du désordres (sur une base soit contractuel soit délictuel)

		Le texte dit que la garantie de parfait achèvement à laquelle l’entrepreneur est tenu. Donc seul l’entrepreneur dont les travaux sont affectés des désordres, peut être actionné. On parle de l’entrepreneur concerné.
		Ex : plombier ne peut voir sa responsabilité engagée pour des problèmes d’électricité
		Il n’y a pas de responsabilité in solidum dans la garantie de parfait achèvement, seul l’entrepreneur concerné est tenu de cette garantie.
		Seul l’entrepreneur est tenu de la garantie de parfait achèvement.
		Les autres constructeurs assimilés constructeurs ou réputés constructeurs, ne sont pas tenus de la garantie de parfait achèvement, il s’agit des architectes, le castor, le vendeur d’immeuble à construire, le bureau d’étude le contrôleur technique.
		Quid quand le désordre trouve sa cause dans l’intervention de plusieurs entrepreneurs de divers corps d’état ?
		Chaque entrepreneur doit reprendre la partie de l’ouvrage qu’il a réalisé et le MO devra agir à l’encontre de chaque locateur d’ouvrage.
		Chaque locateur d’ouvrage dispose alors d’un recours récursoire s’il estime que le désordre est imputable à l’autre locateur d’ouvrage (soit sur du contractuel ou délictuel)

		\subsection{L'étendue de la garantie}

			La garantie couvre les désordres réservés à la réception, ainsi que les désordres qui sont apparus dans l'année qui suit la réception.

			A la lecture littérale de l’article 1792-6 du Code civil :
			-	Désordre réservé à la réception
			-	Désordre qui apparait dans le délai d’un an à compter de la réception, il faut alors une notification, la LRAR n’est pas indispensable mais préférable car il faut se ménager la preuve
			Tous les désordres sont couverts par la garantie de parfait achèvement, quel que soit la nature comme défaut de conformité, vice de construction ou d’une non façon, quel que soit leur origine et quel que soit la gravité, il peut s’agir donc de désordres purement esthétiques ou des désordres plus graves.
			Concernant la gravité, les désordres réservés à la réception sont couverts par la garantie de parfait achèvement, même s’ils atteignent la gravité décennale, même les désordres qui rendent l’ouvrage impropre à sa destination. La décennale ne pourra trouver à s’appliquer car ils étaient apparents, sauf dans un cas où le dommage a été réservé mais s’est révélé dans son ampleur et ses conséquences postérieurement à la réception alors on considère que le dommage était caché et on peut agir en décennale.
			Les désordres réservés sont couverts par la responsabilité contractuelle de droit commun en plus de la garantie de parfait achèvement, un cumul est ici possible, c’est une faveur accordée par les tribunaux.
			Etant précisé que la GPA ne permet la reprise des désordres matériels et pas les immatériels de sorte que si le MO souhaite obtenir l’indemnisation des préjudices annexes liés à un dommage couvert par la GPA il devra agir sur la contractuelle de droit commun.
			Le délai de la GPA est d’un an à compter de la réception.
			Pour les désordres apparus après la réception, notifiés dans l’année qui affectent un élément d’équipement dissociable destiné à fonctionner relèvent de la GPA mais également de la biennale, mais il est préférable d’agir sur le fondement de la biennale que sur la GPA car la biennale permettra d’avoir la reprise du dommage matériel mais également immatériel alors que la GPA ne permettra que la reprise du dommage matériel. Ici donc possibilité d’agir sur les deux fondements.
			S’agissant des désordres apparus après la réception notifiés dans l’année qui rendent l’ouvrage impropre à sa destination qui affectent sa solidité ou affecte les éléments d’équipements indissociables, dans ces cas-là il est possible d’agir sur le fondement de la GPA mais également sur le fondement de la décennale, mais plus avantageux la décennale pour les mêmes raisons.
			Donc la GPA se cumule avec responsabilité contractuelle de droit commun, avec la biennale et la décennale. C’est une particularité de la GPA qui est cumulative. Néanmoins, un intérêt toujours d’agir sur les autres fondements.
			Quelques exclusions :
			Les désordres résultant de l’usure normale de l’ouvrage ou de l’usage : 1792-6 du Code civil : la garantie ne s’étend pas aux travaux nécessaires pour remédier aux effets de l’usure normale ou de l’usage
			Les troubles annexes, les dommages dits consécutifs, les immatériels sont exclus de la GPA alors possible d’agir sur le fondement de la responsabilité contractuelle de droit commun.

			La garantie tend à réparer les désordres en nature par principe, mais la JP a accepté que la réparation puisse intervenir par équivalent.
			Le régime de la GPA permet également d’obtenir la reprise des désordres phoniques. Les désordres acoustiques sont couverts par une garantie particulière qui s’appelle la garantie phonique dont le régime renvoie à celui de la GPA, article L111-11 du Code de la construction et de l’habitation
			On applique donc la GPA lorsque les exigences minimales en matière d’isolation phonique ne sont pas remplies. En pratique, ce texte est peu appliqué car les Tribunaux admettent de faire jouer la responsabilité décennale en cas de trouble d’ordre phonique. A cet égard la garantie décennale peut être mise en œuvre dès lors que l’insuffisance d’isolation phonique caché à la réception rend les locaux impropres à leur destination.
			Donc ce n’est pas parce que les exigences légales et règlementaires en matière phonique ont été respectées, que cela fait obstacle à une action sur le fondement de la responsabilité décennale.
			C’est une particularité que peu de personne ne savent.
			(quitus est un document lorsque la réserve est levée).
			L’intérêt d’engager la responsabilité décennale pour les désordre d’isolation phonique, c’est le délai qui est décennale, et la réparation des préjudices annexes, et surtout derrière il y a l’assurance de responsabilité civile décennale du constructeur.
			La responsabilité contractuelle de droit commun des constructeurs peut également être engagée en cas de désordre acoustique qui résulte d’un défaut de conformité aux stipulations contractuelles, caché à la réception et que pour autant l’ouvrage n’a pas rendu impropre à sa destination.
			A cet égard, l’action doit être introduite dans un délai de 10 ans à compter de la réception, article 1792-4-3 du Code civil et cette action peut être introduite soit par le MO ou l’acquéreur de l’ouvrage sachant que l’acquéreur en l’état futur d’achèvement peut agir à l’encontre du vendeur d’immeuble à construire dans un délai de 5 ans à compter de la découverte du désordre., article 2064 du Code civil
			Le constructeur peut s’exonérer de sa responsabilité en démontrant alors qu’il n’a pas commis de faute ou pour un cas de cause étrangère lorsque l’on est sur responsabilité contractuelle de droit commun.
			La victime peut solliciter une réparation en nature dans la mise en conformité, ou en équivalent. Sachant que l’action sur le fondement de la responsabilité contractuelle de droit commun permet la reprise des dommages matériels et immatériels.


			Il faut une notification, donc il est préférable de faire une lrar.

			Tous les désordres sont couverts, quel que soit leur nature, quel que soit leur origine, quel que soit leur gravité. de sorte que les simples désordres esthétiques sont couverts.

			Les désordres réservés sont couverts par la \gpa, même s'ils atteignent la gravité décennale. C'est à dire : ...

			Dans l'hypothèse où des désordres réservés se sont révélé dans leur gravité postérieurement,  ouvre droit à la décennale (on privilégiera la décennale pour atteindre les assurances)

			La gpa se cumule vec le rcdc, mais egalement avec decennale et biennale.
			Pour les désordres apparus après la réception, notifié dans l'année, alors gpa \& biennale. Il est préfrerable d'agir sur la biennale.

			Pour les... il est possible gpa \& decennale

			Conseil : Il est recommandé 31'

			Exclusion : 1792-6 in fine la garantie ne s'étend pas ...
			De meme les troubles annexes ... on agit alors sur la rcdc

			Par principe réparation en nature, mais jurisprudence accepte réparation en équivalent.

			Les désordres phoniques sont couverts par une garantie particulière, qui renvoie à la gpa L 111-11 cch (cite in extenso). Ce texte est peu appliqué car tribunaux accepte décennale. Des lors que l'insuffisance d'isolation phonique rende impropre les locaux à leur destination. a cet égard, le respect des exigences et obligations légales ne font pas obstacles à la mise en œuvre. L'interet est que le délai est plus long, en plus reparation prejudice annexe, et sutout possibilité de mobiliser les assurances. La rcdc peut etre egalement engagé 40' si elle resulte et que ne rend pas l'ouvrage impropre à sa destination 1791-4-3 soit par le Mo soit l'acquéreur de(Si c'est una cquéreur VEFA il peut agir dans un delai de 5ans à l'encontre du vendeur d'immeuble à construire).
			Le constructeur peut s'exonérer de sa responsabilité en apportant la preuve d'un orig etranger. la victime peut solliciter une ... en nature ou en équivalent, les prejudices matériels et immatériels.

		\subsection{Le régime et les modalités de mise en œuvre de la garantie}

			\subsubsection{Le régime de la garantie}

				\paragraph{Le délai d'action} Un an a compter de la réception. Sachant qu'il s'agit d'un délai de dénonciation et d'action. A defaut l'action du Mo est forclose.

				L'interruption du délai nécessite l'assignation ou les conculsions reconventionnelle. Le délai est interrompu, pas suspendu, donc un nouveau délai d'un an commence à courir à compter de la décision (ordonnance nommant l'expert ou décision définitive).

				Conseil : 47'

				Délai très bref. Cumul avec le rcdc (obligation de résultat).

				Le régime. Elle repose sur le même régime que décennale. Présomption et même cause d'exclusion.

				LE DELAI D’ACTION
				C’est un délai de garantie annale dont le point de départ est la réception. L’action doit être introduite dans un délai d’un an à compter de la réception.
				Il s’agit d’un délai de dénonciation et d’action. L’assignation doit nécessairement être délivrée dans le délai annal ou les conclusions reconventionnelles signifiées dans le délai annal.
				A défaut, l’action du MO est forclose. Le MO ne pourra plus agir sur le fondement de la GPA. L’interruption du délai suppose donc une assignation ou des conclusions reconventionnelles. Une réclamation par LRAR ne suffit à interrompre le délai de garantie.
				Il s’agit d’un délai de forclusion de sorte qu’il est interrompu et non pas suspendu c’est un nouveau délai de même durée qui recommence à courir à compter de la décision de justice.
				Ex : assignation en référé pour la désignation d’un expert, le nouveau délai d’un an commence à courir à compter de l’ordonnance désignant l’expert judiciaire. Si c’est un référé provision, le délai recommence à courir à compter de la décision de référé allouant ou rejetant la demande de provision. Si c’est une assignation au fond le délai recommence à courir à compter de la décision définitive ou de la décision d’appel.
				Ce délai de garantie est très bref. Il vaut mieux donc assigner au fond, le problème est que nous n’aurons pas nécessairement les éléments de preuve du préjudice, de son quantum. Donc il vaut mieux commencer par une assignation en référé pour dans l’année qui suit assigner au fond et demander de sursoir à statuer. Autre possibilité est d’agir au fond et de demander au juge de la mise en état la désignation d’un expert judiciaire. Une fois que le juge du fond est saisi, seul le juge de la mise en état est compétent pour désigner un expert, donc impossible après avoir saisi le fond de faire un référé. C’est pourquoi il vaut mieux commencer par un référé et ensuite assigner au fond avec le sursis à statuer.
				Ce délai est très bref, c’est pour cela qu’il y a un cumul avec la responsabilité contractuelle de droit commun. Ces deux régimes de responsabilité sont donc cumulatifs. On peut donc obtenir l’indemnisation de deux choses différentes. Les immatériels avec la responsabilité de droit commun et les dommages matériels avec la GPA, en sachant que la responsabilité contractuelle de droit commun permet également d’obtenir l’indemnisation des préjudices matériels. L’intérêt est que le délai de la responsabilité de droit commun est de 10 ans à compter de la réception, article 1792-4-3 du Code civil
				Sur le fondement de la responsabilité contractuelle de droit commun pour les désordres réservés il y a une obligation de résultat donc le simple constat du désordre suffit pour mettre en œuvre la responsabilité contractuelle de droit commun.
				REGIME DE LA RESPONSABILITE
				C’est le même régime que celui de la responsabilité décennale, on est donc en présence d’une présomption de responsabilité avec les mêmes causes d’exonération, la cause étrangère, le cas de force majeur, le fait de l victime ou le fait d’un tiers.


			\subsubsection{Les modalités pratiques de mise en œuvre de la garantie}

				1792-6 alinéa 3 à 5.

				Le Mo doit faire appel à l'E lui demandant de reprendre le désordre. S'il fait appel à un tiers, il perd la possibilité d'actionner la gpa.

				Les parties se mettent d'accord sur un calendrier pour réaliser les travaux. Il est donc préférable de se mettre à l'intérieur du calendrier. La norme AFNOR propose des délais 60 jours à compter du pv re
				60 jours à compter de la notification du désordre.

				En cas de carence de l'\E le \Mo doit Mise en demeure .

				Si l'\E ne réagit toujours pas, le \Mo peut faire effectuer les travaux aux frais et risques. La retenue de garantie, si elle a été contractualisée, permettra de financer les travaux. Le risque est que l'\E conteste l'étendue du désordre, le nature des travaux de reprise, le quantum des travaux de reprises. En conséquence, ce mécanisme est très peu mis en œuvre en pratique.

				Si il appartient au Mo et à l'E d'un commun accord de constater contradictoirement la réalisation des travaux. A défaut, il faudra faire constater par voie judiciaire. Encore une fois, processus long qui est peu mis en œuvre en pratique.

				En pratique, comment mettons en œuvre la GPA ?
				Tout est prévu à l’article 1792-6 alinéa 3 à 5 du Code civil
				1° Etape
				Le MO doit faire appel à l’entrepreneur responsable du dommage et lui demander de reprendre le désordre.
				Si le MO fait appel à un tiers avant de mettre en œuvre la GPA, il perd la possibilité d’agir sur le fondement de la GPA, il ne peut pas avoir de suite recours à un tiers, il doit d’abord actionner l’entrepreneur afin que celui-ci reprenne les désordres
				2° Etape
				Les parties se mettent d’accord sur un calendrier de réalisation des travaux.
				Si l’entrepreneur constate la réalité du dommage et décide d’intervenir, les parties doivent donc se mettre d’accord sur un calendrier pour réaliser les travaux.
				Le délai d’un an est un délai de dénonciation et d’action, ce n’est pas un délai dans lequel les travaux doivent être réalisés. Les parties peuvent parfaitement se mettre d’accord pour que les travaux soient réalisés après le délai d’un an. Mais si l’entrepreneur ne réalise pas les travaux le MO perd alors la possibilité d’agir sur le fondement de la GPA qui sera expiré. Il est donc préférable et recommandé au MO de se mettre d’accord avec l’entrepreneur sur un calendrier à l’intérieur du délai annal, comme ça en cas de non-respect du calendrier par l’entrepreneur qui ne reprend pas les dommages, le MO aura encore la possibilité d’agir à son encontre sur le fondement de la GPA.
				La norme AFNOR prévoit des délais de reprise des dommages, elle n’est applicable que si elle a été contractualisée : 60 jours à compter de la réception du PV de réception pour les désordres réservés et 60 jours à compter de la notification des désordres révélés dans l’année de la GPA.
				Les parties doivent se mettre d’accord sur une intervention et les modalités de l’intervention. Si le MO n’est pas d’accord avec ce que propose l’entrepreneur le MO a la possibilité de l’assigner et partir en expertise, et c’est l’expert qui déterminera la solution.
				3° Etape
				En cas de carence de l’entrepreneur, le MO doit le mettre en demeure d’effectuer les travaux. Dans ce cas-là c’est une LRAR qu’il faut lui adresser. Et si malgré la mise en demeure l’entrepreneur ne réagit pas, le MO peut faire effectuer les travaux de reprise aux frais et risques de l’entrepreneur défaillant.
				La problématique est alors le moyen de financer ces travaux par le MO ?
				La retenue légale de garantie peut permettre au MO de pré financer les travaux aux frais et risques de l’entrepreneur principal. Encore faut-il que cette retenue de garantie ait été contractualisée.
				En cas d’un entrepreneur en faillite, la mise en demeure est alors inutile mais toute action à son encontre sera également inutile, alors plus aucune solution.
				On pourrait également faire un référé provision, mais le juge devra constater que le désordre existe et que l’on ait une idée sur son quantum et le coût des travaux de reprise, c qui ne sera pas forcément facile, donc le référé provision est très rare.
				Ce que l’on voit parfois c’est lorsque l’on fait une demande de désignation d’expert, on fait également une demande provision, elle est systématiquement rejetée. Il vaut mieux partir en référé provision après le dépôt du rapport d’expertise parce que là on a un rapport contradictoire avec un expert qui s’est penchée sur le quantum et un juge qui sera donc pleinement éclairé.
				Le risque de faire réaliser les travaux aux risques de l’entrepreneur est que l’entrepreneur puisse contester le quantum des travaux effectués. Donc lorsque l’on demandera la condamnation de l’entrepreneur au montant des travaux réalisés, celui-ci pourra contester le quantum et dire d’une part, je n’ai pas pu apprécier la réalité du désordre et il pourrait contester en disant je conteste donc la réalité des désordres, la nature des désordres et le quantum donc le prix pratiqué par l’entreprise appelé par le MO. Il y a donc pas mal de risques donc en cas de travaux excessifs ceux-ci peuvent rester à la charge du MO, il y a donc un risque qui fait que cette procédure est très peu utilisée, le tribunal contrôle a posteriori le quantum
				En pratique peu utilisée la GPA car soit pas les moyens financiers ou en raison des risques invoqués.
				Si les travaux sont urgents, le MO aura tout intérêt à le faire, mais à ce moment-là le MO prend ce risque en connaissance de cause, donc avec le risque de ne pas être remboursé ou de moindre valeur.

				Si les parties se mettent d’accord sur l’exécution des travaux de reprise, il leur appartient de constater d’un commun accord, à la signature d’un quitus la réalisation des travaux, ou à défaut, la réalisation des travaux doit être constatée judiciairement en cas de désaccord entre les parties, ce qui suppose la désignation d’un expert judiciaire.
