\documentclass[10pt,a4paper,twoside]{report}

\usepackage[utf8]{inputenc}
\usepackage[T1]{fontenc}
%\usepackage{lmodern} % Police un peu plus jolie
\usepackage{enumitem} % Pour personnaliser les listes
\usepackage{graphicx}
\usepackage{eurosym}
\usepackage{xspace}
\usepackage{makeidx}
\usepackage{moredefs}\usepackage[mla]{lips}
\usepackage[autolanguage]{numprint}
\usepackage[left=4.00cm]{geometry}
\usepackage[french]{babel, varioref, isodate}
\usepackage{hyperref}

\author{Adrien \nom{Pelon}
%	\begin{tiny}
%		Avocat associé
%
%		Tél : 01 40 56 08 00
%
%		Fax : 01 42 19 06 35
%
%		\url{apavocats@orange.fr}\\
%
%		53 avenue de Breteuil
%
%		75007 Paris
%
%		\url{www.jpavocats.fr}
%	\end{tiny}
}
\title{Master 2 \\ Droit de l’immobilier et de la construction\\Formation continue\\\bigskip{\Huge Droit de la construction}}

\hypersetup{%
	pdfinfo={%
		Title={Droit de la construction}%
		, Subject={}%Marché \@ReferenceMarche}%
		, Author={Samuel Déom}%
		, Keywords={}%\@ReferenceMarche}%
	}%
	, colorlinks = true% colore, plutot qu'encadre, les liens hypertexte
	, linkcolor = black% colore les liens internes en noir
	, urlcolor = black% colore les liens externes en noir
	, breaklinks = true% autorise les liens à être étendus sur plusieurs lignes
}

% Les acteurs
\newcommand*{\Mo}{Maître de l'Ouvrage\xspace}
\newcommand*{\MO}{\Mo}
\newcommand*{\Moe}{Maître d'Œuvre\xspace}
\newcommand*{\lo}{locateur d'ouvrage\xspace}
\newcommand*{\archi}{architecte\xspace}
\newcommand*{\E}{Entrepreneur\xspace}
\newcommand*{\ep}{entrepreneur principal\xspace}
\newcommand*{\GME}{groupement momentané d'entreprises\xspace}
\newcommand*{\CSPS}{\textsc{csps}\xspace}
\newcommand*{\HLM}{HLM\xspace}
\newcommand*{\SEM}{SEM\xspace}
\makeatletter
\newcommand*{\ST}{\@ifstar{sous-traitance\xspace}{sous-traitant\xspace}}
\makeatother

% Ecrire du texte juridique
\makeatletter
\newcommand*{\assPlen}{\@ifstar{\mbox{ass. plén.}\xspace}{assemblée plénière\xspace}}
\newcommand*{\civUn}{\@ifstar{civ. 1\iere{}\xspace}{première chambre civile\xspace}}
\newcommand*{\civDeux}{\@ifstar{civ. 2\ieme{}\xspace}{deuxième chambre civile\xspace}}
\newcommand*{\civTrois}{\@ifstar{civ. 3\ieme{}\xspace}{troisième chambre civile\xspace}}
\newcommand*{\CourDeCas}{\@ifstar{Cass.\xspace}{Cour de Cassation\xspace}}
\newcommand*{\CA}{\@ifstar{CA\xspace}{Cour d'Appel\xspace}}
\newcommand*{\CE}{\@ifstar{CE\xspace}{Conseil d'État\xspace}}
\makeatother

\newcommand*{\jurisCourDeCas}[3][]{\CourDeCas* #2, \printdate{#3}\ifthenelse{\equal{#1}{}}{}{, \no#1}}% Numéro de chambre, date et en option no de pourvoi
\newcommand*{\jurisCA}[2]{\CA* #1, \printdate{#2}}%Ville, date

\newcommand*{\refArticle}[2]{#1.\,#2}
\newcommand*{\articleCodifie}[2]{\mbox{article~\refArticle{#1}{#2}}}
\newcommand*{\articlesCodifies}[2]{\mbox{articles~\refArticle{#1}{#2}}}
\newcommand*{\ArticleCodifie}[2]{\mbox{Article~\refArticle{#1}{#2}}}
\newcommand*{\ArticlesCodifies}[2]{\mbox{Articles~\refArticle{#1}{#2}}}
\newcommand*{\articleDu}[3][]{\ifthenelse{\equal{#1}{}}{\mbox{article~#2}}{\articleCodifie{#1}{#2}} du #3}
\newcommand*{\articlesDu}[3][]{\ifthenelse{\equal{#1}{}}{\mbox{articles~#2}}{\articlesCodifies{#1}{#2}} du #3}
\newcommand*{\articlesDuEtSuivants}[3][]{\ifthenelse{\equal{#1}{}}{\mbox{articles~#2}}{\articlesCodifies{#1}{#2}} et suivants du #3}
\newcommand*{\ArticleDu}[3][]{\ifthenelse{\equal{#1}{}}{\mbox{Article~#2}}{\ArticleCodifie{#1}{#2}} du #3}
\newcommand*{\ArticlesDu}[3][]{\ifthenelse{\equal{#1}{}}{\mbox{Articles~#2}}{\ArticlesCodifies{#1}{#2}} du #3}
\newcommand*{\ArticlesDuEtSuivants}[3][]{\ifthenelse{\equal{#1}{}}{\mbox{Articles~#2}}{\ArticlesCodifies{#1}{#2}} et suivants du #3}

\makeatletter
\newcommand*{\ca}{\@ifstar{\mbox{C.~assur.}\xspace}{Code des assurances\xspace}}
\newcommand*{\cch}{\@ifstar{CCH\xspace}{Code de la construction et de l'habitation\xspace}}
\newcommand*{\cciv}{\@ifstar{\mbox{C.~civ.}\xspace}{Code civil\xspace}}
\newcommand*{\ccom}{\@ifstar{\mbox{	C.~com.}\xspace}{Code du commerce\xspace}}
\newcommand*{\ccons}{\@ifstar{\mbox{C.~consom.}\xspace}{Code de la consommation\xspace}}
\newcommand*{\ccp}{\@ifstar{CCP\xspace}{Code de la commande publique\xspace}}
\newcommand*{\cpc}{\@ifstar{CPC\xspace}{Code de procédure civile\xspace}}
\newcommand*{\cpi}{\@ifstar{CPI\xspace}{Code de la propriété intellectuelle\xspace}}
\newcommand*{\cu}{\@ifstar{\mbox{C.~urb.}\xspace}{Code de l'urbanisme\xspace}}
\makeatother


\newcommand*{\indexCciv}[1]{}%\index{CodeCivil@Code civil!#1@article #1}}
\newcommand*{\articleCciv}[1]{article~#1 du \indexCciv{#1}\cciv}
\newcommand*{\ArticleCciv}[1]{Article~#1 du \indexCciv{#1}\cciv}
\newcommand*{\articlesCciv}[1]{articles~#1 du \indexCciv{#1}\cciv}
\newcommand*{\ArticlesCciv}[1]{Articles~#1 du \indexCciv{#1}\cciv}
\newcommand*{\articlesCcivEtSuivants}[1]{articles~#1 et suivants du \indexCciv{#1}\cciv}

\newcommand*{\qpc}{question prioritaire de constitutionnalité\xspace}


% Abbréviation ou sigle de la construction
\newcommand*{\ado}{assurance dommage-ouvrage\xspace}
\newcommand*{\arcd}{assurance de responsabilité civile décennale\xspace}
\newcommand*{\bi}{garantie de bon fonctionnement \mbox{(<< biennale >>)}\xspace}
\newcommand*{\ccmi}{contrat de construction de maison individuelle\xspace}
\newcommand*{\cnr}{constructeur non réalisateur\xspace}
\newcommand*{\dec}{garantie décennale\xspace}
\newcommand*{\dgd}{décompte général définitif\xspace}
\newcommand*{\do}{dommage-ouvrage\xspace}
\newcommand*{\garSpec}{garanties spécifiques\xspace}
\newcommand*{\gpa}{garantie de parfait achèvement\xspace}
\newcommand*{\lesGarSpec}{\gpa, \bi, et \dec\xspace}
\newcommand*{\loiST}{loi du \printdate{31/12/1975}\xspace}
\newcommand*{\rcd}{responsabilité civile décennale\xspace}
\newcommand*{\rcdc}{responsabilité contractuelle de droit commun\xspace}
\newcommand*{\vrd}{\textsc{vrd}\xspace}
\newcommand*{\CCMI}{\textsc{ccmi}\xspace}
\newcommand*{\DO}{\textsc{do}\xspace}
\newcommand*{\ERP}{\textsc{erp}\xspace}
\newcommand*{\IGH}{\textsc{igh}\xspace}
\newcommand*{\PC}{permis de construire\xspace}
\newcommand*{\VEFA}{\textsc{vefa}\xspace}

% Ecrire des quantités
\newcommand*{\montant}[1]{\nombre{#1}~\euro}
\newcommand*{\montantTtc}[1]{\montant{#1}~\textsc{ttc}}
\newcommand*{\montantHt}[1]{\montant{#1}~\textsc{ht}}
\newcommand*{\pourcent}[1]{\nombre{#1}\,\%}
\newcommand*{\metreCarre}{m\up{2}\xspace}
\newcommand*{\surface}[1]{\nombre{#1}~\metreCarre}

% Abréviation et régles courantes
\newcommand*{\etc}{\emph{etc.}\xspace}
\newcommand*{\cad}{c'est-à-dire\xspace}
\newcommand*{\lrar}{lettre recommandée avec accusé de réception\xspace}
\newcommand*{\di}{dommages et intérêts\xspace}
\newcommand*{\med}{mise en demeure\xspace}
\newcommand*{\jp}{jurisprudence\xspace}
\newcommand*{\pv}{procès-verbal\xspace}

\newcommand*{\I}{\textsc{i}\xspace}
\newcommand*{\II}{\textsc{ii}\xspace}
\newcommand*{\III}{\textsc{iii}\xspace}

\newcommand*{\nom}[1]{\textsc{#1}}

\newcommand*{\aCompleter}{\colorbox{yellow}{\lips}\index{Trou dans le cours}\xspace}
\newcommand*{\aVerifier}{\colorbox{yellow}{[\textsc{A vérifier}]}\index{A vérifier}\xspace}
\newcommand*{\aValider}{\aVerifier}

% Pour faciliter l'intégration des notes de Julie
\newcommand*{\AD}{action directe}
\newcommand*{\CAD}{\cad}
\newcommand*{\DI}{\di}
\newcommand*{\JP}{\jp}
\newcommand*{\LRAR}{\lrar}
\newcommand*{\MED}{\med}
\newcommand*{\PPE}{principe\xspace}
\newcommand*{\PV}{\pv}
\newcommand*{\RC}{\textsc{RC}}
\newcommand*{\TAV}{troubles anormaux de voisinage\xspace}


% Environnement
\newenvironment*{conseil}{%
	\medbreak
	Conseil :
	\begin{quote}
	\itshape
}{%
	\end{quote}
}

\newenvironment*{exemple}{%
		\smallbreak\itshape
		Exemple :
	}{%
		\smallbreak
}

\newenvironment*{citationArticle}[3][]{%
	\medbreak\noindent\ifthenelse{\equal{#1}{}}{Article~#2}{\ArticleCodifie{#1}{#2}} du {#3} :
	\begin{quote}
		\parskip 6pt
		<< \itshape
	}{%
		>>
		\parskip 0pt
	\end{quote}
}

\newenvironment*{citationArticleLoi}[3][]{%option n° de loi, n° article, date de la loi
	\par\medbreak\textbf{Article #2 de la loi\ifthenelse{\equal{#1}{}}{}{ \no#1} du \printdate{#3}}
	\begin{quote}
		<< \itshape
	}{ >>
	\end{quote}
}

\newenvironment*{citationArticleCciv}[1]{%
	\begin{citationArticle}{#1}{\cciv}
	}{%
	\end{citationArticle}
}

\makeindex

\setcounter{secnumdepth}{3}

\begin{document}
	\maketitle

	\chapter*{Les opérateurs de la construction}
	
	\section*{Le Maître de l'Ouvrage}
	
	\section*{Les Locateurs d'Ouvrage}
		
		\subsection*{L'architecte}
		
		\subsection*{L'entrepreneur}
		
		\subsection*{Les ingénieurs et les bureaux d'études}
		
		\subsection*{Le contrôleur technique}
		
			\subsubsection*{La convention de contrôle}
			
				\paragraph*{Le recours au contrôle}
				
				\paragraph*{L'objet du contrôle}
			
			\subsubsection*{La responsabilité du contrôleur technique}
			
				\paragraph*{À l'égard du maître d'ouvrage}
				
				\paragraph*{À l'égard des autres constructeurs}
		
		\subsection*{Le coordinateur santé et sécurité}
		
			\subsubsection*{L'intervention du coordonnateur}
			
			\subsubsection*{La responsabilité du coordonnateur}
	
	\section*{Les Assureurs Construction}

	%\part*{Le droit commun de la construction}

	\part{Les contrats de construction}

		Le contrat de louage d'ouvrage --- ou contrat d'entrepreneur --- est défini par l'\articleCciv{1710}.

		\begin{citationArticleCciv}{1710}
			Le louage d'ouvrage est un contrat par lequel l'une des parties s'engage à faire quelque chose pour l'autre, moyennant un prix convenu entre elles.
		\end{citationArticleCciv}

		La \CourDeCas{} précise qu'il faut une rémunération et un travail effectué indépendamment.

		% !TEX root = ./droitConstruction.tex

\chapter{Le contrat de l'entrepreneur}

	\section{La qualification du contrat d'entreprise}

		\subsection{Distinction avec le contrat de travail}

			\subsubsection{Le critère : l'absence de subordination}

				Indépendance du \lo{}. Il n'y a pas de lien de subordination. Il décide des moyens à mettre en œuvre pour réaliser sa mission.

				Le \lo{} ne touche pas un salaire, il est payé à la tâche.

			\subsubsection{Intérêt de la distinction}

				Le régime social est différent de celui d'un salarié.

				Le régime de responsabilité est également différent. Le \Mo{} n'est pas un commettant, il n'y a donc pas de relation commis-commettant et par conséquent pas de responsabilité du \Mo{} à l'égard des tiers sur cette base.

				\begin{citationArticleCciv}{1242}
					On est responsable non seulement du dommage que l'on cause par son propre fait, mais encore de celui qui est causé par le fait des personnes dont on doit répondre, ou des choses que l'on a sous sa garde.

					\lips

					Les maîtres et les commettants, du dommage causé par leurs domestiques et préposés dans les fonctions auxquelles ils les ont employés ;

					\lips
				\end{citationArticleCciv}

				L'\E a une obligation de résultat. Le simple constat de l'absence de résultat engage sa responsabilité.

		\subsection{Distinction avec le contrat de mandat}

			\subsubsection{Le critère : l'absence de représentation}

				Le contrat de louage d'ouvrage ne donne pas lieu à représentation.

				\begin{conseil}
					Il faut toutefois faire attention aux cas où un contrat de mandat se superpose.
				\end{conseil}

			\subsubsection{Intérêt de la distinction}

			L'intérêt est double :
			\begin{itemize}
				\item d'une part pour le paiement des travaux commandés.
				\item d'autre part à la réception d'un ouvrage.
			\end{itemize}

			Dans le contrat de louage d'ouvrage le \Mo{} n'est pas lié par les contrats du \lo{}\footnote{Attention toutefois à l'application de la théorie du mandat apparent}, alors que dans le contrat de mandat le mandataire << s'efface >>.

			Dans le cadre d'un contrat de mandat, le mandant est débiteur d'une obligation de moyen. Dans le contrat de louage d'ouvrage, le \lo{} est débiteur d'une obligation de résultat et il est soumis à la responsabilité décennale.

		\subsection{Distinction avec la vente}

			Il s'agit de la distinction la plus délicate.

			\subsubsection{Le critère}

				\paragraph{La construction d'un immeuble\\}

				 Dans cette configuration il faut s'interroger sur l'identité du propriétaire du terrain.

				Si le client est propriétaire, il s'agit d'un contrat de louage d'ouvrage.

				Si le terrain appartient à l'\E{}, il s'agit d'un contrat de vente d'immeuble\footnote{Attention à la \VEFA}.

				\paragraph{Les travaux exécutés sur un immeuble\\}

				Dans ce cas, le critère est celui de la conception des produits fournis.

				Si la conception est le fait du donneur d'ordre et porte sur un produit individualisé, non substituable, alors il s'agit d'un contrat de louage d'ouvrage.

				Si le produit est standard, il s'agit d'un contrat de vente.

				Si le contrat porte sur des travaux et de la fourniture --- adaptation de produits << catalogues >> par exemple --- il faut alors mesurer l'importance de chaque prestation. Si la part de travaux matériels, appréciée d'une manière globale et pas seulement financière, est la plus importante, alors il s'agit d'un contrat de louage d'ouvrage.

			\subsubsection{Intérêt de la distinction}

				Le régime de responsabilité est différent. Dans le cadre d'un contrat de vente, elle s'appuie sur le vice caché. Dans le cadre d'un contrat de louage d'ouvrage elle repose sur l'\articleCciv{1792} et sur les garanties spécifiques.

		\subsection{Distinction avec les contrats de construction}

			La loi de 1967 a créé plusieurs types de contrat de construction.

			\subsubsection{Les types de contrats}

				\paragraph{La vente d'immeuble à construire\\}

				\begin{citationArticleCciv}{1601-1}
					La vente d'immeubles à construire est celle par laquelle le vendeur s'oblige à édifier un immeuble dans un délai déterminé par le contrat.

					Elle peut être conclue à terme ou en l'état futur d'achèvement.
				\end{citationArticleCciv}

				Le critère est celui de la propriété du terrain. Si le terrain appartient au << client >>, il s'agit d'un contrat de louage d'ouvrage. Si le terrain appartient au constructeur, il s'agit d'un contrat de vente d'immeuble à construire.

				L'idée est de protéger l'acquéreur avec un régime d'ordre public favorable au \Mo.

				\paragraph{Le contrat de promotion immobilière\\}

				\begin{citationArticleCciv}{1831-1}
					Le contrat de promotion immobilière est un mandat d'intérêt commun par lequel une personne dite << promoteur immobilier >> s'oblige envers le maître d'un ouvrage à faire procéder, pour un prix convenu, au moyen de contrats de louage d'ouvrage, à la réalisation d'un programme de construction d'un ou de plusieurs édifices ainsi qu'à procéder elle-même ou à faire procéder, moyennant une rémunération convenue, à tout ou partie des opérations juridiques, administratives et financières concourant au même objet. Ce promoteur est garant de l'exécution des obligations mises à la charge des personnes avec lesquelles il a traité au nom du maître de l'ouvrage. Il est notamment tenu des obligations résultant des articles 1792, 1792-1, 1792-2 et 1792-3 du présent code.

					Si le promoteur s'engage à exécuter lui-même partie des opérations du programme, il est tenu, quant à ces opérations, des obligations d'un locateur d'ouvrage.
				\end{citationArticleCciv}

				Le propriétaire du terrain confie au prestataire (le << promoteur immobilier >>) la réalisation d'un programme de construction d'un ou de plusieurs édifices \textbf{pour son compte}. Le promoteur immobilier peut réaliser lui-même tout ou partie des travaux.

				Il s'agit d'un contrat hybride, en partie contrat de louage d'ouvrage, en partie contrat de mandat.

				Le promoteur immobilier est considéré comme un constructeur au sens de l'\articleCciv{1792-1}. Il est donc soumis, en plus de la \rcdc, aux garanties spécifiques : \lesGarSpec. Il est également soumis à l'obligation d'assurance.

				\paragraph{Le contrat de construction des maisons individuelles (\CCMI)}

					est un contrat de louage d'ouvrage. Il est défini par les articles \articleCodifie[L]{231-1} et \articleCodifie[L]{232-1} du \cch.

					On lui applique les règles de droit commun en matière de louage d'ouvrage, auxquelles se rajoute des règles spécifiques visant à favoriser la protection de l'acquéreur.

					Un \ccmi{} doit être conclut impérativement si :
					\begin{enumerate}
						\item le terrain appartient au \Mo{} ;
						\item le contrat porte sur une maison individuelle ne comportant pas plus de deux logements ;
						\item soit l'\E{} s'engage à réaliser le gros œuvre, soit il propose les plans.
					\end{enumerate}

					\subparagraph{Intérêt de la distinction}

					L'intérêt réside une fois de plus dans le régime de responsabilité.

					Le constructeur de maison individuelle est débiteur des \garSpec{}, mais en plus, s'agissant d'un contrat d'ordre public, son formalisme doit être respecté à peine de nullité du contrat et de sanction pénale.

	\section{La formation du contrat}

		Le \Mo{} est libre du nombre de contrat. On parle de << marché en corps d'état séparé >> --- marché qui présente \emph{a priori} un avantage sur le prix et un inconvénient en matière de coordination des travaux --- lorsque celui-ci contractualise avec plusieurs entrepreneurs, et de marché << tout corps d'état >> dans le cas contraire --- marché qui présente \emph{a priori} un avantage en matière de coordination des travaux et un inconvénient sur le prix.

		\subsection{Les conditions de validité du contrat}

		Ces conditions sont fonctions du contrat et de la responsabilité, mais consistent principalement :
		\begin{enumerate}
			\item dans le consentement des parties (dont les trois types de vice sont : l'erreur, la violence et le dol) ;
			\item dans la capacité des parties ;
			\item et dans le caractère licite et certain du contenu du contrat.
		\end{enumerate}

		\begin{conseil}
			On se reportera utilement à consulter les ouvrages sur le droit des obligations qui abondent depuis la réforme de 2016.
		\end{conseil}

		\subsection{Le mode de conclusion du contrat}

			Il s'agit d'un contrat consensuel, librement consenti sans règle de forme, même si en pratique l'écrit est fortement recommandé\footnote{uniquement pour la preuve} notamment en matière d'étendue de la mission. L'écrit est obligatoire si le montant est supérieur à \montant{1 500}. On parle << d'acte d'engagement >>.

			Les contrats sont conclus soit de gré à gré, soit après mise en concurrence.

			\subsubsection{Les marchés de gré à gré}

			Ils sont définis à l'\articleCciv{1110} :
			\begin{quote}
				\itshape Le contrat de gré à gré est celui dont les stipulations sont négociables entre les parties. \lips
			\end{quote}

			\subsubsection{Les contrats conclus après mise en concurrence}

			Ce sont ceux qui résultent d'un appel d'offre.

			Il y a toujours possibilité de poursuivre un appel d'offre par une négociation de gré à gré.

		\subsection{Le contenu du contrat}

			Le contrat doit comporter des précisions quant :
			\begin{enumerate}
				\item au prix ;
				\item aux caractéristiques de l'ouvrage ;
				\item et à ses conditions juridiques et administratives.
			\end{enumerate}
			Par ailleurs, depuis 2016, certaines clauses sont réputées non écrites.

			\subsubsection{Le prix}

				Le prix est un élément essentiel au contrat. A défaut de prix, le contrat peut être requalifié en contrat d'assistance bénévole qui n'ouvre pas le droit aux \garSpec.

				\paragraph{Les règles générales}

				\emph{Quid} en absence de stipulation du prix ?

				Avant la réforme des obligations, le juge pouvait fixer le prix. Depuis la réforme, le prix est fixé par le l'\E (le créancier) en application de l'\articleCciv{1165}. Une contestation est possible en cas d'abus.

				Les clauses de variation du prix sont licites, tant en matière d'indexation --- variation du prix entre son évaluation et son paiement --- qu'en matière d'actualisation --- variation du prix entre la signature du contrat et le début des travaux.

				\paragraph{Le prix dans le marché à forfait}\index{MarcheForfait@Marché à forfait}

					Il s'agit du type de prix le plus courant. Il s'agit d'un prix fixé à l'avance et globalement pour des travaux dont la nature et la consistance sont précisément définies. Il est défini aux articles 1793 et 1794 du \cciv.

					\begin{citationArticleCciv}{1793}
						Lorsqu'un architecte ou un entrepreneur s'est chargé de la construction à forfait d'un bâtiment, d'après un plan arrêté et convenu avec le propriétaire du sol, il ne peut demander aucune augmentation de prix, ni sous le prétexte de l'augmentation de la main-d'œuvre ou des matériaux, ni sous celui de changements ou d'augmentations faits sur ce plan, si ces changements ou augmentations n'ont pas été autorisés par écrit, et le prix convenu avec le propriétaire.
					\end{citationArticleCciv}

					\begin{citationArticleCciv}{1794}
						Le maître peut résilier, par sa seule volonté, le marché à forfait, quoique l'ouvrage soit déjà commencé, en dédommageant l'entrepreneur de toutes ses dépenses, de tous ses travaux, et de tout ce qu'il aurait pu gagner dans cette entreprise.
					\end{citationArticleCciv}

					\subparagraph{Les conditions du marché à forfait}

					Elles sont au nombre de trois.

					\begin{enumerate}\index{MarcheForfait@Marché à forfait!Conditions}
						\item \textbf{Le prix global et définitif}. S'agissant d'un contrat consensuel, il n'y a pas de condition de forme. Toutefois, s'il est précisé que le prix est forfaitaire ou si \lips.

						Le marché à forfait est compatible avec l'indexation ou l'actualisation du prix.

						La jurisprudence admet la mixité des prix (en partie forfaitaire, en partie autre chose $\dots$)

						\item \textbf{La construction d'un bâtiment}. Il faut noter que les cours d'appel sont souple sur l'interprétation de ce critère, alors que la \CourDeCas est rigide à ce propos.	Par exemple, cette dernière a refusé le caractère de bâtiment au terrassement d'un bassin ou à la construction d'une piscine. Mais elle l'a reconnu pour une rénovation importante\footnote{\jurisCourDeCas[16-14317]{civ 3\ieme{}}{27/4/2017}}.

						\item \textbf{L'existence d'un << plan arrêté et convenu >>}. On entend par plan tous les documents contractuels qui définissent le bâtiment à réaliser, y compris un devis ou tout autre document descriptif.

						Le << plan arrêté et convenu >> doit l'être avec le propriétaire du sol.
					\end{enumerate}

					L'\articleCciv{1793} n'est pas d'ordre public, les parties peuvent donc convenir de ne pas s'y soumettre même si les conditions sont réunies. De manière similaire, il est possible de convenir de s'y soumettre alors même que les conditions ne sont pas réunies. Il faut cependant que l'intention des parties soit clairement exprimé\footnote{A titre d'exemple : arrondir le prix n'a pas été jugé suffisant --- \jurisCourDeCas{civ. 3\ieme{}}{25/3/2014}.}.

					Il est également possible d'aménager la sortie du forfait. Par exemple, le contrat peut prévoir des travaux supplémentaire possible avec un mandat donné à l'\archi.

					\subparagraph{La stabilité du marché à forfait\\}

					Le prix d'un marché à forfait est stable quel qu'en soit le motif\footnote{\jurisCourDeCas[06-18876]{civ. 3\ieme{}}{18/12/2007}}. Il ne peut pas être unilatéralement modifié même au motif d'un changement ou d'une erreur dans la description des travaux.

					Toutefois, le \Mo a pu obtenir remboursement des trop perçus au motif d'une erreur de métrés et dans le cas d'un contrat qui prévoyait leur vérification, au titre de la bonne foi et de l'obligation de conseil de l'\E.

					La réforme du droit des contrats a introduit dans le \cciv un nouvel article 1195\footnote{Ordonnance \no 2016-131 du 10 février 2016 - art. 2} :
					\begin{quote}
						\itshape Si un changement de circonstances imprévisible lors de la conclusion du contrat rend l'exécution excessivement onéreuse pour une partie qui n'avait pas accepté d'en assumer le risque, celle-ci peut demander une renégociation du contrat à son cocontractant. Elle continue à exécuter ses obligations durant la renégociation.

						En cas de refus ou d'échec de la renégociation, les parties peuvent convenir de la résolution du contrat, à la date et aux conditions qu'elles déterminent, ou demander d'un commun accord au juge de procéder à son adaptation. A défaut d'accord dans un délai raisonnable, le juge peut, à la demande d'une partie, réviser le contrat ou y mettre fin, à la date et aux conditions qu'il fixe.
					\end{quote}
					Cet article s'applique-t-il aux marchés à forfait ? Les marchés à forfait y échappent-ils en vertu de l'adage \emph{specialia generalibus derogant} ? En tout état de cause, il est possible d'écarter contractuellement l'application de cet article car il n'est pas d'ordre public.

					\bigskip Le principe est donc que le prix est intangible. L'exception est qu'il est possible de sortir du marché à forfait. Il faut distinguer si les travaux sont nécessaires ou non à la perfection de l'immeuble.
					\begin{itemize}
						\item \textbf{Les travaux supplémentaires nécessaires} sont ceux indispensables à la réalisation de l'ouvrage selon les règles de l'art --- par exemple : des travaux de soutènement --- ou indispensable à la sécurité de l'immeuble --- par exemple : mise en place de garde-corps.

						Ils doivent par principe être compris dans le forfait. L'\E ne peut pas demander une augmentation du prix, sauf si le \Mo est d'accord pour payer\footnote{Le juge peut déduire des éléments qui lui sont présentés que le \Mo était implicitement d'accord de manière non équivoque (volonté non équivoque visible).}.

						\item \textbf{Les travaux supplémentaires non nécessaires} sont les autres travaux.

						Par principe ils ne donnent pas lieu à paiement, sauf accord du \Mo (intangibilité du prix). Néanmoins leur paiement est possible :
						\begin{itemize}
							\item \textbf{conformément à l'\articleCciv{1793}}, s'il y a accord écrit préalable du \Mo\footnote{Et non du \Moe ou du \Mo délégué en l'absence de mandat express} sur les travaux et sur leur prix --- il est à noté que l'inscription des travaux sur un compte rendu de chantier ne suffit pas\footnote{Il serait possible d'imaginer invoquer la théorie << l'enrichissement injustifié >>, mais il serait vain de l'invoquer pour des travaux non nécessaires.} ;

							\item par appréciation souveraine du juge \textbf{en cas d'acceptation express et non équivoque \emph{a posteriori} des travaux par le \Mo}\footnote{Il s'agit d'une création praetorienne, invoquable dès lors que le \Mo a accepté sans équivoque les travaux supplémentaires après leur réalisation.} --- il a été ainsi jugé que << le paiement \lips vaut acceptation des travaux >> ;

							\item par appréciation souveraine du juge \textbf{en cas de bouleversement de l'économie du contrat}\footnote{Il n'y a pas de règle, mais 25 à \pourcent{30} semble être une règle tacite.}, notamment si le \Mo apporte en cours de travaux des modifications substantielles au marché initial d'une telle ampleur que celui-ci les a implicitement accepté\footnote{\jurisCourDeCas{civ. 3\ieme{}}{19/01/2017}}.
						\end{itemize}
					\end{itemize}

				\paragraph{Le prix dans le marché au métré\\}

				Le règlement est effectué en appliquant des prix unitaires, soit des prix unitaires spécialement établis --- on parle alors de << bordereau de prix >> --- soit des tarifs préexistant --- on parle dans ce cas de << série de prix >> --- soit sur un devis.

				Le prix global n'est pas défini à la signature mais devient définitif à la fin du marché.

				Il faut néanmoins apporter kla preuve de l'accord du \Mo, et l'on parle dans ce cas de << commande >> lorsque cet accord est antérieur aux travaux, et << d'acceptation >> lorsqu'il est postérieur.

				\paragraph{Le prix dans les marchés sur dépenses contrôlées\\}

				Également appelé << cost + fit contract >>. Les travaux sont rémunérés sur les dépenses réelles majorées d'un certain pourcentage pour les frais généraux.

				Ce type de contrat est utilisé lorsqu'il n'y a aucune visibilité sur les coûts des travaux --- par exemple : le tunnel sous la Manche. Il s'agit d'un contrat très risqué pour le \Mo, car l'aléa repose sur lui.

				Généralement, on fixe un prix cible et on prévoit comment l'acrt entre le prix réel et ce prix cible sera réparti, que l'écart soit favorable ou défavorable.

			\subsubsection{Les caractéristiques de l'ouvrage}

				Le contrat précise la nature, la consistance et la qualité des prestations. Cette composante s'apprécie avec l'ensemble des documents annexés ou visés dans le contrat.

				\begin{citationArticleCciv}{1166}
					Lorsque la qualité de la prestation n'est pas déterminée ou déterminable en vertu du contrat, le débiteur doit offrir une prestation de qualité conforme aux attentes légitimes des parties en considération de sa nature, des usages et du montant de la contrepartie.
				\end{citationArticleCciv}

			Par ailleurs, cette aspect peut évoluer grâce à des avenants.

			\subsubsection{Les conditions juridiques et administratives du marché}

			Le contrat peut fixer les conditions de réalisation des travaux et les modalités d'exécution du marché. Il peut renvoyer pour cela à des documents externes comme la norme AFNOR.

			Le contrat peut imposer des délais, une date butoir, ou encore une période butoir. Il peut prévoir des causes légitimes de suspension --- intempéries, grèves, \etc
			Les critères de la force majeure\footnote{L’événement doit avoir un caractère extérieur, imprévisible et inévitable (anciennement irrésistible). Cf. l'\articleCciv{1218}.} n'entrent pas en ligne de compte dans les causes de suspension légitime, mais les cas de force majeure peuvent faire partie des causes de suspension légitime.

			Le juge doit appliquer les stipulations contractuelles.

			\subsubsection{Les clauses réputées non écrites}

			\begin{citationArticleCciv}{1170}
				Toute clause qui prive de sa substance l'obligation essentielle du débiteur est réputée non écrite.
			\end{citationArticleCciv}

			A ce titre, sera réputée non écrite une cluse limitant ou exonérant de réparation.

			\bigskip Par ailleurs, il y a possibilité de requalification d'un contrat en contrat d'adhésion.
			\begin{citationArticleCciv}{1110}
				\lips

				Le contrat d'adhésion est celui qui comporte un ensemble de clauses non négociables, déterminées à l'avance par l'une des parties.
			\end{citationArticleCciv}


			\begin{citationArticleCciv}{1171}
				Dans un contrat d'adhésion, toute clause non négociable, déterminée à l'avance par l'une des parties, qui crée un déséquilibre significatif entre les droits et obligations des parties au contrat est réputée non écrite.

				L'appréciation du déséquilibre significatif ne porte ni sur l'objet principal du contrat ni sur l'adéquation du prix à la prestation.
			\end{citationArticleCciv}

	\section{Les effets du contrat}

		\subsection{Les obligations de l'\E}

			\subsubsection{L'obligation d'information et de conseil}

				Cette obligation est fondée sur le qualités professionnelles du \lo. Sa compétence est présumée totale et absolue. Il doit connaitre les règles de l'art, même non codifiées.

				Le \lo doit informer, conseiller et se renseigner --- notamment sur l'attente du \Mo. Il doit se renseigner sur la finalité de l'ouvrage, quand bien même il y a un \Moe. Toutefois, le \lo n'est pas tenu d'informer de ce qui est évident.

				Il est soumis à une obligation de moyen. La preuve est à la charge du \lo, sauf dans certains cas :
				\begin{itemize}
					\item si le \Mo est plus compétent que le constructeur\footnote{Appréciation souveraine du juge du fonds.} --- par exemple si s'agit d'un ouvrage très spécifique comme une clinique vétérinaire ;

					\item s'il y a plusieurs intervenants.
				\end{itemize}

				Il appartient au \lo de refuser des travaux dangereux ou inefficace. Il doit attirer l'attention sur les erreurs du \Moe, mettre en garde contre les propres choix du \Mo --- en particulier sur les conséquences négatives.

				Si les manquements à ses obligations se traduisent par un désordre, il y a basculement de la \rcdc à celles spécifiques des constructeurs. L'obligation de moyen se mue alors en obligation de résultat. Elle est renforcée en l'absence d'\archi.

				\bigskip Par ailleurs, il est bien sûr tenu par l'obligation d'information pré-contractuelle.

				\begin{citationArticleCciv}{1112-1}
					Celle des parties qui connaît une information dont l'importance est déterminante pour le consentement de l'autre doit l'en informer dès lors que, légitimement, cette dernière ignore cette information ou fait confiance à son cocontractant.

					Néanmoins, ce devoir d'information ne porte pas sur l'estimation de la valeur de la prestation.

					Ont une importance déterminante les informations qui ont un lien direct et nécessaire avec le contenu du contrat ou la qualité des parties.

					Il incombe à celui qui prétend qu'une information lui était due de prouver que l'autre partie la lui devait, à charge pour cette autre partie de prouver qu'elle l'a fournie.

					Les parties ne peuvent ni limiter, ni exclure ce devoir.

					Outre la responsabilité de celui qui en était tenu, le manquement à ce devoir d'information peut entraîner l'annulation du contrat dans les conditions prévues aux articles 1130 et suivants.
				\end{citationArticleCciv}


			\subsubsection{L'obligation de conservation des existants}

				Le \lo doit vérifier au préalable que les travaux sont compatibles avec les existants.

				\paragraph{En cas de sinistre avant la réception} on applique la << théorie des risques >> des articles 1788 et 1789 du \cciv.

				\subparagraph{Le \lo fournit la matière} la perte est donc pour l'\E jusqu'à la réception, sauf s'il a mit en demeure le \Mo de recevoir la chose.

				\begin{citationArticleCciv}{1788}
					Si, dans le cas où l'ouvrier fournit la matière, la chose vient à périr, de quelque manière que ce soit, avant d'être livrée, la perte en est pour l'ouvrier, à moins que le maître ne fût en demeure de recevoir la chose.
				\end{citationArticleCciv}

				\subparagraph{Le \lo fournit le travail} il n'est donc responsable qu'en cas de faute. Néanmoins la jurisprudence considère qu'il y a présomption de faute, qui n'est toutefois pas irréfragable.

				\begin{citationArticleCciv}{1789}
					Dans le cas où l'ouvrier fournit seulement son travail ou son industrie, si la chose vient à périr, l'ouvrier n'est tenu que de sa faute.
				\end{citationArticleCciv}

				Cette théorie s'applique également si il y a perte de l'ouvrage avant la réception.

				\paragraph{En cas de sinistre après la réception des travaux} La jurisprudence a admis l'application de l'\articleCciv{1792} dans deux hypothèses :
				\begin{enumerate}
					\item si les désordres sont causés aux existants indissociables des travaux neufs ;

					\item s'il n'est pas possible de déterminer si les dommages proviennent des travaux neufs ou des existants\footnote{Cette hypothèse vise principalement les cas d'effondrement.}.
				\end{enumerate}

				Dans les autres cas, on applique le droit commun, l'\E n'est tenu que par sa faute.

			\subsubsection{L'obligation d'exécution conforme}

				L'\E doit exécuter de qui a été commandé et qui a été stipulé dans le contrat.

				\paragraph{Le respect des délais convenus} implique que la livraison de l'ouvrage doit se faire à la date déterminée par le contrat --- l'\E étant tenu à une obligation de résultat.

				A défaut de précision, le contrat doit être réalisé dans un délai raisonnable, sauf :
				\begin{itemize}
					\item si la cause est légitime,
					\item si force majeure,
					\item si faute du \Mo --- retard dans les autorisations, modification constante du projet \etc,
					\item si faute d'un tiers --- mais pas d'un co-\lo.
				\end{itemize}
				Dans le cas contraire, des pénalités sont possibles si prévues au contrat. S'il n'ya pas de pénalités prévues, il reste possible d'en toucher sur la base :
				\begin{itemize}
					\item soit des << illisible >> dommages,
					\item soit des troubles de jouissance.
				\end{itemize}

			Les pénalités sont des clauses pénales, donc réductibles par le juge en vertu de l'\articleCciv{1231-5}\footnote{\itshape Lorsque le contrat stipule que celui qui manquera de l'exécuter paiera une certaine somme à titre de dommages et intérêts, il ne peut être alloué à l'autre partie une somme plus forte ni moindre.

				Néanmoins, le juge peut, même d'office, modérer ou augmenter la pénalité ainsi convenue si elle est manifestement excessive ou dérisoire.

				Lorsque l'engagement a été exécuté en partie, la pénalité convenue peut être diminuée par le juge, même d'office, à proportion de l'intérêt que l'exécution partielle a procuré au créancier, sans préjudice de l'application de l'alinéa précédent.

				Toute stipulation contraire aux deux alinéas précédents est réputée non écrite.

				Sauf inexécution définitive, la pénalité n'est encourue que lorsque le débiteur est mis en demeure.}. On demande plutôt dans ce cas des dommages et intérêts. En pratique on trouve des clauses de tolérances.

				\paragraph{Le respect des prestations et de la qualité convenues}

		\subsection[L'obligation du \Mo]{L'obligation du \Mo : le paiement du prix}

			Les obligations du \Mo sont au nombre de quatre :
			\begin{itemize}
				\item il doit être de bonne foi ;
				\item il doit informer l'\E, notamment sur la nature du sol ;
				\item il ne doit pas gêner --- tant en s'immisçant dans les travaux qu'en changeant d'avis de manière intempestive ;
				\item mais surtout, \textbf{il doit payer le prix} des travaux.
			\end{itemize}

			\subsubsection{Les modalités du paiement du prix}

				Le \Mo doit payer le prix convenu aux époques convenues.

				L'\articleCodifie[L]{111-3-1} du \cch relatif au paiement entre professionnel pose le principe du versement d’acompte et rappelle les délais maximums de l'\articleCodifie[L]{441-6} du Code du commerce. Il intègre le délai de vérification du \Moe dans le délai de paiement.

				En cas de non paiement \textbf{la suspension des travaux est autorisée}.

				La norme AFNOR prévoit un échéancier, un état de situation et un \dgd.

				Des pénalités de retard sont possibles.

			\subsubsection{La retenue de garantie}\index{RetenueGarantie@Retenue légale de garantie}\label{retenueLegaleDeGarantie}

				La retenue de garantie est une technique développée par les Maîtres d'Ouvrage et consacrée par la loi \no 71-584 du \printdate{16/7/1971} tendant à réglementer les retenues de garantie en matière de marchés de travaux définis au 3\ieme{} alinéa de l'\articleCciv{1779}.

				Elle a pour objet de << \textit{satisfaire, le cas échéant, aux réserves faites à la réception par le maître de l'ouvrage} >>. La jurisprudence a étendu ses effets aux non façons, mais a refusé de la faire jouer pour les retards ou abandons de chantiers\footnote{Il est toujours possible de provoquer une réception, notamment pour la faire jouer.}.

				Elle est limitée à \pourcent{5} du montant des acomptes successifs, et ne s'applique que si elle a été contractualisée. Les disposition de la loi du \printdate{16/7/1971} sont d'ordre public, elles s'appliquent donc en bloc dès lors que la retenue de garantie a été contractualisée.

				Les sommes doivent être consignées entre les mains d'un tiers, et il est possible d'y substituer une caution équivalente.

				Sa mise en jeux requière trois conditions cumulatives :
				\begin{enumerate}
					\item l'existence d'une réception,
					\item assortie de réserves,
					\item non levées par l'\E.
				\end{enumerate}

				Délai d'un an à compter de la réception, sauf opposition motivée de la part du \Mo auprès du consignataire.

			\subsubsection{La garantie de paiement de l'entrepreneur}

				Il s'agit d'une création de la loi \no94-475 du \printdate{10/6/1994} qui a introduit l'\articleCciv{1799-1}.

				Elle garantie à l'\E le paiement du prix. Elle ne porte que sur les travaux exécutés. Suite à l'arrêt du \printdate{1/12/2004} de la 3\ieme{} chambre civile de la \CourDeCas, elle est d'ordre public et il n'est pas besoin qu'elle soit contractualisée pour qu'elle s'applique.

				\begin{citationArticleCciv}{1799-1}
					Le maître de l'ouvrage qui conclut un marché de travaux privé visé au 3\degres{} de l'article 1779 doit garantir à l'entrepreneur le paiement des sommes dues lorsque celles-ci dépassent un seuil fixé par décret en Conseil d'État.

					Lorsque le maître de l'ouvrage recourt à un crédit spécifique pour financer les travaux, l'établissement de crédit ne peut verser le montant du prêt à une personne autre que celles mentionnées au 3\degres{}  de l'article 1779 tant que celles-ci n'ont pas reçu le paiement de l'intégralité de la créance née du marché correspondant au prêt. Les versements se font sur l'ordre écrit et sous la responsabilité exclusive du maître de l'ouvrage entre les mains de la personne ou d'un mandataire désigné à cet effet.

					Lorsque le maître de l'ouvrage ne recourt pas à un crédit spécifique ou lorsqu'il y recourt partiellement, et à défaut de garantie résultant d'une stipulation particulière, le paiement est garanti par un cautionnement solidaire consenti par un établissement de crédit, une entreprise d'assurance ou un organisme de garantie collective, selon des modalités fixées par décret en Conseil d'État. Tant qu'aucune garantie n'a été fournie et que l'entrepreneur demeure impayé des travaux exécutés, celui-ci peut surseoir à l'exécution du contrat après mise en demeure restée sans effet à l'issue d'un délai de quinze jours.

					Les dispositions du présent article ne s'appliquent pas aux marchés conclus par un organisme visé à l'article L. 411-2 du code de la construction et de l'habitation, ou par une société d'économie mixte, pour des logements à usage locatif aidés par l'État et réalisés par cet organisme ou cette société.
				\end{citationArticleCciv}

				\paragraph{Le champs d'application de la garantie}

					Elle ne s'applique que si le montant dépasse un certain seuil, et elle ne bénéficie que à l'\E.

					\subparagraph{Les marchés concernés} Ce sont les marchés de travaux privés commandés par un \Mo professionnel ou profane, dont le prix, déduction faite des arrhes et des acomptes, est supérieur à \montant{12 000}\footnote{Décret du \printdate{30/7/1999}.}. Les travaux commandés par des sociétés d'\HLM ou des \SEM sont exclus.

					\subparagraph{Les débiteurs de la garantie} Le débiteur est le \Mo, qu'il soit professionnel ou profane. La loi du \printdate{1/2/95} a exclus certaines dispositions pour le \Mo profane. Il n'est tenu que d'une forme de garantie spécifique : le crédit spécifique.

					\subparagraph{Les bénéficiaires de la garantie} Le bénéficiaire est l'\E qui contracte, et donc ni l'\archi ni le bureau d'étude technique. Elle bénéfice également aux sous-traitants.

				\paragraph{Le mécanisme de la garantie}

					Il peut être de trois types :
					\begin{enumerate}
						\item si le \Mo sollicite un crédit spécifique, alors versement du crédit directement à l'\E ;
						\item si non, le \Mo peut fournir une garantie particulière --- une garantie à première demande par exemple ;
						\item si non, le \Mo peut mettre en place un cautionnement solidaire.
					\end{enumerate}

					\subparagraph{Le versement direct du crédit} Il n'est possible que si le crédit est spécifique, c'est-à-dire exclusivement et en totalité destiné aux travaux. Le versement se fait même si le crédit ne finance qu'en partie les travaux.

					Sa mise en œuvre se fait uniquement sur demande écrite du \Mo. c'est une délégation de paiement.

					\subparagraph{Les autres garanties}

					Le 3\ieme{} alinéa de l'\articleCciv{1799-1} ne s'impose qu'aux professionnels. Il peut consister en :
					\begin{enumerate}
						\item stipulation particulière --- consignation du prix, hypothèque, garantie à première demande, \etc ;
						\item un cautionnement solidaire --- sans privilège de discussion.
					\end{enumerate}

					Seule l'absence totale de garantie est sanctionnée.

				\paragraph{Le moment auquel la garantie doit être fournie} Il n'y a pas de moment imposé, elle peut être demandée à tout moment, même après la réalisation des travaux, même après le réception ou la résilitaion du marché.

				\paragraph{La sanction du défaut de garantie} L'\articleCciv{1799-1} prévoit dans son 3\ieme{} alinéa la possibilité pour l'\E de sursoir à l'exécution du contrat.

				Dans le cas où :\index{GarantiePaiement@Garantie de paiement!Conditions@Conditions de sursi à l'exécution du contrat}
				\begin{enumerate}
					\item le \Mo ne propose pas de garantie,
					\item l'\E demeure impayé,
					\item après mise en demeure ;
				\end{enumerate}
				l'\E peut saisir le juge pour ordonner de constituer un garantie --- en référé --- sous astreinte.

	\section{La fin du contrat}

		\subsection{L'interruption du contrat en cours d'exécution}

			\subsubsection{La nullité du contrat}

				Le contrat peut être interrompu par la nullité du contrat (en cas d’incapacité, du vice du consentement et contenu illicite)

				\textbf{Prescription en nullité} : 5 ans à compter de la conclusion du contrat


				En construction les demandes de nullité sont rares car souvent il y a eu des commencements d’exécution, les conséquences de la nullité sont donc trop importantes et risquées.

				Il existe quand même la nullité du contrat de sous-traitance qui est intéressant (cf ci-après) : l'article 14 de la loi \printdate{31 /12/1975}, le sous-traitant peut demander la nullité du contrat de louage d’ouvrage si l’entrepreneur principal n’a pas fourni de garantie de paiement --- différent de la garantie de paiement du << \Mo >> prévue à l'\articleCciv{1799-1}.
				Ici, il s’agit d’une cause de nullité propre au contrat de sous-traitance.



			\subsubsection{La résolution du contrat}\index{Resolution@Résolution du contrat}\label{resolutionContrat}

				\paragraph{Les cas de résolution du contrat}

					\subparagraph{La résolution par le jeu d'une clause résolutoire}

					\subparagraph{La résolution unilatérale}

							L'\articleCciv{1226} prévoit que le créancier peut à ses risques et périls, résoudre le contrat par voie de notification avec au préalable mise en demeure de satisfaire à ses engagements.


							Alors le créancier est en droit de résoudre le contrat qu’il notifie à son débiteur avec les raisons qui la motive.


							Le débiteur peut saisir le juge pour contester la résolution alors le créancier doit démontrer les raisons sérieuses.


					\subparagraph{La résolution judiciaire}

							La résolution peut être en toute hypothèse demandée en justice. Elle est prononcée par la juge.

					\subparagraph{La résolution en cas de force majeure}\index{CasForceMajeure@Cas de force majeure}

						L'\articleCciv{1218} prévoie dans son deuxième alinéa : la suspension du contrat en cas d'empêchement provisoire, et la résolution en cas d'empêchement définitif.

						\begin{citationArticleCciv}{1218}
							Il y a force majeure en matière contractuelle lorsqu'un événement échappant au contrôle du débiteur, qui ne pouvait être raisonnablement prévu lors de la conclusion du contrat et dont les effets ne peuvent être évités par des mesures appropriées, empêche l'exécution de son obligation par le débiteur.

							Si l'empêchement est temporaire, l'exécution de l'obligation est suspendue à moins que le retard qui en résulterait ne justifie la résolution du contrat. Si l'empêchement est définitif, le contrat est résolu de plein droit et les parties sont libérées de leurs obligations dans les conditions prévues aux articles 1351 et 1351-1.
						\end{citationArticleCciv}

						La force majeure est appréciée par les juges pour chaque cas, l’appréciation se faisant \emph{in concreto} donc selon les circonstances, chaque cas étant un cas particulier.

						Trois critères cumulatifs permettent d'identifier juridiquement la force majeure. ce sont, depuis la réforme\footnote{Avant la réforme du droit des contrats de 2016, il s'agissait de : imprévisibilité, irrésistibilité et extériorité} :
						\begin{itemize}
							\item \textbf{Extériorité} \emph{échappant au contrôle du débiteur}. Ce critère s’apprécie généralement par rapport à une personne. La personne concernée n’est en rien responsable de la survenance de l’événement. L’événement est totalement indépendant de ce qu’il souhaite, de sa volonté. L’événement ne doit en rien pouvoir être imputé à la personne.

							\item \textbf{Imprévisibilité} \emph{qui ne pouvait être raisonnablement prévu}. L’événement concerné ne doit, par aucun moyen, pouvoir être anticipé ou prévu.

							\item \textbf{Inévitabilité} \emph{dont les effets ne peuvent être évités par des mesures appropriées}. Le caractère inévitable d’un événement est primordial pour que la force majeure soit juridiquement reconnue. Cela signifie qu’il doit être impossible d’éviter l'événement. Ou plus précisément, il est impossible d’éviter ses conséquences. Elle est caractérisée dès lors que les conséquences de l’événement surviennent malgré le fait que tout ait été mis en œuvre pour les réduire ou les éviter. Malgré toutes les précautions, les conséquences sont inévitables.
						\end{itemize}

					\subparagraph{La résolution en cas de circonstance imprévisibles}

						La réforme des obligations de 2016 a consacré la théorie de l’imprévision, qui était tirée de la théorie des sujétions imprévues en droit public admis, en l'encadrant par
l'\articleCciv{1195}.

						\begin{citationArticleCciv}{1195}
							Si un changement de circonstances imprévisible lors de la conclusion du contrat rend l'exécution excessivement onéreuse pour une partie qui n'avait pas accepté d'en assumer le risque, celle-ci peut demander une renégociation du contrat à son cocontractant. Elle continue à exécuter ses obligations durant la renégociation.

							En cas de refus ou d'échec de la renégociation, les parties peuvent convenir de la résolution du contrat, à la date et aux conditions qu'elles déterminent, ou demander d'un commun accord au juge de procéder à son adaptation. A défaut d'accord dans un délai raisonnable, le juge peut, à la demande d'une partie, réviser le contrat ou y mettre fin, à la date et aux conditions qu'il fixe.
						\end{citationArticleCciv}

						La partie doit continuer à exécuter ses obligations pendant la négociation.


						En cas de refus ou échec de négociation, les parties peuvent convenir de la résolution du contrat avec possibilité de demander au juge d’un commun accord de procéder à son adaptation.


						Les dispositions de l'\articleCciv{1195} ne sont pas d’ordre public. Il faut donc vérifier ce qu’il y a dans le contrat, le risque d’imprévision peut peser sur chacune des parties et donc pas de nature à remettre en cause le contrat, mais il faut l’écarter conventionnellement.


				\paragraph{Les effets de la résolution du contrat}

					En application de l'\articleCciv{1229}, <<~\emph{la résolution met fin au contrat} >> :
					\begin{itemize}
						\item soit dans les conditions prévues par la clause résolutoire ;
						\item soit à la date de la réception par le débiteur de la notification faite par le créancier ;
						\item soit à la date prévue par le juge.
					\end{itemize}

					Si les prestations échangées ne trouvent leur utilité que dans l’exécution complète du contrat résolu, alors la résolution oblige les parties à des restitutions réciproques.


					Mais si l’utilité se trouve au fur et à mesure de l’exécution du contrat, alors pas de restitution pour la période antérieure à la résolution.

					Alors on parle de résiliation du contrat, avec effet sur le contrat que pour l’avenir.


					Donc pour les dommages d’ouvrage essentiellement résiliation, sauf si les travaux n’ont pas commencé alors résolution.


					Les restitutions sont encadrées par les articles 1352 à 1352-9 du \cciv. L'\articleCciv{1352-8} concerne notamment la restitution en cas de prestation de service.


			\subsubsection{La résiliation du contrat}\index{Resiliation@Résiliation du contrat}\label{resiliationContrat}

				C’est un anéantissement du contrat pour le futur. Plusieurs cas de résiliation prévues par le code ou conventionnellement.

				\paragraph{La résiliation par le décès de l'entrepreneur}

					L'\articleCciv{1795} dispose que le contrat de louage d’ouvrage est dissout de plein droit par la mort de l’ouvrier, de l’architecte ou de l’entrepreneur



					L'\articleCciv{1795} n’est pas d’ordre public et donc peut être écartée par les parties.
La norme AFNOR en fait une cause de résiliation de plein droit aux torts du défaillant.


					L'\articleCciv{1796} prévoit les conséquences de la résiliation pour le décès.
Si contractant est une personne morale, il ya alors renvoi à la liquidation judiciaire qui emporte résiliation du contrat de louage d’ouvrage.



				\paragraph{La résiliation par le jeu d'une clause de résiliation de plein droit}

					Les parties peuvent insérer dans le contrat de louage d’ouvrage des clauses de résiliation de plein droit. La clause doit identifier clairement les modalités de mise en jeu de cette clause, préciser le comportement sanctionné, les conditions de la résiliation --- avec ou pas \med infructueuse et pendant combien de temps --- et les conséquences de la résiliation --- avec ou sans pénalités, avec ou sans \di.


					La norme AFNOR érige comme cause de plein droit : l’abandon de chantier, l'interruption de plus de 6 mois imputable au \Mo, la tromperie grave sur la qualité ou l’exécution des travaux.



				\paragraph{Le cas particulier de la résiliation unilatérale du marché à forfait}

					L'\articleCciv{1794} stipule que si les conditions du marché à forfait sont remplies ou si les parties s’y soumettent, le \Mo peut résilier de sa seule volonté le marché à forfait, en dédommageant l’\E de toutes ses dépenses, de tous ses travaux et de tout ce qu’il aurait pu gagner pour son entreprise (manque à gagner).


					En pratique la résiliation unilatérale de l’\articleCciv{1794} est peu mise en œuvre.


					\bigskip La résiliation et résolution ne sont pas les seules sanctions envisageables en cas d’inexécution :
					\begin{itemize}
						\item Le co-contractant peut refuser ou suspendre ses obligations en application des articles 1219 et 1220 du Code civil : exception d’inexécution.

						\item L’exécution forcée peut être invoquée, mais l'\articleCciv{1221} ne la rends pas possible en cas de disproportion manifeste entre le coût pour le débiteur et l’intérêt pour le créancier.

						Le \Mo a la possibilité de faire exécuter l’obligation par un tiers au frais du co-contractant.


						C’est une possibilité expressément offerte en cas de garantie de parfait achèvement, mais sous contrôle du juge.


						\item Le \Mo peut également solliciter une réduction du prix prévue à l'\articleCciv{1223}. Le créancier peut après \med accepter une exécution imparfaite du contrat et solliciter une réduction du prix.
Le créancier peut demander réparation des préjudices résultant de l’inexécution du contrat, soit des \di, cela peut être cumulé avec l’acceptation de l’exécution imparfaite.

					\end{itemize}


		\subsection{La fin du contrat par le prononcé de réception}

			La réception marque la fin du contrat de louage d’ouvrage. C’est le terme de la relation contractuelle.


			\medskip Ensuite on bascule dans la responsabilité.


			\medskip La réception n’impose pas l’achèvement des travaux. Cette position de la \CourDeCas est bancale car considère que la réception met fin aux relations contractuelles mais la construction ne sera jamais parfaite. Il y a des tolérances (exemple de tolérance dans le DTU).

			\medskip La réception marque le point de départ des garanties spécifiques. Il peut donc y avoir réception d’un ouvrage inachevé.


			L’\articleCciv{1792-6}, sur la responsabilité des constructeurs, définit la réception : acceptation de l’ouvrage avec ou sans réserve, soit à l’amiable ou à défaut judiciairement. Elle doit être contradictoire.


			C’est un acte juridique unilatéral avec des conséquences considérables, le \Mo donne quitus à l’entrepreneur sur l’exécution de ses travaux. S’il y a des désordres, non-conformité, il faut les réserver, de manière exhaustive, tous les désordres, les non façons malfaçons, vices de construction apparents doivent être listés.


			Si trop de malfaçons, l’ouvrage ne peut être reçu.


			Il y a également la réception tacite qui est une création prétorienne.


			La réception rend le prix du marché exigible, à l’exception le cas échéant de la retenue légale de garantie. Le prix est celui initialement convenu + les pénalités de retard éventuelles.


			La réception a pour effet de transférer les risques de l’ouvrage, après la réception, le risque pèse sur le \Mo.


			La réception purge les désordres apparents à défaut de mention dans le \pv de réception, le \Mo ne pourra plus obtenir réparation de ces désordres sur une quelconque fondement à l’encontre de l’entrepreneur.


			Celui qui invoque l’apparence du désordre devra le prouver.


			La \jp a conscience du caractère sévère de la purge, on apprécie donc l’apparence aux yeux d’un \Mo profane, au regard des désordres apparents et ses conséquences.


			Tous les désordres cachés sont couverts par les garanties spécifiques et de droit commun.


			La réception marque le point départ des garanties spécifiques et des responsabilités contractuelles de droit commun comme pour les dommages intermédiaires.


			Elle marque également le point de départ des assurances obligatoires.


			Enfin, elle permet la mise en œuvre de la retenue légale de garantie. Ces sommes consignées ou cautionnées pour lever les désordres réservés, donc nécessairement une réception.

		\chapter{Le contrat de l'architevte (ou contrat de maîtrise d'œuvre)}
		% !TEX root = ./droitConstruction.tex

\chapter{Le contrat de sous-traitance}

	Article 1710 : c’est un contrat de louage d’ouvrage. Il est signé entre deux locateurs d’ouvrage : l’entrepreneur principal et un sous-traitant. Principes du consensualisme et de la liberté contractuelle président à la conclusion du contrat de sous-traitant, sauf pour \CCMI où les contrats de sous-traitance doivent être rédigés par écrit en vertu de l'\articleDu[L]{231-13}{\cch}.

	De principe pas de régime du marché à forfait\footnote{Puisque l'Entrepreneur Principal n'est en principe pas propriétaire du sol} sauf si les parties y conviennent conventionnellement.


	Avant la loi du \printdate{31/12/1975} pas de réglementations spécifiques sur la sous-traitance, avant c’était les dispositions du louage d’ouvrage qui s’appliquaient. Or pas de garantie pour le sous-traitant pour le règlement.


	Désormais réglementation particulière de la sous-traitance


	Loi \printdate{31/12/1975} : principal objectif de garantir le paiement du sous-traitant, en revanche cette loi ne traite pas de la problématique de responsabilité du sous-traitant, juste article 1 « entrepreneur principal sous-traite sous sa responsabilité » donc intervention de la \jp qui définit tout le régime de responsabilité du sous-traitant



\section{Les conditions de la sous-traitance}

	\begin{citationArticleLoi}[75-1334]{1}{31/12/1975}
		Au sens de la présente loi, la sous-traitance est l'opération par laquelle un entrepreneur confie par un sous-traité, et sous sa responsabilité, à une autre personne appelée sous-traitant l'exécution de tout ou partie du contrat d'entreprise ou d'une partie du marché public conclu avec le maître de l'ouvrage.
	\end{citationArticleLoi}

	Cette loi concerne tant les marchés publics que privés.

	\subsection{Les conditions générales de sous-traitance}

		\subsubsection{La succession de deux contrat d'entreprise}

			Il faut nécessairement succession de deux contrats d’entreprise, le premier entre \E et Entrepreneur Principal, et un deuxième contrat entre l’Entrepreneur Principal et le Sous-traitant.


			La qualification du premier contrat est donc primordiale.


			Au même titre, il faut s’intéresser à la qualification du deuxième contrat (pas mandat, vente, travail)

			S’agit-il vraiment d’un sous-traitant ?


			\paragraph{La qualification du contrat en cas de réalisation de travaux d'ordre matériel ou intellectuel} La qualification du second contrat ne pose pas de problème en cas de travaux matériels.


				\subparagraph{La sous-traitance de prestation intellectuelle est-elle possible ?}
				En 1984 la 3\ieme{} chambre civile a considérée que oui c’est possible, donc le deuxième contrat peut porter sur des travaux d’ordre intellectuel.

				\begin{exemple}
					architecte, bureau d’étude
				\end{exemple}

				L'article 37 du code de déontologie des architectes interdit la sous-traitance du projet architectural. A défaut, le \Mo n’est pas tenu de payer les sommes exposés par le \Moe\footnote{\jurisCourDeCas[16-15958]{\civTrois*}{27/04/2017}}.

			\paragraph{La qualification du contrat en cas de fourniture de matériaux}

				\subparagraph{Peut-il y avoir sous-traitance lorsque le co-contractant se contente de fournir des matériaux ?}
				Si que fourniture de matériaux deux qualifications possibles : vente ou louage d’ouvrage
.

				\jurisCourDeCas[83-16675]{\civTrois*}{5/02/1985}  : critère de la conception de l’ouvrage ou des matériaux et de la spécialité des fournitures


				Si fourniture sur plan du donneur d’ordre et spécificité pour le chantier alors louage d’ouvrage alors sous-traitance, il faut que les fournitures aient été réalisés sur les plans de l’entrepreneur donneur d’ordre et que le produit soit individualisé et non substituable, produit spécifiquement réalisé pour le chantier

				\subparagraph{Quid en cas de fourniture de matériaux et de travaux réalisés sur place ?}

				Si fourniture de matériaux et prestation, qualification de louage d’ouvrage si vraiment travaux sur place.


				Mais si les travaux réalisés consistent à de simple adaptation de produit standard, alors contrat de vente avec prestation de service.


				Les juges doivent apprécier l’importance de chaque prestation, ici appréciation souveraine des juges du fonds.
				La loi de 1975 est d’ordre public, mais son interprétation est jurisprudentielle.

				Pour éviter le contrat de vente, il faut une spécificité des matériaux pour le chantier



			\paragraph{La qualification du contrat en cas de prestation de service ou de fourniture de main-d'œuvre}

				Le second contrat qui porte sur une prestation de service ou sur fourniture de main d’œuvre est un contrat de sous-traitance ou juste louage de chose et de fourniture de main d’œuvre ?

				\begin{exemple}
					Quid pour la mise en place d’échafaudage ?
Pas un contrat de louage d’ouvrage.

				\end{exemple}

				\subparagraph{Si prestation de service} est sous-traitant à 2 conditions :

				\begin{itemize}
					\item Autonomie du personnel de prestataire donc absence du lien de subordination

					\item Participation directe du prestataire de service à l’acte de construire

				\end{itemize}
				On considère que celui qui réalise l’échafaudage ne participe pas directement à l’acte de construire par apport de conception d’industrie ou de matière, juste met à disposition le matériel adapté

				Il s’agit d’un contrat de louage de chose avec mise à disposition de main d’œuvre.


				\subparagraph{Si fourniture de personnel} un tel fournisseur ne peut pas invoquer la loi de 1975 sauf si équipe autonome avec une tâche précise.


				Donc si prêt de matériel comme grue ou de personnel, alors pas de sous-traitance



		\subsubsection{L'autonomie de l'intervenant}

			Pour qu’il ait contrat de sous-traitance il faut nécessairement une absence de subordination, le sous-traitant doit donc être autonome.


			En pratique, il va recevoir des instructions de l’Entrepreneur Principal mais choisi librement les moyens pour les mettre en œuvre. Donc relation hiérarchique mais pas de subordination.


		\subsubsection{L'étendue de la sous-traitance}


			En droit privé la sous-traitance peut être totale. Il faut alors regarder le contrat pour vérifier si contrat de sous-traitance ou de cession de contrat.

			Il s'agit d'une appréciation souveraine du juge du fonds.

			Cependant, en droit positif français, la sous-traitance est majoritairement partielle.


			En droit public, seule sous-traitance partielle est possible.



		\subsubsection{La date de conclusion du contrat de sous-traitance}

			Il n’y a pas de condition de date pour le contrat de sous-traitance, il n’est pas obligatoire qu’il soit signé après, on peut le signer avant le contrat principal dès lors qu’il est assorti d’une condition suspensive de régularisation du contrat principal.

			Le \Mo peut passer un contrat << tout corps d’état >> avec un constructeur qui va sous-traiter les lots. Dans ce cas-là, il va signer des contrats de sous-traitance sous condition suspensive de signer le contrat principal, pour ensuite signer le contrat principal.
Cela permet au constructeur de proposer une offre juste.


			Attention, différence de la sous-traitance avec la co-traitance dans le groupement momentané d’entreprise, ici le \Mo va conclure avec le groupement qui n’a pas de personnalité morale donc contrat de louage d’ouvrage mais les entrepreneurs entre eux sont en co-traitance. Ici la loi de 1975 ne s’applique pas.


			On peut interdire le contrat de sous-traitance dans un contrat de droit privé.



	\subsection{Les conditions spécifiques à la sous-traitance en matière de contrat de \ccmi}

		Le contrat de \ccmi obéit à réglementation particulière, qui distingue deux types de contrats selon qu'il soit avec ou sans fourniture de plan :
		\begin{itemize}
			\item \articlesDuEtSuivants[L]{231-1}{\cch} ;
			\item  \articlesDuEtSuivants[L]{232-1}{\cch} ;
		\end{itemize}

		Le \ccmi est un contrat d’entreprise, un contrat de louage d’ouvrage.


		Les contrats passés par le constructeur de maisons individuelles avec les entreprises sont des contrats de sous-traitance. La loi du \printdate{31/12/1975} s’applique. Aux terme de l'\articleDu[L]{231-13}{\cch} le constructeur est tenu de conclure par écrit le contrat de sous-traitance et le contrat doit contenir les justificatifs de garantie de paiement. Il s'agit d'une condition de forme, donc un écrit avant tout travaux.


		La sanction est pénale : 2 ans de prison et \montant{18.000} d’amende.  Il en va de même si écrit mais pas de justificatif de garantie de paiement.


\section{Une loi d'ordre public}

	La loi du \printdate{31/12/1975} est d’ordre public, son article 15 dispose en effet que sont << {\itshape nuls et de nul effet, quelle qu'en soit la forme, les clauses, stipulations et arrangements qui auraient pour effet de faire échec aux dispositions de la présente loi}.
>>

	Le sous-traitant ne peut pas renoncer aux droits conférés par la loi de 1975\footnote{\jurisCourDeCas{\civTrois*}{9/7/2003}}.
De même, interdiction toute renonciation de la garantie\footnote{\jurisCourDeCas{\civTrois*}{14/09/2017}}.
	\subparagraph{Quid en cas de contrats internationaux ?}
	Deux grandes idées :

	\begin{itemize}
		\item les dispositions d'ordre public s’appliquent aux parties françaises alors même que la construction serait réalisée à l’étranger. Donc entrepreneur principal doit fournir la garantie de paiement à son sous-traitant\footnote{\jurisCourDeCas{\civTrois*}{14/10/1992}}.
		\item s'agissant d'une loi de police au sens de la convention de Rome, elle doit  être respectée par toute personne si la construction est réalisée en France\footnote{\jurisCourDeCas{ch. mixte}{30/11/2007} ;  \jurisCourDeCas{\civTrois*}{25/02/2009}}.
	\end{itemize}


\section{La protection du sous-traitant dans les marchés privés}

	La loi protège le sous-traitant contre la faillite de l’\ep.


	La loi de 1975 ne traite que des problèmes financiers et non de responsabilité.


	La loi le protège face au concours des créanciers en cas de liquidation judiciaire. Le sous-traitant est alors dans une situation à part, la loi lui offrant un autre débiteur que l’\ep.



	\subsection{La protection principale du sous-traitant}

		La loi impose aux partenaires du sous-traitant des obligations particulières.

		\subsubsection{Les obligations imposés à l'\ep}

			\paragraph{L'obligation de faire accepter le sous-traitant et faire agréer ses conditions de paiement\\}

				\begin{citationArticleLoi}[75-1334]{3}{31/12/1975}
					L'entrepreneur qui entend exécuter un contrat ou un marché en recourant à un ou plusieurs sous-traitants doit, au moment de la conclusion et pendant toute la durée du contrat ou du marché, faire accepter chaque sous-traitant et agréer les conditions de paiement de chaque contrat de sous-traitance par le maître de l'ouvrage ; l'entrepreneur principal est tenu de communiquer le ou les contrats de sous-traitance au maître de l'ouvrage lorsque celui-ci en fait la demande.

					Lorsque le sous-traitant n'aura pas été accepté ni les conditions de paiement agréées par le maître de l'ouvrage dans les conditions prévues à l'alinéa précédent, l'entrepreneur principal sera néanmoins tenu envers le sous-traitant mais ne pourra invoquer le contrat de sous-traitance à l'encontre du sous-traitant
				\end{citationArticleLoi}

				\subparagraph{Le contenu de l'obligation} L’\ep doit demander au \Mo d’exprimer son accord sur la présence du sous-traitant sur le chantier et ses conditions de paiement.
				Il s'agit d'une obligation de résultat, mais l’obligation ici est de présentation pour faire agréer, mais pas d’obtenir l'agrément.


				Si le contrat principal interdit la sous-traitance, alors l’agrément sera obligatoire. Mais le contrat principal est silencieux la \jp a considéré que le caractère discrétionnaire du refus est limité par un éventuel abus de droit\footnote{\jurisCourDeCas{\civTrois*}{2/02/2005} : caractère discrétionnaire jusqu’à l’abus de droit}.
				Le \Mo peut donc refuser un sous-traitant pour des motifs raisonnables.


				En cas de sous-traitance en chaine, la qualité de \Mo reste sur le \Mo mais la qualité d'\ep se décale. Chaque sous-traitant devant présenter et faire agréer son propre sous-traitant et lui accorder une garantie de paiement.
				Il n’y a pas de limite sauf disposition contractuelle contraire.

				En pratique les entrepreneurs principaux ne présentent pas leur sous-traitant pour ne pas révéler les marges, d'autant plus que les sanctions sont limitées.


				\subparagraph{La sanction de l'obligation : l'impossibilité pour l'entrepreneur principal d'invoquer le contrat de sous-traitance}

					Dans un arrêt du \printdate{13/04/1998} la 3\ieme{} chambre civile de la \CourDeCas a interprété et considérer que le sous-traitant non agréé ne peut pas à la fois se prévaloir du contrat de sous-traitance pour le paiement, et le rejeter quant à son obligation de résultat. Le sous-traitant demeure donc tenu à son obligation de résultat.

					L'hypothèse dans laquelle le sous-traitant peut valablement invoquer la sanction en cas d'absence d'agrément pour échapper à une obligation est celle dans laquelle le sous-traitant cherche à éviter des pénalités de retard.

			\paragraph{L'obligation de fournir au sous-traitant une garantie de paiement, soit avant, soit au plus tard au moment de la conclusion du contrat}

				L’obligation de l’\ep est de présenter et faire agréer, mais pas d’obtenir l’agrément, et de fournir une garantie de paiement.

				\subparagraph{Le contenu de l'obligation}

					L’entrepreneur doit fournir une garantie de paiement à son sous-traitant avant, ou au plus tard en même temps, que la conclusion du contrat de sous-traité.

					\bigbreak Deux garanties alternatives qui s’imposent à l’\ep :

					\begin{enumerate}
						\item \textbf{Caution personnelle et solidaire.}

						La caution doit être personnelle. La \CourDeCas fait une lecture stricte de l’article 14, la caution doit être accordée nominativement au sous-traitant. L’acte de caution doit indiquer le nom du sous-traitant et le montant de son marché.

						Cette décision semble condamner le système de caution flotte \cad une caution générale pour tous les contrats pour une année donné par un établissement bancaire et financier. cependant, dans un arrêt du \printdate{20/6/2012}\footnote{\jurisCourDeCas[11-18463]{\civTrois*}{20/6/2012}}, la troisième chambre civile a admis ce mécanisme à des conditions strictes : un accord cadre conclu entre une entreprise et un établissement bancaire, l’entreprise fait connaître sous forme d’un avis les contrats de sous-traitance sur lesquels elle entend user l’accord cadre en précisant le montant et l’opération, et la banque donne une attestation au nom du sous-traitant.

						La caution doit être solidaire donc pas de privilège de discussion \cad elle ne doit pas obliger le sous-traitant à agir en priorité contre le débiteur principal.

						En pratique, cette solution est peu utilisée car couteuse.

						\item 	\textbf{Délégation de paiement.}

						Ce mécanisme repose sur l'\articlesDu{1336}{\cciv} : l’\ep --- le déléguant --- peut déléguer au \Mo --- le délégué --- le paiement au sous-traitant --- le délégataire. En clair : l’\ep va demander au \Mo de payer le sous-traitant.

						Dans une délégation parfaite, le déléguant disparait, il y a novation, mais ce n’est pas notre cas, dans la délégation imparfaite, il y a un débiteur de plus. Il n’y a pas de novation, elle créé un second rapport. \aVerifier

						Il faut nécessairement l’accord du \Mo qui ne serait résulter de la simple acceptation du sous-traitant.

						La délégation de paiement peut se combiner avec la garantie de paiement de l'\articleDu{1799-1}{\cciv}

						\end{enumerate}
						Dans les faits la délégation de paiement est peu mise en œuvre malgré ses intérêts qui sont :
						\begin{itemize}
							\item Absence de concours avec les autres créanciers de l’\ep car le sous-traitant a un nouveau débiteur : le \Mo.

							\item Inopposabilité des exceptions du contrat principal et des rapports entre le déléguant et le délégataire.

							En application de l'\articleDu{1336}{\cciv}, le délégué --- le \Mo --- ne peut, sauf stipulation contraire, opposer au délégataire --- le sous-traitant --- aucune exception tirée de son rapport avec le déléguant ou des rapports entre ce dernier et le délégataire. Le \Mo ne peut donc pas refuser de payer le sous-traitant pour des raisons tirées de son contrat avec l’\ep, ni invoquer des exceptions provenant du contrat entre l'\ep et le sous-traitant.
							\aVerifier Il y a inopposabilité des exceptions du contrat principal et du sous-traité, et l'on peut uniquement opposer les exceptions de leur relation sur un fondement délictuel\footnote{\jurisCourDeCas{\civTrois*}{07/06/2018} concernant l’inopposabilité des exceptions en matière de délégation de paiement pour les contrats de sous-traitance}.
						\end{itemize}


						\bigbreak En matière de \CCMI, le législateur a créé une troisième branche de garantie : l'\articleDu[L]{331-13}{\cch} fourni au sous-traitant toute forme de crédit donnant l’assurance de paiement. \aVerifier


						\bigbreak La question de constitutionnalité de l'article 14 a été posée, et la \CourDeCas a jugé qu'il n'avait pas lieu de renvoyer au Conseil constitutionnel une \qpc qui ne présentait pas de caractère sérieux dès lors que le législateur a prévu une alternative gratuite par la délégation de paiement\footnote{\jurisCourDeCas{\civTrois*}{10/06/2014}}.



						\bigbreak {\bfseries Jusqu'à quand le sous-traitant peut-il réclamer à l'entrepreneur principal une garantie de paiement ?}
						Il faut d'abord souligner que le sous-traitant ne peut pas valablement renoncer à la garantie de paiement de l'article 14 de la Loi dès lors que les dispositions que cet article renferme sont d'ordre public.

						Le sous-traitant peut solliciter auprès de l'entrepreneur principal une garantie de paiement tant qu'il n'a pas été intégralement réglé des sommes qui lui sont dues en vertu du contrat de sous-traitance.

				\subparagraph{La sanction de l'obligation : la nullité du contrat de sous-traitance}

					L'article 14 prévoit que la sanction du défaut de garantie est la nullité du contrat de sous-traitance, le sous-traité, si aucune garantie n’est mise en œuvre.


					C’est une nullité relative prescrite sous 5 ans à compter de la conclusion du contrat\footnote{\jurisCourDeCas{\civTrois*}{20/02/2002}}.

					La nullité du contrat de \ST* peut être demandée par le \ST alors qu’il a intégralement été payé. Dans ce cas l’intérêt est d’échapper aux pénalités de retard ou obtenir une sortie du forfait\footnote{\jurisCourDeCas{Civ}{18/07/2001}}.

\aCompleter

					Dans le cas du \CCMI, l'\articleDu[L]{231-13}{\cch} prévoit que le constructeur doit fournir dans le contrat, le justificatif de la caution ou de la délégation. La sanction est pénale\footnote{\ArticleDu[L]{241-9}{\cch}}.


			\paragraph{L'interdiction de nantir ou céder à une banque une créance supérieure à celle qui résulte des travaux qu'il effectue personnellement pour le compte du \Mo}

				L'article 13-1 de la \loiST stipule que l’\ep ne peut céder ou nantir qu’à concurrence du contrat avec \Mo.

				\subparagraph{Le contenu de l'obligation}

					L'\ep peut céder ou nantir la totalité de sa créance s’il constitue une caution personnelle ou solidaire au profit de son \ST.

					\begin{exemple}
						dans le cadre d'un marché de \montant{1 000 000} mais avec \montant{600 000} de \ST*, l'\ep ne peut nantir ou céder que \montant{400 000}.
					\end{exemple}

					Cette disposition interdit au banquier d’accepter une cession ou créance supérieur au montant de son obligation



				\subparagraph{La sanction de l'obligation : l'inopposabilité de la cession de la créance excessive}

					Inopposabilité de la cession excessive, si la banque reçoit du \MO une somme supérieure de la part de l’\ep.\aVerifier

					\begin{exemple}
						dans le cadre d'un marché marché de \montant{1 000 000}, l'\ep soustraite pour \montant{600 000} et effectue \montant{400 000} de travaux personnellement, il peut donc céder ou nantir \montant{400 000}. S'il cède  \montant{500 000}, il y a une créance excessive de \montant{100 000},  et le \ST peut agir à l’encontre de la banque pour obtenir \montant{100 000} de celle-ci.
					\end{exemple}

					On s’est aperçu que les garanties de paiement n’étaient pas efficaces donc mise en place d’obligations à l’égard du MO



		\subsubsection{Les obligations imposées au \Mo}

			L'article 14-1 de la loi \printdate{6/1/1986} \aVerifier stipule que le \Mo doit vérifier que les obligations de l’entrepreneur ont été respectées, car il détient le pouvoir économique sur l’opération.

			\paragraph{Le champs d'application des obligations du \Mo}

				\subparagraph{La nature du contrat} La loi s'applique pour les marchés privés et publics.

				\subparagraph{Conditions d’application }
Il s’agit de conditions cumulatives :\index{SousTraitance@\ST!Conditions}
				\begin{enumerate}
					\item \textbf{Le type de prestation} : l'article 14-1 s’applique aux travaux de bâtiment, à cet égard la \CourDeCas a une vision extensive et les travaux de démolition peuvent avoir la notion juridique de bâtiment\footnote{\jurisCourDeCas{\civTrois*}{24/09/2014}\aVerifier}.


					Elle s’applique également à la sous-traitance industrielle


					\item 	\textbf{La qualité du \MO} : l'article s'applique aux \MO personnes morales et personnes physiques professionnels. Ainsi, le \MO personne physique qui fait construire pour lui-même et sa famille n’est pas tenue des obligations.


					\item 	\textbf{Le \MO doit avoir connaissance de l’existence du \ST sur le chantier}. La connaissance est un fait juridique qui se prouve par tout moyen. C’est le ST qui doit prouver que le MO savait qu’il intervenait sur le chantier. Pour cela il peut se fonder sur l’indication du ST sur le CR de chantier, la présence de véhicule avec logo sur le chantier.

					\item 	\textbf{Le \MO ne doit pas avoir payé intégralement l’\ep} au jour de la connaissance de la présence du \ST sur le chantier

				\end{enumerate}


				Le \ST n’a pas d’obligation de se déclarer auprès du \MO, il ne peut se voir reprocher de ne pas s’être rapproché du \MO. Néanmoins on peut recommander au \ST de se manifester au \MO pour bénéficier des garanties de la loi.
En effet,  aux termes de l'article 14-1, le \MO en est tenu, dès qu’il en a connaissance, alors même que le chantier est terminé ou que le \ST n’est pas ou plus sur le chantier, tant que le marché n’est pas soldé. Cet article s'applique en cas de \ST* en chaine, le \MO restant toujours le même.


			\paragraph{L'étendue des obligations}

				Il existe 2 types obligations sur le \MO, qui sont prévues aux
article 3 et 14.

				\subparagraph{Première obligation}
				Dans le cas d'un \ST qui n’a pas été présenté ni ses conditions de paiement agréées, celui-ci doit mettre en demeure l’\ep de respecter ses obligations.

				Le \MO est alors libre d’accepter ou non le \ST sauf abus de droit.

				\subparagraph{Deuxième obligation}

				Si le \MO a accepté ou agréé le \ST il doit exiger de l’\ep de fournir une garantie de paiement.

				Obligation de vérification de la caution bancaire, et vérification de la preuve de la transmission au \ST de l’identité de l’établissement financier.

				Obligation de résultat du \MO, pas juste \MED, il doit faire pression sur l’entrepreneur principal d’obtenir la preuve de la garantie de paiement, par la menace de résiliation du contrat, le \MO n’est pas fautif s’il résilie le contrat car c’est en raison de la faute de l’\ep.

			\paragraph{La sanction des obligations : la responsabilité pour faute du \Mo}

				\subparagraph{Responsabilité pour faute du \MO.} Le \MO engage sa responsabilité délictuelle à l’égard du \ST s’il n’obtient pas la preuve de garantie de paiement ou de délégation.

				Il appartient au \ST de caractériser la faute, le préjudice, le lien de causalité.

				\subparagraph{Faute}
				Caractérisée à défaut de \MED de l’\ep.


				La \CourDeCas a pu juger que le \MO engage sa responsabilité, après \med restée infructueuse, s’il ne met pas tout en œuvre pour obtenir la preuve des garanties nécessaires.


				\subparagraph{Préjudice}



				La question s’est posée de savoir si le dommage consistait en la perte de chance d'être payé ou le non-paiement.
 La \JP a considéré que le préjudice est l’impossibilité d’être payé, et donc que le dommage réside dans le fait que le \ST n’a pas pu être payé..

				Il y alors réparation intégrale de sa créance.

				Le \ST peut agir à l’encontre du \MO sans justifier l’impossibilité de recouvrir sa créance à l’égard de l’entrepreneur principal, c’est une action directe sur le \MO.

				En cas d’annulation du contrat de \ST*, le règlement de \DI peut être fixé à hauteur du montant des travaux réalisés\footnote{\jurisCourDeCas{\civTrois*}{18/02/2015}}

				\subparagraph{Lien de causalité}

				La faute du \MO qui a privé \ST du bénéfice de garantie qu’il lui aurait permis d’être intégralement payé.

				\bigbreak \textbf{Attention : limite à l'action du \ST} Le \ST ne peut plus agir s’il a été intégralement payé sauf si nullité et que le montant des travaux dépasse le marché à forfait\footnote{\jurisCourDeCas{\civTrois*}{1/1/1}\aVerifier}

				Elle tient à la fois à la date où le \MO a connaissance de l’existence du \ST sur le chantier \textbf{et} sur le montant des sommes qu’il doit encore à l’entrepreneur principal le jour de la connaissance.

				Si le \MO doit \montant{300 000} à l’entrepreneur au moment où il a connaissance du \ST sur le chantier, alors l’action du \ST sur le \MO repose sur cette assiette, même s’il le \ST avait une créance de \montant{500 000}.


				Attention si entre le jour où il a connaissance de la présence du \ST et l’action du \ST le \MO a payé l’entrepreneur il peut être condamné a payé 2 fois, la connaissance a pour effet de cristalliser les sommes entre ces mains.\aVerifier

				Il aura la possibilité de se retourner contre l’entrepreneur principal sur le fondement de la répétition de l’indu mais si en faillite alors il devra déclarer sa créance.


				Prescription de l’action du \ST sur le \MO est de 5 ans à compter de la connaissance de l’existence du \ST.

				Le \MO peut voir sa responsabilité engagée si en cas de condamnation, alors il a une action récursoire sur le maitre d’œuvre qu’il s’est abstenu d’alerter le \MO de la présence du \ST sur le chantier\footnote{\jurisCourDeCas[09-11562]{\civTrois*}{10/02/2010} : responsabilité du Maitre d’œuvre faute d’avoir alerté la présence de \ST ; \jurisCourDeCas[13-24892]{\civTrois*}{10/12/2014} : responsabilité du Maitre d’œuvre engagée si abstenu d’avertir le \MO de la présence du \ST \textbf{et} faute de lui avoir conseillé de respecter ses obligations \cad de se le faire présenter à l’acceptation et à l’agrément}.



	\subsection{La protection subsidiaire du sous-traitant : l'action directe}

		Les article 11 à 13
sont d'ordre public, il n'est pas possible de renoncer à l’action directe. Le sous-traitant y bénéficie dès lors que les conditions sont réunies.

		\subsubsection{Les conditions de l'action directe}

			\paragraph{Les marchés concernés}

				Ils sont visés à l'article 11 de la loi : à tous les contrats de sous-traitant qui ne rentrent pas dans le champ d’application du paiement direct.

				Le paiement direct est prévu par le titre 2 de la loi en son article 4 : aux marchés publics passés en application de l’ordonnance du 23 Juillet 2015 d’un montant supérieur à 600 € = champ d’application du paiement direct. \aVerifier

				\medbreak Ce qui permet de déterminer le champ d’application de l’action directe :

				\begin{itemize}
					\item Les marchés privés

					\item Les marchés publics d’un montant inférieur à \montant{600}
					\item Les marchés conclus par une personne publique d’un montant inférieur à \montant{600}

				\end{itemize}

			\paragraph{Les conditions relatives à l'auteur de l'action directe}

				L'\AD ne bénéficie qu’au sous-traitant accepté et dont les conditions de paiement ont été agréées par le \MO.

				dans l''article 3 de la loi de 1975, le législateur a précisé que l'acceptation et l'agrément sont une condition de l’action directe.

				Ces conditions sont en pratique très souvent non remplies, et cela prive le sous-traitant d’agir sur ce fondement. C’est donc première limite fondamentale.

				La \CourDeCas en est consciente et a développé une \JP favorable au \ST. Le \ST peut faire l’objet d’un agrément tacite et tardif
:
				\begin{itemize}
					\item \textbf{L’agrément tacite} est caractérisée lorsqu’il existe un acte de volonté non équivoque du \MO\footnote{\jurisCourDeCas[16-10.719]{\civTrois*}{18/05/2017} : une MED de fournir une garantie principale de paiement adressé par le MO à l’entrepreneur suivi d’effet accompagné \aVerifier vaut acceptation du \ST (arrêt d’espèce)}. Il faut donc un acte positif, une simple attitude passive étant insuffisante.

					Cet agrément tacite peut être aménagée contractuellement. Ainsi l'article 4.4.1 de la norme AFNOR stipule que si le \MO n’a pas répondu dans un délai de 15 jours après la \med de l'\ep par \LRAR, il y a alors acceptation tacite.

					\item \textbf{L'agrément tardif} a été accepté par la troisième chambre civile. L'agrément reste possible jusqu’au moment où l’action directe est exercée alors même que l’\ep est en faillite.

					En pratique si le \ST exerce l’\AD et que le \MO ne relève pas l’absence d’agrément, les juges n’ont pas à relever d’office cette absence. Donc ne pas soulever ce moyen de défense équivaut à une acceptation tardive.
				\end{itemize}

				L’action directe bénéficie au \ST direct et au \ST en chaîne, donc conditions cumulatives. Le \MO demeure toujours le même le \ST de second rang peut agir à l’égard du \MO sur le fondement de l’\AD s’il a été accepté et agréé selon les conditions déjà vues.


				L’établissement financier qui a fourni le cautionnement selon l’article 14 de la loi du \printdate{31/12/1975 } peut exercer l'\AD\footnote{\jurisCourDeCas[16-10719]{\civTrois*}{18/05/2017}}, avec toujours la condition d’acceptation et d’agrément, l’établissement étant subrogé dans les droits du \ST.



			\paragraph{Les conditions de mise en œuvre}

				L'article 12, premier alinéa, de la \loiST, l'\AD s'exerce après \MED de l’\ep par le \ST.

				il y a trois conditions cumulatives :

				\begin{enumerate}
					\item Le \ST doit mettre en demeure l’entrepreneur principal de payer. Même s’il est en liquidation judiciaire (Civ 3°)
\aVerifier

					En la forme d’une LRAR, un acte d’huissier. Une lettre simple ne suffit pas.

					\item Le \ST doit envoyer une copie de la \MED au \MO. Cela a pour effet de bloquer entre les mains du \MO l’argent qu’il doit encore à l’\ep.

					\item La \MED doit demeurer infructueuse à l’expiration d’un délai d’un mois.
				\end{enumerate}

				L’action directe subsiste même si l’entrepreneur principal est en liquidation ou redressement judiciaire\footnote{Article 12 alinéa 3}.


		\subsubsection{Les effets de l'action directe}

			Ils sont précisés à l’article 13 de la \loiST.

			\paragraph{Les créances garanties}

				Seules sont garanties les créances de travaux stipulés dans le contrat de \ST et qui bénéficient au \MO.

				Sur un chantier il y a des travaux supplémentaires qui sont par principe pas prévus au contrat initial. L'\AD st possible s’ils sont opposables au \MO au sens de l’\articleDu{793}{\cciv}.


			\paragraph{L'assiette de l'action directe}

				L’action est limitée au somme restant due par le MO à l’entrepreneur principal, moins les sommes dues par l’\ep au \MO


				\subparagraph{Les sommes dues par le \MO}
				Ce sont les sommes dues en vertu du marché au jour de la réception de la \MED, toute cause confondue au contrat de marché principal.

				Le \ST ne peut pas réclamer les sommes dues par le \MO à l’entrepreneur principal au titre d’un autre contrat, juste au titre du contrat principal qui a fait l’objet du contrat de \ST*\footnote{\jurisCourDeCas[16-10719]{\civTrois*}{18/05/2017}}.

				\subparagraph{Les sommes dues par l’entrepreneur au \MO}
				Si l’entrepreneur n’a pas respecté les délais donc possibilité d’opposer au \ST les pénalités de retard, donc opposabilité des exceptions tirées de la relation \MO - \ep.

				Cela peut être aussi le cas pour non-respect du marché (non façon ou malfaçon), le \MO peut aussi faire jouer l’exception de compensation.

				Si plusieurs \ST exercent en même temps une \AD à l'encontre du \MO pour un montant supérieur qu’il doit à l’entrepreneur principal alors répartition à part égale entre les \ST, en proportion de leur créance respective, à défaut c’est « le premier arrivé ».



			\paragraph{Le versement direct du crédit au sous-traitant}

				Le \ST peut se faire verser directement par la banque le montant du crédit consenti au \MO\footnote{Article 12, alinéa 4 de la \loiST}.

\section{La protection du sous-traitant dans les marchés de travaux publics}

	Il s'agit principalement du mécanisme du paiement direct.

	\subsection{Les sous-traitants directs}

		\aCompleter

		\subsubsection{Le paiement direct}

			\aCompleter

			\paragraph{Les conditions du paiement direct}

				\aCompleter

			\paragraph{Les modalités du paiement direct}

				\aCompleter

		\subsubsection{Les autres garanties}

			\aCompleter

	\subsection{Les sous-traitants en chaine}

		\aCompleter

\section{La fin du contrat de sous-traitance}

	Le contrat peut être annulé pour défaut de capacité, ou vice de consentement, ou contenu illicite.

	\paragraph{Nullité} Le défaut de garantie de paiement par l’entrepreneur principal est une cause de nullité spécifique.

	\paragraph{Résolution} Par application d’une clause de résolution de plein droit. Aussi résolution unilatérale ou judiciaire possible. Également en cas de force majeure ou de circonstances imprévisibles.


	Cf. section \ref{Resolution}

	\paragraph{Résiliation} Le contrat de \ST* peut être résilié par application d’une clause de résiliation de plein droit ou décès de l’une des parties

	Le sous-traité peut également être frappé de caducité en cas de disparition du contrat principal.


	Cf. section \ref{Resiliation}

	\paragraph{Caducité} La caducité est définie aux articles 1186 et 1187 du \cciv.

	\begin{citationArticle}{1186}{\cciv}
		Un contrat valablement formé devient caduc si l'un de ses éléments essentiels disparaît.

		Lorsque l'exécution de plusieurs contrats est nécessaire à la réalisation d'une même opération et que l'un d'eux disparaît, sont caducs les contrats dont l'exécution est rendue impossible par cette disparition et ceux pour lesquels l'exécution du contrat disparu était une condition déterminante du consentement d'une partie.

		La caducité n'intervient toutefois que si le contractant contre lequel elle est invoquée connaissait l'existence de l'opération d'ensemble lorsqu'il a donné son consentement.
	\end{citationArticle}

	\begin{citationArticle}{1187}{\cciv}
		La caducité met fin au contrat.

		Elle peut donner lieu à restitution dans les conditions prévues aux articles 1352 à 1352-9.
	\end{citationArticle}

	La caducité met fin au contrat et donne lieu à restitution dans les condition de l'\articleDu{1352-8}{\cciv} : en valeur à la date à laquelle la prestation a été fournie.


	Le quantum des restitutions consécutives à la caducité du contrat de \ST doit correspondre au coût de la prestation réalisée sans tenir compte du coût contractuellement convenu et sans tenir compte de la valeur de l’ouvrage réalisé.


	\paragraph{Réception} Le contrat de \ST* peut également prendre fin par le prononcé de la réception.



\section{Les règles de responsabilités liées à la sous-traitance}

	C’est la \JP qui a dégagé ce régime.

	\subsection{La responsabilité du sous-traitant à l'égard de l'\ep}

		\subsubsection{Nature et fondement de la responsabilité}

			Le \ST est un co-contractant de l’\ep donc responsabilité contractuelle.
Ce sont donc les règles du contrat d'entrepreneur qui s'applique, et en particulier l'\articleDu{1788}{\cciv} sur la perte de la chose.

			\begin{citationArticle}{1788}{\cciv}
				Si, dans le cas où l'ouvrier fournit la matière, la chose vient à périr, de quelque manière que ce soit, avant d'être livrée, la perte en est pour l'ouvrier, à moins que le maître ne fût en demeure de recevoir la chose.
			\end{citationArticle}

			\paragraph{Obligation de conseil} Le \ST est tenu d’une obligation de conseil à l’égard de l’\ep il ne doit pas se contenter de suivre les instructions de l’\ep s’il les estime contraires aux règle de l’art et le conseiller s’il estime devoir procéder différemment.


			\paragraph{Les \garSpec} Le \ST n’est pas débiteur des garanties spécifiques des constructeurs (\lesGarSpec), seuls les contractants avec le \MO y sont débiteurs.


			Mais il est possible de contractualiser les garanties spécifiques des constructeurs, et dans ce cas l’entrepreneur principal à la possibilité d'agir à l’encontre du \ST sur le fondement de ces garanties spécifiques.


			A défaut, le \ST n’est débiteur que de la responsabilité contractuelle, qui suppose en principe un lien de causalité : existence d’une faute, d'un préjudice et lien de causalité entre la faute et le préjudice\index{LienCausalité@Lien de causalité}.
Cependant la \JP a considéré que le \ST est soumis à une obligation de résultat. Il n'est donc pas nécessaire de démontrer une faute, un simple dommage suffit à engager sa responsabilité\footnote{\jurisCourDeCas{\civTrois*}{02/02/2017} ; \jurisCourDeCas{\civTrois*}{26/04/2006}}.

			L'obligation de résultat subsiste même si le \ST n'est pas accepté et agréé, et même si le \ST n’a pas été payé.

			\medbreak Le \ST est responsable en cas de dommage, de défaut de conseil, de retard de travaux, vice de construction ou de défaut de conformité.

			En cas de condamnation de l’entrepreneur principal sur les \TAV il peut se retourner contre le \ST s’il démontre la preuve d’une faute du \ST\footnote{\jurisCourDeCas{\civTrois*}{26/04/2006}}. %alors action récursoire.\aVerifier

			La responsabilité du fournisseur du \ST à l’égard de l’\ep est de nature contractuelle.



		\subsubsection{Le régime de la responsabilité}

			Il s'agit du délai de droit commun. Le recours d’un constructeur contre un autre constructeur ou un \ST
repose sur l'\articleDu{2224}{\cciv} : 5 ans à compter du jour où le premier a connu ou aurait dû connaitre les faits lui permettant de l’exercer.

			La \CourDeCas fixe le point de départ de la prescription le jour où l’entrepreneur principal a été actionné par le \MO. Soit la date de l’assignation en référé expertise, soit la date d’assignation au fond\footnote{Néanmoins, \jurisCourDeCas{\civTrois*}{13/09/2006} fixe le point de départ au jour où le dommage s’est manifesté à l’égard du \MO. Cet arrêt date toutefois d'avant la réforme du droit des obligations et de la réforme de la prescription et des décisions qui fixe le point de départ au jour de l’assignation}.


			L’action récursoire de l’entrepreneur principal sur le fondement des \TAV, alors 5 ans à compter de l’action\footnote{\jurisCourDeCas[18-25915]{\civTrois*}{16/01/2020} --- 3 arrêts attendus rendu le même jour concernant les \TAV et l'action récursoire}.


			\textbf{Pour les sinistres avant réception} on s’interroge entre l’application du délai de droit commun : \articleDu{2224}{\cciv}, et celui du \ccom : \articleDu[L]{110-4}{\ccom}. Dans tous les cas, le délai est de cinq ans.

			Les causes d’exonération sont les causes étrangères\index{CausesEtrangeres@Causes étrangères}\label{causesEtrangeres} : cas de force majeure, fait d’un tiers ou fait de la victime.



	\subsection{La responsabilité du sous-traitant à l'égard du \Mo}

		\subsubsection{Nature et fondement de la responsabilité}

			La \civUn a pu juger qu'il s'agit d'une responsabilité de nature contractuelle car présence d'une chaine homogène de contrat. Cependant, la \civTrois a jugé qu'il s'agit d'un responsabilité délictuelle. En \assPlen\footnote{\jurisCourDeCas{\assPlen}{12/07/1991}, arrêt \nom{Besse}}, la \CourDeCas a tranché en faveur de la nature délictuelle de la responsabilité du \ST.

			L’engagement de la responsabilité du \ST impose la démonstration : d’une faute, d’un préjudice et d’un lien de causalité entre les deux.

			Le \MO ne peut pas invoquer une obligation de résultat mais la faute peut trouver son origine dans le sous-traité

			La \CourDeCas considère qu’en dépit de l’effet relatif des contrat, le tiers peut invoquer un manquement contractuel dès lors que ce manquement lui a causé un préjudice, la faute contractuelle devient une faute délictuelle à l’égard d’un tiers AP 06/10/2006 AP 13/01/2020 qui est juste une confirmation.


			Le \MO peut alors se prévaloir au plan délictuel de l’inexécution du contrat entre l’entrepreneur principal et son exécutant ainsi que des fautes d’inexécution.

			La faute contractuelle du \ST à l’égard de l’entrepreneur principal est donc une faute délictuelle à l'égard du \Mo.

			Civ 3° 18/05/2017, le seul manquement à une obligation de résultat est impropre à caractériser l’existence d’une faute délictuelle

			En matière de \TAV le \MO dispose d’une action récursoire à l’encontre du ST, c’est une action fondée sur le fondement du TAV si le MO a indemnisé le voisin, alors il est subrogé dans les droits du voisin pour agir à l’encontre du ST.

			L’action est en revanche de nature délictuelle si le MO n’a pas indemnisé le voisin (le mécanisme de subrogation intervient qu’en cas de paiement) le MO doit alors apporter la preuve d’une faute.

			Le MO peut rechercher la responsabilité du ST sur le fondement de la responsabilité des produits défectueux.1245 et suivants du code civil

			La responsabilité du fournisseur du ST à l’égard du \MO est de nature délictuelle Civ 3° 21/11/2001



		\subsubsection{Le régime de la responsabilité}

			\paragraph{Prescription}
			Par 10 Ans à compter de la réception des travaux pour les dommages relevant de la garantie décennale

			2 Ans à compter de la réception des travaux 1792-3

			Sinon 1792-4-3 Code civil

			Ici on s’intéresse à la nature du désordre et pas sur le fondement juridique.

			Ici pas réception de l’ouvrage mais des travaux. La réception doit-elle intervenir par travaux ou sur l’ouvrage ?

			En pratique la question ne se pose pas toujours mais dans la rédaction du contrat « délai à compter de la réception de l’ouvrage ou des travaux dans la globalité »

			L’action récursoire du MO à l’encontre du ST dont l’activité a engendré un TAV, doit être engagée 5 ans à compter du jour où le MO a assigné le voisin.

			Pour les sinistres avant réception : 1224 et L210-4 du Code de commerce : 5 Ans à compter de la survenance du sinistre

			Les causes d’exonération sont les causes étrangères\footnote{Cf. \vref{causesEtrangeres}}.


	\subsection{La responsabilité de l'\ep à l'égard du \Mo}

		\subsubsection{Nature et fondement de la responsabilité}

			Ici responsabilité contractuelle

			L’entrepreneur principal sous-traite sous sa responsabilité et est donc responsable des fautes de son sous-traitant.

			La simple faute du ST à l’égard de l’entrepreneur principal suffit à engager la responsabilité de l’entrepreneur principal à l’égard du MO

			Civ 3° 11/05/2006

			L’entrepreneur principal est responsable des fautes de son ST à l’égard du MO et de ses acquéreurs successifs.

			La responsabilité de l’entrepreneur peut être recherché sur le fondement spécifiques des constructeurs.

			En matière de TAV le MO peut agir à l’encontre de l’entrepreneur principal sur le fondement délictuel du TAV s’il a payé la victime, et s’il n’a pas indemnisé le voisin alors que sur le fondement contractuel.



		\subsubsection{Le régime de la responsabilité}

			La prescription

			Le délai d’action varie en fonction de la nature du dommage subi par le \MO.


			Si les conditions des garanties spécifiques sont réunies alors il doit agir sur ce fondement et alors la forclusion est de 10 /2 / 1 ou 10 ans pour tous les autres 1792-4-3 du Code civil


			Pour les désordres avant réception, alors dans un délai de 5 ans à compter de la manifestation du dommage


			Les causes d’exonération sont les causes étrangères\footnote{Cf. \vref{causesEtrangeres}}.



	\subsection{La responsabilité de l'\ep à l'égard des tiers}

		\subsubsection{Nature et fondement de la responsabilité}

			L’\ep n’est pas délictuellement responsable des fautes de son \ST à l’égard des tiers\footnote{\jurisCourDeCas{\civTrois*}{8/09/2009} ; \jurisCourDeCas{\civTrois*}{22/09/2010} : l’entrepreneur principal n’est pas responsable des dommages au tiers par le ST dès lors qu’il n’est pas son commettant}. S’il sous-traite sous sa responsabilité c’est à l’égard du \MO mais pas à l’égard des tiers.


			Mais il est possible de mettre en cause la responsabilité de l’entrepreneur principal si le tiers démontre une faute personnelle de celui-ci, par exemple un défaut de surveillance. Alors le tiers peut invoquer la réalisation défectueuse de son contrat\footnote{\jurisCourDeCas{\civTrois*}{27/03/2008}}.


		\subsubsection{Le régime de la responsabilité}

			Prescription : 5 Ans à compter de la manifestation du dommage 2224 du Code civil

			Les causes d’exonération sont les causes étrangères\footnote{Cf. \ref{causesEtrangeres}}.

	\subsection{La responsabilité du sous-traitant à l'égard des tiers}

		Tiers : toute personne extérieure au chantier et tout intervenant sur le chantier extérieur à la relation de sous-traitance (différent du tiers voisin, ici tiers au chantier).

		\subsubsection{Nature et fondement de la responsabilité}

		Les voisins peuvent agir à l’encontre du \ST sur le fondement des \TAV, aux côtés de l’entrepreneur principal ou non, aux côtés du \MO ou non
.

		Il appartiendra au demandeur de démontrer un lien d’imputabilité entre l’activité du \ST et le dommage.


		\bigbreak Le tiers voisin peut aussi agir sur le fondement délictuel de l'\articleDu{1241-1}{\cciv}, mais en pratique il agira sur le fondement des  \TAV car plus simple car sinon il faut démontrer une faute, un lien de causalité, et un préjudice. Dans les \TAV juste un lien d’imputabilité


		\bigbreak Le voisin victime peut agir sur le fondement de l'\articleDu{1242}{\cciv} sur la responsabilité du fait des choses, TAV aussi

		Les colocataire d’ouvrage peuvent agir à son encontre sur le fondement de la responsabilité délictuelle en démontrant la faute


		\subsubsection{Le régime de la responsabilité}

		Prescription : 5 Ans à compter de la manifestation du dommage 2224 du Code civil

		Les causes d’exonération sont les causes étrangères


	\part{La responsabilité des constructeurs et des fabricants}

		% !TEX root = ./droitConstruction.tex

\chapter{Les garanties spécifiques des constructeurs}

Les \garSpec sont : la décennale, la biennale et la \gpa.

\section{Le point de départ des garanties : la réception}

	\index{Reception@Réception!Definition@Définition}\label{Reception@Réception}\begin{citationArticleCciv}{1792-6}
		La réception est l'acte par lequel le maître de l'ouvrage déclare accepter l'ouvrage avec ou sans réserves. Elle intervient à la demande de la partie la plus diligente, soit à l'amiable, soit à défaut judiciairement. Elle est, en tout état de cause, prononcée contradictoirement.

		\lips
	\end{citationArticleCciv}

	Il ne faut pas confondre la réception et la livraison. La livraison correspond à la prise de possession de l'ouvrage, c'est une notion que l'on retrouve surtout en \VEFA ou \aCompleter.

	La réception intervient entre le \Mo et le locateur d’ouvrage. Le \Mo déclare accepte l’ouvrage avec ou sans réserve. A l’amiable ou à défaut judiciairement, et en tout état de cause contradictoirement.

	A l’absence de réception, pas possible de mobiliser la retenue légale de garantie qui est conditionnée avec la réception avec réserve.


	En l’absence de réception, seule la responsabilité contractuelle de droit commun peut être mobilisée, 5 ans à compter de la survenance du dommage. Il s'agit d'une responsabilité de nature contractuelle nécessitant en principe : faute, préjudice et lien de causalité. La \CourDeCas considère cependant que les entrepreneurs sont tenus d’une obligation de résultat.


	Le \Moe est tenu d’une obligation de moyen donc démonstration d’une faute.


	\jurisCourDeCas{\civTrois*}{24/05/2006} : prescription de 10 ans à compter de l’apparition des dommages avant réception, mais depuis réforme de la prescription donc à priori 5 Ans.


	\subsection{Les modalités de la réception}

		\subsubsection{Le moment de la réception}

			La réception intervient à l’achèvement des travaux mais il ne s’agit pas d’une condition de la réception, il est possible de réception des travaux achevés \jurisCourDeCas{\civTrois*}{30/01/2019}


			L’intérêt est de faire jouer les assurances, \do ou responsabilité civile décennale.


			Lorsqu’il y a succession d’entreprise sur le même lot, il est recommandé à l’entreprise de provoquer la réception pour ventiler les responsabilités par rapport au prédécesseur. S’il accepte l’ouvrage en l’état alors il engage sa responsabilité.


			\paragraph{La réception est unique.}

			La question se pose de savoir si la réception est unique pour l’ouvrage en son entier ou pour chacune des entreprises.


			Il s'agit d'une question délicate car le marché peut être conclu en lot séparé.


			Si le marché est << tout corps d’état >>, alors l'unicité de la réception ne fait pas de difficulté. Il convient cependant de préciser dans le contrat \ST* que la réception des travaux du \ST interviendra avec celle de l'/ep.


			\medbreak Si l’ouvrage est réalisé par tranche, la \JP admet la réception par tranche, mais unique par tranche.


			La difficulté apparait lorsque le contrat est conclu << corps d’état séparé. >> Au regard de l’esprit du texte, il n'est \emph{a priori} pas possible de réaliser une réception par locataire d’ouvrage alors réception distincte pour chaque type de travaux.


			Néanmoins, la norme AFNOR P0301 admet la réception par entrepreneur sauf en cas d’entrepreneur groupé.


			La réception partielle par lot n’est pas prohibée par la loi. En revanche, la \civTrois a jugé en 2017 que la réception partielle à l’intérieur d’un lot n’est pas possible.


			\medbreak La \CourDeCas a admis que si les travaux sont réalisés par palier successif, il est alors possible réceptionner par palier. \aValider



		\subsubsection{Le formes de la réception}

			L'\articleDu{1792-6}{\cciv} prévoit que la réception intervient à la demande de la partie la plus diligente soit à l’amiable soit judiciaire.


			Il y a donc deux type de réception prévue par la loi : expresse ou judiciaire.

			Par ailleurs, la jurisprudence qui a découvert une troisième forme de réception dite tacite

			\paragraph{Réception expresse}\index{Reception@Réception!Expresse@Réception expresse}\label{receptionExpresse}

			\subparagraph{Contenu}

			Les parties manifestent expressément dans un document écrit leur volonté non équivoque d’accepter l’ouvrage tel qu’il a été réalisé.

			La réception est prononcée entre locataire d’ouvrage et le \MO --- ou une personne mandatée spécifiquement par celui-ci.

			En pratique l’entrepreneur et le \MO (souvent assisté au maître d’œuvre) se rendent sur place et vont faire le tour de l’ouvrage et lister dans document écrit les réserves : non façon, malfaçon et vices de construction

			\subparagraph{Caractère contradictoire}

			L’exigence de la contradiction ne nécessite pas la signature formelle du \PV, dès lors que la présence est bien établie lors de la réception.

			La \CourDeCas a pu jugé que l’absence de l’entrepreneur ne saurait priver du caractère contradictoire le \PV pour le \MO, s’il a été valablement convoqué aux opérations de conception\footnote{\jurisCourDeCas{\civTrois*}{7/3/2019} : une \LRAR 4 jours avant la réception avec en plus télécopie suffit à considérer le caractère contradictoire}.

			\subparagraph{Date de la réception expresse}


			Elle figure sur le document établi, le procès-verbal.

			Donc la réception sera prononcée au jour apposé sur le \PV. On peut tout à fait rétroagir la réception, en indiquant une date antérieure à la date d'élaboration du procès-verbal.

			Des aménagements conventionnels de la réception expresse sont également envisageables.

			Ces aménagements ont lieu généralement au moment de la conclusion du contrat de louage d’ouvrage, mais ils aussi possible par avenant.

			Cela peut consister à une pré-réception \CAD une première visite --- les OPR opération préalable à la réception --- pour qu’il y ait au moment de la réception beaucoup moins de réserves.

			\medbreak Autre possibilité : aménager la réception expresse en cas de refus de l’une des parties. L’idée est d’éviter la situation de blocage. On a une partie qui veut réceptionner l’ouvrage, si le contrat prévoit qu’en cas de convocation de l’autre partie par voie d’huissier la réception pourra intervenir et en cas de procès-verbal de constat dressé par huissier notifié à l’autre partie alors point de départ de la réception est la notification du constat.

			\paragraph{Réception tacite}\index{Reception@Réception!Tacite@Réception tacite}\label{receptionTacite}

			Il s'agit d'une création prétorienne non visée par le texte. C’est une réception amiable, non contentieuse, mais les parties ne savaient pas qu’il fallait réceptionner.

			Dans cette hypothèse, la \CourDeCas va constater la réception tacite. Elle ne fait que constater que la réception a eu lieu, ce qui est différent de la réception judiciaire qui est prononcée.

			Le juge recherche si le \MO par son comportement a manifesté sa volonté non équivoque de recevoir l’ouvrage. La réception tacite n’est pas envisageable lorsque le contrat impose une réception expresse.

			\subparagraph{Conditions d'existences}


			le juge va chercher à caractériser la volonté non équivoque du \MO d’accepter l’ouvrage\footnote{\jurisCourDeCas{\civTrois*}{13/07/2017}}.

			L’appréciation du comportement du \MO relève de l’appréciation souveraine des juges du fonds avec un faisceau d’indice :
			\begin{itemize}
				\item La prise de possession est un élement important mais non déterminant. La simple prise de possession ne suffit pas à caractériser la réception tacite.

				\item Le paiement de l’intégralité du prix.

				\item La manifestation de la volonté du \MO de déterminer les travaux.

				\item L’envoi d’une lettre de réclamation.
			\end{itemize}

			La prise de possession des lieux et le paiement du prix présume la volonté non équivoque du \MO de réceptionner l’ouvrage\footnote{\jurisCourDeCas{\civTrois*}{20/04/2017} paiement de \pourcent{95} du prix (comme la retenue légale de garantie)}.

			A défaut de prise de possession et de paiement intégral du prix la réception tacite ne peut être caractérisée\footnote{\jurisCourDeCas{\civTrois*}{13/07/2017}}.

			La réception tacite d’un \MO d’un immeuble d’habitation n’est pas soumis à la constatation de l’habitabilité et l’achèvement de l’ouvrage. %25/01/2011

			\subparagraph{Date de réception}

			IL appartient au juge de fixer la date. Le juge doit fixer la date de la réception tacite, une fois constaté la volonté non équivoque de réceptionner l’ouvrage.

			Généralement, à la date de prise de possession de l’ouvrage, ou à la date du paiement intégral, date d’envoi des réserves. Les parties peuvent conventionnellement fixer la date de la réception tacite

			L'absence de contestation dans un délai de 2 mois  vaut réception tacite\footnote{\jurisCourDeCas{\civTrois*}{04/04/2019}}.


			\paragraph{Réception judiciaire}


			En cas de situation conflictuelle entre \MO et locataire d’ouvrage, entrainant, par exemple Le \MO refuse de se rendre à la réunion de réception.

			Dans ce cas là, la partie la plus diligente a la faculté de saisir le juge pour prononcer la réception. Sachant que le refus n’a pas besoin d’être abusif, il faut juste un refus

			Seule les parties au contrat ont qualité pour demander la réception judiciaire.

			Très souvent le juge ordonnera une expertise et demande à l’expert de déterminer si l’ouvrage est en état d’être reçu\footnote{\jurisCourDeCas{\civTrois*}{12/10/2017} : un ouvrage en état d’être reçu est ouvrage utilisable conformément à sa destination et qui n'est pas être affecté de défaut ou vice substantiel. Il s'agit d'une noton différente de celle de l'achèvement de l’ouvrage, \jurisCourDeCas{\civTrois*}{18/10/2018} : il suffit que l’ouvrage soit en état d’être habité} et à quel date.

			Elle peut être prononcée avec ou sans réserve\footnote{\jurisCourDeCas{\civTrois*}{17/10/2019}}.

			\textbf{Date de la réception} : date proposée par l’expert à laquelle l'ouvrage est en état d’être reçu, ou en état d’être habité, ou date d’abandon de chantier, ou date d’entrée dans les lieux, cela est apprécié par le juge.

			Les dispositions applicables au \CCMI n’empêchent pas une réception judiciaire\footnote{\jurisCourDeCas{\civTrois*}{21/11/2019}}.

	\subsection{Les effets de la réception à l'égard des désordres affectant l'ouvrage}



		\subsubsection{Les désordres apparents réservé}

		Lorsqu'il y a les désordres apparents réservés --- que la récetion soit express, tacite ou judiciaire, la formulation de réserve permet de mettre en oeuvre la retenue légale et d’agir sur le fondement de la garantie de parfait d’achèvement.

		A l’origine les dommages réservés n’étaient repris que sur la base de la garantie de parfait achèvement d’un an.
		Il n’était pas possible la biennale et décennale et RC.

		La \CourDeCas a admis de cumuler la responsabilité de droit commun concurremment avec la garantie de parfait achèvement. Pour les autres \garSpec c’est un principe de non cumul.

		Ce cumul est permis car le délai d’un an de la \gpa est trés court alors après le délai 10 an possible d’agir sur le \RC pour les désordres réservés à la réception.

		La garantie de parfait achèvement vise la reprise des seuls dommages. Les dommages annexes seront pris en charge par la \RC, avec la preuve d’une faute.
		La \civTrois ajugé que les locateur d’ouvrage sont tenus d’une obligation de résultat, pour les désordres réservés jusqu’à la levée des réserves, même pour la contractuelle.

		Pour le parfait achèvement il y a une présomption de responsabilité, donc même pour la RC.
		Deuxième évolution favorable au \MO

		Possibilité d’invoquer la responsabilité décennale pour les désordres réservés à la réception qui se sont révélés postérieurement dans leur ampleur et leur importance, Civ 3°12/10/1994

		\textbf{Attention} : il n'y pas d’assurance responsabilité décennale pour les désordres réservés.
		Assurance \DO peut couvrir en cas de \MED de l’entrepreneur.


		\subsubsection{Les désordres apparents non réservés}

			Ils sont alors purgés. Aucune action sur le fondement de la garantie parfait achèvement, ni sur le biennal, ni sur la décennale, si RC car la C.Cass considère que ce désordre est purgé (possible manquement à la mission d’assistance de l’architecte ou Maître d’œuvre, si mission spécifique d’assistance aux opérations de réception.)

		\subsubsection{Les désordres cachés}

		Lors de la réception, s’ils sont cachés ils ne peuvent être réservés.

		La réception constitue le point de départ des désordres intermédiaires, décennale, parfait achèvement, biennal, RC
		Si les désordres se manifestent dans l’année qui suit la réception, possible sur le parfait achèvement qui se cumule avec RC ou biennale ou décennale si la gravité le permet.
		Si le désordre se manifeste Réception, alors sur le fondement biennal, décennale et RC. Alors la réception marque le point de départ de la mobilisation de l’assurance.


\section{La garantie décennale}

	1792 et s du Code civil

	\subsection{Les conditions de mise en œuvre de la garantie décennale}

		\subsubsection{Les conditions communes aux trois garanties spécifiques}\index{GarantiesSpecifiques@\garSpec!ConditionsCommunes@Conditions communes}\label{garantiesSpecifiquesConditions}

			Il y a quatre conditions communes aux trois garanties :
			\begin{enumerate}
				\item un ouvrage de construction,
				\item un ouvrage recu,
				\item un dommage caché,
				\item un lien d'imputabilité.
			\end{enumerate}

			\paragraph{Un ouvrage de construction}

				1792 : tout constructeur d’un ouvrage est responsable

				\subparagraph{La définition de la notion d'ouvrage}\index{Ouvrage}

				La définition est jurisprudentielle.

				D’une construction sur le sol ou en sous-sol. Notion plus large que celle d’édifice ou de bâtiment. Il doit s’agir d’une construction ancrée dans le sol, nature immobilier

				Ex : mobil home, tout dépend s’il est ancré sur le sol, la chape sera un ouvrage
				Péniche : pas ouvrage, abri piscine repliable non plus

				L’ouvrage doit donc avoir une nature immobilière, et relever une technique de construction (différent de la technique industrielle notamment pour les éléments d’équipement

				Certains éléments d’équipement peuvent être considérés comme des ouvrages (chaudière, cheminée)

				\subparagraph{La problématique des travaux sur existant}

				PPE / de la responsabilité contractuelle de droit commun car si on est au-delà de 10 ans sur les existants pas de décennale
				EXCEPTION : JP a admis dans certaines hypothèses les dommages sur existant des garanties décennales
				1.	La garantie décennale peut trouver à s’appliquer quand les travaux sur existants constituent en eux même un ouvrage comme des travaux de rénovation lourde qui modifient la structure, redistribue les pièces, et donc constitue un ouvrage alors entraine décennale.
				3 hypothèses :
				i.	Travaux de rénovation lourdes
				ii.	Travaux avec apports d’éléments nouveaux : travaux qui nécessitent apport de matériaux ou éléments nouveaux, constituent en eux même des ouvrages entrainant la décennale
				iii.	Ravalement étanche constitue eux même un ouvrage entrainant responsabilité décennale ( il y a des décisions du juge du fond qui apprécie l’imperméabilisation comme un ouvrage ce qui n’est pas la position de la C.Cass)
				2.	Décennale couvre désordres causés aux existants s’ils sont indissociables des travaux neufs et si les désordres des existants sont la conséquence des désordres aux travaux neufs. Dès lors si on ne peut pas déterminer la cause exacte des dommages, alors la décennale joue.
				3.	Si on ne peut pas déterminer si les désordres proviennent de l’existant ou des travaux neuf
				4.	Désordre d’origine dès lors qu’il porte sur un élément d’équipement installé sur existant lorsqu’il porte atteinte à la destination de son ouvrage 14/06/2017 ; 14/09/2017 ; 26/10/2017
				Ex : cheminée : avant il fallait que la cheminée soit un ouvrage pour entrainer décennale sur l’existant


			\paragraph{Un dommage}

				\subparagraph{Un dommage à l'ouvrage}

				Il faut un dommage dont la cause et l’origine sont indifférentes, alors même qu’il provient d’un vice du sol.

				Etude de sol obligatoire aux actes de vente, mais pas de sanction

				Le constructeur engage alors sa responsabilité.

				Si cela vient d’un défaut de conformité ou d’une faute, c’est bien mais c’est indifférent.

				Il est indifférent que la cause des désordres n’ait pu être déterminée avec précision (différent de l’imputabilité)

				La non-conformité n’entre pas le champ de la garantie décennale alors responsabilité contractuelle de droit commun. Toutefois, dans un sens favorable au MO si la conformité a pour conséquence une atteinte à la destination ou à la solidité de l’immeuble alors action décennale possible.
				Ex:: menuiserie extérieure en bois au lieu de PVC et donc infiltrante.

				\subparagraph{Un dommage caché à la réception}

					Le dommage doit donc être caché, c’est à l’entrepreneur de rapporter la preuve du caractère apparent du dommage, en ce qu’il doit être apparent aux yeux d’un profane dans son ampleur et ses conséquences.

				\subparagraph{Un dommage apparu dans le délai de la garantie}

					Le dommage doit apparaître dans le délai de la garantie que la victime souhaite invoquer.

					\begin{description}
						\item[Les dommages futurs] Dommage qui est judiciairement dénoncé à l’intérieur du délai décennal, qui n’a pas la gravité décennale au jour où il est judiciairement dénoncé mais qu’il atteindra à coup sûr dans le délai décennal.

						C.Cass Civ 29/01/2003 : le dommage doit atteindre à coup sûr la gravité décennale dans le délai décennal.
						C.Cass Civ 3° 23/10/2013 ou 28/02/2018


						\item[Les désordres évolutifs] Celui qui est apparu après le délai de 10 ans, est la conséquence du désordre dénoncé dans le délai décennal.

						La C.Cass admet que l’assignation initiale interrompt pour les dommages dénoncés dans l’assignation mais aussi dans leur aggravation postérieure
						-	Désordre initial dénoncé dans le délai de garantie
						-	Désordre initial doit avoir atteint la gravité décennale dans le délai de la garantie
						-	Nouveau désordre doit constituer une aggravation, une suite ou une conséquence du désordre initial donc ce qui exclut les désordres nouveaux sans lien avec le précédent
						Civ 3° 2006 : il faut donc que le siège du dommage soit le même ouvrage
						En pratique ils sont peu nombreux

					\end{description}




			\paragraph{Un lien d'imputabilité entre le dommage et l'activité du constructeur}

			La mise en œuvre des responsabilités spécifiques du constructeur, il faut établir un lien d’imputabilité Civ 3°  20/05/2015

			Il faut une imputabilité du dommage à l’activité du constructeur. Différent du lien de causalité qui renvoie à la faute


		\subsubsection{La gravité décennale}\index{GraviteDecennale@Gravité décennale}\label{graviteDecennale}

		Le dommage doit soit affecté la solidité de l’ouvrage ou le rendre impropre à sa destination, ou atteindre la solidité d’un élément d’équipement indissociable

		L’origine est indifférente (vice de construction ou vice de sol), une de ces conditions suffit.


			\paragraph{Une atteinte à la solidité de l'ouvrage}

			Qui remet en cause la sécurité physique de l’ouvrage, fissure structurelle, les fissures esthétiques rentrent dans le champ de la RC de droit commun, ou possible atteinte à la destination de l’ouvrage alors décennale

			Généralement l’atteinte à la solidité ne fait pas débat


			\paragraph{Une atteinte à la destination de l'ouvrage}

			L’ouvrage doit être impropre à sa destination.
			-	L’atteinte doit être appréciée au regard de la destination première de l’ouvrage, CAD celle à laquelle on peut raisonnable s’attendre. Une maison d’habitation doit être habitable
			-	Au regard de la destination contractuelle, CAD décidée par les partie Civ 10/10/1992
			Les constructeurs ont l’obligation de se renseigner sur la destination exacte de l’ouvrage.
			L’impropriété à la destination ne suppose pas que le risque soit déjà réalisé, à cet égard la JP a considéré que l’impropriété peut être régularisé en cas de risque d’inondation rendant impropre à la destination
			En cas de mauvais fonctionnement de ballon d’eau chaude dans une prison / Le non respect des règles parasismique Civ 3° 11/05/2011
			Mais Civ 3° 05/07/2018, si les juges du fonds que le défaut des normes parasismique ne peut être engagée si le risque n’est pas déterminé
			-	Même si les désordres n’affectent une partie de l’ouvrage à sa destination
			-	Alors même que les normes réglementaires sont respectées
			C.Cass 27/10/2006 21/09/2011, le respect des normes légales et réglementaires n’empêchent pas la caractérisation du désordre décennale entrainant une impropriété
			-	Atteinte à la destination de l’élément d’équipement : Indifférent que l’élément soit dissociable ou non, et indifférent si existant ou non (à vérifier)
			17/06/2017 ; 14/09/2017 ; 26/10/2017

			En matière de performance énergétique, impropriété retenue que si dommage lié au produit, à la conception et mise en œuvre de l’ouvrage L111-13-1 du CCH, que si surconsommation est strictement encadré par ces destinations.
			Ex : erreur d’implantation, règles de sécurité, défaut de nivellement d’un terrain de tennis, dysfonctionnement d’une pompe à chaleur.


			\paragraph{Une atteinte à la solidité d'un éléments d'équipement indissociable de l'ouvrage}

			13/02/2020 : refus de qualification d’élément d’équipement à un enduit de façade dès lors qu’il n’est pas destiné à fonctionner (on avait cette condition sur le biennal, c’est la première fois en décennale)
			Donc désormais 1792-2 ne peut s’appliquer que si l’élément d’équipement en cause est destiné à fonctionner (mue par un dynamisme propre)
			Elément doit être indissociable de l’ouvrage 1792-2 al 2 : indissociabilité quand la dépose, démontage ou remplacement ne peut s’effectuer sans détérioration de l’ouvrage.
			Ex : un carrelage fixé avec mortier c’est indissociable, avec colle c’est dissociable
			Radiateur scellé dans le mur c’est indissociable, juste fixé alors dissociable.
			L’élément ne doit pas avoir une fonction professionnelle, car les éléments à vocation professionnelle soit exclue de la décennale.


	\subsection{Les redevables de la garantie décennale}

		\subsubsection{Les constructeurs}

			\paragraph{Les personnes réputées constructeurs}

			L'\articleDu{1792}{\cciv} répute constructeur trois type de personnes :
			\begin{enumerate}
				\item Les personnes liées au \MO par un contrat de louage d’ouvrage : architecte, entrepreneur, technicien, contrôleur technique, \etc

				\textbf{Attention} L'\articleDu[L]{111-24}{\cch} stipule que le contrôleur technique n'est soumis que dans les limites de sa mission \CAD qui ne peut être tenu que s’il est prouvé que le fait à l’origine du dommage entrait dans le champ de ses missions et il ne peut être tenu vis-à-vis des constructeurs qu’à concurrence de sa part de responsabilité donc en cas de condamnation \emph{in solidum}. Il ne supporte donc pas l’insolvabilité des autres.

				Les sous-traitants ne sont pas visés car ils ne concluent pas directement avec le \MO, et ne sont ni réputés ni assimilés constructeur.

				\item Les personnes vendant après achèvement un immeuble qu’elles ont construit ou fait construire (exclu VEFA \aValider) cela vise « le castor » ou faut construire et revend dans un délai de 10 ans.

				Pour le castor, le point de départ ne sera pas la réception, mais l’achèvement des travaux attesté par tout moyen comme la date à laquelle l’ouvrage était utilisable et propre à sa fonction, la DAT peut être un moyen.

				Le fait que les désordres étaient apparents ou connus au moment de la vente, car ce qui compte est la réception ou de l’achèvement, le vendeur ne peut pas se prévaloir d’une clause de garantie car l'\articleDu{1792}{\cciv} est d’ordre public, alors clause réputée non écrite

				La \JP considère que la garantie décennale n’est pas exclusive de la garantie des vices cachés.

				\item Personne bien qu’agissant en qualité de mandataire du propriétaire de l’ouvrage, accomplisse une mission assimilable à un locateur d’ouvrage. Ici pour palier la fraude de personne qui se réfugie sous un mandat, comme le promoteur de l'\articleDu{1831-1}{\cciv}, le gérant d’une société d’Attribution, \etc

				De même le \MO délégué peut voir sa responsabilité décennale engagée s’il s’est comporter en locateur d’ouvrage.

				Est réputé constructeur le constructeur de maison individuelle.
			\end{enumerate}



			\paragraph{Les personnes assimilées constructeurs}

			Le vendeur d’immeuble à construire : est assimilé constructeur, responsabilité biennale, décennale à l’égard de l’acquéreur

			Le vendeur d’immeuble à rénover : L262-3 al 3 CCH, débiteur de la biennale et décennale


		\subsubsection{Les fabricants d'éléments pouvant entrainer la responsabilité solidaire}

		La C.Cass appelle à la suppression de ces dispositions, renvoi aux ouvrages pour ce passage.

			\paragraph{La notion d'EPERS}

				\aCompleter

			\paragraph{Mise en œuvre et nature de la responsabilité}

				\aCompleter

	\subsection{Les bénéficiaires de la garantie décennale}

		\subsubsection{Le \Mo}

			C’est la personne physique ou morale pour le compte de laquelle les travaux sont réalisés

		\subsubsection{L'acquéreur de l'ouvrage}

		Il est expressément visé par 1792 du Code civil. Bien qu’il ne soit pas visé par 1792, les sous-acquéreurs sont également visés tant que le délai décennale n’est pas expiré, ce qui donne qualité et intérêt à agir.

		Sauf clause contraire, l’acquéreur peut agir pour les désordres apparus avant l’acquisition. L’intérêt est d’être propriétaire au jour de l’action. L’acquéreur de l’ouvrage aura alors intérêt et qualité.

		Sauf clause contraire où le vendeur se serait réservé le droit d’agir contre le constructeur, notamment si le MO pour vendre le bien a procédé aux réparations, s’il prouve avoir supporté les conséquences du dommage dans son patrimoine (trouble de jouissance, perte spécifique comme financière)


		\subsubsection{Les copropriétaires et le syndicat des copropriétaires}

		PPE :
		-	Syndicat est compétent pour solliciter la réparation des dommages affectant les parties communes, et non pas le syndic. Le syndic doit être habilité pour agir en justice, sauf urgence comme expertise. Depuis 2018, les tiers ne peuvent plus invoquer le défaut d’habilitation désormais seuls les copropriétaires peuvent invoquer ce défaut d’habilitation
		-	Chaque copropriétaire est habilité à intérêt et qualité, compétent, en cas de dommage affectant ses parties privatives

		TEMPERAMENT
		-	Le syndicat des copro peut solliciter la réparation des désordres des parties privatives si par la généralisation ils affectent la conservation de l’immeuble Civ 3° 12/05/1993

		Le syndicat peut aussi solliciter la réparation des dommages des parties communes et privatives de manière indivise. Dans cette hypothèse l’intérêt du délai par le syndicat profite aux copropriétaires pour la réparation des préjudices personnels Civ 3° 31/03/2004

		-	Chaque copropriétaire peut agir si l’atteinte au partie commune lui porte préjudice personnellement


		\subsubsection{Les locataires et les crédits-preneurs}

		LES LOCATAIRES

		Civ 3° reconnait la qualité de MO que le propriétaire de l’ouvrage et les acquéreurs successifs.
		Le locataire ne peut pas par principe agir sur le fondement de 1792 du Code civil

		Sauf dans le bail, une cession des droits sur le fondement des articles 1792 du Code civil

		A défaut de cession de droit, les seuls fondements possibles à l’encontre des constructeurs par le locataire, responsabilité contractuelle de droit commun si c’est le locataire qui a commandé et payé les travaux, délictuelle si c’est le bailleur qui a commandé et payé le constructeur.

		Le locataire bénéficie d’une action à l’encontre de son bailleur 1221 du Code civil : garantie des vices et défauts de la chose louée, le bailleur aura alors une action récursoire à l’encontre des constructeurs par garantie spécifique ou RC

		CREDIT PRENEUR : crédit bail

		Avant la levée de l’option d’achat, le crédit preneur est assimilé à un locataire donc ne bénéficie pas de 1792 et suivants, sauf cession de droit dans contrat de bail avec clause de subrogation Civ 3° 16/05/2001

		Après la levée d’option, il devient propriétaire, il peut donc agir sur le fondement de la garantie décennale tant que l’on est dans les délais.


	\subsection{Le régime de responsabilité}

	La garantie décennale est exclusive de toute responsabilité, il faut alors nécessairement agir sur ce fondement dès lors que les conditions sont remplies, et ce à peine d’irrecevabilité.

		\subsubsection{La présomption de responsabilité}

		La garantie décennale repose sur une présomption de responsabilité. Pas besoin de caractériser une faute, une absence de faute est indifférente, c’est une responsabilité de plein droit.

		Impossible de déroger à cette présomption 1792-5 : toute clause contraire est réputée non écrite.

		Les moyens de défense sont alors absence de réception, absence d’ouvrage, caractère apparent, lien d’imputabilité


		\subsubsection{Les causes d'exonération}

		Force majeur, fait d’un tiers, fait du MO
		1.	La force majeure
		Irrésistibilité imprévisibilité et extériorité. Rarement admise par la JP. Un arrêté de catastrophique naturelle ne suffit pas, sauf des calamités exceptionnelles ont été admises
		Nv critère : doit empêcher l’exécution par la force majeure On ne sait pas si cela sera plus admis ou pas par la JP
		Les effets : 1218 du Code civil si l’empêchement est temporaire, l’exécution de l’obligation est suspendue avec résolution possible. Si empêchement définitif, résolution de plein droit et les parties sont libérées de leurs obligations dans les conditions de 1351 du Code civil
		2.	Le faut d’un tiers (Tiers au chantier)
		Aucun intervenant réputé constructeur ne peut se prévaloir de la faute d’un autre intervenant pour exonérer sa responsabilité
		I ne peut pas se prévaloir des vices des matériaux mais peut agir du fournisseur en garantie
		Il ne peut pas se prévaloir de l’insuffisance des règles de l’art ou des DTU pour s’exonérer de sa responsabilité.
		Quelques exemple d’exonération : acte de vandalisme d’un tiers à l’ouvrage
		3.	Le fait du MO

		Les constructeurs invoquent souvent le fat du MO comme exonératoire mais rarement admis.
		a.	L’immixtion du MO notoirement compétent.
		Ici le MO est notoirement compétent s’immisce dans la construction de l’ouvrage.
		Il fait nécessairement un acte d’immixtion, il faut un acte positif
		Ex : le MMO choisi des matériaux et techniques de constructions. Mais s’il donne juste son accord ce n’est pas suffisant, il doit imposer les matériaux ou participation à la conception de l’ouvrage
		Le MO doit notoirement compétent CAD avoir des connaissances techniques comme un architecte
		Médecin, véto.. seront considérés comme compétents sur la partie concernant leur activité

		Conséquence de cette immixtion, il y a partage de responsabilité car l’entreprise est tenue de son obligation de conseil, dans certains cas exonération totale, soumis à l’appréciation du juge du fond.
		b.	L’acceptation délibéré des risques par le MO
		Le MO passe délibérément outre les conseils et avertissements du constructeur.
		Conditions :
		-	Le constructeur doit avoir présenté les risques dans leur ampleur et leur conséquence au MO
		-	Malgré cela le MO doit décider de poursuivre de l’opération sans tenir compte des conseils et avertissements du constructeur
		-	Pas nécessaire que le MO soit notoirement compétent, il suffit qu’il soit pleinement informé
		Ex : pose de carrelage inadéquate ou matériaux inadapté.
		Pas d’acceptation délibéré des risques si absences de recours à un maître d’œuvre, et la volonté de faire des économies n’est pas en soi une acceptation délibérée des risques
		Conséquences :
		-	Partage de responsabilité car il appartenait au constructeur de refuser les travaux qu’il savait inefficace Civ 3° 21/05/2014
		-	Exonération totale dans cas exceptionnel c’est plus rare

		c.	Mauvaise utilisation de l’ouvrage
		Elle se situe après réception, en ce cas exonération de la responsabilité du constructeur
		Ex : dépassement des charges d’un plancher
		Il y a également une absence d’entretien de l’ouvrage. A ce moment là il n’y a pas d’imputabilité


		\subsubsection{La durée de la garantie décennale}

			\paragraph{Le délai de garantie} Il est donné par l'\articleCciv{1792-4-1}.

			Cet article prévoit que << {\itshape toute personne physique ou morale dont la responsabilité peut être engagée en vertu des articles 1792 à 1792-4 du présent code est déchargée des responsabilités et garanties pesant sur elle, en application des articles 1792 à 1792-2, après dix ans à compter de la réception des travaux ou, en application de l'article 1792-3, à l'expiration du délai visé à cet article} >>. Le délai est donc de 10 ans à compter de la réception.

			\paragraph{L'interruption du délai de garantie}

				\subparagraph{Les modes d'interruption du délai de garantie}

				Depuis la réforme de la prescription du \printdate{17/06/2008}, il semblerait que le délai de garantie ne puisse être interrompu que par deux actes :
				\begin{itemize}
					\item la citation en justice,
					\item et la signification de conclusion reconventionnelle.
				\end{itemize}

				La reconnaissance de responsabilité qui était reconnue par les tribunaux comme mode interruptif du délai, antérieurement à la réforme de 2008, ne constituerai plus une cause valable d'interruption puisque l'\articleCciv{2240}\footnote{<<  {\itshape La reconnaissance par le débiteur du droit de celui contre lequel il prescrivait interrompt le délai de prescription.} >>} ne vise que les délais de prescription, or le délai de la décennale est un délai de \textbf{forclusion}.

				Il faut comprendre par << citation en justice >> ou assignation, tant les assignations au fond\footnote{
					L’article 750 du Code de procédure civile dispose : « La demande en justice est formée par assignation \lips ». L’assignation est donc un acte introductif d’instance, qui peut marquer le début d’un procès.

					Le demandeur assigne le défendeur à comparaître, c’est-à-dire à se présenter, devant une juridiction définie. L’assignation se distingue de la requête, qui est un acte introductif d’instance par lequel le demandeur saisit le tribunal pour qu’il convoque directement les parties.

					On parle d’assignation « au fond » lorsque le juge devant lequel les parties sont citées à comparaître va se prononcer sur tous les aspects de droit et de procédure de l’affaire. L’assignation « au fond » est également appelée « assignation à toutes fins ». On l’oppose à l’assignation en référé, qui ne concerne que certains points à trancher dans l’urgence.
				}
				que les assignations en référé.  Par exemple : la demande d'expertise en référé est suffisante.

				La citation doit préciser sans équivoques les désordres dont la réparation est demandée. Elle doit viser l'ensemble des responsables : \E, \archi, \Moe ; sachant qu'il est important de les mentionner parce que l'assignation de l'assureur n'a pas de caractère interruptif à l'égard du constructeur assuré --- mais uniquement à l'égard de l'assureur. Si le constructeur n'est pas assigné, il n'y aura pas d'action à l'encontre de celui-ci.

				En vertu de l'\articleCciv{2241}\footnote{<< {\itshape La demande en justice, même en référé, interrompt le délai de prescription ainsi que le délai de forclusion.

					Il en est de même lorsqu'elle est portée devant une juridiction incompétente ou lorsque l'acte de saisine de la juridiction est annulé par l'effet d'un vice de procédure.} >>}

					Particularité en matière de copro : délai interrompu par chaque copropriétaire lorsque le dommage affecte les parties privatives mais également pour les dommages qui affectent les parties communes dès lors que le dommage cause un préjudice personnel. Le délai peut également être interrompu par le syndicat des copro lorsque le dommage affecte les PC mais aussi compétent pour interrompre les délais lorsque le dommage affecte PP et engendre un trouble collectif. Les désordres doivent alors causer les mêmes préjudices à l’ensemble des copropriétaires et doit leur préjudicier de la même manière
					07/09/2011 27/09/2000 c’est donc le juge qui doit caractériser ces éléments pour apprécier l’intérêt à agir du syndicat
					Le syndicat peut interrompre si désordre affecte PC et PP de manière indivisible alors l’interruption du délai réalisé par le syndicat profite au copropriétaire personnellement pour son préjudice personnel. Attention à la problématique de l’habilitation du syndic pour agir au fond (pas le cas en référé) mais en cas de procédure au fond, le syndic doit être habilité au préalable ou a postériori pour agir au fond sinon irrecevable. Cela ne peut pas être soulevé d’office par le juge mais possible par le défenseur. Désormais cela ne peut être soulevé que par un copropriétaire (article 55 suite à la réforme) L’autorisation doit clairement et précisément énumérer un désordre visé, mais la JP dit que pas nécessaire (17 :01 :2019 décret précis) il faut indiquer les réserves, les désordres dans l’habilitation mais la JP dit que pas besoin de mentionner précisément les personnes visées (défendeur) dès lors qu’elles sont déterminables (civ 3° 23/01/2020)
					Néanmoins, conseil résolution visant toutes les personnes où on est sur

					-	Conclusions reconventionnelles : elles sont interruptives du délai de forclusion dans les mêmes conditions que l’assignation donc envers les personnes à l’encontre de qui on souhaite interrompre les délais, elles doivent viser les désordres. Le syndic peut être habilité pour réaliser des demandes reconventionnelles, sauf s’il s’agit d’un moyen de défense


				\subparagraph{Les effets de l'interruption du délai}

				C’est un délai de forclusion, cela a un impact important. Les citations en justice et signification des conclusions reconventionnelles, interrompent le délai de garantie (pas suspension) donc un nouveau délai de même durée qui recommence à courir à partir de l’issue de l’instance.

				L’ordonnance COVID est une suspension des délais mais pas interruptif, avec reprise à la fin de la crise. La formulation est prorogation du délai, si le délai de forclusion expire pendant la crise sanitaire. En réalité il est réputé avoir été fait dans les délais
				Si c’est une assignation 10 ans qui recommence à courir à compter de l’ordonnance désignant l’expert, si c’est au fond, à l’issu de la première instance ou de l’appel, donc jusqu’à que le litige trouve une issue.
				En assignation en référé provision, délai recommence le jour de l’ordonnance
				A compter du prononcé de la décision et non pas de la signification.
				Pour l’expertise, les mesures d’expertises suspendent les délais de prescription selon l’article 2239 du Code civil, mais Civ 3/06/2015 la suspension 2239 pas applicable au délai de forclusion donc pas applicable aux garanties spécifiques.


		\subsubsection{L'obligation \emph{in solidum} et les recours des constructeurs entre eux}

			En cas d'insolvabilité d'un \lo, elle se répartie entre tous les autres \lo solvables. L'insolvabilité ne pèse pas sur la victime mais sur les autres ...

			Les constructeurs dont l’activité est à l’origine de dommage peuvent être condamnés in solidum à indemniser le MO si le MO le sollicite. Toute clause visant à écarter cette solidarité serait réputée non écrite, en application de 1792-5 du Code civil (toute clause contraire aux garantie spécifiques est réputée non écrites)

			La clause qui a pour effet d’écarter cette solidarité serait valable que dans l’hypothèse où une responsabilité contractuelle est recherchée mais pas sur les garanties spécifiques\footnote{\jurisCourDeCas{\civTrois*}{14/02/2019} ; \jurisCourDeCas{\civTrois*}{17/10/2019}}

			L’obligation in solidum permet au MO de s’adresser pour le tout à n’importe quel constructeur. C’est un avantage pour le MO, car il a la possibilité de s’adresser à l’un quelconque des constructeurs pour obtenir le paiement de la totalité des paiements des \DI qui lui ont été accordés par le juge. Il s’adresse la plupart du temps au constructeur ou débiteur le plus solvable comme l’assurance. De même il est possible de demander une condamnation in solidum avec le ST, il pourrait s’adresser au ST mais pas généralement le plus solvable.
			Le constructeur qui a été actionné par la victime a une action récursoire à l’encontre des co-constructeurs coobligés, action fondée sur le droit commun Civ 3°08/02/2012
			Plusieurs hypothèses envisageables :
			-	Elle peut être de nature contractuelle s’il existe un contrat entre les parties (exemple : vendeur immeuble à construire et locateur d’ouvrage, mais ici le vendeur d’immeuble à construire bénéficie des garanties spécifiques à l’égard des locateurs d’ouvrage alors pas besoin de démontrer une faute)
			-	Action de nature aussi délictuelle dans les hypothèses où il n’existe pas de lien contractuel entre les coobligés comme co-locateur d’ouvrage entre eux, entre ST entre eux ou locateur d’ouvrage et ST, alors responsabilité délictuelle avec démonstration d’une faute, préjudice et lien de causalité, le rapport d’expertise est alors probant qui déterminera la faute de chacun et la part de responsabilité de chacun.
			Les délais d’action entre constructeurs en fonction des différentes hypothèses :
			-	Si action de nature contractuelle 1792 et suivants sont enfermés dans les délais de 2 et 10 ans à compter de la réception, en fonction de la garantie spécifique qui est invoquée

			Pour les autres actions contractuelles, quand on ne peut pas agir sur le fondement des garanties spécifiques, elles sont enfermées dans un délai de 10 ans à compter de la réception article 1792-4-3 du Code civil, dans la relation de MO et constructeur.

			Pour actions entre entrepreneur principal et ST, voir précédemment.

			-	Action de nature délictuelle : ici cas entre co-locateur d’ouvrage non lié par un contrat entre eux, ou entre ST ou entre ST et locateur d’ouvrage, prescription 5 ans à compter de la manifestation du dommage ou de son aggravation 2224 du Code civil. La Cour de cassation a refusé d’applique l’article 1792-4-3. Pur certains auteurs sa formulation était générale et s’appliquait aux actions entre constructeurs entre eux. D’autres auteurs pensaient que les actions récursoires ne devaient pas être enfermées dans un délai de 10 ans, ce qui a été retenu par la C.Cass, c’est le délai de droit commun qui s’applique et donc 5 ans à compter de la manifestation du dommage ou de son aggravation, l’article 1792-4-3 ne profite qu’au MO.
			18-25.915 Civ 3°16/01/2020
			Qu’est ce qui fixe la manifestation du dommage ou son aggravation ?
			Les tribunaux fixent couramment Le point de départ du délai de 5 ans à l’assignation du constructeur par le MO ou la victime, Civ 3° 16/01/2020 18-25.915
			Attention à l’arrêt de la Civ 3° 13/09/2006 : ici la C.Cass a jugé que le point de départ de l’action du constructeur à l’encontre d’un autre constructeur devait être fixé au jour du dommage s’est manifesté à l’égard du MO et non pas au jour où le constructeur a été assigné, donc 5 ans à compter du dommage à l’égard du MO. Décision non favorable au constructeur car elle enferme l’action récursoire du constructeur dans un délai dont il ne pouvait avoir connaissance. C’est une décision qui existe mais plus applicable
			Concernant le quantum de l’action récursoire :
			Le co-débiteur du MO n’agit à l’encontre des autres co-débiteurs que pour obtenir le remboursement de leur part de responsabilité personnelle, donc l’action est limitée dans ce que doit chacun des co-obligés au titre de la condamnation.
			Ex : 3 constructeurs co-obligés à hauteur d’1/3. Un constructeur a payé le tout, il devra se retourner sur chacun pour obtenir un tiers. Il ne peut pas se retourner à l’égard de l’un pour obtenir les 2/3.
			En cas d’insolvabilité d’un locateur d’ouvrage, l’insolvabilité se répartie entre tous les autres, la part de l’insolvable se répartie entre tous les autres solvables. L’insolvabilité pèse sur les personnes condamnées in solidum.

	\subsection{La réparation du dommage}

		\subsubsection{La preuve du dommage}

			Bien que le \lo soit tenu d'une présomption de responsabilité, ou encore sur une obligation de résultat, il incombe à la victime d'apporte la preuve du préjudice.

			Bien que le constructeur soit tenue d’une présomption de responsabilité, ou sur une obligation de résultat (pour responsabilité contractuelle de droit commun), il faut démontrer le préjudice, trouble de jouissance, perte financière, malfaçon… la victime doit en rapporter la preuve.

			La victime doit rapporter la preuve de la réalité du préjudice et de son étendue. Cette preuve est généralement rapportée grâce à l’expertise judiciaire, avec aussi nécessité de déterminer les solutions de reprises.
			Article 145 du Code de procédure civile : désignation expertise en référé, avec missions types
			Alors agir en référé devant le TJ du lieu de situation de l’immeuble avec désignation expert avec mission
			L’expertise doit être menée contradictoirement sous peine de nullité du rapport. C’est important car le contradictoire de l’expertise le rapport va s’imposer à tous et pourra servir de base essentielle, principale à la victime pour obtenir réparation. La Cour de cassation, Ch Mixte 28/09/2012, prévoit que l’irrégularité affectant le déroulement de l’expertise régi par les règles de nullité des règles des actes de procédure alors pas inopposabilité du rapport mais sa nullité, si on en démontre le grief. Si une partie n’a pas été convoquée, c’est une cause de nullité s’il en démontre le grief de la victime.
			Le MO victime doit attrait aux opérations d’expertise l’ensemble des constructeurs dont la responsabilité est susceptible d’être engagée et assurances.
			Le juge n’est pas lié par les conclusions de l’expert.
			Après dépôt du pré-rapport possible dire, mais il faut le faire de préférence avant le pré-rapport. Après le pré-rapport dires récapitulatifs.
			Deux précisions concernant l’opposabilité des rapports d’expertise :
			-	Concernant les rapports d’expertise amiable, le juge peut fonder décision sur rapport d’expertise amiable, si régulièrement versé au débat et à la discussion des parties et pas l’unique moyen de preuve. Donc pas de décision exclusivement sur ce rapport. C.Cass Ch Mixte 28/09/2012 11-18.710 Civ 2° 13/09/2018 ici expertise amiable non contradictoire.
			-	Concernant les rapports d’expertise judiciaire, les conclusions sont opposables aux parties à l’expertise même non convoquées dès lors que le contenu claire et précis a été débattu contradictoirement devant la juridiction 28/09/2012 Ch mixte 11-11.381

			Les juges peuvent fonder leur décision sur rapport d’expertise judiciaire même si défendeur n’a pas été partie aux opérations d’expertise à condition que régulièrement versé au débat et soumis à la discussion contradictoire et pas unique moyen des parties car devient comme un rapport amiable, alors même raisonnement que le précédent, CAD si versé aux débats pas unique élément de preuve et librement débattu et discuté Civ 2° 7/09/2017 16-15.531


		\subsubsection[L'étendue de la réparation]{L'étendue de la réparation : le principe de réparation intégrale}

			Le principe : le dommage, tout le dommage, rien que le dommage.

			\paragraph{La réparation de tout le dommage}

				\subparagraph{La réparation des dommages affectant l'ouvrage} Le principe est simple le \Mo doit être placé dans la situation qui aurait été la sienne si le dommage ne s'était pas produit. A cet égard, le \Mo peut demander la remise en l'état de l'ouvrage à l'identique. Il peut donc faire reprendre la fissure mais également la peinture.

				Un abattement pour vétusté est-il possible ? Le \Mo a droit \aCompleter même s'il bénéficie d'une plus-value de fait. La jurisprudence est le refus d'un abattement, probablement parce qu'il n'est pas possible de construire vieux. elle a refusée de \aCompleter\footnote{Contrairement à la doctrine administrative} au du principe de réparation intégral. Exemple du drainage périphérique \aCompleter.

				Le principe de réparation intégral \aCompleter. C'est l'assureur qui va payer.

				Le \Mo pouvait demander la démolition-reconstruction quel que soit le coups. Depuis la réforme ... 1280 cciv, qui vient entériner des jurisprudence dès lors qu'il 1221 cciv.

				la réparation doit être pérenne, s'ils s'avèrent inadapté, la victime peut intenter une nouvelle  sans que ceux 3eme 12/5/99

				Le PPE : le MO doit être placé dans la situation si le dommage ne s’était pas produit. Le MO victime peut demander la remise en l’état de l’ouvrage à l’identique de sorte que si le dommage consiste en une fissure, reprise de la fissure et de la peinture.
				Pas de prise en compte d’un coefficient de vétusté, la JP c’est le refus d’affecter un coefficient de vétusté, de prendre en compte l’état de l’ouvrage au moment du désordre, donc reconstruction à neuf.
				La JP a refusé de faire jouer la théorie de l’enrichissement injustifié car contraire au principe de réparation intégrale. En marché public cette théorie est acceptée.
				La réparation doit comprendre la réparation d’élément non prévu à l’origine si cela est nécessaire pour supprimer le désordre.
				Ex : suppression de drainage obligation pour réparation intégrale ou création d’un mur de soutènement
				Ce PPE s’applique pour toutes les garanties spécifiques
				Le MO pouvait demander une déconstruction et reconstruction totale de l’ouvrage si cela était nécessaire, sur le fondement de la responsabilité décennale avec le principe de réparation intégrale
				Mais depuis réforme droit des contrats, exécution en nature doit être proportionnelle 1221 du Code civil
				La réparation doit être pérenne, si la réparation est inadaptée, la victime a une nouvelle action à / des constructeurs, une nouvelle action, sans qu’on peut lui opposer le premier délai décennale Civ 3° 12/05/1999


				\subparagraph{La réparation des troubles annexes} Le principe est très clair, sur le fondement de ???. Il doit également ouvrir ???

				Il s'agit des troubles de jouissances, des manques à gagner, des pertes de loyer, des pertes d'exploitation, des frais médicaux en cas de dommages corporels, des préjudices moraux, \etc L'assurance ne couvrira pas les préjudices annexes, mais seulement les dommages à l'ouvrage.

				Sur le fondement de garantie décennale, la réparation intégrale permet la réparation des troubles annexes\index{TroublesAnnexes@Troubles annexes}
				Il s’agit des troubles de jouissance, des manques à gagner, les pertes de loyer, les troubles moraux
				La garantie décennale pour les préjudices annexes est obligatoire mais pas l’assurance qui ne porte que sur l’ouvrage.
				Dommages matériels consécutifs + dommages immatériels
				La responsabilité contractuelle couvre les dommages matériels et immatériels.


				\subparagraph{La prise en compte de la TVA} \label{tvaIndemnite}\index{Indemnite@Indemnité!TVA@prise en compte de la TVA} L'indemnité versée est-elle ttc ou ht ? Si la victime récupère la tva, il s'agit d'une indemnité ht, sinon il s'agit d'une indemnité ttc.

				La preuve de la non suggestion à la tva est à la charge de la victime\jurisCourDeCas{\civTrois*}{6/11/2007}.

				Le taux de tva est celui en vigueur au jour ou le juge statut.

			\paragraph{La réparation du seul dommage} Il n'est pas possible d'obtenir des améliorations déguisée. Ainsi, si le désordre concerne des tuiles, il n'est pas possible de demander le remplacement des tuiles par des ardoises, mais uniquement par des nouvelles tuiles. De même, en cas d'insuffisance de chauffage, il n'est pas possible de demander une climatisation réversible qui permet également le froid.

		\subsubsection{Les formes de la réparation}

			Elle peut intervenir en nature ou en équivalent ?

			En nature c'est la reprise physique du dommage. elle peut être fait par l'auteur du dommage ou par un tiers.

			Par équivalent : des dommages et intérêts a

			La victime peut imposer au constructeur la réparation en nature. Par contre, le constructeur ne peut pas imposer une réparation en nature \jurisCourDeCas{\civTrois*}{28/9/2005}.

			Il est possible de combiner les deux.

\section{La Garantie biennale}

	1792-3 garantie de bon fonctionnement des éléments d’équipement durant 2 ans à compter de la réception.

	Il est possible d’étendre le délai contractuellement. Très peu usé, même cette garantie car souvent rend impropre à la destination

	Il s'agit d'une garantie résiduelle qui est peu retenue.

	\subsection{Les conditions de mise en œuvre de la garantie}

		\subsubsection{Les conditions communes aux trois garanties spécifiques}

		Ainsi que vu en \vref{garantiesSpecifiquesConditions} que les quatres conditions communes sont :
		\begin{enumerate}
			\item un ouvrage de construction,
			\item un ouvrage recu,
			\item un dommage caché,
			\item un lien d'imputabilité.
		\end{enumerate}

		Il ne faut pas que les éléments d’équipement professionnels n’aient pas de fonction exclusivement professionnel, sinon sont exclus des garanties spécifiques

		Il ne faut pas que pro exclusivement -> exclu

		\subsubsection{Les conditions propres à la garantie biennale}\index{GarantiesBi@\bi!ConditionsCommunes@Conditions}\label{garantiesBiConditions}

		Il faut une atteinte un element d'equipement dissociable, \cad sans enlèvement de matière ni détérioration de l'ouvarge.

		Il ne faut pas que le desordre atteigne la gravité décennale

		Il faut que l'élément d'équipement aie vocation à fonctionner, il doit être mu par un mécanisme propre. \jurisCourDeCas[12-12016]{\civTrois*}{13/2/2013} dans ce cas on rentre dans la responsabilité contractuelle.

		Il faut que le désordre intervienne dans le délais et que l'action soit introduite dans le délais également.

		4 Conditions :
		-	Il faut une atteinte à un élément d’équipement dissociable de l’ouvrage
		Une atteinte à un élément d’équipement dissociable à l’ouvrage CAD qu’il peut être retiré de l’ouvrage sans atteinte à l’ouvrage et sans enlèvement de matière. Il doit donc être dissociable.
		-	Le désordre ne doit pas atteindre la gravité décennale
		-	Il faut que l’élément d’équipement ait vocation à fonctionner, il doit être mue par un dynamisme propre
		Pendant très longtemps, la JP a admis de faire jouer la garantie biennale pour des éléments inertes.
		Mais, depuis un arrêt du 13/02/2013 12-12.016, la Cour de cassa a exclu les éléments inertes du champ d’application de la garantie biennale, cela a été confirmé de nombreuses fois, appel 11/09/2013 (pour le carrelage) C.Cass 18/02/2016 (revêtement végétal d’étanchéité) alors la biennale n’est pas applicable alors responsabilité contractuelle
		-	Il faut que le désordre apparaisse dans le délai de la garantie dans le délai de 2 ans, et que l’action soit introduite également dans ce délai.



		\subsection{Le régime et les modalités de mise en œuvre de la garantie biennale}

			L'action doit être introduite par les même bénéficiaire : \Mo,

			Ceux sont les même redevables :

			Elle repose également sur le même régime de responsabilité. elle repose sur une présomption de responsabilité. On retrouve les même causes d'exonérations, la même nature de délai. De même, il est possible de demander une condamnation \emph{in solidum} des lors que le dommage peut être imputé à plusieurs constructeurs.

			De même, si les conditions de la biennale sont remplies il faut nécessairement pas \rcdc

			Enfin, le principe de réparation intégral va trouver à s'appliquer ??? en nature ou en équivalent.

			L’action doit être introduite par les mêmes bénéficiaire de la garantie décennale, le MO les acquéreurs successifs. Et les mêmes redevables de la garantie
			La garantie repose sur le même régime de responsabilité de la garantie décennale ainsi la responsabilité biennale repose sur une présomption de responsabilité avec les mêmes causes d’exonération, la même nature de délai, c’est un délai de forclusion qui ne peut être interrompu que par citation ou conclusions reconventionnelles, interruption pas de suspension.
			Possible de demander condamnation in solidum dès lors que plusieurs constructeurs sont responsables
			La responsabilité biennale est exclusive de tout autre régime de responsabilité. Si les conditions de la biennale sont remplies, il faut nécessairement agir sur la biennale. Donc si désordre 3 ans alors que les conditions sont remplies, il n’est pas possible de basculer sur une autre garantie, sauf si atteinte à la destination alors décennale.
			Le principe de réparation intégrale se trouve à s’appliquer, en nature ou en équivalent, sur les dommages matériels et annexes
			Ex : volet roulant qui ne fonctionne pas en position ouverte


\section{La garantie de parfait achèvement}

1792-6 cciv cite
Donc couvre que si les conditions communes au 3 catégories :
Exception : désordre apparents mais réservé.

	C’est une garantie annale fondée sur l’\articleDu{1792-6}{\cciv}. La garantie de parfait achèvement couvre les désordres réservés à la réception et apparus dans un délai d’un an, et ne peut être mise en œuvre que lorsque les conditions communes aux 3 garanties spécifiques sont remplies (un ouvrage de construction – réception – dommage caché (exception à la garantie de parfait achèvement pour les désordres réservés) – lien d’imputabilité)

		\subsection{Le débiteur de la garantie}

		Le texte par el de l'E. Seul l'entrepreneur concerné. Les autres constructeurs ou assimilé constructeur ne sont pas concernés par la gpa. pas de castor, pas d'archi.

		Il n'y a pas d'in solidum, sauf si plusieurs E ont concourus au désordre. Chaque E doit reprendre sa partie.

		Il dispose cependnt d'une action récursoire envers les autres E qui pourraient être à l'orgine du désordres (sur une base soit contractuel soit délictuel)

		Le texte dit que la garantie de parfait achèvement à laquelle l’entrepreneur est tenu. Donc seul l’entrepreneur dont les travaux sont affectés des désordres, peut être actionné. On parle de l’entrepreneur concerné.
		Ex : plombier ne peut voir sa responsabilité engagée pour des problèmes d’électricité
		Il n’y a pas de responsabilité in solidum dans la garantie de parfait achèvement, seul l’entrepreneur concerné est tenu de cette garantie.
		Seul l’entrepreneur est tenu de la garantie de parfait achèvement.
		Les autres constructeurs assimilés constructeurs ou réputés constructeurs, ne sont pas tenus de la garantie de parfait achèvement, il s’agit des architectes, le castor, le vendeur d’immeuble à construire, le bureau d’étude le contrôleur technique.
		Quid quand le désordre trouve sa cause dans l’intervention de plusieurs entrepreneurs de divers corps d’état ?
		Chaque entrepreneur doit reprendre la partie de l’ouvrage qu’il a réalisé et le MO devra agir à l’encontre de chaque locateur d’ouvrage.
		Chaque locateur d’ouvrage dispose alors d’un recours récursoire s’il estime que le désordre est imputable à l’autre locateur d’ouvrage (soit sur du contractuel ou délictuel)

		\subsection{L'étendue de la garantie}

			La garantie couvre les désordres réservés à la réception, ainsi que les désordres qui sont apparus dans l'année qui suit la réception.

			A la lecture littérale de l’article 1792-6 du Code civil :
			-	Désordre réservé à la réception
			-	Désordre qui apparait dans le délai d’un an à compter de la réception, il faut alors une notification, la LRAR n’est pas indispensable mais préférable car il faut se ménager la preuve
			Tous les désordres sont couverts par la garantie de parfait achèvement, quel que soit la nature comme défaut de conformité, vice de construction ou d’une non façon, quel que soit leur origine et quel que soit la gravité, il peut s’agir donc de désordres purement esthétiques ou des désordres plus graves.
			Concernant la gravité, les désordres réservés à la réception sont couverts par la garantie de parfait achèvement, même s’ils atteignent la gravité décennale, même les désordres qui rendent l’ouvrage impropre à sa destination. La décennale ne pourra trouver à s’appliquer car ils étaient apparents, sauf dans un cas où le dommage a été réservé mais s’est révélé dans son ampleur et ses conséquences postérieurement à la réception alors on considère que le dommage était caché et on peut agir en décennale.
			Les désordres réservés sont couverts par la responsabilité contractuelle de droit commun en plus de la garantie de parfait achèvement, un cumul est ici possible, c’est une faveur accordée par les tribunaux.
			Etant précisé que la GPA ne permet la reprise des désordres matériels et pas les immatériels de sorte que si le MO souhaite obtenir l’indemnisation des préjudices annexes liés à un dommage couvert par la GPA il devra agir sur la contractuelle de droit commun.
			Le délai de la GPA est d’un an à compter de la réception.
			Pour les désordres apparus après la réception, notifiés dans l’année qui affectent un élément d’équipement dissociable destiné à fonctionner relèvent de la GPA mais également de la biennale, mais il est préférable d’agir sur le fondement de la biennale que sur la GPA car la biennale permettra d’avoir la reprise du dommage matériel mais également immatériel alors que la GPA ne permettra que la reprise du dommage matériel. Ici donc possibilité d’agir sur les deux fondements.
			S’agissant des désordres apparus après la réception notifiés dans l’année qui rendent l’ouvrage impropre à sa destination qui affectent sa solidité ou affecte les éléments d’équipements indissociables, dans ces cas-là il est possible d’agir sur le fondement de la GPA mais également sur le fondement de la décennale, mais plus avantageux la décennale pour les mêmes raisons.
			Donc la GPA se cumule avec responsabilité contractuelle de droit commun, avec la biennale et la décennale. C’est une particularité de la GPA qui est cumulative. Néanmoins, un intérêt toujours d’agir sur les autres fondements.
			Quelques exclusions :
			Les désordres résultant de l’usure normale de l’ouvrage ou de l’usage : 1792-6 du Code civil : la garantie ne s’étend pas aux travaux nécessaires pour remédier aux effets de l’usure normale ou de l’usage
			Les troubles annexes, les dommages dits consécutifs, les immatériels sont exclus de la GPA alors possible d’agir sur le fondement de la responsabilité contractuelle de droit commun.

			La garantie tend à réparer les désordres en nature par principe, mais la JP a accepté que la réparation puisse intervenir par équivalent.
			Le régime de la GPA permet également d’obtenir la reprise des désordres phoniques. Les désordres acoustiques sont couverts par une garantie particulière qui s’appelle la garantie phonique dont le régime renvoie à celui de la GPA, article L111-11 du Code de la construction et de l’habitation
			On applique donc la GPA lorsque les exigences minimales en matière d’isolation phonique ne sont pas remplies. En pratique, ce texte est peu appliqué car les Tribunaux admettent de faire jouer la responsabilité décennale en cas de trouble d’ordre phonique. A cet égard la garantie décennale peut être mise en œuvre dès lors que l’insuffisance d’isolation phonique caché à la réception rend les locaux impropres à leur destination.
			Donc ce n’est pas parce que les exigences légales et règlementaires en matière phonique ont été respectées, que cela fait obstacle à une action sur le fondement de la responsabilité décennale.
			C’est une particularité que peu de personne ne savent.
			(quitus est un document lorsque la réserve est levée).
			L’intérêt d’engager la responsabilité décennale pour les désordre d’isolation phonique, c’est le délai qui est décennale, et la réparation des préjudices annexes, et surtout derrière il y a l’assurance de responsabilité civile décennale du constructeur.
			La responsabilité contractuelle de droit commun des constructeurs peut également être engagée en cas de désordre acoustique qui résulte d’un défaut de conformité aux stipulations contractuelles, caché à la réception et que pour autant l’ouvrage n’a pas rendu impropre à sa destination.
			A cet égard, l’action doit être introduite dans un délai de 10 ans à compter de la réception, article 1792-4-3 du Code civil et cette action peut être introduite soit par le MO ou l’acquéreur de l’ouvrage sachant que l’acquéreur en l’état futur d’achèvement peut agir à l’encontre du vendeur d’immeuble à construire dans un délai de 5 ans à compter de la découverte du désordre., article 2064 du Code civil
			Le constructeur peut s’exonérer de sa responsabilité en démontrant alors qu’il n’a pas commis de faute ou pour un cas de cause étrangère lorsque l’on est sur responsabilité contractuelle de droit commun.
			La victime peut solliciter une réparation en nature dans la mise en conformité, ou en équivalent. Sachant que l’action sur le fondement de la responsabilité contractuelle de droit commun permet la reprise des dommages matériels et immatériels.


			Il faut une notification, donc il est préférable de faire une lrar.

			Tous les désordres sont couverts, quel que soit leur nature, quel que soit leur origine, quel que soit leur gravité. de sorte que les simples désordres esthétiques sont couverts.

			Les désordres réservés sont couverts par la \gpa, même s'ils atteignent la gravité décennale. C'est à dire : ...

			Dans l'hypothèse où des désordres réservés se sont révélé dans leur gravité postérieurement,  ouvre droit à la décennale (on privilégiera la décennale pour atteindre les assurances)

			La gpa se cumule vec le rcdc, mais egalement avec decennale et biennale.
			Pour les désordres apparus après la réception, notifié dans l'année, alors gpa \& biennale. Il est préfrerable d'agir sur la biennale.

			Pour les... il est possible gpa \& decennale

			Conseil : Il est recommandé 31'

			Exclusion : 1792-6 in fine la garantie ne s'étend pas ...
			De meme les troubles annexes ... on agit alors sur la rcdc

			Par principe réparation en nature, mais jurisprudence accepte réparation en équivalent.

			Les désordres phoniques sont couverts par une garantie particulière, qui renvoie à la gpa L 111-11 cch (cite in extenso). Ce texte est peu appliqué car tribunaux accepte décennale. Des lors que l'insuffisance d'isolation phonique rende impropre les locaux à leur destination. a cet égard, le respect des exigences et obligations légales ne font pas obstacles à la mise en œuvre. L'interet est que le délai est plus long, en plus reparation prejudice annexe, et sutout possibilité de mobiliser les assurances. La rcdc peut etre egalement engagé 40' si elle resulte et que ne rend pas l'ouvrage impropre à sa destination 1791-4-3 soit par le Mo soit l'acquéreur de(Si c'est una cquéreur VEFA il peut agir dans un delai de 5ans à l'encontre du vendeur d'immeuble à construire).
			Le constructeur peut s'exonérer de sa responsabilité en apportant la preuve d'un orig etranger. la victime peut solliciter une ... en nature ou en équivalent, les prejudices matériels et immatériels.

		\subsection{Le régime et les modalités de mise en œuvre de la garantie}

			\subsubsection{Le régime de la garantie}

				\paragraph{Le délai d'action} Un an a compter de la réception. Sachant qu'il s'agit d'un délai de dénonciation et d'action. A defaut l'action du Mo est forclose.

				L'interruption du délai nécessite l'assignation ou les conculsions reconventionnelle. Le délai est interrompu, pas suspendu, donc un nouveau délai d'un an commence à courir à compter de la décision (ordonnance nommant l'expert ou décision définitive).

				Conseil : 47'

				Délai très bref. Cumul avec le rcdc (obligation de résultat).

				Le régime. Elle repose sur le même régime que décennale. Présomption et même cause d'exclusion.

				LE DELAI D’ACTION
				C’est un délai de garantie annale dont le point de départ est la réception. L’action doit être introduite dans un délai d’un an à compter de la réception.
				Il s’agit d’un délai de dénonciation et d’action. L’assignation doit nécessairement être délivrée dans le délai annal ou les conclusions reconventionnelles signifiées dans le délai annal.
				A défaut, l’action du MO est forclose. Le MO ne pourra plus agir sur le fondement de la GPA. L’interruption du délai suppose donc une assignation ou des conclusions reconventionnelles. Une réclamation par LRAR ne suffit à interrompre le délai de garantie.
				Il s’agit d’un délai de forclusion de sorte qu’il est interrompu et non pas suspendu c’est un nouveau délai de même durée qui recommence à courir à compter de la décision de justice.
				Ex : assignation en référé pour la désignation d’un expert, le nouveau délai d’un an commence à courir à compter de l’ordonnance désignant l’expert judiciaire. Si c’est un référé provision, le délai recommence à courir à compter de la décision de référé allouant ou rejetant la demande de provision. Si c’est une assignation au fond le délai recommence à courir à compter de la décision définitive ou de la décision d’appel.
				Ce délai de garantie est très bref. Il vaut mieux donc assigner au fond, le problème est que nous n’aurons pas nécessairement les éléments de preuve du préjudice, de son quantum. Donc il vaut mieux commencer par une assignation en référé pour dans l’année qui suit assigner au fond et demander de sursoir à statuer. Autre possibilité est d’agir au fond et de demander au juge de la mise en état la désignation d’un expert judiciaire. Une fois que le juge du fond est saisi, seul le juge de la mise en état est compétent pour désigner un expert, donc impossible après avoir saisi le fond de faire un référé. C’est pourquoi il vaut mieux commencer par un référé et ensuite assigner au fond avec le sursis à statuer.
				Ce délai est très bref, c’est pour cela qu’il y a un cumul avec la responsabilité contractuelle de droit commun. Ces deux régimes de responsabilité sont donc cumulatifs. On peut donc obtenir l’indemnisation de deux choses différentes. Les immatériels avec la responsabilité de droit commun et les dommages matériels avec la GPA, en sachant que la responsabilité contractuelle de droit commun permet également d’obtenir l’indemnisation des préjudices matériels. L’intérêt est que le délai de la responsabilité de droit commun est de 10 ans à compter de la réception, article 1792-4-3 du Code civil
				Sur le fondement de la responsabilité contractuelle de droit commun pour les désordres réservés il y a une obligation de résultat donc le simple constat du désordre suffit pour mettre en œuvre la responsabilité contractuelle de droit commun.
				REGIME DE LA RESPONSABILITE
				C’est le même régime que celui de la responsabilité décennale, on est donc en présence d’une présomption de responsabilité avec les mêmes causes d’exonération, la cause étrangère, le cas de force majeur, le fait de l victime ou le fait d’un tiers.


			\subsubsection{Les modalités pratiques de mise en œuvre de la garantie}

				1792-6 alinéa 3 à 5.

				Le Mo doit faire appel à l'E lui demandant de reprendre le désordre. S'il fait appel à un tiers, il perd la possibilité d'actionner la gpa.

				Les parties se mettent d'accord sur un calendrier pour réaliser les travaux. Il est donc préférable de se mettre à l'intérieur du calendrier. La norme AFNOR propose des délais 60 jours à compter du pv re
				60 jours à compter de la notification du désordre.

				En cas de carence de l'\E le \Mo doit Mise en demeure .

				Si l'\E ne réagit toujours pas, le \Mo peut faire effectuer les travaux aux frais et risques. La retenue de garantie, si elle a été contractualisée, permettra de financer les travaux. Le risque est que l'\E conteste l'étendue du désordre, le nature des travaux de reprise, le quantum des travaux de reprises. En conséquence, ce mécanisme est très peu mis en œuvre en pratique.

				Si il appartient au Mo et à l'E d'un commun accord de constater contradictoirement la réalisation des travaux. A défaut, il faudra faire constater par voie judiciaire. Encore une fois, processus long qui est peu mis en œuvre en pratique.

				En pratique, comment mettons en œuvre la GPA ?
				Tout est prévu à l’article 1792-6 alinéa 3 à 5 du Code civil
				1° Etape
				Le MO doit faire appel à l’entrepreneur responsable du dommage et lui demander de reprendre le désordre.
				Si le MO fait appel à un tiers avant de mettre en œuvre la GPA, il perd la possibilité d’agir sur le fondement de la GPA, il ne peut pas avoir de suite recours à un tiers, il doit d’abord actionner l’entrepreneur afin que celui-ci reprenne les désordres
				2° Etape
				Les parties se mettent d’accord sur un calendrier de réalisation des travaux.
				Si l’entrepreneur constate la réalité du dommage et décide d’intervenir, les parties doivent donc se mettre d’accord sur un calendrier pour réaliser les travaux.
				Le délai d’un an est un délai de dénonciation et d’action, ce n’est pas un délai dans lequel les travaux doivent être réalisés. Les parties peuvent parfaitement se mettre d’accord pour que les travaux soient réalisés après le délai d’un an. Mais si l’entrepreneur ne réalise pas les travaux le MO perd alors la possibilité d’agir sur le fondement de la GPA qui sera expiré. Il est donc préférable et recommandé au MO de se mettre d’accord avec l’entrepreneur sur un calendrier à l’intérieur du délai annal, comme ça en cas de non-respect du calendrier par l’entrepreneur qui ne reprend pas les dommages, le MO aura encore la possibilité d’agir à son encontre sur le fondement de la GPA.
				La norme AFNOR prévoit des délais de reprise des dommages, elle n’est applicable que si elle a été contractualisée : 60 jours à compter de la réception du PV de réception pour les désordres réservés et 60 jours à compter de la notification des désordres révélés dans l’année de la GPA.
				Les parties doivent se mettre d’accord sur une intervention et les modalités de l’intervention. Si le MO n’est pas d’accord avec ce que propose l’entrepreneur le MO a la possibilité de l’assigner et partir en expertise, et c’est l’expert qui déterminera la solution.
				3° Etape
				En cas de carence de l’entrepreneur, le MO doit le mettre en demeure d’effectuer les travaux. Dans ce cas-là c’est une LRAR qu’il faut lui adresser. Et si malgré la mise en demeure l’entrepreneur ne réagit pas, le MO peut faire effectuer les travaux de reprise aux frais et risques de l’entrepreneur défaillant.
				La problématique est alors le moyen de financer ces travaux par le MO ?
				La retenue légale de garantie peut permettre au MO de pré financer les travaux aux frais et risques de l’entrepreneur principal. Encore faut-il que cette retenue de garantie ait été contractualisée.
				En cas d’un entrepreneur en faillite, la mise en demeure est alors inutile mais toute action à son encontre sera également inutile, alors plus aucune solution.
				On pourrait également faire un référé provision, mais le juge devra constater que le désordre existe et que l’on ait une idée sur son quantum et le coût des travaux de reprise, c qui ne sera pas forcément facile, donc le référé provision est très rare.
				Ce que l’on voit parfois c’est lorsque l’on fait une demande de désignation d’expert, on fait également une demande provision, elle est systématiquement rejetée. Il vaut mieux partir en référé provision après le dépôt du rapport d’expertise parce que là on a un rapport contradictoire avec un expert qui s’est penchée sur le quantum et un juge qui sera donc pleinement éclairé.
				Le risque de faire réaliser les travaux aux risques de l’entrepreneur est que l’entrepreneur puisse contester le quantum des travaux effectués. Donc lorsque l’on demandera la condamnation de l’entrepreneur au montant des travaux réalisés, celui-ci pourra contester le quantum et dire d’une part, je n’ai pas pu apprécier la réalité du désordre et il pourrait contester en disant je conteste donc la réalité des désordres, la nature des désordres et le quantum donc le prix pratiqué par l’entreprise appelé par le MO. Il y a donc pas mal de risques donc en cas de travaux excessifs ceux-ci peuvent rester à la charge du MO, il y a donc un risque qui fait que cette procédure est très peu utilisée, le tribunal contrôle a posteriori le quantum
				En pratique peu utilisée la GPA car soit pas les moyens financiers ou en raison des risques invoqués.
				Si les travaux sont urgents, le MO aura tout intérêt à le faire, mais à ce moment-là le MO prend ce risque en connaissance de cause, donc avec le risque de ne pas être remboursé ou de moindre valeur.

				Si les parties se mettent d’accord sur l’exécution des travaux de reprise, il leur appartient de constater d’un commun accord, à la signature d’un quitus la réalisation des travaux, ou à défaut, la réalisation des travaux doit être constatée judiciairement en cas de désaccord entre les parties, ce qui suppose la désignation d’un expert judiciaire.

		% !TEX root = ./droitConstruction.tex

\chapter{La responsabilité de droit commun des constructeurs}

Il existe deux types de responsabilité de droit commun, la responsabilité contractuelle (s’il existe un contrat) et délictuelle (en l’absence de contrat).

\section{La \rcdc des constructeurs}

	Ce sont les hypothèses de responsabilités fondées sur les articles 1103 du Code civil : les conventions légalement formées tiennent lieu de loi entre les parties, et 1231-1 (ancien 1134 et 1147) du Code civil.

	\textbf{Précisions liminaires} :
	\paragraph{Principe de non cumul.} Dès lors que les conditions de la décennale ou biennale sont remplies, il faut nécessairement agir sur ces fondements. Donc au-delà du délai de 2 ans pour la biennale, il n’est pas possible d’agir sur le fondement de la contractuelle et sauver le dossier.

	En revanche, la responsabilité contractuelle peut être invoquée de manière subsidiaire

	La seule exception est le cumul avec la GPA
	\paragraph{Elle est transmissible.} L’action sur le fondement de la responsabilité contractuelle de droit commun est transmissible aux acquéreurs aux acquéreurs et sous acquéreurs de l’ouvrage, au même titre que les garanties spécifiques, la limite étant le délai de 10 ans à compter de la réception (parfois 5 ans) ici on est en prescription et pas en forclusion
	\paragraph{Causes exonératoires.} Les causes d’exonération de responsabilité de droit commun sont identiques à celles des garanties spécifiques, il faut une cause étrangère : le fait d’un tiers, la faute de la victime \MO (il existe plusieurs hypothèses à savoir le mauvais usage, immixtion du MO notoirement compétent, et acceptation délibérée du risque donc le MO a pleinement été informé donc en connaissance de cause, délibérément le MO a accepté le risque), la force majeure.
	cause étrangère : force majeure, fait d'un tiers (mais pas co locateur d'ouvrage), faute du Mo (mauvais usage = acceptation délibéré (= pleinement informé) du risque)
	La faute d’un co-locateur d’ouvrage n’est pas une cause d’exonération de responsabilité contractuelle de droit commun du constructeur, ici le tiers est extérieur à l’opération de construction
	\paragraph{Hors assurance obligatoire.}	Les assurances obligatoires ne couvrent pas les hypothèses de responsabilité contractuelle de droit commun, mais le constructeur peut souscrire une assurance facultative

	\subsection{Les désordres n'affectant pas un << ouvrage >>}

		Les peintures ne constitue jamais un ouvrage dans la jurisprudence.

		Les constructeurs sont tenus d'une obligation de résultat, de sorte qu'ils ne peuvent s'exonérer de leur responsabilité du fait d'une cause extérieure.

		délai 1792-4-3 s'applique car Mo

	\subsection{Les désordres affectant des travaux commandés par une personne autre que le propriétaire}

		Les désordres affectant des travaux autres que le Mo. Donc lorsque les travaux sont commandés par le locataire civ 3 1/7/09. Dans ce cas seule la rcdc peut etre actionnée.

		L'E est tenu d'une obligation de résultat, le simple constat des désordres suffit. Prescrit à compter de 5 ans 2224 du c. civ à compter de la manifestation des dommages. Civ 3 16/1/20 18-21895.

		Les titulaires de baux emphytheotique assimilé Mo (on prévoit dans le contrat).

	\subsection{Les désordres apparus avant la réception}

		Seule la rcdc est applicable avant la réception.

		Les locateurs d'ouvrages sont néanmoins redevable d'une obligation de résultat.  cas ci 3 eme ???. Le simple constat du désordre suffit. Le vendeur immeuble à construire. Les archi sont débiteurs d'une obligation de moyen, il faut donc apporter la preuve d'une faute.

		Avant réception, tous les dommages doivent être repris et les préjudices indemnisées doivent être indemnisé.

		Prescription 24/4/06 ccas considère que les desordres se prescrive à partir de la manifestation du dommage

	\subsection{Les désordres réservés à la réception}

		ca civ 3 13/12/1995 : cumul gpa et rcdc

		Dans un arrêt du 2/2/17 la 3 c civ a jugé que l'obligation de résultat persiste jusqu'à la levée des réserves. L'intérêt demander l'indemnisation des préjudices annexes en plus de la reprise des dommages

	\subsection{Les dommages intermédiaires}

		C'est une catégorie importante de la resp contra. Il s'agit d etous ls dommages cachés à la réception, qui ne . Hypo des garantie construceur réunie :
			...
		Mais sans garvité decennale. Par exemple : un décordre affectant le ravalement si il n'y a pas ou si ces désordes sont simplement esthétque. On y compte galement des désordres dans les tuiles, des défaus de carrelge.
		La gravité du désordre ets indifférent dès lors qu ela gravité décennale n'est pas atteinte.

		Il n'est pas possible ... en cas de mauvais fonctionnement d'un équipement dissociable. Il serait vain d'invoquer la théorie de dommages interméiaires pour échapper à la forclusion biennale.


		Mise en œuvre Les \lo ont une obligation de moyen. Les c civ 3eme 11 5 204 6 10 2010 il faut apporter la preuve de la faute du constructeur
			faute + préjudice + lien de causalité.
			Cette faute ne peut pas résulter simplement d el'obligation de résultat de livrer un ouvrage exempt de vice.

			L'\E sous-traite sous sa responsabilité. La simple faute du sous traitant suffit à engager la responsabilité de l4EP à l'egar d du Mo cas civ 3 1 5 2006

			La responsabilité de est engagé également à lagrd de sprop. succesif d el'ouvrage.
			Sont également débiteur :
				le viac - pour faute prouvée ...
				la personne qui vends après achèvement qu'elle a construit ou fait construire

			Prescription : 10 ans à compter de la réception 1792-4-3

	\subsection{Les défauts de conformité cachés à la réception}

		Cette hypo correspond au non respect de stipulation contractuelle dont il ne découle pas un dommage.

		Si d'un défaut de conforrmité dont il ne découla pas un dommage de gravité décennale.

		Ex : différence de superficie, non respect des plans

		Les constructeurs sont tenus d'une obligation de résultat. le simple constat de la non conformité suffit à engager leur responsabilité.

		presci : 1792-4-3 10 ans

		Les défaut de conformité apparents à la réception non réservé sont purgés.

	\subsection{La violation d'obligations contractuelles dont il ne découle pas de désordre à l'ouvrage lui-même}

		Hypo : le retard
		il s'agit d'une obligation de résultat, à la quelle il  ne peut échapper qu'en apportant la preuve d'une cause étrangère (force majeur, fait d'un tiers, faute de la victime)

		Il est recommandé de prévoir des clauses de suspension du délai d'exe., te donc de déterminer des cause légitimes de suspension -

		Sanctionné par des dommage set interets évalués sur la base du préjudice subi (pertes financières, etc.).

		des clauses de pénalité de retard sont possible. cen 'est pas parceque le contrat ne prevoit pas de pénalité de retard, que les d\&i ne sont pas dus. Si le contrat stipule des pénalités de retard ... il s'agit de clause pénale, le juge peut les réduires ou les augmenter. Le juge apprecie souverainement le quantum du préjudice.

		Hypo : manquement à l'obligation d'info et de conseil. le \lo est tenu d'une . Le contenu dépend de chaque intervenant et de sa compétence.

		c'est une obligation de moyen, de sorte qu'il revient au débiteur, le c/lo, qu'il a rempli.

		La sanction dépend de la nature du préjudice subit.

		S'il découle un dommage 1792 et suivant, alors dans ce cas la responsabilité doit être recherche sur le fondement de la responsabilité decennale. Ex; defaut d'info sur le choix des matérieux

		Si le préjudice n'est pas un dommage à l'ouvrage mais par exemple un surcout ou una llongement, alors le Mo pourra obtenir sur rcdc. Ex : Moe ne pas avoir vérifié la qualification ou la solvailité, s'il a commis une erreur sur l'estimation.
		le /lo s'il n'a pas averti le Mo sur le risque d etrouble sur les avoisinants

		Prescri 10 ans 1792-4-3

		Hypo : manquement à l'assitance à la reception du Moe ;

 		Il faut que le contrat le prévoit.

		S'il ne le liste pas les réserves. Obligation de moyen c'est au Mo

		Prescri : 10 ans 1792-4-3

		Hypo : Dommages causés aux ouvrages existants.

		A priori ils reeléevnt de la rcdc.

		Avant Obligation de cosnerver les existant et de . Sa faute est présum

		5 an à comptre de la manifetstaion 2224 cciv

		Après : ...

		Presci 10 1792-4-3, ou 5 ans car pas de réception. Me Pelon penche sur 10 ans.

		En pratique rare compte tenu de la juris favorable au Mo\footnote{non cumul} qui fait bénéficier de la décennale dans différents cas :
		 pas possible de détreiner si la cause provient des
		 provent des travaux net travaux neuf et existant sont techniquement indivisibles
		 trvx sur existant consttuen en eux meme un ouvrage (renovation lourde ravalement étanche)
		 élément équipement sur existant qui porte atteinte à la destination de l'ouvrage dans son ensemble (cheminée)

	\subsection{Le dol du constructeur}

		La juris le constr "sauf faute extérieure au contrat" contractuellement  tenu à l'agrd du Mo de sa faute dolosive. Il y a dol lorsque de propos délibéré, meme sans intention de nuire, le contructeut viole par dissimulation ou par fraude ses obligations contractueles 27 1 2001 27 3 2013

		Dans l'hypothèse du dol, le construc n'a pas necessaireemnt mais il a la volonté de le cacher. 8 9 2009 la cas a admis l'existence du dol en cas de travaux desastreux, i e contraire aux règles de l'art et aux précotion elementaire. En reveanche un défaut de surveillance des soustraitants ,ene suffit pas à caractèriser.

		Le dol est intersseant en ce qi permet le Mo au-dela des délais de prescription et de forclusion. Pres 5 à compter de la survenance. elle est transmissible avec l'ouvrage, de sorte un sous acquéreur est recevable à agir.

		Le dol n'est couvert par aucune assurance

	\subsection{Les désordres affectant ou provenant d'un élément d'équipement à vocation professionnelle}

		Aux termes 1792-7 les éléments ... exclusive ... ne bénéficie ni de la décennale ni de la biennale. Et ce même si leur défaillance rend l'ouvrage dans son ensemble impropre.

		Si dommage à l'éléments d'équipement lui même = vice caché

		Les dommages causé par l'éléments d'équipement = rcdc

\section{La responsabilité délictuelle du droit commun des constructeurs}

	Ce Les actions fondées sur les inconvénients anormaux de voisinage... 1240 et suivants du cciv

	\subsection{Les actions fondées sur les inconvénients anormaux de voisinage}

		Origine praterorinne Nul ne doit causer à autrui de trouble excedant les inconvenient normaus de voisinage.

		Responsabilité de plein droit qui ne requiert pas la preuve d'une faute.

		\subsubsection{Caractérisation du trouble anormal de voisinage}


			\paragraph{Les conditions de l'action}

				Elle suppose une relation de voisinage. Il n'existe pas de def legale de la notion de voisinage. Ni juris, de sorte que ce sont les tribunaux qui vont au cas par cas sur la base de la proximité géo.

				Deuxime condition : trouble anormal. Un simple trouble ne suffit pas à ouvrir droit à indenisation. Il papartiet au demandeur de caratériser l'anomraliré. Il doit demontrer que les nuisance subies, ressenties, excede les inconvenients normaux de voisinage. Ex : la tondeuse normal le dimanche, anorml

				il s'git d'une question defait soumise à l'appreciation souveraine des juges du fonds.

				L'anormalité du trouble peut etre caractérisée alors même qe des dispos legales, conventionnelel ou reglementaire sont réunies. Ainsi l'obtention d'un PC ne suffit pas car sous réserve des droits des tiers. De meme regelemnt de copro, permis de démolir, lotissement.

				3eme : trouble continu. Un trouble episodique, fugace n'est pas sufffisant. La preuve du caractere continu incombe à la victime

			\paragraph{La diversité des troubles}

				\subparagraph{Les troubles de jouissance dus au chantier}

					Il s'agi de tous les troubles causés bruit odeurs poussieres. Il faut qu'il revetent une certiane gravité, qu'il appartient à la victile de caratériser.

				\subparagraph{Les dommages matériels subis par l'immeuble voisin}

					Il s'agit de tous les dommages matériels causé par le chantier. Le plus souvent terrassement démolition fondation gros œuvre. référé préventif intérêt car on va pouvoir imputer les desordres au chantier et plus précisément à l'entre présente.

					Pratiquement : assigner l'ensemble des voisins, le juge nomme un expert. La constitution d'un avocat est obligatoire. Intervention volontaire possible.

				\subparagraph{Les troubles de jouissance dus à la construction}

					La construction. Grande diversité : dues aux dimensions à l'implantation de la construction, ouverture de vues obturation d'ouverture, diminution d el'ensoleillement 16 2 2019, 19 7 2017 attention il n'existe pas de droit acquis à l'ensoleillement, il appartient au juge \emph{in concreto} de la situation pour caractériser l'existence d'un trouble. civ 3 17 5 2018 -  21 10 2009 dans un lotissement
					diminution de la vue sur le paysage environnant et dommage d'ordre esthétique (perte totale de la vue sur la seine 26 5 2016)

		\subsubsection{Les spécificités du régime de l'action pour trouble anormal de voisinage en matière de construction}

			\paragraph{L'auteur de l'action} Peuvent agir toutes les personne s qui subissent directement et personnellement le trouble. il peut s'agir du proprietaire

			\paragraph{Les defendeurs} Peuvent voir le responsabilité engagées :
				le \Mo alors même qu'il n'est plus proprio du fonds --- 21 5 2008 : le Mo est responsable de plein droit. Il est considéré comme l'auteur intellectuel du dommage.
				les \E dès lors qu'ils interviennent matériellement sur le chantier, qu'ils soient ou non liés au Mo par un contrat. Ainsi, l'\E général, le titulaire d'un simple lot, les sous-traitants
				les Moe, les bureau xd'étude et les controleurs techniques

			Pour les constructeurs, il faut établi un lien d'imputabilité entre le trouble et les travaux litigieux 21 5 2008 c'est à dire une relation de cause directe (inverser la phrase ...). Le demandeur doit démontrer que l'activité de l\E est à l'origine matérielle du trouble, ou le cas échéant qu'il a commis une erreur dans le contrôle et a surveillance des travaux de son sous-traitant.

			pour les Moe, les lbureaux ... a responsabil d'une relation de cause directe entre els troubles et la réalisation des missions confiées civ 3 9 2 2011 ce qui revient à exiger la preuve d'une faute.

			\paragraph{Le délai d'action} L'action se prescrit par 5 ans à compter de la manifestation du dommage ou son aggravation. 16 1 2020 ... réduite à 5 ans à compter la

			en cas d'empiétement 30 ans 2272 cCiv

			\paragraph{La sanction du trouble} Le demandeur peut solliciter une réparation en nature jusqu'à la démolition (sauf disproportion) ou en équivalent (dommages et intérêts)

		\subsubsection{Les actions récursoires entre co-auteurs}

			\paragraph{Les recours contre le \lo}

				\subparagraph{Le fondement de l'action} Il faut envisager deux hypo selon que le Mo indemnise ou non les victimes.

					Si le Mo indemnise la victime il est subrogé dans les droits de la victime. ce qui le dispense d'apporter la preuve d'une faute c civ 3 21 5 2008

					Si le Mo n'a pas indemnisé la victime. Par exemple : . Dans ces cas il 'est pas subrogé dans les droits de la victime, il ne pourra agir que dans en fonction du lien de droit qui l'unit avec le responsable.

					Attention une exception : en cas de stipulation contractuelle entre le Mo et ses locateurss d'ouvrages --- garantie intégrale même sans faute par exemple, qui sera simplement appliquée 10102 du c civ.

				\subparagraph{Les cause exonératoires de responsabilité : la faute du \Mo}

					Les peuvent invoquer la faute du (acceptation délibéré des risques et immixtion fautive du \Mo notoirement compétent (pas de plein droit 19eme chambre  appel de paris 26 11 2008))

			\paragraph{Les recours entre locateurs d'ouvrage}

				Ce recours sera de nature contractuelle ou délictuelle selon qu'il existe ou non un contrat entre eux. La jurisprudence refuse de faire bénéficier du mécanisme de la subrogation au lo qui aurait indemnisé la victime.

				Dans les deux cas cela nécessite de la preuve du faute 26 4 2006. la charge dela dette se répartie en fonction de la faute de chacun c'est le mécanisme du recours \emph{in solidum}. En l'absence de faute, la répartition se fera par part virile.

	\subsection{Les autres actions en responsabilité délictuelle}

		Actions fondées sur 1240 et suivnats du cciv



		%\subsubsection{L'action du \Mo contre un sous-traitant} c'est une précision liminaire

		Ne pas oublier que l'action du \Mo à l'encontre du \ST sont délictuelle

		\subsubsection{Les actions menées par un tiers au chantier}

			Hypo 1 :1240 action fondée sur la faute d'un constructeur : faute + +

			LA faute peut etre prouvée par tout moyen (ex : absence de cloture et électrocution)

			Conseil : agir sur le trouble anormal de voisinage

			assemblée plénière 6/10/2006 confirmé 13/1/2020

			18/5/17 il faut prouver une faute, le simple manqueme

			Prescription 5 ans à compter de la manifestation du dommage.

			Hypo 2 : 1342 al 1 \& 2. action fondée sur la garde de l'immeuble

			Invoquée en cas de chute d'objet, de matériaux, d'un

			Conseil : traoble anmral de voisinage

			Prescription 5 ans à compter de la manifestation du dommage.

			Hypo 3 : action commis-comettant. 1242 al 5. Aller voir les ouvrages

			Hypo 3 : dommage caus par la ruine du bâtiment. 1244 Aller voir les ouvrages

		\subsubsection{Les actions délictuelles entre locateurs d'ouvrage}

			Ce sont les actions entre \lo non liés entre eux par un contrat. Ex : les \E{}s entre eux ST entre eux, un \E avec

			pas de présomption de responsablité de 1792 et suivant = nécessité : faute + préjudice + lien de causalité

			Quels hypothèses ?
			\begin{itemize}
				\item Action récursoire après condamnation \emph{in solidum}.
				\item si Entrepreneur dégrade les travaux d'un autre locateur d'ouvrage.
				\item Histoire du forfait top peu élevé à cause du \Moe...
			\end{itemize}

			Conseil : toujours mettre la petite phrase que l'\E doit vérifier les métrés.

			Prescrit par 5 ans à compter de la manifestation du dommage ou de son aggravation. Les tribunaux fixent communément le point de départ au jour de l'assignation du constructeur par le \Mo.

			Attention : Arret civ 3 13/9/06, point de départ au jour où le dommage s'est manifesté à l'égard du \Mo. elle a été remise en cause depuis 16/1/2020, probablement car de nature à remettre en cause ...

		\subsubsection{La responsabilité du fait des produits défectueux}

			Article 1245 Cciv et suivants.

			Le régime de responsabilité garantie les dommages causé par un produit défectueux, \cad un produit qui n'offre pas la sécurité à laquelle on peut légitiment s'attendre.

			1245-2 constitue un produi .. même ... immeuble

			ne couvre que les dommages portant à la personne ou a un bien, autre que le produit défectueux lui-même, dès lors qu'ils excedent \montant{500}.

			il peut s'agir d'un défaut de conception, de fabrication d'un matériaux de construction  d'une partie composante ou d'un élément d'équipement de la construction.

			elle s'applique au producteur du produit \cad du fabriquant ou ... importateur ... à l'encontre de tous ceux qui incorporent le produit à l'immeuble ... Ne s'applique pas aux personnes qui peuvent être recherché sur le fondement 1792 et suivant (garanties spécifiques), ni au vica sur la responsabilité 1646-1. Expressément prévu par 1245-5.

			Mise ne œuvre ne suppose la preuve d'une faute, << le producteur est responsable ... >>. Dommage + défaut + lien de causalité.

			Action encadré par un double délai 10 ans de la mise en circulation du produit / à l'intérieur de ce délai 3 ans à compte de la date. ???? 1245-10.

		\chapter{La responsabilité des fabricants}

	\part{Les assurances construction}

		% !TEX root = ./droitConstruction.tex

\chapter{Les assurances obligatoires}

	Il existe deux types d'assurances obligatoires :
	\begin{itemize}
		\item l'assurance de responsabilité, assurance de responsabilité civile décennale qui couvre l’activité des constructeurs ;
		\item l'assurance de chose l'\ado qui couvre l’immeuble assuré.
	\end{itemize}

	\medbreak La dernière grande réforme est du \printdate{4/1/1978} complétée notamment par l'ordonnance du \printdate{8/6/2005}

	\medbreak Les principaux textes sont :
	\begin{itemize}
		\item les \articlesDuEtSuivants[L]{241-1}{\ca}, pour l'\arcd ;
		\item les \articlesDuEtSuivants[L]{242-1}{\ca}, pour l'\ado ;
		\item les \articlesDuEtSuivants[L]{241-1}{\ca}, en ce qui concerne les dispositions communes.
	\end{itemize}


\section{Le principe de l'assurance obligatoire}

	\subsection{Le champs commun des assurances obligatoires}


		\subsubsection{Les travaux couverts par l'obligation d'assurance}

			Avant l'ordonnance \printdate{8/1/2005} l'obligation d'assurance ne portait que sur les travaux de bâtiment.
			Seuls les locateurs d’ouvrage qui réalisaient des travaux de bâtiment devaient souscrire une assurance obligatoire, c’était très restrictif.

			De sorte que la \JP a progressivement et considérablement élargi le champ d’application de l’obligation d’assurance en soumettant tous les travaux réalisés en faisant appel aux techniques de travaux de bâtiment.
			Ainsi, les travaux portant sur un mur de soutènement, les stations de métro, les \vrd, les dalles de plafonds, \etc rentraient dans le champ d’application des assurances obligatoires car c’étaient des travaux réalisés en faisant appel au technique de travaux de bâtiment.

			Le législateur est intervenu conscient que la formule utilisée était restrictive, depuis ordonnance du \printdate{8/6/2005} l’obligation porte sur les travaux de construction, notion plus large que celle de bâtiment.
			Cette notion inclus tous les ouvrages immobiliers, \CAD tout ce qui est ancré dans le sol, donc tout ce qui est ouvrage est construction au sens décennale et doit être couvert.

			L'\articleDu[L]{243-1-1}{\ca}liste des exclusions :
			\begin{itemize}
				\item les ouvrages maritimes, lacustre, fluviaux, les ouvrages infrastructure routière, portuaire, aéroportuaire, héliportuaire, ferroviaire, \etc, bref l'ensemble des ouvrages de génie civil ;
				\item les voiries, les parcs de stationnement, les canalisations, les ouvrages sportifs non couverts, les ouvrages de production, de stockage, et de distribution d’énergie, les ouvrages de télécommunication, et les éléments d’équipement.
			\end{itemize}

			Si les ouvrages ou les éléments d’équipement visés dans la deuxième exception, constituent un accessoire à un ouvrage soumis à l’obligation d’assurance alors ils sont soumis à une obligation d’une assurance

		\subsubsection{Les dommages couverts par l'obligation d'assurance}

			\paragraph{Les dommages de nature décennale, provenant d'un sinistre, affectant l'ouvrage}

				Les assurances obligatoires ne couvrent que les dommages à l’ouvrage lui-même ce qui exclut les dommages consécutifs : les dommages annexes \CAD les dommages incorporels, les dommages aux tiers et les dommages aux mobiliers.

				Il faut que le dommage soit survenu à l'occasion d'un sinistre, ce qui exclut les non façons et les défauts de conformité.
				En revanche couvre les travaux non prévus à l'origine (\emph{i.e.} absence d'ouvrage) mais qui aurait été indispensable pour éviter le sinistre.
				\begin{exemple}
					Dans le cas d'une inondation au sous-sol à cause d’une absence de drainage qui aurait dû être prévu à l’origine, on a bien un sinistre accidentel avec l’inondation du sous-sol, dans ce cas-là les assurances obligatoires couvrent les travaux non prévus à l’origine mais qui auraient été indispensables pour éviter le sinistre
				\end{exemple}

				L'assurance obligatoire ne couvre que les dommages de nature décennale, dont le critère déterminant est la gravité décennale, de sorte que les dommages survenus pendant les travaux sont exclus. C’est le critère déterminant

				\medbreak Ces critères :
				\begin{enumerate}
					\item dommage affectant l'ouvage,
					\item provenant d'un sinistre,
					\item de gravité decennale ;
				\end{enumerate}
				sont cumulatifs. Il faut donc un dommage à l’ouvrage lui-même, survenu à l’occasion d’un sinistre, de nature décennale, sur un ouvrage reçu (les dommages intermédiaires sont exclus)

				\medbreak Le fondement de la responsabilité est indifférent, dans le cadre des actions récursoires entre constructeurs et de son assurance, les assurances de responsabilité civile décennale ne peuvent pas dénier leur garantie au motif que le fondement de la responsabilité de leur assuré est délictuel, ce qui compte c’est le dommage et la gravité du dommage, alors l’assurance peut être tenue à garantir son assuré\footnote{\jurisCourDeCas[17-13833]{\civTrois*}{8/11/2018} : le fondement juridique de la responsabilité est indifférent}.

				\subparagraph{Cas des ouvrages existants}
				Aux termes de l'\articleDu[L]{243-1-1 \II}{\ca} << {\itshape ces obligations d'assurance ne sont pas applicables aux ouvrages existants avant l'ouverture du chantier, à l'exception de ceux qui, totalement incorporés dans l'ouvrage neuf, en deviennent techniquement indivisibles}. >>

				L’interprétation traditionnellement admise de cette disposition a conduit à n'intégrer dans le champ de l’assurance obligatoire que les dommages sur les existants techniquement indivisibles des ouvrages neufs. Seuls les dommages affectant les existants techniquement indivisibles avec les ouvrages neufs étaient couverts par les assurances obligatoires.
				Néanmoins dans des arrêts \printdate{15/6/2017} et \printdate{14/9/2017}, la \CourDeCas a admis que la garantie décennale pouvait être mobilisée en cas d’atteinte à un élément d’équipement dissociable ou non d’origine ou installée sur existant dès lors qu’il rend l’ouvrage en son ensemble impropre à sa destination, alors relève de la responsabilité décennale

				La \civTrois a tiré les conséquences de la jurisprudence et a soumis les éléments d'équipement installés sur existant\footnote{\jurisCourDeCas{\civTrois*}{26/10/2017}}

				Le legislateur a tenté de contrer cette jurisprudence à l'occasion de la loi ELAN en réécrivant l'\articleDu[L]{243-1-1 \II}{\ca}, car elle très favorable aux \Mo et engage la responsabilité personnelle des artisans. Le Conseil constitutionnel a cependant déclaré le texte inconstitutionnel faute de lien avec le projet de loi initial.

				Aujourd’hui c’est toujours cette \JP qui s’applique, et les artisans restent soumis à cette obligation d’assurance.

			\paragraph{Les dommages affectant les travaux régulièrement déclarés}

				Seuls les dommages affectant des travaux exactement déclarés par l’assuré à l’assureur sont couverts par la police d’assurance. Cela s'explique par le fait que la prime est calculée sur le risque auquel s'expose l'assureur. Le \lo, ou le \MO, doit déclarer le chantier, la construction envisagée, pour que l’assureur puisse déterminer son risque.

				\medbreak En cas de déclaration inexacte ou en cas d’absence de déclaration, les juges peuvent soit :
				\begin{itemize}
					\item annuler la police d'assurance si l’assurance a été intentionnellement trompée  --- fraude visée par l'\articleDu[L]{113-8}{\ca} --- et dans cette hypothèse les primes payées restent acquises à l’assureur sans indemnité , la nullité du contrat étant opposable à la victime ;
					\item faire application du mécanisme de la réduction proportionnelle de l'\articleDu[L]{113-9}{\ca}\footnote{Mécanisme de reduction de l'indemnité en proportion du taux de prime payé par rapport au taux de prime qui aurait du être appliqué si les risques avaient été complètement et exactement déclarés. en pratique les juges entérinent les calculs proposés par l’assureur.}.
				\end{itemize}
				Le mécanisme de la réduction proportionnelle s'applique également à la victime, ce qui est trés pénalisant pour celle-ci. Lorsque le chantier n’est pas du tout déclaré la victime ne pourra rien obtenir.

	\subsection{La preuve de la souscription d'une assurance}

		La forme de la preuve est consacrée à l’\articleDu[L]{243-2}{\ca} : << {\itshape les personnes soumises aux obligations prévues par les articles L. 241-1 à L. 242-1 du présent code doivent être en mesure de justifier qu'elles ont satisfait auxdites obligations} >>. La preuve se fait par la production d'une attestation d'assurance. Elle doit être jointe aux devis et factures (sans sanction).

		Un arrêté du \printdate{05/01/2016} fixe le modèle d’attestation. Les mentions minimales concernent :
		\begin{itemize}
			\item le nom,
			\item l’adresse et les coordonnées complètes de l’assureur,
			\item l’identification de l’assuré,
			\item le \no de contrat,
			\item sa période de validité,
			\item la date d’établissement de l’attestation,
			\item les activités déclarées et assurées,
			\item l’adresse et le coût de l’opération déclarée,
			\item et la date d’ouverture de chantier
		\end{itemize}
		L’attestation doit donc fournir des informations exactes sur l’étendue des garanties offertes et ne pas égarer le \MO.
		Etant précisé que les exclusions et limitations de garantie doivent être mentionnées.

		L’auteur de l’attestation est l’assureur.

		\subparagraph{S’agissant du moment de la preuve}
		Toute personne physique ou morale soumise aux obligations d’assurance doivent être en mesure de justifier qu’elles ont souscrit un contrat d’assurance à l’ouverture de tout chantier.

		Le \MO peut demander à tout intervenant à l’acte de construire de justifier qu’il satisfait à ses obligations d’assurance à tout moment, après la période de validité pour l’année suivante.


	\subsection{Les sanctions du défaut d'assurance}

		Le défaut d'assurance est sanctionné.

		\paragraph{Sanction d'ordre pénal}
			L'\articleDu[L]{243-3}{\ca} prévoit 6 mois d'emprisonnent et \montant{75 000} d'amende.

			Les sanctions pénales ne s’appliquent pas aux personnes physiques qui construisent pour occuper le logement ou des membres de sa famille. de sorte que particulier qui construit pour lui même n'encourt pas de sanction pénale.

		\paragraph{Sanction d'ordre civil} Il existe également des sanctions d'ordre civil :
		\begin{itemize}
			\item Le \Mo engage sa responsabilité civile s’il n’a pas souscrit de \DO.

			\item L’architecte engage sa responsabilité civile s’il engage des entreprises non assurées ou s’il ne conseille pas le \MO de souscrire une \DO

			\item L’entrepreneur locateur d’ouvrage, engage certes sa responsabilité civile à l’égard du MO, mais en tant que gérant il engage également sa responsabilité personnelle s'il n'assure pas sa société \footnote{\jurisCourDeCas{\civTrois}{10/3/2016} : faute détachable de ses fonctions}.

			\item Le notaire engage sa responsabilité civile s'il ne vérifie pas l'existence ou la réalité des assurances obligatoires. Son devoir de conseil doit le conduire à attirer l’attention de l’acquéreur sur l’absence d’assurance, notamment sur les risques encourus en cas d’absence d’assurance. A défaut, le notaire engage sa responsabilité civile.

			Lorsqu’un acte translatif de la propriété ou de la jouissance intervient avant l’expiration du délai de 10 ans, l'\articleDu[L]{243-2}{\ca} impose au notaire de faire mention de l’existence ou de l’absence de l’assurance de responsabilité civile et d'\ado dans le corps de l’acte ou en annexe.

			\item Le vendeur engage sa responsabilité civile s’il n’a pas souscrit de \DO.

			\item Le syndic engage sa responsabilité civile s’il n’a pas proposé au syndicat de copropriétaire la souscription d’une \ado où s’il a proposé des locateurs d’ouvrage non assurés.
		\end{itemize}


	\subsection{L'obligation d'assurer}

		Ceux qui se voient refuser une police peuvent saisir le bureau central de tarification (BCT) qui va déterminer un taux de prime et imposer à une compagnie d'assurance d'assurance d'assurer le \lo.


\section{L'assurance dommages-ouvrage}

	il s'agit d'une assurance de chose, donc applicable en dehors de toute recherche de responsabilité

	L'\articleDu[L]{242-1}{\ca} stipule qu'elle doit être souscrite avant le début du chantier. Mais le législateur n’a prévu aucune sanction. Donc, en l'absence de sinistre, il est toujours possible au \Mo de chercher une police.

	Ici on paie une prime au début ce ne sont pas des échéances mensuelles.


	\subsection{Le domaine de l'assurance dommages-ouvrage}



		\subsubsection{Les personnes assujetties}

			Le principe est posé par l'\articlesDu[L]{242-1}{\ca} : << {\itshape toute personne physique ou morale qui, agissant en qualité de propriétaire de l'ouvrage, de vendeur ou de mandataire du propriétaire de l'ouvrage, fait réaliser des travaux de construction, doit souscrire avant l'ouverture du chantier, pour son compte ou pour celui des propriétaires successifs, une assurance garantissant, en dehors de toute recherche des responsabilités, le paiement de la totalité des travaux de réparation des dommages de la nature de ceux dont sont responsables les constructeurs au sens de l'article 1792-1, les fabricants et importateurs ou le contrôleur technique sur le fondement de l'article 1792 du code civil.} >>

			De sorte que toute personne qui construit ou fait construire un ouvrage --- y compris le castor, le vendeur d’immeuble à construire, le promoteur immobilier, le syndicat de copropriétaires, le copropriétaire --- doit souscrire une \DO\footnote{Le constructeur de maison individuelle ne souscrit pas une \DO, c’est le \MO qui souscrit une \DO. Le constructeur souscrit une assurance décennale.}

			L’assurance peut être souscrite par le \MO ou par son mandataire.

			\subparagraph{Exception} Ne sont pas soumis à l’obligation d’assurance  :
				\begin{itemize}
				\item l'État ;
				\item les personnes morales assurant la maitrise d’ouvrage dans le cadre d’un contrat de partenariat public privé ;
				\item les personnes morales de droit public ou privé remplissant les conditions cumulatives suivantes\footnote{\ArticleDu[L]{242-1}{\ca}} :
					\begin{itemize}
						\item l'activité du \Mo doit être importante, ce qui est le cas lorsque que sont remplies 2 des 3 conditions suivantes :
						\begin{enumerate}
							\item le chiffre d'affaire du dernier exercice doit être supérieur à 12,8 millions d'euros,
							\item le nombre moyen de salarié doit être supérieur à 250,
							\item le total du dernier bilan doit être supérieur à 6,2 millions d'euros ;
						\end{enumerate}
					\item les travaux doivent être réalisé pour le compte du \Mo (donc la \DO demeure si en vue de la vente) ;
					\item l'ouvrage réalisé doit être à usage autre que l'habitation.
					\end{itemize}
				\end{itemize}

		\subsubsection{Les bénéficiaires de la garantie}

			La DO est une assurace de chose. elle est souscrite pour le compte

			La police se transmet en même temps que la cnstruction aux acquéreurs successifs. elle bénéficie donc au propriétaire de l'ouvrage au moment ...

			en cas de vente régularisée entre la et le règlement de l'indemnité. c'est l'acquéreur même sii \jurisCourDeCas{\civTrois}{15/9/2016}. Sauf clause contraire. De surcroit le vendeur qui a supporté le coups des réparations après la vente... en qualité de subrogé ... \jurisCourDeCas{\civUn}{21/2/1995}.

			L'assurance \do ne bénéficie pas au locataire.

		\subsubsection{L'étendue de la garantie}



			\paragraph{Le quantum de la garantie}

				L'assureur doit prefinancer tous les travaux de nature à mettre fin au désordre 7/12/2005. L'\ado couvre donc les couts des travaux de reprise des domamges de nature decenalle affectant les ouvrages, les élement d'éqi techniqu

				elle ne couvre pas les dommages annexes.

				les franchises sont interdites depuis L 242-1 \ca prévoit que << totalité >>

				L 243-9 \ca : peuvent comporter des plafonds de garantie, de sorte qu'il est possible autre que l'habitation. Un décret R 243-3 \ca vient préciser le montant de plafond la garantie ne peut être inférieur au total de la construction déclaré par le \Mo ou si 150 M eris HT, le plafond est à 150 M.

				Seul 3 cas d'exclusion de garantie :
				\begin{enumerate}
					\item fait intentionnel\footnote{Caratérisée lorsque l'assuré à voulu par con comportement le dommage tel qu'il s'est réalisé} ou dol\footnote{Suppose la preuve que l'assuré est su que son comportement est ... (civ 2 1h00 16-23103), mais en pratique la civ 3 ne distingue pas le dol de la faute intentionnelle} L 113-1 du \ca\footnote{Car dans ce cas, il n'y a pas d'aléa} ;
					\item l'effet de l'usure ou de l'usage normal, et du défaut d'entretien ;
					\item la cause étrangère (fait de guerre\etc)
				\end{enumerate}
				Les exclusions de garante sont opposables à la victime. Ce qui peut ... si le constructeur est en faillite.

				\paragraph{Prise en compte de la TVA} Cela dépend \index{TVA} Même solution que décennale. Il appartient d'apporter la preuve qu'il ne récupère pas la TVA s'il demande une indemnité TTC.

			\paragraph{Le point de départ de la garantie}

				\subparagraph{Principe} L'\ado prend normalmeent effet qu'après la gpa, \cad un an après la réception. Elle prend fin à compter de l'expérition d'une priode de 10

				2 excepti 8 aliné L242-1 \ca :
				\begin{enumerate}
					\item avant la réception en cas de résiliation du contrat de la pour inexécution hypo de l'abandon de chantier.
					\begin{itemize}
						\item avant réception
						\item dommage de nature décennale
						\item mise en demeure de l'\E restée infructueuse\footnote{suaf si elle s'avère impossible ou inutile en raison de la cessation d'activité de l\E 23/6/1998, notamment à la suite d'une liquidation judiciaire --- mais reste nécessaire en cas de rdressement judiciaire avec poursuite d'activité} (l'assignation vaut mise en demeure)
						\item contrat doit être résilié (judiciaire ou unilatérale), étant précisé que la liquidation emporte 13/2/2020
					\end{itemize}
					\item après réception, lorsque l'\E n'a pas executé ses obligations malgré une mise en demeure Hypo de l'\E défaillant dans le cadre de la gpa
					\begin{itemize}
						\item une réception
						\item dommage de nature décennale réservé ou apparu dans le délai d'1 an
						\item mise en demeure de l'\E restée infructueuse
					\end{itemize}
				\end{enumerate}

a 243-1 annexe 2 A : obligation de l'assurée

	\subsection{La mise en œuvre de l'assurance dommages-ouvrage et la gestion des sinistres}

		Comme en Amour, tout commence par une déclartion...

		\subsubsection{La déclaration de sinistre}

			L'assuré doit par une déclaration de sinistre, et non pas pas par une assignation. Si avant ou dans les 60 jour, il perd

			2f formes reconnues : \lrar ou remise contre recepissé

			Elle doit impérativement être régularisée dans le délai biennal à compter du sisnistre L114-1 du \ca

			la date du sinstré avant récpetion = resiliation du marché 13/2/202
			après réception, date à laquelle le \Mo a eu connaissance du sinistre

			Le délai de 2 ans n'est pas enfermé dans le délai décennal Si il assigne l'assureur dans le délai mais pos cela ne constitue pas une faute 31/3/2004. En pratique déclarer immédiatement, car c'est un moyen d'éviter que l'assureur puisse reprocher de l'avoir privé du mécanisme de la subrogation (L121-12 du \ca)\footnote{... de sorte que pour illustration :  8/2/2018 17-10010}.

			délai biennale est un délai de prescription. Il peut être interrompu par une des causes ordinaires l'interruption de la prescription\footnote{= forclusion + reconnaissance} :
			\begin{enumerate}
				\item citation en justice
				\item reconventionnelle
				\item reconnaissance de responsabilité
			\end{enumerate}
			Mais
			+ un cas propre à l'assurance L114-2 : \lrar
			+ courrier electronique
			+ ???

			Donc c'est un nouveau délai de même durée qui recommence à courir. En cas d'assignation en référé expertise le délai est suspendu jusqu'à la remise du rapport définitif (L). Si le reliquat est inférieur à 6 mois.

			Le contenu est précisé dans les << clauses types de 2009 >> annexe 2 de A 243-1

			L'assureur dispose d'un délai de 10 jours pour lui notifier une demande de renseignement.

		\subsubsection{La prise de position de l'assureur sur le principe de sa garantie}

			L'assureur dispose d'un délai de 60 jours à compter de la déclaration de sinistre.

			L'assureur désigne un expert, sauf dans 2 cas :
			\begin{itemize}
				\item lorsque la ... est manifestement injustifié
				\item lorsque le cout des travaux de reprise est inférieur à \montantTtc{1 800}
			\end{itemize}
			Il dispose alors d'un délai de 15 jours pour notifier un refus de garantie ou une offre d'indemnité.
			L'assuré peut contester la position de l'assurer pour

			dans les autres cas le recours à l'expertise est obligatoire. L'expert peut faire l'objet d'une récusation ...

			Rapport préliminaire dans lequel il va relater les circonstances ... sur lequel l'assureur prendra position. Il peut le cas échénat préconsier des mesures conservatoires. Il doit être communiqué préalablement à la notification,, ou au plus trad lors de cette notification.

		\subsubsection{La proposition d'indemnisation de l'assureur}

			L'assureur qui a admis la mobilisation de sa garantie doit proposer une indemnité à compter de son acceptation.

			l'expert va rédiger un rapport définitif comportant des proposition ... et estimation concernant ...

			L'assureur doit notifier le rapport définitif ...

			En cas de difficultés exceptionnelles dues à la nature ou à l'importance du sinistre, l'assureur peut proposer à l'assuré la fixation d'un délai supplémentaire pour déterminer. Elle doit êre fondée exclusivement sur des considérations d'ordre technique et être motivée. Le délai sp est subordonné et ne peut excéder 135 jours (en plus des 30)

			la proposition de l'assureur peut revêtir un caractère provisionnel.

		\subsubsection{Le versement de l'indemnité}

			L'assureur est libre d'accepter ou de refuser. Sa réponse n'est pas enfermée dans un délai.

			Si l'as refuse, il peut lui demander de verser les 3/4 de la somme proposée. Dans cette hypo versement dans les 15 jours. L'assuré peut ensuite assigné son assureur pour obtenir un coplément d'indemnité.

			S'il accepte la proposition d'un indemnité. Il doit verser le montant convenu dans les 15 jours.

			... . A défaut d'accord l'assuré peut saisir le juge.

		\subsubsection{L'affectation de l'indemnité versée}

			Le \Mo est obligé d'affecter le montant de l'indemnité à la reprise des désordres civ 3 17/12/2003

			l'assuré s'engage à autoriser l'assureur à constater ... des travx ... A 243-1 annexe 2 du \ca

			L'assureur peut obtenir la restitution des sommes non affectées ou excédant le cout réel des travaux.

	\subsection{Les sanctions de l'assureur en cas de non respect de ses obligations}

		Ces sanctions sont cumulatives et limitatives, de sorte qu ela victime ne pourra pas demander de \di supplémentaires. L242-1

		L'ado doit garantir l'efficacité des trvx prefinancés. A defaut il engage sa responsabilité civile personnelle. C'est-à-dire que si les travaux . La victime pourra alors non pas grâce à la do, mais grâce à la responsabilité de l'ado. Bien évidemment. délais de droit commn (5 à compter de la manifetstaion)

		\subsubsection{Les sanctions en cas de dépassement des délais}

			Ces sanctions de trois ordres :
			\begin{itemize}
				\item \textbf{L'impossibilité de refuser la garantie}, de sorte que l'assureur ne pourra invoquer le défaut de caractère décennal, ni la nullité du contrat  ou le mécanisme de la réduction d'aléa, il ne pourra pas non plus invoquer la prescription biennale sauf si la déclaration intervient plus de deux ans après la fin décennale\footnote{\jurisCourDeCas{\civTrois*}{20/6/2012}} ou si \~25'\jurisCourDeCas{\civTrois*}{20/6/2012}

				L... les délais 60 30/90 15\jurisCourDeCas{\civTrois*}{30/6/2016}

				\item \textbf{La possibilité pour l'assuré d'engager les dépenses nécessaires} après l'avoir notifié à l'assureur \articleDu{L}{242-1 alinéa 5}{Code des assurances}. Viens en des déalis ... De sorte que le \Mo peut commander les travaux de reprise.

				L'assureur peut contester le chiffrage devant les tribunaux.

				\item \textbf{La majoration de l'indemnité versée} \articleDu{L}{242-1 alinéa 5}{Code des assurances}. L'indemnité versée est de plein droit CITE .... cette sanction s'apllique que le \Mo ai ou non les
			\end{itemize}

		\subsubsection{Les sanctions en cas d'offre manifestement insuffisante}

			L242-1 alinéa 5 cCA

			\begin{itemize}
				\item \textbf{La possibilité pour l'assuré d'engager les dépenses nécessaires}
				\item \textbf{La majoration de l'indemnité versée}
			\end{itemize}

		\subsubsection{La sanction en cas de défaut de notification du rapport préliminaire}

			Les textes prévoient que l'ado doit notifier .. le rapport préliminaire à la victime. La jurisprudence à dcouvert : l'impossibilité pour l'assureur de refuser sa garantie

			 \jurisCourDeCas{\civTrois*}{24/9/2013}

	\subsection{Les action récursoires de l'assureur dommages-ouvrage}

		L'ado qui indemnise la victime est subrogé dans ses droits et peut exercer une action récursoire contre les constructeurs responsables et leurs assureurs.

		Dans la mesure où il y a subrogation, il récupère tous les droits et actions, il pourra agir. Quand bien même .. il pourra récupérer auprès des constructeurs --- et peut être --- il pourra agir sur fondement des \garSpec ou sur le fondement contractuel \jurisCourDeCas{\civTrois*}{13/7./2016}

		Il pourra obtenir le remboursement de toutes les sommes versées sauf celles qui lui sont imputable : les pénalités.

		CONVENTION CRAC. elle prévoit également une absence de recours

\section{L'assurance de responsabilité civile décennale}

	Elle est visée au L 242-1 et al du CA

	\subsection{Le domaine de l'assurance de responsabilité civile décennale}



		\subsubsection{Les personnes assujetties}

			Doivent être couverte par une \rcd  toute personne physique ou morale dont la responsabilité
			archi
			entrepreneur
			techniicien
			vendeur apres acheveement (qu'il est contruit ou fait construire)
			promoteur 1931-1
			le vendeur à c (CNR)
			contructeur de
			fabricant d'epers

			sous traitant mais assurance facultative

		\subsubsection{Les activités et les procédés techniques d'exécution déclarés}

			L'assurance ne couvre que les travaux afférant aux secteurs d'activité déclarés.

			lorsqu'il va

			néanmoins la \CourDeCas a jugé que ... les trvx implique carrelage.

			l'assureur ne couvre que les procédés techiques d'exe décalrés. En l'espèce, \civTrois* a considéré que l'entre ... n'est pas couverte

		\subsubsection{L'étendu de la garantie}

			\paragraph{Le quantum de la garantie}

				... couvre le coût de nature décennale affectant l'ouvrage à la réalisation duquel l'assuré a contribué. cela exclu les préjudices annexes\index{préjudices annexes}

				\subparagraph{Franchise}\label{rcFranchise} Il est possible de prévoir des franchises, mais elles sont inopposables à la victime.
				De sorte que la victime sera indemnisé de la totalité de son préjudice, et l'assureur se retournera contre son assuré pour obtenir le remboursement de la franchise.

				\subparagraph{Plafonds} Il est possible de prévoir un plafonds sauf si ... (idem DO renoie), le montant etc. ou à 150M si > 150 M R243-3 CA

				\subparagraph{Exclusion de garantie} : clauses types A 241-1 et 1 243-1 du CA. ces clauses prévoient qu'il n'est pas possible d'exclure du champ de garantie certains type de trvx (contructeur de mi par exemple) ou certains types d'ouvrage (par exmeple les constructions ouvertes). Il n'est pas non plus possible d'exclure certaine techiques de construction. De même une clause qui aurait pour effet d'exclure certains types de sinistre serait réputée non écrite (exemple : limité aux seuls défauts de solidité). c'est également valable pour certains type de contrat, ainsi ... car elle aurait pour effet ...

				Le législateur a voulu assurer la garantie la plus large possible.

				Il n'est pas possible autre que celles L 113-1 CA ... Les exclusions de garanties sont opposables à la victime.

				\subparagraph{La déchéance de garantie}\label{rcDecheance} Elle est prévue par les clause types qui prévoient que << l'assuré est déchu en cas d'inobservation inexcusable des règles de l'art >>. La déchéance n'st pas opposable à la victime.

			\paragraph{L'application de la garantie dans le temps}



				\subparagraph{Le point de départ de la garantie}

					La garantie s'applique aux trvx aynat fait l'objet d'une ouverture de chnatier pendant.
					Il s'agit d'une date unique applicable à l'ensemble des travx de constru.
					Ce sont les clausetiers qui précisent à quoi correpondent.

					Une distinction doit être opéré selon PC ou non. Si PC = DOC, sinon soit date du prmeir OS soit à défaut de la date efective de commencement des trvx.

					en pratique certains locateurs d'ouvrage commencent avant la date d'ouverture, ou postérieurement. Dans ces cas là les clause type précisent que pour l'assuré post la date à prendre en compte est celle où il commence effectivement. Pour l'assuré qui réalise ses prestations antérieurement et qui a cessé sont activité à cette date, date à prendre ne compte est celle de la signature de son marché ou à défaut la date de tout acte pouvant être considérée comme le point de départ de ses prestations.

				\subparagraph{L'expiration de la garantie}

					dernier alinéa

					<< Tout contrat d'assurnce est réputée comporter une clause >>

					de sorte qu'il y a un maintient de la garantie ... sans prime subséquente. La prme est calculée en tenant compte de ce maintient de la garantie.

	\subsection{La mise en œuvre de l'assurance de responsabilité civile décennale}


		\subsubsection{La mise en œuvre de l'assurance de responsabilité civile décennale par le \Mo}

			C'est l'hyp de l'action directe. Action dircete de la victime à l'encontre. le \Mo peut actionner directement sans qu'il soit nécessaire de mettre en œuvre la do; Il peut également sans nécessairement avoir à mettre en cause l'assuré. \jurisCourDeCas{\civTrois*}{7/11/2013}

			\jurisCourDeCas{civ.}{28/3/1939} \jurisCourDeCas{civ. 2\ieme{}}{11/6/2009}

			la \CourDeCas a également admis que la victime puisse au-delà du délai de garantie, tant que l'assureur reste exposé au recours dès lors que valablement actionné. C'est-à-dire ...Le \Mo peut donc agir  2 ans à compter de la réclamation\footnote{\jurisCourDeCas{\civUn*}{11/03/1986} ; \jurisCourDeCas{\civDeux*}{13/9/2017} ; \jurisCourDeCas{\civTrois*}{26/11/2003}} Ainsi lorsque

			\textbf{Attention} : un arrêt sévère du \printdate{29/3/2018} de la \CourDeCas \civTrois* qui a jugé que l'assignation de l'assureur en sa qualité d'ado n'interromp pas ... en sa qualité dc

		\subsubsection{L'action récursoire de l'assureur dommages-ouvrage contre l'assureur de responsabilité civile décennale}

			L'ado est subrogé

			l'action doit impérativement exercé dans le mê délai. Les réglement entre assureur CRAC

		\subsubsection{La mise en œuvre de l'assurance de responsabilité civile décennale par un constructeur}

			\paragraph{La mise en œuvre de l'assurance par le constructeur assuré}

				... peut se retourner contre son propre assureur.

				Soit il attend d'être condamné soit, dans le cadre de la même de sorte que in solidum

				Il doit déclarer le sinsitre dans le délai prévu par les polices , sans que ce délai puisse être inférieur à 5 jours. Néanmoins un déclaration tardive ne peut pas entrainer la déchéance des droits de l'assuré sauf si l'assureur parvient à démontrer que le retard slui a causé un préjudice (ex: l'a privé de ses recours).

				le constructeur doit intenter L114-1 CA : 2 ans à compter (au fonds ou en expertise)

			\paragraph{La mise en œuvre de l'assurance par un constructeur autre que l'assuré}

				Le constructeur qui a indemnisé le \Mo, suite à une condamnation \emph{in solidum}, peut se retourner . Il pourra obtenir le remboursement de tout ou partie en fonction du partage de responsabilité arrêté par le juge.

				Le constructeur devra diviser ses recours

	\subsection{Les actions récursoires de l'assureur de responsabilité civile décennale}

		L'assureur de responsabilité civile peut se retourner contre son assuré :
		\begin{itemize}
			\item en cas de déchéance (cf. \ref{rcDecheance}) ;
			\item en cas de franchise (cf. \ref{rcFranchise}) ;
		\end{itemize}

	1:42 Il peut également ... notamment sur le fondement des vices cachés. il est subrogé dans les droits de son assuré.

		\chapter{Les assurances facultatives}

	\part{Le droit spécial de la construction}

		\chapter{Le contrat de vente d'immeuble à construire}
		\chapter{Le contrat de vente d'immeuble à rénover}
		\chapter{Le contrat de construction de maisons individuelles}

	\part{Annexes}

		% !TEX root = ./droitConstruction.tex

\chapter{Définition}

\section{Responsabilité civile}

  En droit français, une distinction doit être faite entre deux types de responsabilités civiles ne reposant pas sur les mêmes textes de loi : la responsabilité contractuelle et la responsabilité délictuelle. Celles-ci ne peuvent pas être cumulées pour un même dommage.

\section{Responsabilité contractuelle}\index{ResponsabiliteContractuelle@Responsabilité contractuelle}\label{responsabiliteContractuelle}

  La responsabilité contractuelle consiste en la réparation d'un dommage causé par l'inexécution ou le retard dans l'exécution d'un contrat. Les conditions de cette mise en jeu reposent sur les dispositions de l'\articleDu{1231-1}{\cciv}.

  \paragraph{Fondement}
  La responsabilité contractuelle est fixée par plusieurs articles du Code civil :
  \begin{itemize}
    \item L'\articleDu{1103}{\cciv} dispose que « \emph{les contrats légalement formés tiennent lieu de loi à ceux qui les ont faits} ».
    \item L'\articleDu{1104}{\cciv} prévoit que « \emph{les contrats doivent être négociés, formés et exécutés de bonne foi} ».
    \item L'\articleDu{1193}{\cciv} dispose que « \emph{les contrats ne peuvent être modifiés ou révoqués que du consentement mutuel des parties, ou pour les causes que la loi autorise} ».
  \end{itemize}

  Ainsi, une fois les termes du contrat fixé, celui-ci doit être respecté par chacune des parties. L'on notera également que « \emph{les contrats obligent non seulement à ce qui y est exprimé, mais encore à toutes les suites que leur donnent l'équité, l'usage ou la loi}\footnote{\ArticleDu{1194}{\cciv}}. »

  Enfin, les juges du fond ont le pouvoir de rechercher quelle a été « la commune intention des parties » afin de trancher un litige\footnote{\jurisCourDeCas{sect. réunies}{2/2/1808}}. S'agissant de la fin du contrat, elle intervient dès lors que les obligations ont été intégralement exécutées.

  Le contrat prend fin par un accord des parties : elles peuvent défaire ce qu'elles ont fait.

  Il existe également la résiliation pour inexécution qui peut donner lieu à contentieux.

  \paragraph{Mise en œuvre et sanction de la responsabilité contractuelle}

  \subparagraph{Sur la mise en œuvre de la responsabilité contractuelle}

  La mise en œuvre de la responsabilité contractuelle repose sur la démonstration d'une inexécution de la part d'un des deux cocontractants.

  Le contrat risque alors la résolution ou la résiliation\footnote{La résolution est l'anéantissement rétroactif du contrat. La résiliation opère rupture du contrat mais uniquement pour l'avenir.}.

  Une clause prévoyant la résiliation en cas d'inexécution est couramment insérée dans les contrats commerciaux.

  Il y est généralement prévu que la résiliation interviendra aux termes d'une lettre recommandée valant mise en demeure d'avoir à exécuter la ou les obligations contractuelles, demeurée sans effet, suivant un préavis fixé dans la convention.

  En tout état de cause et, dès lors que l'inexécution persiste, le contrat peut être résilié même en l'absence d'une telle clause.

  En effet, l'\articleDu{1224}{\cciv} distingue trois modes de résolution :
  \begin{itemize}
    \item la résolution conventionnelle (clause résolutoire) ;
    \item la résolution par notification (résolution unilatérale aux risques et périls de son auteur) ;
    \item la résolution judiciaire.
  \end{itemize}


  \subparagraph{Sur la réparation de l'inexécution}

  Pour faire valoir ses droits à réparation, celui qui s'estime lésé doit faire valoir :
  \begin{itemize}
    \item un fait fautif ou générateur de responsabilité (l'inexécution contractuelle) ;
    \item un lien de causalité ;
    \item un dommage (ou préjudice) subi.
  \end{itemize}

  La réparation peut intervenir par le biais de l'exécution forcée, de la résolution du contrat, de la diminution de prix, d'une demande de dommages et intérêts\footnote{\articleDu{1217}{\cciv}}. Les sanctions qui ne sont pas incompatibles peuvent être cumulées ; des dommages et intérêts peuvent toujours s'y ajouter.

  La réparation de l'inexécution n'empêche pas l'octroi de dommages et intérêts supplémentaires. Tel est le cas, en effet, selon l'\articleDu{1231-1}{\cciv} : « Le débiteur est condamné, s'il y a lieu, au paiement de dommages et intérêts, soit à raison de l'inexécution de l'obligation, soit à raison du retard dans l'exécution, s'il ne justifie pas que l'exécution a été empêchée par la force majeure. »

\section{Responsabilité délictuelle}

  La responsabilité délictuelle repose sur une obligation générale consistant à devoir réparer le dommage causé à autrui. Par rapport à la responsabilité contractuelle, la responsabilité délictuelle est envisageable « par défaut », c'est-à-dire lorsque le dommage causé ne résulte pas d'une inexécution d'un contrat conclu entre l'auteur et la victime du dommage. L'obligation de réparation est prévue au sein de l'\articleDu{1240}{\cciv}. Cette responsabilité recouvre notamment les dommages causés du fait des choses dont on a la garde ou du fait des personnes dont on répond\footnote{\ArticleDu{1242}{\cciv}}.

		% !TEX root = ./droitConstruction.tex

\chapter{Assurance constructeur non réalisateur}

La loi du 4 janvier 1978, dite « loi Spinetta », instaure des obligations en matière d'assurance construction, aussi bien pour le constructeur professionnel que pour le particulier maître d'ouvrage.

Souvent, le constructeur est assimilé à celui qui conçoit l'ouvrage (architecte, concepteur), ou à celui qui le réalise (entreprise, réalisateur, etc).

Pourtant, la loi en a une acception beaucoup plus large, puisque notre droit de la construction soumet à la responsabilité décennale les intervenants du bâtiment, mais aussi les personnes qui ne construisent pas, mais font construire des ouvrages pour autrui ou en vue de la vente (article 1792-1 du Code civil).

C'est cette dernière catégorie de personnes que désigne l'expression « constructeurs non réalisateurs ». Cette catégorie regroupe des professionnels, mais aussi certains particuliers.

\section{Personnes concernées}

  \subsection{Constructeurs non réalisateurs professionnels}

    Il s'agit principalement :
    \begin{itemize}
      \item du vendeur d'immeubles achevés (par exemple un promoteur), article 1831-1 du Code civil ;
      \item du vendeur d'immeubles à construire (vente en état futur d'achèvement, article 1646-1 du Code civil) ;
      \item du maître d'ouvrage délégué ;
      \item du marchand de biens (qui revend après rénovation ou reconstruction, si l'opération est assimilable à des travaux de construction) ;
      \item du lotisseur-aménageur (notamment pour les ouvrages de viabilité ou VRD).
    \end{itemize}

  \subsection{Vendeurs non professionnels}

    Sont concernés les particuliers qui revendent, dans la période de 10 ans après la réception, un ouvrage qu'ils ont construit ou fait construire.

\section{Constructeur non réalisateur : une obligation d'assurance décennale}

  Les constructeurs non réalisateurs sont tenus de souscrire un contrat garantissant leur responsabilité civile décennale lorsqu'elles font réaliser des ouvrages soumis à obligation d'assurance.

  Cette obligation résulte de l'article L 241-2 du Code des assurances, qui dispose que : « Celui qui fait réaliser pour le compte d'autrui des travaux de construction doit être couvert par une assurance de responsabilité garantissant les dommages visés aux articles 1792 et 1792-2 du Code civil. »

  Dommages garantis
  Ceux-là mêmes qui sont énoncés par les articles 1792 et 1792-2 du Code civil : « Tout constructeur d'un ouvrage est responsable de plein droit envers le maître ou l'acquéreur de l'ouvrage, des dommages, même résultant d'un vice du sol, qui compromettent la solidité de l'ouvrage, ou qui, l'affectant dans un de ses éléments constitutifs ou l'un de ses éléments d'équipement, le rendent impropre à sa destination. ».

  L'article 1792-2 précise les conditions dans lesquelles la présomption de responsabilité s'étend aux dommages affectant la solidité des éléments d'un ouvrage.

  \section{Assurance constructeur non réalisateur : différences avec la garantie « dommages-ouvrage »}

L'assurance dommages ouvrage a pour but de garantir, en dehors de toute recherche de responsabilité, le paiement des travaux de réparation des dommages subis.

Effectivement, quand vous l'avez souscrite avant l'ouverture du chantier, votre contrat profite à l'acquéreur de votre maison, bien que ce ne soit pas lui le signataire.

En effet, l'assurance dommage-ouvrage, comme son nom l'indique est une assurance de « dommages » (par opposition à une assurance de responsabilité), qui est attachée à un bien déterminé désigné au contrat.

Toutefois, du fait de la dualité du principe instauré par la loi Spinetta, qui repose sur la complémentarité de deux systèmes poursuivants des buts différents, on peut avoir le sentiment que ces deux assurances font « doublon ». Il n'en est rien.

En réalité, comme l'illustre parfaitement un arrêt de la Cour d'appel de Montpellier : « La finalité d'une assurance dommages-ouvrage et celle d'une responsabilité civile décennale du constructeur sont fondamentalement différentes » (Montpellier, 12 nov. 2002, Juris-Data n° 2002-245579).

Puis, poursuivant, la Cour précise : « L'assurance responsabilité civile décennale a pour but de garantir la responsabilité de plein droit dont tout constructeur d'un ouvrage est redevable envers le maître ou l'acquéreur de l'ouvrage pour les dommages même résultant d'un vice du sol qui compromettent la solidité de l'ouvrage ou qui l'affectant dans l'un de ses éléments constitutifs ou de ses éléments d'équipement le rendent impropre à sa destination ».

Or, « une assurance dommage ouvrage a pour but de garantir, en dehors de toute recherche de responsabilité, le paiement de la totalité des travaux de réparation des dommages de la nature de ceux dont sont responsables les constructeurs au sens de l'article 1792-1 du Code civil. »

Cette assurance vient donc compléter l'assurance dommages ouvrage. Elle a pour but de garantir le constructeur non réalisateur de son obligation d'assurance décennale conformément à la loi n° 78.12 du 4 janvier 1978 dite « loi Spinetta ».

\section{Assurance constructeur non réalisateur : cas pratique}

Vous faites construire une maison. Au bout de 6 ans, donc avant le terme légal de 10 ans, vous la vendez.

Les garanties de l'assurance dommages ouvrage que vous avez souscrite profitent à l'acquéreur pour les 4 ans restants.

Mais la vente du bien immobilier que vous avez fait construire vous donne automatiquement la qualité de « constructeur non réalisateur », faisant ainsi peser sur vous une responsabilité décennale envers les acquéreurs successifs.

C'est cette responsabilité que la garantie constructeur non réalisateur vient couvrir. Ce risque reste largement méconnu des propriétaires immobiliers qui vendent le bien qu'ils ont fait édifier.

Il ne s'agit pas à proprement parler d'un contrat spécifique, mais d'une option complémentaire à la garantie dommage ouvrage qui doit normalement être souscrite à l'ouverture du chantier.

Attention : la souscription de cette garantie optionnelle est légalement obligatoire, et son absence peut éventuellement paralyser ou retarder la vente d'un bien immobilier, dans la mesure où le notaire est en droit de l'exiger.

Enfin sachez que cette option (dont le coût est modeste au regard de celui de la dommages ouvrage) doit être souscrite à la signature du contrat dommages ouvrage, car il est quasiment impossible d'obtenir cette garantie une fois la construction réalisée.

En ce qui concerne les constructeurs non réalisateurs professionnels, la souscription s'impose bien évidemment.


	\clearpage\pdfbookmark[-1]{Table des matières}{toc}
	\setcounter{tocdepth}{4}
	\tableofcontents

	\clearpage\pdfbookmark[-1]{Index}{index}
	\printindex

\end{document}
