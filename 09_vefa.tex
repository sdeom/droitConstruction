% !TEX root = ./droitConstruction.tex

\chapter{Le contrat de vente d'immeuble à construire}

Le contrat de vente d'immeuble à construire (VIC) est régime institué par la loi \no67-3 du \printdate{3/1/1967} en vue, notamment, de protéger les acquéreurs.

Il est codifiée dans le code civil aux articles 1601-1 à 1601-4, et dans le Code de la construction et de l'habitation aux articles \articleCodifie[L]{261-10} et suivants.

\begin{citationArticleCciv}{1601-1}
	La vente d’immeuble à construire est celle par laquelle le vendeur s’oblige à édifier un immeuble dans un délai déterminé par le contrat.
\end{citationArticleCciv}

\medskip \noindent Trois éléments sont ainsi nécessaires pour caractériser une vente d’immeuble à construire :
\index{VenteImmeubleAConstruire@Vente d'immeuble à construire!Conditions}
\begin{itemize}
	\item la vente du terrain sur lequel doit être construit l’immeuble ;
	\item l’édification d’une construction par le vendeur ;
	\item le respect dans délai d’édification déterminé.
\end{itemize}

\medskip Il existe deux types de contrat de vente d’immeuble à construire : la vente à terme et la vente en l’état futur d’achèvement.

\paragraph{La vente à terme (VAT)}
\begin{citationArticleCciv}{1601-2}
	La vente à terme est le contrat par lequel le vendeur s'engage à livrer l'immeuble à son achèvement, l'acheteur s'engage à en prendre livraison et à en payer le prix à la date de livraison
\end{citationArticleCciv}

\textbf{Le paiement au vendeur se fait uniquement à la fin des travaux.}

Le vendeur doit donc assumer la trésorerie de l’opération (= recours à un prêt).


\paragraph{La vente en l’état futur d’achèvement (VEFA)}
\begin{citationArticleCciv}{1601-3}
	La vente en l’état futur d’achèvement est le contrat par lequel le vendeur transfère immédiatement à l’acquéreur ses droits sur le sol ainsi que la propriété des constructions existantes.

	Les ouvrages à venir deviennent la propriété de l’acquéreur au fur et à mesure de leur exécution ; l’acquéreur est tenu d’en payer le prix à mesure de l’avancement des travaux.

	Le vendeur conserve le pouvoir de maître de l’ouvrage jusqu’à la réception des travaux.
\end{citationArticleCciv}

{\bfseries Le transfert de propriété des ouvrages s'opère au fur et à mesure de leur exécution et le paiement au vendeur se fait au fur et à mesure de l’édification de l’immeuble.
}

\medskip Le législateur a dissocié la qualité de propriétaire (du sol et des ouvrages) et celle de maître d'ouvrage, qui est conservée par le vendeur jusqu'à la livraison des travaux.

Les prérogatives qu'il conserve comportent ainsi le pouvoir de choisir les intervenants à l’acte de construire et de procéder à la réception des travaux.

\medskip Depuis la loi ELAN du \printdate{23/11/2011}\index{ELAN@Loi ELAN}, l’acquéreur peut se réserver l’exécution de certains travaux, après la livraison.

\medskip Ce mécanisme bien plus favorable au vendeur que la VAT de sorte qu’il représente \pourcent{99} des VIC signées.

On ne traitera alors que du contrat de VEFA.

\section{La formation du contrat de vente en l'état futur d'achèvement}

	Le régime différera selon que l’on se trouve dans le secteur libre ou dans le secteur protégé.

	\subsection{La formation du contrat dans les secteurs protégé}

		\subsubsection{La définition du secteur protégé}

			La délimitation du champ d’application du secteur protégé résulte du 1\ier{} alinéa de l'\articleCodifie[L]{261-10} du \cch :

			\begin{citationArticle}[L]{261-10}{\cch}
				Tout contrat ayant pour objet le transfert de propriété d'un immeuble ou d'une partie d'immeuble à usage d'habitation ou à usage professionnel et d'habitation et comportant l'obligation pour l'acheteur d'effectuer des versements ou des dépôts de fonds avant l'achèvement de la construction doit, à peine de nullité, revêtir la forme de l'un des contrats prévus aux articles 1601-2 et 1601-3 du code civil, reproduits aux articles L. 261-2 et L. 261-3 du présent code. Il doit, en outre, être conforme aux dispositions des articles L. 261-11 à L. 261-14 ci-dessous.
			\end{citationArticle}

			Ainsi relèvent du secteur protégé les ventes qui :
\index{SecteurProtege@Secteur protégé!Conditions}
			\begin{itemize}
				\item portent sur des locaux à usage d'habitation ou mixtes ;
				\item et qui sont assorties de l'obligation d'effectuer des versements avant l'achèvement de la construction.
			\end{itemize}

			Ces deux conditions sont cumulatives.


			\paragraph{L’immeuble doit être à usage d’habitation ou mixte, c'est-à-dire professionnel et d’habitation.}

			Le législateur ne fait pas de différence entre les logements.

			Ainsi peu importe :
			\begin{itemize}
				\item qu’il s’agisse d’une résidence principale ou secondaire ;
				\item que les locaux soient destinés à être loués, ou à être occupé à titre gratuit, voire rester vide dans l’attente d’une occupation quelconque.
			\end{itemize}

			L'usage doit être apprécié par rapport à la « partie d'immeuble » objet du contrat de vente : le lot vendu doit être à usage d'habitation ou à usage mixte professionnel et d'habitation.

			La vente d'un lot à destination de commerce dans un immeuble dont tous les autres lots sont affectés à l'habitation échappe aux dispositions du secteur protégé.

			En revanche, la vente d'un local d'habitation dans un immeuble dont le surplus est commercial est susceptible d'entrer dans le cadre du secteur protégé (à moins qu'il puisse apparaître comme accessoire de locaux commerciaux).

			Echappent au secteur protégé les contrats qui ont pour objet la vente d’un lot :
			\begin{itemize}
				\item en totalité à usage commercial ;
				\item en totalité à usage professionnel ;
				\item à usage mixte d’habitation et commercial.
			\end{itemize}


			\paragraph{L’acquéreur doit effectuer des versements ou des dépôts avant l'achèvement.}

			Il s’agit de la seconde condition posée par l'\articleDu[L]{261-10}{\cch} pour que s'applique le régime du secteur protégé.

			Sont visés tous versements au profit du vendeur, sous quelque forme que ce soit.

			Peu importe la qualification de ces versements.

			\bigskip \textbf{Précision} : dès lors que ces 2 conditions sont réunies, l’acte translatif de propriété doit nécessairement revêtir la forme d’une vente d’immeuble à construire (VEFA ou VAT).

			A défaut, l’acte encours la nullité\footnote{Il s’agit d’une nullité relative (\jurisCourDeCas[16-22095]{\civTrois*}{4/10/2018})}.


		\subsubsection{La conclusion du contrat préliminaire}

			La conclusion d’un contrat préliminaire n’est pas obligatoire quelque soit le secteur (libre ou protégé).

			L’\articleDu[L]{261-15}{\cch} dispose en effet que le contrat de VEFA « peut » être précédé d’un avant-contrat.

			Toutefois, lorsque l’acquéreur souhaite réaliser une partie des travaux, le contrat préliminaire de réservation devient obligatoire --- la loi ELAN a en effet imposé que le contrat définitif précise « la description des travaux dont l’acquéreur se réserve l’exécution lorsque la vente est précédée d’un contrat préliminaire comportant la clause prévue au II de l’article L 261-15 $\dots$ »\footnote{\ArticleDu[L]{261-11}{\cch} modifié}.

			Dans les autres cas, la signature d’un contrat préliminaire est facultative.

			Cependant, dans le secteur protégé, si les parties conviennent de régulariser un avant-contrat, il ne peut alors s’agir que d’un contrat préliminaire répondant aux exigences de l’\articleDu[L]{261-15}{\cch}.

			\begin{citationArticle}[L]{261-15}{\cch}
				La vente prévue à l'article L. 261-10 peut être précédée d'un contrat préliminaire par lequel, en contrepartie d'un dépôt de garantie effectué à un compte spécial, le vendeur s'engage à réserver à un acheteur un immeuble ou une partie d'immeuble.

				Ce contrat doit comporter les indications essentielles relatives à la consistance de l'immeuble, à la qualité de la construction et aux délais d'exécution des travaux ainsi qu'à la consistance, à la situation et au prix du local réservé.

				\lips

				\textbf{Est nulle toute autre promesse d'achat ou de vente}.
			\end{citationArticle}


			\paragraph{Les règles de forme}

				Le contrat préliminaire est un contrat \emph{sui generis} qui doit répondre à des conditions de formes posées aux \articlesDuEtSuivants[R]{261-25}{\cch}.

				Ces règles tendent à assurer la protection de chacune des parties.

				\subparagraph{L'exigence d'un écrit}

					Il s’agit d’une condition de validité du contrat. Le contrat doit être \textbf{formalisé par écrit}.
					Le principe est posé par l’\articleDu[R]{261-27}{\cch} aux termes duquel :

					\begin{quote}
						« {\itshape Le contrat préliminaire est établi par écrit ; un exemplaire doit être remis au réservataire avant tout dépôt de fonds. \lips} ».
					\end{quote}

					Il peut s’agir d’un acte sous seing privé ou d’un acte authentique.


				\subparagraph{L'exigence de mentions impératives}

					Le contrat préliminaire doit contenir certaines \textbf{mentions impératives, dont l’absence est sanctionnée par la nullité du contrat}\footnote{Combinaison des \articlesCodifies[L]{ 261-15} et \refArticle[R]{261-25} et suivants du \cch}.

					Il s’agit d’indications techniques ou financières relatives :
					\begin{itemize}
						\item à la construction;
						\item au prix et à ses modalités de paiement ;
						\item au dépôt de garantie ;
						-\item à la date à laquelle la vente pourra être conclue.
					\end{itemize}

					L’objectif est de renseigner et éclairer l’acquéreur.
					\bigskip \begin{enumerate}

						\item \textbf{Mentions relatives à la construction}

							Aux termes de l’\articleDu[L]{261-15}{\cch}, le contrat préliminaire « {\itshape doit comporter les indications essentielles relatives à la consistance de l'immeuble, à la qualité de la construction et aux délais d'exécution des travaux ainsi qu'à la consistance et à la situation \lips du local réservé} ».

							Lorsque le réservataire souhaite réaliser lui-même des travaux, le contrat préliminaire doit comporter certaines mentions supplémentaires\footnote{\ArticleDu[L]{261-15}{\cch}} :

							\begin{itemize}
								\item \textbf{La consistance de l’immeuble.}

								\begin{citationArticle}[R]{261-25}{\cch}
									Le contrat préliminaire doit indiquer
									\begin{itemize}
										\item la surface habitable approximative de l'immeuble faisant l'objet de ce contrat,
										\item le nombre de pièces principales et l'énumération des pièces de service, dépendances et dégagements.
										\item S'il s'agit d'une partie d'immeuble, le contrat doit en outre préciser la situation de cette partie dans l'immeuble.
									\end{itemize}
								\end{citationArticle}

								Le vendeur doit donc indiquer dans le contrat préliminaire :
								\begin{itemize}
									\item le numéro du bâtiment, éventuellement de la cave, du parking, du garage, \etc ;
									\item  le type d’appartement ;
									\item le nombre et la répartition des pièces ;
									\item l’étage ;
									\item le numéro du lot de copropriété ;
									\item la surface habitable ;
									\item  \etc
								\end{itemize}


							\item \textbf{La qualité de la construction.}

								\begin{citationArticle}[R]{261-25}{\cch}
								\lips	la qualité de la construction est suffisamment établie par une \textbf{note technique sommaire indiquant la nature et la qualité des matériaux et des éléments d'équipement}.

								Si le contrat porte sur une partie d'immeuble, cette note technique doit contenir également l'indication des équipements collectifs qui présentent une utilité pour la partie d'immeuble vendue.

								Cette note technique doit être \textbf{annexée au contrat}
								\end{citationArticle}

								Aux termes de l'\articleDu[R]{261-25}{\cch}, il est donc nécessaire de joindre une notice descriptive sommaire.

							\item \textbf{Les délais d'exécution des travaux.}

								Le vendeur doit insérer dans le contrat préliminaire une clause relative aux délais d’exécution des travaux.

								Il n'est pas nécessaire d'indiquer une date d'achèvement ; une durée prévisionnelle des travaux suffi (on peut prévoir que l’achèvement de l’immeuble interviendra environ (X) mois après le début des travaux).

							\item \textbf{Les travaux réservés.}

								Lorsque le réservataire se réserve la réalisation de travaux, le contrat préliminaire doit comporter une clause en caractères très apparents stipulant que l'acquéreur accepte la charge, le coût et les responsabilités qui résultent de ces travaux.

								Les travaux ne pourront être réalisés qu’après la livraison du bien\footnote{\articleDu[L]{261-15}{\cch}}.

								La nature des travaux dont l'acquéreur peut se réserver l'exécution est précisée par un décret du 25 juin 2019\footnote{\ArticleDu[R]{261-13-1}{\cch}}.

								Il s’agit des travaux de finition des murs intérieurs, de revêtement ou d'installation d'équipements de chauffage ou sanitaires, et le cas échéant, du mobilier pouvant les accueillir.

								Un arrêté du ministre chargé du logement fixe la liste limitative des travaux concernés et détermine leurs caractéristiques\footnote{Arrêté du 28 octobre 2019} :
								\begin{quote}

								« {\itshape \textbf{Article 1 :}

								\medskip La liste limitative des travaux réservés par l'acquéreur mentionnée à l'article R. 261-13-1 du code de la construction et de l'habitation est la suivante :

								1° L'installation des équipements sanitaires de la cuisine et, le cas échéant, du mobilier pouvant les accueillir ;

								2° L'installation des équipements sanitaires de la salle de bains ou de la salle d'eau et, le cas échéant, du mobilier pouvant les accueillir ;

								3° L'installation des équipements sanitaires du cabinet d'aisance ;

								4° La pose de carrelage mural ;

								5° Le revêtement du sol à l'exclusion de l'isolation ;

								6° L'équipement en convecteurs électriques, lorsque les caractéristiques de l'installation électrique le permettent et dans le respect de la puissance requise ;

								7° La décoration des murs.

								\medskip Sont exclus les travaux relatifs aux installations mentionnées au a de l'article R.111-3 \emph{[installation d'alimentation en eau potable et d'une installation d'évacuation des eaux usées]}.

								\bigskip \textbf{Article 2 :}

								\medskip Ces travaux respectent les caractéristiques suivantes :

								-	Ils sont sans incidence sur les éléments de structure ;

								-	Ils ne nécessitent pas d'intervention sur les chutes d'eau, sur les alimentations en fluide et sur les réseaux aérauliques situés à l'intérieur des gaines techniques appartenant aux parties communes du bâtiment ;

								-	Ils n'intègrent pas de modifications sur les canalisations d'alimentation en eau, d'évacuation d'eau et d'alimentation de gaz nécessitant une intervention sur les éléments de structure ;

								-	Ils ne portent pas sur les entrées d'air ;

								-	Ils ne conduisent pas à la modification ou au déplacement du tableau électrique du logement.} »

								\end{quote}
						\end{itemize}


						Enfin, afin d’éclairer l’acquéreur, il est recommandé d’apporter des précisions sur les responsabilités encourues et les éventuelles assurances à souscrire (DO et RCD).


						\item \textbf{Mentions relatives au prix et à ses modalités de paiement.}

						L’\articleDu[R]{261-26}{\cch} , impose l’insertion dans le contrat préliminaire de stipulations relatives au prix et à ses modalités de paiement.


						\begin{itemize}
							\item L’avant contrat doit indiquer le prix auquel sera proposée la vente de l'immeuble

								\medskip Le prix stipulé peut n’être que prévisionnel.
								Il peut évoluer :
								\begin{itemize}
									\item à l’initiative du réservant, en raison de contraintes budgétaires,

									Exemple : modifications apportées, volontairement ou non, à l'opération de construction

									\item à l’initiative de l’acquéreur, en cas de demande de travaux supplémentaires (amélioration des prestations contractuelles, demande de prestations complémentaires, \etc).
								\end{itemize}

								Ces modifications passent par la conclusion d’avenant au contrat préliminaire.

								\medskip \textbf{Attention} : lorsque le réservataire se réserve la réalisation de travaux, le contrat préliminaire doit préciser (articles L 261-15 et R 261-26 CCH) :
								\begin{itemize}
									\item Le prix du local réservé décomposé comme suit :
									\begin{enumerate}[label=\alph*)]
										\item le prix du local réservé,
										\item le coût des travaux dont l'acquéreur se réserve l'exécution ; ceux-ci doivent être décrits et chiffrés par le vendeur,
										\item Le coût total de l'immeuble égal à la somme du prix convenu et du coût des travaux mentionnés aux a et b.
									\end{enumerate}

									\item Le délai dans lequel l'acquéreur peut revenir sur sa décision de se réserver l'exécution des travaux.
								\end{itemize}

								\medskip \textbf{Précisions} : Si l'acquéreur renonce à exécuter lui-même les travaux, le vendeur est tenu de les exécuter ou de les faire exécuter.

								L’acquéreur doit notifier sa décision au vendeur par lettre recommandée avec demande d'avis de réception ou par lettre recommandée électronique dans le délai stipulé au contrat préliminaire\footnote{\ArticleDu[R]{261-13-2}{\cch}}.


							\item Le contrat préliminaire doit indiquer, le cas échéant, les modalités de la révision du prix dans les limites et conditions prévues aux \articlesCodifies[L]{261-11-1} et \refArticle[R]{261-15} du \cch\footnote{\ArticleDu[R]{261-26}{\cch}}.

								Cf. infra


							\item Le cas échéant, l’avant contrat doit contenir « l’indication des prêts que le réservant déclare qu'il fera obtenir au réservataire ou dont il lui transmettra le bénéfice » ainsi que le ou les prêt auxquels l’acquéreur entend avoir recours.

								Dans cette hypothèse, le contrat préliminaire est passé sous condition suspensive de l’obtention des prêts.
						\end{itemize}


						\item \textbf{Mentions relatives au dépôt de garantie.}

						Le contrat préliminaire doit obligatoirement reproduire les dispositions des articles R. 261-28 à R.  261-31 », relatifs au dépôt de garantie (article R. 261-27 du CCH, in fine).

						L’objectif est ici encore d’assurer une parfaite information des réservataires.

						Sanction : nullité du contrat préliminaire.

						Contenu des articles :

						\begin{citationArticle}[R]{261-28}{\cch}
							Le montant du dépôt de garantie ne peut excéder \pourcent{5} du prix prévisionnel de vente si le délai de réalisation de la vente n'excède pas un an ; ce pourcentage est limité à \pourcent{2} si ce délai n'excède pas deux ans. Aucun dépôt ne peut être exigé si ce délai excède deux ans
						\end{citationArticle}

						\begin{citationArticle}[R]{261-29}{\cch}
							Le dépôt de garantie est fait à un compte spécial ouvert au nom du réservataire dans une banque ou un établissement spécialement habilité à cet effet ou chez un notaire. Les dépôts des réservataires des différents locaux composant un même immeuble ou un même ensemble immobilier peuvent être groupés dans un compte unique spécial comportant une rubrique par réservataire
						\end{citationArticle}

						\begin{citationArticle}[R]{261-30}{\cch}
							Le réservant doit notifier au réservataire le projet d'acte de vente un mois au moins avant la date de la signature de cet acte
						\end{citationArticle}

						\begin{citationArticle}[R]{261-31}{\cch}
							Le dépôt de garantie est restitué, sans retenue ni pénalité au réservataire :
						\begin{enumerate}[label = \alph*)]
							\item Si le contrat de vente n'est pas conclu du fait du vendeur dans le délai prévu au contrat préliminaire ;
							\item Si le prix de vente excède de plus de \pourcent{5} le prix prévisionnel, révisé le cas échéant conformément aux dispositions du contrat préliminaire. Il en est ainsi quelles que soient les autres causes de l'augmentation du prix, même si elles sont dues à une augmentation de la consistance de l'immeuble ou à une amélioration de sa qualité ;
							\item Si le ou les prêts prévus au contrat préliminaire ne sont pas obtenus ou transmis ou si leur montant est inférieur de \pourcent{10} aux prévisions dudit contrat ;
							\item Si l'un des éléments d'équipement prévus au contrat préliminaire ne doit pas être réalisé ;
							\item Si l'immeuble ou la partie d'immeuble ayant fait l'objet du contrat présente dans sa consistance ou dans la qualité des ouvrages prévus une réduction de valeur supérieure à \pourcent{10}.
						\end{enumerate}

						Dans les cas prévus au présent article, le réservataire notifie sa demande de remboursement au vendeur et au dépositaire par lettre recommandée avec demande d'avis de réception.

						Sous réserve de la justification par le déposant de son droit à restitution, le remboursement intervient dans le délai maximum de trois mois à dater de cette demande
						\end{citationArticle}


						\item \textbf{Mention relative à la date à laquelle la vente pourra être conclue\footnote{\ArticleDu[R]{261-26}{\cch}}.}

						Le contrat préliminaire doit indiquer la date envisagée pour la conclusion du contrat définitif.

						La jurisprudence admet traditionnellement la possibilité de prévoir un délai et non pas nécessairement une date précise.
					\end{enumerate}

				\textbf{Précision} : la nullité du contrat de réservation --– quelqu’en soit a cause --- est sans incidence sur la validité de l’acte de vente\footnote{ \jurisCourDeCas[18-11707 ]{\civTrois*}{21/3/2019} : le contrat de réservation étant facultatif, l’acquéreur ne saurait obtenir l’annulation des actes de vente et de prêt sur le seul fondement de la nullité du contrat préliminaire}.

			\paragraph{Les règles de fonds}

				Elles résident dans les deux obligations majeures souscrites par les parties.

				\subparagraph{L'obligation du vendeur : la réservation du bien}

					Aux termes de l’\articleDu[L]{261-1}{\cch}, le vendeur s’engage à réserver à l’acheteur « un immeuble ou une partie de l’immeuble ».

					Il s’engage simplement à offrir une priorité d’achat des locaux à l’acquéreur.
					Il ne s’engage nullement à conclure le contrat définitif.

					\medskip Cette obligation est double :
					\begin{itemize}
						\item Le réservant qui entend poursuivre son opération doit proposer un contrat définitif au réservataire.

							A défaut il est dans l’obligation de restituer le dépôt de garantie versé par le réservataire au moment de la signature de l’avant-contrat.

							Il peut également être tenu à des dommages et intérêts s’il a commis une faute.

							Exemple : refus injustifié de proposer un contrat définitif au réservataire (\jurisCourDeCas{\civTrois*}{11/6/1987})  :
							\begin{quote}
								« {\itshape Le refus injustifié de la part du réservant d'offrir à son cocontractant la vente des biens réservés aux conditions stipulées ne peut donner lieu qu'à l'allocation de dommages-intérêts au réservataire en plus de la restitution du dépôt de garantie} ».
							\end{quote}

							La Cour de cassation refuse en revanche de prononcer la réalisation forcée de la vente des biens et droits immobiliers objet du contrat de réservation.

							L’importance du préjudice subi est souverainement appréciée par les juges du fond.


						\item Le réservant doit proposer à l’acquéreur un contrat définitif conforme au contrat préliminaire.

						Les caractéristiques de la construction précisées dans le contrat préliminaire doivent, en principe, être reprises dans le contrat définitif.

						Le réservant peut toutefois modifier ou adapter son opération.

						Dans ce cas :
						\begin{itemize}
							\item le projet d'acte de vente doit alors faire ressortir les modifications par rapport au contrat préliminaire, de façon à ce que le réservataire en soit parfaitement informé et se décide en conséquence\footnote{\jurisCourDeCas{\civTrois*}{20/1/1981}}.

							\item le réservataire peut refuser de conclure le contrat définitif si les changements font apparaître une différence anormale par rapport aux prévisions du contrat préliminaire.
						\end{itemize}

						Dans cette hypothèse, le réservataire pourra obtenir la restitution du dépôt de garantie et des dommages et intérêts s’il n’a pas été prévenu suffisamment à l’avance des modifications envisagées.

						Si, malgré les changements, le réservataire signe le contrat définitif, il ne pourra obtenir des dommages et intérêts que si, au moment de sa signature, il a émis des réserves sur les modifications intervenues.
					\end{itemize}

				\subparagraph{L'obligation de l'acquéreur : le versement d'un dépôt de garantie}

					\begin{enumerate}
						\item \textbf{La seule obligation que contracte l’acquéreur au stade du contrat préliminaire est de verser un dépôt de garantie}

						Le dépôt de garantie constitue la contrepartie de la réservation\footnote{\ArticleDu[L]{261-15}{\cch}}.

						L’acquéreur ne s’engage nullement à signer le contrat définitif.

						Cependant, il s’expose à la perte du dépôt de garantie en cas de refus non justifié de signer le contrat définitif, c'est-à-dire en l’absence de différence anormale par rapport au contrat préliminaire.


					\item\textbf{Le montant du dépôt varie en fonction du délai prévu pour la réalisation de la vente.}

						\begin{citationArticle}[R]{261-28}{\cch}
							Le montant du dépôt de garantie ne peut excéder :
						\begin{itemize}
							\item \pourcent{5} du prix prévisionnel de vente si le délai de réalisation de la vente n'excède pas un an
							\item ou bien \pourcent{2} si ce délai n'excède pas deux ans.
						\end{itemize}

						Enfin aucun dépôt ne peut être exigé si ce délai excède deux ans
						\end{citationArticle}

						Le dépôt de garantie doit être déposé sur un compte spécial ouvert au nom du réservataire dans une banque ou un établissement spécialement habilité à cet effet ou chez un notaire\footnote{\ArticleDu[R]{261-29}{\cch}}.

						Il ne peut en aucun cas être fait entre les mains du réservant.

						L’\articleDu[L]{261-15}{\cch} précise que les fonds déposés sont « {\itshape indisponibles, incessibles et insaisissables jusqu'à la conclusion du contrat de vente} ».

						Le législateur a cependant admis, par mesure de simplification, que « {les dépôts des réservataires des différents locaux composant un même immeuble ou un même ensemble immobilier peuvent être groupés dans un compte unique spécial comportant une rubrique par réservataire} ».

					\item \textbf{Sort du dépôt de garantie.}

						\begin{itemize}
							\item Si l’acquéreur signe le contrat définitif, le dépôt de garantie est acquis au réservant.

							\item Si les parties ne donnent pas de suite au contrat préliminaire, le sort du dépôt de garantie dépendra du motif du refus.

								Le dépôt de garantie doit être restitué dans son intégralité au déposant\footnote{\ArticleDu[L]{261-15}{\cch}} :
								\begin{itemize}
									\item si le contrat n'est pas conclu du fait du vendeur ;
									\item si la condition suspensive d’obtention de prêt n'est pas réalisée ;
									\item si le contrat proposé fait apparaître une différence anormale par rapport aux prévisions du contrat préliminaire.
								\end{itemize}

								Ces hypothèses sont explicitées par l’\articleDu[R]{261-31}{\cch} :
								\begin{quote}
									« {\itshape Le dépôt de garantie est restitué, sans retenue ni pénalité au réservataire :
									\begin{enumerate}[label=\alph*)]
										\item Si le contrat de vente n'est pas conclu du fait du vendeur dans le délai prévu au contrat préliminaire ;

										\item Si le prix de vente excède de plus de 5 p. 100 le prix prévisionnel, révisé le cas échéant conformément aux dispositions du contrat préliminaire. Il en est ainsi quelles que soient les autres causes de l'augmentation du prix, même si elles sont dues à une augmentation de la consistance de l'immeuble ou à une amélioration de sa qualité ;

										\item Si le ou les prêts prévus au contrat préliminaire ne sont pas obtenus ou transmis ou si leur montant est inférieur de 10 p. 100 aux prévisions dudit contrat ;

										\item Si l'un des éléments d'équipement prévus au contrat préliminaire ne doit pas être réalisé ;

										\item Si l'immeuble ou la partie d'immeuble ayant fait l'objet du contrat présente dans sa consistance ou dans la qualité des ouvrages prévus une réduction de valeur supérieure à 10 p. 100 >>
									\end{enumerate}}
								\end{quote}

								Cet article précise, par ailleurs, les modalités de la restitution du dépôt de garantie :
								\begin{itemize}
									\item le réservataire doit notifier sa demande de remboursement au vendeur et au dépositaire par lettre recommandée avec demande d'avis de réception ;
									\item le remboursement doit ensuite intervenir dans un délai maximum de trois mois à compter de cette demande, sous réserve de la justification par le déposant de son droit à restitution
								\end{itemize}
						\end{itemize}
					\end{enumerate}

			\paragraph{La purge du droit de rétractation}

				L’\articleDu[L]{271-1}\cch{} ménage au profit de l’acquéreur en état futur d’achèvement un droit de rétractation :
				\begin{quote}
					« {\itshape \textbf{Pour tout acte ayant pour objet la construction ou l'acquisition d'un immeuble à usage d'habitation}, la souscription de parts donnant vocation à l'attribution en jouissance ou en propriété d'immeubles d'habitation ou la vente d'immeubles à construire ou de location-accession à la propriété immobilière, \textbf{l'acquéreur non professionnel peut se rétracter dans un délai de dix jours à compter du lendemain de la première présentation de la lettre lui notifiant l'acte}.

					\medskip\textbf{Cet acte est notifié à l'acquéreur par lettre recommandée avec demande d'avis de réception} ou par tout autre moyen présentant des garanties équivalentes pour la détermination de la date de réception ou de remise.

					\medskip\textbf{La faculté de rétractation est exercée dans ces mêmes formes. }

					\medskip Lorsque l'acte est conclu par l'intermédiaire d'un professionnel ayant reçu mandat pour prêter son concours à la vente, cet acte peut être remis directement au bénéficiaire du droit de rétractation. Dans ce cas, le délai de rétractation court à compter du lendemain de la remise de l'acte, qui doit être attestée selon des modalités fixées par décret.

					\medskip \textbf{Lorsque le contrat constatant ou réalisant la convention est précédé d'un contrat préliminaire} ou d'une promesse synallagmatique ou unilatérale, \textbf{les dispositions figurant aux trois alinéas précédents ne s'appliquent qu'à ce contrat} ou à cette promesse.
}

					\lips ».
				\end{quote}

				Le vendeur professionnel est obligé d’informer l’acquéreur de son droit de rétractation et de le purger.

				\bigskip Conditions de purge du droit de rétractation :
				\begin{itemize}
					\item En cas de régularisation d’un contrat préliminaire, la purge du droit de rétractation s’effectue au stade de l’avant contrat.

					\item Seuls les acquéreurs non professionnels bénéficient du droit de rétractation.

						« L'acquéreur non professionnel » est celui qui conclut un contrat en dehors de son activité professionnelle\footnote{Position de la jurisprudence}.

						Quid des personnes morales ?
						\begin{quote}
							Une SCI dont l'objet social est l'acquisition, l'administration et la gestion par location ou autrement de tous immeubles et biens immobiliers meublés et aménagés, n’est pas un acquéreur non professionnel dès lors que l'acte d’acquisition a un rapport direct avec cet objet social\footnote{ \jurisCourDeCas[11-18774]{\civTrois*}{24/10/2012} ; \jurisCourDeCas[13-20002]{\civTrois*}{16/09/2014} pour une SCI familiale}.
						\end{quote}

					\item L’avant contrat doit être notifié par lettre recommandée ou équivalent à l’acquéreur.

					Attention : La notification n’est régulière que si la lettre est remise à son destinataire ou à un représentant muni d’un pouvoir à cet effet.

					A défaut, le délai de rétractation ne commence pas à courir\footnote{\jurisCourDeCas[18-10772]{\civTrois*}{21/3/2019} pour un recommandé signé par le conjoint}.

					\item La rétractation doit être exercée dans les mêmes formes que la notification, dans le délai de 10 jours à compter du lendemain de la première présentation de la notification de l’acte.
				\end{itemize}


		\subsubsection{La conclusion du contrat définitif}


			Le contrat définitif dans le secteur protégé est soumis à un ensemble de règles impératives qui résulte de la combinaison des \articlesCodifies[L]{261-11} à \refArticle[L]{261-14}, \refArticle[R]{261-11} à \refArticle[R]{261-16} et \refArticle[R]{261-30} du \cch.

			\paragraph{L'exigence d'un acte authentique}

				L'\articleDu[L]{261-11}{\cch}, en son 1\ier{} alinéa prévoit que « {\itshape le contrat doit être conclu par acte authentique} ».

				Les juges n’hésitent pas à prononcer la nullité du contrat en cas d'inobservation de cette règle\footnote{\jurisCA{Versailles}{23/10/1992}, Korsakoff c/ Crédit Lyonnais : Juris-Data \no006185}.

				L'\articleDu[R]{261-30}{\cch} impose que le projet d’acte authentique soit notifié à l'acquéreur, un mois au moins avant la signature de cet acte.
				L’objectif est ici de permettre au réservataire de comparer les dispositions du contrat définitif avec celles de l’avant-contrat.

				À défaut, l'acquéreur peut légitimement refuser de signer la vente définitive et obtenir le remboursement de son dépôt de garantie (mais également, le cas échéant, des dommages et intérêts).

				L'acquéreur n'est toutefois pas tenu par le délai de l'\articleDu[R]{261-30} {\cch} de sorte qu’il peut décider de signer la vente avant son expiration.

			\paragraph{Le contenu de l'acte authentique}

				Le contenu de l'acte authentique est précisé par les dispositions de l’\articleDu[L]{261-11}{\cch}.

				\begin{enumerate}
					\item Aux termes de l'\articleDu[L]{261-11, alinéa 1\ier{} a)}{\cch}, l'acte authentique de vente doit préciser « \emph{l\textbf{a description de l'immeuble ou de la partie d'immeuble vendu}} ».

					\medbreak Il doit, par ailleurs, « \emph{comporter en annexe, ou par référence à des documents déposés chez un notaire, les indications utiles relatives à la consistance et aux caractéristiques techniques de l'immeuble} ».

					A ce stade la description de l’immeuble donnée par le vendeur est définitive et non plus prévisionnelle comme celle contenue dans le contrat préliminaire.

					Elle lie le vendeur qui s’oblige à réaliser les travaux de construction conformément aux indications fournies.

					Le vendeur peut néanmoins se ménager, dans le contrat, une certaine marge de manœuvre en prévoyant qu’il fournira « \emph{les matériaux ou éléments d’équipement de telle marque ou similaire} ».

					Cette précision permet au vendeur de substituer un matériau ou un élément d’équipement similaire à celui initialement prévu.

					La faculté de substitution est cependant encadrée.

					Dans une réponse ministérielle \no21990 du \printdate{12/11/1992}, le ministère de l’équipement a, en effet, précisé que :

					\begin{quote}
						« Sous réserve de l'appréciation souveraine des tribunaux, \textbf{la faculté} que se réserve le vendeur ou le constructeur de modifier l'un des éléments du contrat ne \textbf{doit être utilisée qu'en cas de nécessité absolue, voire de force majeure} justifiant l'impossibilité d'approvisionnement et le produit utilisé doit strictement être équivalent en qualité et en valeur à celui qui était prévu.

						A défaut, l'acquéreur dispose d'une action en résolution du contrat ou en minoration du prix en application des règles du droit commun, notamment l'article 1304-2 du Code civil \emph{[nullité des clauses potestatives]}, sans préjudice de dommages et intérêts s'il y a lieu.

						Il est à souligner que, pour éviter des difficultés en cette matière, les contractants ont intérêt, d'une part, à préciser dans l'acte d'origine les cas où la clause peut être mise en jeu et, d'autre part, à prévoir la signature d'un avenant au contrat en cas de nécessité de remplacement d'une prestation prévue par une autre équivalente ».
					\end{quote}

					La description de l’immeuble au stade du contrat définitif doit être plus complète que celle contenue dans le contrat préliminaire.

					Elle doit comprendre\footnote{\articleDu[R]{261-13}{\cch}} :
					\begin{itemize}
						\item des plans de l'immeuble vendu, coupes et élévations avec les cotes utiles et l'indication des surfaces de chacune des pièces et des dégagements (1\ier{} alinéa) ;

						\item si l’immeuble est compris dans un ensemble immobilier, un plan faisant apparaître le nombre de bâtiments de cet ensemble, leur emplacement et le nombre d'étages de chacun d'eux (2\ieme{} alinéa) ;

						\item une notice descriptive précisant les caractéristiques techniques du bien --- la notice doit être conforme à un modèle type agréé par arrêté ministériel\footnote{Arrêté du 10 mai 1968, Journal Officiel 29 Juin 1968} ---  les 4\ieme{}et 5\ieme{} alinéas précisant que :

						« {\itshape Ces documents s'appliquent au local vendu, à la partie de bâtiment ou au bâtiment dans lequel il se trouve et aux équipements extérieurs et réseaux divers qui s'y rapportent.

						Un plan coté du local vendu et une notice indiquant les éléments d'équipement propres à ce local doivent être annexés au contrat de vente }».
					\end{itemize}


					\item Le contrat définitif doit comporter \textbf{l'indication du prix et les modalités de son paiement}\footnote{\articleDu[L]{261-11 alinéa 1\ier{} b)}{\cch}}.

					Le prix est librement fixé par le vendeur qui n'est pas lié par les prévisions du contrat préliminaire.

					Le vendeur peut modifier le prix stipulé dans l’avant-contrat pour tenir compte du coût des travaux supplémentaires ou des améliorations apportées aux éléments d'équipement par rapport aux prévisions initiales ou même tout simplement pour augmenter la rentabilité de l'opération de construction.

					Toutefois si cette augmentation dépasse \pourcent{5}, le réservataire est libre de renoncer à l'opération et a droit au remboursement intégral de son dépôt de garantie

					Le contrat définitif doit préciser les modalités d’échelonnement du prix ainsi que ces conditions de révision (voir infra).

					\item Le contrat définitif doit préciser le \textbf{délai de livraison de l'immeuble}\footnote{\articleDu[L]{261-11, alinéa 1er c)}{\cch}}.

					Le délai peut être stipulé sous forme d’une période : 4T2019…

					Tout dépassement de ce délai est susceptible d'être sanctionné par l’allocation de dommages et intérêts ou par la résolution judiciaire du contrat, en application de l'article 1228 du \cciv\footnote{ancien 1184 alinéa 2}.


					\item 	L'acte authentique de vente doit comporter des indications relatives à la garantie financière d’achèvement ou de remboursement.


					\begin{citationArticle}[L]{261-11 alinéa 1er d)}{\cch (issue de l’ordonnance du 3 octobre 2013)}
						Lorsqu'il revêt la forme prévue à l'article 1601-3 du code civil, reproduit à l'article L. 261-3 du présent code \emph{[VEFA], [le contrat doit comporter]} la justification de la garantie financière prescrite à l'article L. 261-10-1, l'attestation de la garantie étant établie par le garant et annexée au contrat.
					\end{citationArticle}

					\begin{citationArticle}[L]{261-10-1}{\cch}
						Avant la conclusion d'un contrat prévu à l'article L. 261-10, le vendeur souscrit une garantie financière de l'achèvement de l'immeuble ou une garantie financière du remboursement des versements effectués en cas de résolution du contrat à défaut d'achèvement.
					\end{citationArticle}

					Cf infra.


					\item  Le contrat définitif doit comporter, le cas échéant, la description des travaux dont l'acquéreur se réserve l'exécution.


					\item 	Lorsque l'immeuble à construire est un immeuble en copropriété, le vendeur doit remettre à l'acquéreur un exemplaire du règlement de copropriété, lors de la signature du contrat\footnote{\articleDu[L]{261-11 5\ieme{} alinéa}{\cch*}}.

					Le règlement doit lui avoir été communiqué préalablement, mais aucun délai n'est fixé.


					\item 	Le contrat définitif doit, le cas échéant, contenir une condition suspensive d’obtention de crédit.

					Le contrat est en effet soumis aux dispositions des \articlesDuEtSuivants[L]{312-15}{\ccons} relatifs au crédit, dès lors que l’acquéreur peut être qualifié de consommateur.

					La plupart des directives communautaires définissent le consommateur, d’une part comme une personne physique, et d’autre part comme une personne agissant pour son usage pouvant être considéré comme étranger à son activité professionnelle.


					L’\articleCodifie[L]{312-17} du même Code, précise que, dans l’hypothèse ou l’acquéreur n’aurait pas recours à un crédit, il doit apposer une mention spéciale en ce sens au contrat.


					\item 	Le vendeur doit, s’il a recours à un prêt spécial à la construction de la part du Crédit foncier de France (ou du Comptoir des entrepreneurs), tenir à la disposition de l'acquéreur une copie du plan de financement de la construction faisant apparaître les éléments de l'équilibre financier de l'opération au vu desquels a été prise la décision de prêt\footnote{\articlesCodifies[L]{261-11}, sixième alinéa et \refArticle[R]{261-16} du \cch*}.
				\end{enumerate}

				\bigbreak\noindent\textbf{Précisions :}

				\begin{itemize}
					\item L'\articleDu[L]{261-16}{\cch} dispose que toute clause du contrat de vente contraire aux dispositions des \articlesCodifies[L]{261-11} à \refArticle[L]{261-15} et à celle des articles 1642-1 et 1646-1 du \cciv\footnote{relatifs aux garanties de vices}, est réputée non écrite.

					\item Les mentions imposées par l’\articleDu[L]{261-11}{\cch} sont prescrites à peine de nullité du contrat (6\ieme{} alinéa \emph{in fine} de l’article).

					Il s’agit d’une nullité relative qui ne peut être soulevée que par l’acquéreur et avant l’achèvement des travaux.
				\end{itemize}



			\paragraph{La purge du droit de réflexion}

				L’acquéreur dispose d’un droit de réflexion de 10 jours à compter de la notification du projet d’acte authentique, s’il n’a pas déjà bénéficié d’un droit de rétractation.

				\begin{citationArticle}[L]{2271-1 dernier alinéa}{\cch}
					Lorsque le contrat constatant ou réalisant la convention est dressé en la forme authentique et n'est pas précédé d'un contrat préliminaire ou d'une promesse synallagmatique ou unilatérale, l'acquéreur non professionnel dispose d'un délai de réflexion de dix jours à compter de la notification ou de la remise du projet d'acte selon les mêmes modalités que celles prévues pour le délai de rétractation mentionné aux premier et troisième alinéas. En aucun cas l'acte authentique ne peut être signé pendant ce délai de dix jours
				\end{citationArticle}

				Pratiquement, ce délai ne profitera qu’à l’acquéreur dont l’acquisition n’a pas été précédée d’un contrat préliminaire.

				La doctrine semble considérer que l’acquéreur ne peut pas renoncer au délai de 10 jours.

				\bigbreak\textbf{Attention} : en cas d’annulation du contrat préliminaire, la purge du droit de rétractation intervenue à ce stade est effacée, de sorte que la protection bénéficiant à l’acquéreur au titre de l’\articleDu[L]{271-1}{\cch} se reporte sur le contrat définitif de vente.

				Ainsi, le contrat définitif pourra être annulé à la demande de l’acquéreur si le notaire n’a pas pris soin de purger le délai de réflexion avant de recevoir l’acte authentique de vente.


	\subsection{La formation du contrat dans le secteur libre}

		\subsubsection{La faculté de conclure un avant contrat}

			Dans le secteur libre, comme dans le secteur protégé, la conclusion d’un avant-contrat est facultative.

			Rien n’empêche vendeur et acquéreur de conclure directement le contrat de \VEFA.


			\bigbreak Toutefois, en pratique, la conclusion d’un contrat de \VEFA dans le secteur libre est souvent précédée d’un avant-contrat, compte tenu de son utilité :
			\begin{itemize}
				\item L’avant-contrat permet tout d’abord au vendeur d’apprécier le marché potentiel de son opération.

				Il s’agit pour le vendeur de déterminer si son produit trouvera facilement acquéreur, si celui-ci plait à la clientèle ; c’est un révélateur de l’intérêt que porte le public au programme de construction projeté.

				\item La conclusion d’avant-contrat permettra également au vendeur d’obtenir plus aisément un financement de la part des établissements de crédit.
			\end{itemize}





			\bigbreak L’avant-contrat ne fait l’objet d’aucune réglementation spécifique dans le secteur libre.

			C’est le principe de liberté contractuelle qui s’applique : les parties sont libres de conclure l’avant-contrat de leur choix.

			\medbreak On pourrait ainsi envisager de conclure :
			\begin{itemize}
				\item une promesse unilatérale de vente,
				\item une promesse synallagmatique de vente,
				\item un pacte de préférence,
				\item ou encore un contrat préliminaire.
			\end{itemize}

			Rien n’interdit en effet aux parties de se soumettre volontairement aux dispositions du Code de la construction et de l’habitation.

		\subsubsection{La conclusion du contrat définitif}

			Les parties sont libres de choisir le montage contractuel qu’elles souhaitent, et peuvent décider de conclure un contrat de vente d’immeuble à construire.

			Dans cette hypothèse les règles édictées par le \cch ne seraient que supplétives de volonté.

			L’acte authentique n’est même pas obligatoire.

			\bigbreak Toutefois, à défaut d’écrit, les parties seront dans l’impossibilité de satisfaire à l’exigence de publicité foncière\footnote{exigé par l’article 28 du décret du 4 janvier 1955} :

			\begin{quote}
				« {\itshape Sont obligatoirement publiés au bureau des hypothèques de la situation des immeubles :

				1\degres Tous actes, même assortis d'une condition suspensive, et toutes décisions judiciaires, portant ou constatant entre vifs :

				a) Mutation ou constitution de droits réels immobiliers autres que les privilèges et hypothèques, qui sont conservés suivant les modalités prévues au code civil} ».
			\end{quote}

			\paragraph{Sanction du défaut de publicité} inopposabilité de l’acte.


\section{L'exécution du contrat de vente en l'état futur d'achèvement}

	\subsection{Les obligation du vendeur}

		\subsubsection{L'obligation d'information}

			Le consentement de l’acquéreur doit être éclairé.

			Le vendeur d’immeuble à construire doit donc informer l’acquéreur des inconvénients et des nuisances de l’immeuble.

			Il doit également le renseigner sur l’existence et l’impact des modifications survenues par rapport aux documents initiaux.

			\paragraph{Sanctions}

			Le plus souvent ce sont des dommages et intérêts, mais dans les cas les plus graves ce peut être la nullité de la vente pour dol\footnote{\jurisCA{Paris}{23/09/1994}}.


		\subsubsection{L'obligation de fournir une garantie de remboursement ou d'achèvement dans le secteur protégé}

			Lorsque l’opération de \VEFA est située dans le secteur protégé, il appartient au vendeur de fournir à l’acquéreur une garantie d’achèvement ou de remboursement.

			Avant l’ordonnance du \printdate{3/10/2013} cette garantie pouvait revêtir deux formes : intrinsèque ou extrinsèque.

			Cependant, depuis cette ordonnance la garantie ne peut être qu’extrinsèque pour les opérations dont la demande de permis de construire a été déposée à compter du \printdate{1/1/2015}.

			\begin{citationArticle}[L]{261-10-1}{\cch}
				Avant la conclusion d'un contrat prévu à l'article L. 261-10, le vendeur souscrit une garantie financière de l'achèvement de l'immeuble ou une garantie financière du remboursement des versements effectués en cas de résolution du contrat à défaut d'achèvement.
			\end{citationArticle}

			Il existe deux types de garanties extrinsèques en fonction de leur objet :
			\begin{itemize}
				\item le remboursement des sommes versées par l’acquéreur,
				\item ou le financement des travaux nécessaires à l’achèvement.
			\end{itemize}

			La garantie extrinsèque doit être délivrée par une banque, un établissement financier habilité à faire des opérations de crédit immobilier, une entreprise d’assurance agréée à cet effet ou une société de caution mutuelle\footnote{\articleDu[R]{261-17}{\cch}}.


			\paragraph{La forme de la garantie}

				\subparagraph{La garantie de remboursement} \ArticleDu[R]{261-22}{\cch*}

				La garantie de remboursement revêt la forme d'une convention de cautionnement aux termes de laquelle la caution s'oblige envers l'acquéreur, solidairement avec le vendeur, à lui rembourser les versements effectués.

				La garantie de remboursement ne peut être mise en œuvre qu’en cas de résolution de la vente pour défaut d’achèvement.

				Elle est peu utilisée.

				\subparagraph{La garantie d’achèvement} \ArticleDu[R]{261-21}{\cch*}

				La garantie d’achèvement a pour objet de financer l’intégralité des travaux nécessaires à l’achèvement de l’immeuble.

				\medbreak Elle peut prendre la forme :
				\begin{itemize}
					\item soit d'une ouverture de crédit par laquelle celui qui l'a consentie s'oblige à avancer au vendeur ou à payer pour son compte les sommes nécessaires à l'achèvement de l'immeuble ;

					\item soit d'une convention de cautionnement aux termes de laquelle la caution s'oblige envers l'acquéreur, solidairement avec le vendeur, à payer les sommes nécessaires à l'achèvement de l'immeuble.
				\end{itemize}


				\subparagraph{Faculté de substitution des garanties}

				Le vendeur et le garant ont la faculté, au cours de l'exécution du contrat de vente, de substituer la garantie d'achèvement, à la garantie de remboursement ou inversement, à la condition que cette faculté ait été prévue au contrat de vente.

				Cette substitution doit être notifiée à l'acquéreur.


			\paragraph{La mise en œuvre pratique de la garantie financière d’achèvement}\index{GarantieFinanciereAchevement@Garantie financière d’achèvement}

				\subparagraph{Le déclenchement}

				La garantie financière d'achèvement peut être mise en œuvre par l'acquéreur en cas de défaillance financière du vendeur, caractérisée par une absence de disposition des fonds nécessaires à l'achèvement de l'immeuble\footnote{\articleDu[L]{261-10-1}{\cch}}.

				\subparagraph{Le rôle du garant}

				Le garant peut se substituer au vendeur défaillant (rare) ou faire désigner un administrateur \emph{ad hoc} par ordonnance sur requête\footnote{\articleDu[L]{261-10-1, alinéa 3}{\cch}}.

				L'administrateur \emph{ad hoc} dispose des pouvoirs du maître de l'ouvrage ; il a pour mission de faire réaliser les travaux nécessaires à l'achèvement de l'immeuble.

				Il peut réaliser toutes les opérations qui y concourent et procéder à la réception de l'ouvrage, au sens de l'article 1792-6 du \cciv.

				Il est réputé constructeur au sens de l'article 1792-1 du même code et dispose, à ce titre, d'une assurance de responsabilité en application de l'\articleDu[L]{241-2}{\ca}.

				Sa rémunération est à la charge du garant.

				\subparagraph{Versement du prix}

				Lorsque sa garantie est mise en œuvre, le garant financier de l'achèvement de l'immeuble est seul fondé à exiger de l'acquéreur le paiement du solde du prix de vente, même si le vendeur fait l'objet d'une procédure collective.


			\paragraph{La fin de la garantie}

			L’engagement financier du garant prend fin à l’achèvement de la construction tel que définie à l'\articleDu[R]{261-1}{\cch} :

			\begin{quote}
				« {\itshape L’immeuble \lips est réputé achevé \lips lorsque sont exécutés les ouvrages et sont installés les éléments d’équipement qui sont indispensables à l’utilisation, conformément à sa destination, de l’immeuble faisant l’objet du contrat, à l'exception des travaux dont l'acquéreur se réserve l'exécution en application du II de l'article L. 261-15.

				Pour l’appréciation de cet achèvement, les défauts de conformité avec les prévisions du contrat ne sont pas pris en considération lorsqu’ils n’ont pas un caractère substantiel, ni les malfaçons qui ne rendent les ouvrages ou éléments d’équipement impropres à leur utilisation.} »
			\end{quote}

			L’\articleDu[R]{261-1}{\cch} dispose en outre que « {\itshape la constatation de l’achèvement n’emporte par elle-même ni reconnaissance de la conformité aux prévisions du contrat, ni renonciation aux droits que l’acquéreur tient de l’article 1642-1 du Code Civil} ».


			\medbreak La constatation de l’achèvement est faite\footnote{\articleDu[R]{261-24}{\cch*}}  :
			\begin{itemize}
				\item soit par une personne désignée dans les conditions prévues à l'article R. 261-2 (personne qualifié désignée par ordonnance sur requête),

				\item soit par un homme de l'art ou un organisme de contrôle indépendant.
			\end{itemize}

			Il peut s’agir du maître d’œuvre d’exécution.

			Toutefois, lorsque le vendeur assure lui-même la maîtrise d'œuvre, la constatation est faite par un organisme de contrôle indépendant.


		\subsubsection{La délivrance de l'immeuble }

			Cette obligation est double.

			Le vendeur doit délivrer :
			\begin{itemize}
				\item un immeuble conforme aux caractéristiques contractuelles,
				\item dans le délai d’édification contractuel.
			\end{itemize}


			\paragraph{Le respect des caractéristiques contractuelles de l'immeuble}

				Principe : le vendeur est tenu de délivrer un bien dont les caractéristiques correspondent à la commande.

				Il s’agit d’une obligation de résultat.

				\subparagraph{La notion de défaut de conformité}

					\begin{enumerate}
						\item  Le vendeur doit édifier et livrer un immeuble conforme aux caractéristiques prévues dans le contrat en termes de quantité et de qualité.

						Cela concerne aussi bien les surfaces, les matériaux, les équipements \etc

						Il est donc essentiel que le vendeur répercute dans les contrats de louage d’ouvrage, les spécifications du contrat de vente.

						Le vendeur peut, par ailleurs, insérer dans le contrat de \VEFA des clauses de tolérance pour les surfaces, les mesures diverses ou les matériaux.

						Précision : La délivrance du certificat de conformité par l’autorité publique atteste de la conformité de l’ouvrage aux prescriptions d’urbanisme mais pas du respect des obligations contractuelles incombant aux parties.


						\item Le défaut de conformité doit être distingué du vice de construction.

						Selon le Professeur \nom{Perinet-Marquet} « \emph{le défaut de conformité suppose le non respect de prescription contractuelles, même s’il n’en découle pas un désordre} ».

						Cette notion est à distinguer du vice, qui se caractérise par « \emph{la mauvaise exécution d’un travail qui se voulait, a priori conforme} ».

						La distinction entre les deux notions est parfois très délicate ; elle relève de la compétence des juges du fond.

						La distinction n’a aujourd’hui que peu d’importance pour les désordres apparents (couvert par même régime) mais revêt encore une importance pour les désordres cachés (régimes des actions distincts).
					\end{enumerate}


				\subparagraph{L'appréciation de la conformité}

					La conformité de la construction s’apprécie par rapport à tous les documents contractuels ; notion pour laquelle la jurisprudence adopte une conception extensive.

					Ainsi, la jurisprudence considère que la conformité de l’immeuble s’entend :
					\begin{itemize}
						\item du respect du contrat de vente et de ses annexes,
						\item des dispositions du permis de construire,
						\item des engagements contenus dans le contrat préliminaire,
						\item des documents publicitaires lorsque leur présentation et leur précision ont exercés une réelle influence sur le consentement de l’acquéreur.
						\item les devis, les plans, l’état descriptif de division, le cahier des charges, le règlement de copropriété qui sont visés aux actes de vente et qui ont valeur contractuelle.
					\end{itemize}

					\bigbreak En cas de contradiction :
					\begin{itemize}
						\item les mentions du contrat définitif priment celles du contrat préliminaire,
						\item les mentions de la notice descriptive annexée à l'acte authentique priment celles du permis de construire, des documents publicitaires et de la notice descriptive sommaire annexée au contrat de réservation\footnote{\jurisCourDeCas[16-16627]{\civTrois*}{18/05/2017}}.
					\end{itemize}

				\subparagraph{La sanction du défaut de conformité}

					En cas de non conformité par rapport aux prévisions contractuelles, l’acquéreur peut au choix\footnote{par application des dispositions de l'article 1228 du Code civil} :
					\begin{itemize}
						\item exiger la mise en conformité de la chose vendue,
						\item demander une indemnisation,
						\item demander la résolution du contrat avec dommages- intérêts.
					\end{itemize}

					La résolution du contrat ne sera prononcée que dans les cas d’inexécution les plus graves (réduction importante de la surface).

					Les tribunaux sont souverains pour apprécier la gravité du défaut de conformité et déterminer la forme de réparation préférable.

					Prescription de l’action :
					\begin{itemize}
						\item En cas de défaut de conformité apparent, 1 an à compter de la plus tardive des deux dates que constituent la réception ou l’expiration d’un délai d’un mois à compter de la prise de possession,
						\item En cas de défaut de conformité caché, 5 ans à compter de la révélation\footnote{\ArticleCciv{2224}}.
					\end{itemize}


					\textbf{Attention} : s’il découle un dommage de la non-conformité il faudra actionner la garantie décennale ou biennale.


			\paragraph{Le respect du délai contractuel d'édification de l'immeuble}

				La date à prendre en compte pour apprécier le respect du délai d’édification est la date de livraison.

				Elle est souvent stipulée sous forme d’une période : 4T2015, au plus tard le 30 septembre 2018…

				Pratiquement, les vendeurs d’immeuble à construire stipulent dans l’acte de vente des causes légitimes de suspension du délai de livraison (intempéries, grève générale de la profession, faillite \etc)

				\textbf{Sanctions} :
				\begin{itemize}
					\item l’allocation de dommages et intérêts ;
					\item ou résolution du contrat si le retard laisse mal augurer de l’achèvement de la construction ou s’avère excessif \footnote{\jurisCA{Paris}{16/06/1998}, RD imm., 1999, 661, obs. Saint-Alary-Houin ; \jurisCA{Paris}{25/03/1999}, Constr. Urb., 1999, 234, obs. Sizaire}.
				\end{itemize}


		\subsubsection{La garantie de l'immeuble}

			\paragraph{La garantie des vices et défauts de conformité apparents}

				La garantie des vices et défauts de conformité apparents couvre tous les vices et défauts de conformité apparent à la livraison ou apparu dans le mois suivant, quelque soit leur gravité : des désordres simplement esthétiques au plus grave\footnote{\jurisCourDeCas[90-13320]{\civTrois*}{26/02/1992}}.

				L’apparence s’apprécie par référence à un acquéreur moyen dépourvu de connaissances techniques particulières.

				La garantie est exclusive de toute notion de faute ; seul compte le caractère apparent du désordre le jour de la réception ou dans le délai d’un mois à compter de la prise de possession\footnote{\jurisCourDeCas[11-27486]{\civTrois*}{4/12/2012}}.

				\subparagraph{Le délai d’action}

				Aux termes des articles 1642-1 et 1648 du Code civil combinés, le vendeur d’immeuble à construire est débiteur de la garantie des vices de construction et des défauts de conformité apparents, pendant une période d’un an à compter de la plus tardive des deux dates que constituent :
				\begin{itemize}
					\item la réception des travaux avec ou sans réserves, intervenue entre le locateur d’ouvrage et le vendeur ;

					\item l’expiration d’un délai d’un mois après la prise de possession par l’acquéreur.

						On parle également de livraison de l’immeuble.\index{Livraison}

						Elle intervient entre le vendeur et l’acquéreur.

						Elle est le plus souvent matérialisée par la signature d’un \pv de livraison dans lequel les parties listent les réserves et par la remise des clefs.
				\end{itemize}


				La Cour de cassation tend à faire du délai d’un mois, visé à l’\articleCciv{1642-1}, un délai d’apparition du vice et non pas un délai de dénonciation :
				\begin{quote}
					« {\itshape l'acquéreur est recevable pendant un an à compter de la réception des travaux ou de la prise de possession des ouvrages à intenter contre le vendeur l'action en garantie des vices apparents, même dénoncés postérieurement à l'écoulement du délai d'un mois après la prise de possession.} »\footnote{\jurisCourDeCas[98-20250]{\civTrois*}{22/03/2000}}
				\end{quote}

				L’acquéreur dispose donc d’un délai d’un an à compter soit de la réception, soit de l’expiration du délai d’un mois pour introduire son action.

				Le délai de l’\articleCciv{1648} est un délai de forclusion.

				Il ne peut être qu’interrompu (assignation ou conclusions) mais jamais suspendu\footnote{\jurisCourDeCas[14-15796]{\civTrois*}{3/6/2015}}.

				C’est alors un nouveau délai d’un an qui recommence à courir à compter de la décision\footnote{\jurisCourDeCas[18-17856]{\civTrois*}{11/7/2019}}.

				La responsabilité de droit commun n’est pas cumulative avec la garantie des vices et défauts de conformité apparents\footnote{\jurisCourDeCas[14-14706]{\civTrois*}{3/6/2015}, 629}.

				\subparagraph{La sanction du désordre}

				L’existence de vices ou de défauts de conformité apparents peut être sanctionnée par une diminution du prix ou par la résolution de la vente, sauf toutefois si le vendeur s’oblige à les réparer \footnote{\ArticleCciv{1642-1}, 2\ieme{} alinéa}.

				La proposition de reprise du constructeur doit être pertinente et opportune ; à défaut l’acquéreur peut obtenir une diminution du prix\footnote{\jurisCourDeCas[18-16182]{\civTrois*}{7/3/2019}}.

				La Cour de cassation a également admis le principe d’une réparation en équivalent ; de même que l’allocation de dommages et intérêts en cas de préjudices annexes\footnote{\jurisCourDeCas{\civTrois*}{2/3/2005} : pour un trouble de jouissance}.


				\bigbreak Précisions en cas de résolution (quelqu’en soit la cause) :
				\begin{itemize}
					\item La résolution de la vente, dans le secteur protégé, entraîne le jeu de la garantie de remboursement.

					\item Le montant des dommages et intérêts est, en principe, apprécié par le juge mais le contrat peut fixer son quantum.

					Le montant ne peut toutefois pas dépasser \pourcent{10} du prix :

					\ArticleDu[L]{261-14}{\cch}  : « {\itshape le contrat ne peut stipuler forfaitairement, en cas de résolution, le paiement, par la partie à laquelle elle est imputable, d'une indemnité supérieure à 10 p. 100 du prix.} >>

					Les parties conservent néanmoins la faculté de demander la réparation du préjudice effectivement subi (même article).
				\end{itemize}

			\paragraph{La garantie des désordres cachés}

				\subparagraph{Les actions de l'acquéreur à l'encontre du vendeur}

					\begin{enumerate}
						\item L’\articleCciv{1646-1} assimile le vendeur en l’état futur d’achèvement à un constructeur :
						\begin{quote}
							« {\itshape Le vendeur d'un immeuble à construire est tenu, à compter de la réception des travaux, des obligations dont les architectes, entrepreneurs et autres personnes liées au maître de l'ouvrage par un contrat de louage d'ouvrage sont eux-mêmes tenus en application des articles 1792, 1792-1, 1792-2 et 1792-3 du présent code.

							Ces garanties bénéficient aux propriétaires successifs de l'immeuble.} »
						\end{quote}

						Le vendeur en l’état futur d’achèvement peut donc voir sa responsabilité engagée par son cocontractant et par les acquéreurs successifs de l’immeuble, sur le fondement des \articlesDuEtSuivants{1792}{\cciv}.

						Le régime des garanties décennale et biennal exposé précédemment s’applique : présomption de responsabilité, cause exonératoire, délai d’action, sanction du désordre…

						L’\articleDu{1646-1}{\cciv} prévoit cependant « {\itshape qu’il n'y aura pas lieu à résolution de la vente ou à diminution du prix si le vendeur s'oblige à réparer les dommages définis aux articles 1792, 1792-1 et 1792-2 du présent code et à assumer la garantie prévue à l'article 1792-3}. »


						\item  Le vendeur d’immeuble à construire est tenu de la garantie des désordres acoustiques de l’\articleDu[L]{111-11}{\cch}.

						\item Il est également tenu des désordres intermédiaires\footnote{\ArticleDu{1231-1}{\cciv} – responsabilité pour faute prouvée (souci d’économie) --- \jurisCourDeCas[08-13239]{\civTrois*}{4/6/2009} ; \jurisCourDeCas[08-22062]{\civTrois*}{8/9/2010} ; \jurisCourDeCas[11-28376]{\civTrois*}{13/2/2013}}.

						\item  En revanche, le vendeur d’immeuble à construire n’est pas tenu de la \gpa à laquelle lui est substituée la garantie des vices et défauts de conformité apparents.

					\end{enumerate}

				\subparagraph{Les actions à l'encontre des locateurs d'ouvrage}

					Le vendeur d’immeuble à construire, l’acquéreur et les sous-acquéreurs peuvent agir à l’encontre des locateurs d’ouvrage, véritable responsables des désordres, sur le fondement :
					\begin{itemize}
						\item de la garantie de parfait achèvement ;
						\item des garanties décennale et biennale
;
						\item de la responsabilité contractuelle de droit commun \footnote{\jurisCourDeCas{\civTrois*}{28/2/1996} : « {\itshape les acquéreurs jouissent de tous les droits et actions attachés à la chose qui appartenaient à leur auteur et dispose contre les locateurs d’ouvrage, d’une action contractuelle fondée sur un manquement à leurs obligations} ».} ;
						\item de la responsabilité délictuelle de droit commun à l’encontre des sous-traitants.
					\end{itemize}


			\subsubsection{L'obligation de souscrire les assurances construction obligatoires}

				\begin{enumerate}
					\item L’\articleDu[L]{242-1}{\ca}  impose au vendeur en l’état futur d’achèvement de souscrire une \ado.
					\begin{quote}
						« {\itshape Toute personne physique ou morale qui, agissant en qualité de propriétaire de l'ouvrage, de vendeur ou de mandataire du propriétaire de l'ouvrage, fait réaliser des travaux de construction, doit souscrire avant l'ouverture du chantier, pour son compte ou pour celui des propriétaires successifs, une assurance garantissant, en dehors de toute recherche des responsabilités, le paiement de la totalité des travaux de réparation des dommages de la nature de ceux dont sont responsables les constructeurs au sens de l'article 1792-1, les fabricants et importateurs ou le contrôleur technique sur le fondement de l'article 1792 du code civil.} »

					\end{quote}

					\textbf{Précision} : l’assurance dommages-ouvrage ne doit couvrir que les travaux que le vendeur d’immeuble à construire s’est engagé à réaliser, à l’exclusion des éventuels travaux réservés par les acquéreurs.


					\item Les \articlesCodifies{L}{241-1} et \refArticle{L}{241-2} du même code impose au vendeur la souscription d’une assurance de responsabilité civile décennale.

					\begin{quote}
						« {\itshape Toute personne physique ou morale, dont la responsabilité décennale peut être engagée sur le fondement de la présomption établie par les articles 1792 et suivants du code civil, doit être couverte par une assurance.} »

					\end{quote}

					et,

					\begin{quote}
						« {\itshape Celui qui fait réaliser pour le compte d'autrui des travaux de construction doit être couvert par une assurance de responsabilité garantissant les dommages visés aux articles 1792 et 1792-2 du code civil et résultant de son fait.

						Il en est de même lorsque les travaux de construction sont réalisés en vue de la vente. }»
					\end{quote}

				\end{enumerate}

	\subsection{L'obligation de l'acquéreur : le paiement du prix de vente}

		L’acquéreur a une obligation majeure : le paiement du prix convenu --- le cas échéant après révision --- aux époques convenues.

		\subsubsection{La révision du prix dans le secteur protégé}

			Le prix stipulé dans le contrat définitif peut être assorti d'une clause d'indexation.
\aVerifier
			Il s'agit d'une simple faculté.

			Le contrat doit expressément indiquer si le prix est ou non révisable et, dans l'affirmative, les modalités de sa révision\footnote{\ArticleDu[L]{261-11}{\cch*}}).


			\bigbreak L'indexation du prix, dans le secteur protégé, est limitée.
			\begin{itemize}
				\item La révision du prix ne peut être calculée qu'en fonction de la variation de l'indice national du bâtiment tous corps d'État dénommé << BT01 >> publié mensuellement au journal officiel et défini à l'\articleDu[R]{261-15}{\cch}\footnote{\ArticleCodifie{L}{261-11-1} et \articleDu[R]{261-15}{\cch*}}.

				La variation prise en compte résulte de la comparaison du dernier indice publié au jour de la signature du contrat et du dernier indice publié à la date de chaque paiement.

				\item L’indexation n’est en outre admise que dans la limite de \pourcent{70} de la variation de cet indice.
			\end{itemize}


		\subsubsection{L'échelonnement des paiements dans le secteur protégé}

			Dans le secteur protégé, le versement du prix est réglementé par les articles 1601-2 et 1601-3 du \cciv --- reproduits dans le Code de la construction et de l'habitation aux \articlesCodifies{L}{261-2} et \refArticle{L}{261-3} --- et par l'\articleDu[L]{261-12}{\cch*}.

			L'\articleDu[L]{261-10}{\cch} rend ces dispositions impératives et sanctionne leur violation par la nullité de l’acte.

			Ainsi :
			\begin{itemize}
				\item L’acquéreur est tenu de payer le prix de l’immeuble au fur et à mesure de l'avancement des travaux\footnote{\ArticleDu{1601-3}{\cciv}}.

				\item Tout versement avant la signature de l’acte, ou avant la date à laquelle la créance est exigible est interdit\footnote{\ArticleDu[L]{261-12, 1\ier{} alinéa}{\cch}}.

				Toute personne qui exige ou accepte un versement en violation de ces dispositions encourt une peine d'emprisonnement de deux ans et d'une amende de \montant{9 000} ou de l'une de ces deux peines seulement\footnote{\ArticleDu[L]{263-1}{\cch}}.


				\item Les versements sont plafonnés\footnote{\ArticleDu[R]{261-14}{\cch}} :
					\begin{quote}
						« { \itshape Les paiements ou dépôts ne peuvent excéder au total :
						\begin{itemize}
							\item \pourcent{35} du prix à l'achèvement des fondations ;
							\item \pourcent{70} du prix à la mise hors d'eau ;
							\item \pourcent{95} à l'achèvement de l'immeuble\footnote{Tel que défini à l’\articleDu[R]{261-1}{\cch} et constatée dans les conditions de l’\articleDu[R]{261-24}{\cch}}
						\end{itemize}.}»
					\end{quote}

				Les versements doivent être déterminés en fonction de l'avancement réel des travaux, dans la limite des plafonds prévus.

				Le contrat peut prévoir des paliers intermédiaires en fonction des différents stades de la construction, dans la limite des plafonds :
				\begin{itemize}
					\item $\dots$ \% à l'achèvement des fondations ;
					\item $\dots$ \% au plancher haut du rez-de-chaussée ;
					\item $\dots$ \% au plancher haut du dernier étage ;
					\item $\dots$ \% au hors d’eau ;
					\item 	\etc
				\end{itemize}

				\medbreak\textbf{Attention} : Le cumul des différents versements ne peut excéder les plafonds définis à l'\articleDu[R]{261-14}{\cch} : \pourcent{35}  à l'achèvement des fondations, \pourcent{70} au hors d'eau, \pourcent{95} à l'achèvement de l'immeuble.

				Toutes clauses contraires à l’échelonnement du prix encourent la nullité.


				\item « Le solde est payable lors de la mise du local à la disposition de l'acquéreur ; toutefois il peut être consigné en cas de contestation sur la conformité avec les prévisions du contrat»\footnote{ArticleDu[R]{261-14 2\ieme{} alinéa}{\cch}}.

				Le solde correspond théoriquement à \pourcent{5} du prix.

				«\emph{ La mise du local à disposition de l'acquéreur} » correspond soit la livraison (remise des clés), soit jour à partir duquel le vendeur aura mis l’acquéreur en mesure d'en prendre possession.
			\end{itemize}


		\subsubsection{Les sanctions de défaut de paiement}

			Elles sont d’ordre civil.

			\medskip Les parties peuvent avoir stipulé :
			\begin{itemize}
				\item des \textbf{pénalités de retard},	qui ne peuvent excéder \pourcent{1} par mois\footnote{\ArticleDu[R]{261-14}{\cch}} ;

				\item une \textbf{clause de résolution de plein droit} en cas de non-paiement.

					Le mécanisme de sa mise en jeu est prévu à l’\articleDu[L]{261-13}{\cch}.

					Elle ne produira effet qu’un mois après une sommation ou un commandement de payer, resté infructueux.

					Pendant ce délai d’un mois, l’acquéreur peut solliciter des délais de paiement conformément aux dispositions de l’\articleCciv{1343-5}.

					Les effets de la clause de résolution de plein droit sont suspendus pendant le cours des délais octroyés.

					La clause est réputée n'avoir jamais joué si le débiteur se libère dans les conditions déterminées par le juge.
			\end{itemize}

			Si la sommation ou le commandement de payer demeure infructueux, le vendeur peut faire constater la résolution du contrat par le juge.
