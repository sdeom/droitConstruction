% !TEX root = ./droitConstruction.tex

\chapter{Le contrat de sous-traitance}

	Article 1710 : c’est un contrat de louage d’ouvrage. Il est signé entre deux locateurs d’ouvrage : l’entrepreneur principal et un sous-traitant. Principes du consensualisme et de la liberté contractuelle président à la conclusion du contrat de sous-traitant, sauf pour \CCMI où les contrats de sous-traitance doivent être rédigés par écrit en vertu de l'\articleDu[L]{231-13}{\cch}.

	De principe pas de régime du marché à forfait\footnote{Puisque l'Entrepreneur Principal n'est en principe pas propriétaire du sol} sauf si les parties y conviennent conventionnellement.


	Avant la loi du \printdate{31/12/1975} pas de réglementations spécifiques sur la sous-traitance, avant c’était les dispositions du louage d’ouvrage qui s’appliquaient. Or pas de garantie pour le sous-traitant pour le règlement.


	Désormais réglementation particulière de la sous-traitance


	Loi \printdate{31/12/1975} : principal objectif de garantir le paiement du sous-traitant, en revanche cette loi ne traite pas de la problématique de responsabilité du sous-traitant, juste article 1 « entrepreneur principal sous-traite sous sa responsabilité » donc intervention de la \jp qui définit tout le régime de responsabilité du sous-traitant



\section{Les conditions de la sous-traitance}

	\begin{citationArticleLoi}[75-1334]{1}{31/12/1975}
		Au sens de la présente loi, la sous-traitance est l'opération par laquelle un entrepreneur confie par un sous-traité, et sous sa responsabilité, à une autre personne appelée sous-traitant l'exécution de tout ou partie du contrat d'entreprise ou d'une partie du marché public conclu avec le maître de l'ouvrage.
	\end{citationArticleLoi}

	Cette loi concerne tant les marchés publics que privés.

	\subsection{Les conditions générales de sous-traitance}

		\subsubsection{La succession de deux contrat d'entreprise}

			Il faut nécessairement succession de deux contrats d’entreprise, le premier entre \E et Entrepreneur Principal, et un deuxième contrat entre l’Entrepreneur Principal et le Sous-traitant.


			La qualification du premier contrat est donc primordiale.


			Au même titre, il faut s’intéresser à la qualification du deuxième contrat (pas mandat, vente, travail)

			S’agit-il vraiment d’un sous-traitant ?


			\paragraph{La qualification du contrat en cas de réalisation de travaux d'ordre matériel ou intellectuel} La qualification du second contrat ne pose pas de problème en cas de travaux matériels.


				\subparagraph{La sous-traitance de prestation intellectuelle est-elle possible ?}
				En 1984 la 3\ieme{} chambre civile a considérée que oui c’est possible, donc le deuxième contrat peut porter sur des travaux d’ordre intellectuel.

				\begin{exemple}
					architecte, bureau d’étude
				\end{exemple}

				L'article 37 du code de déontologie des architectes interdit la sous-traitance du projet architectural. A défaut, le \Mo n’est pas tenu de payer les sommes exposés par le \Moe\footnote{\jurisCourDeCas[16-15958]{\civTrois*}{27/04/2017}}.

			\paragraph{La qualification du contrat en cas de fourniture de matériaux}

				\subparagraph{Peut-il y avoir sous-traitance lorsque le co-contractant se contente de fournir des matériaux ?}
				Si que fourniture de matériaux deux qualifications possibles : vente ou louage d’ouvrage
.

				\jurisCourDeCas[83-16675]{\civTrois*}{5/02/1985}  : critère de la conception de l’ouvrage ou des matériaux et de la spécialité des fournitures


				Si fourniture sur plan du donneur d’ordre et spécificité pour le chantier alors louage d’ouvrage alors sous-traitance, il faut que les fournitures aient été réalisés sur les plans de l’entrepreneur donneur d’ordre et que le produit soit individualisé et non substituable, produit spécifiquement réalisé pour le chantier

				\subparagraph{Quid en cas de fourniture de matériaux et de travaux réalisés sur place ?}

				Si fourniture de matériaux et prestation, qualification de louage d’ouvrage si vraiment travaux sur place.


				Mais si les travaux réalisés consistent à de simple adaptation de produit standard, alors contrat de vente avec prestation de service.


				Les juges doivent apprécier l’importance de chaque prestation, ici appréciation souveraine des juges du fonds.
				La loi de 1975 est d’ordre public, mais son interprétation est jurisprudentielle.

				Pour éviter le contrat de vente, il faut une spécificité des matériaux pour le chantier



			\paragraph{La qualification du contrat en cas de prestation de service ou de fourniture de main-d'œuvre}

				Le second contrat qui porte sur une prestation de service ou sur fourniture de main d’œuvre est un contrat de sous-traitance ou juste louage de chose et de fourniture de main d’œuvre ?

				\begin{exemple}
					Quid pour la mise en place d’échafaudage ?
Pas un contrat de louage d’ouvrage.

				\end{exemple}

				\subparagraph{Si prestation de service} est sous-traitant à 2 conditions :

				\begin{itemize}
					\item Autonomie du personnel de prestataire donc absence du lien de subordination

					\item Participation directe du prestataire de service à l’acte de construire

				\end{itemize}
				On considère que celui qui réalise l’échafaudage ne participe pas directement à l’acte de construire par apport de conception d’industrie ou de matière, juste met à disposition le matériel adapté

				Il s’agit d’un contrat de louage de chose avec mise à disposition de main d’œuvre.


				\subparagraph{Si fourniture de personnel} un tel fournisseur ne peut pas invoquer la loi de 1975 sauf si équipe autonome avec une tâche précise.


				Donc si prêt de matériel comme grue ou de personnel, alors pas de sous-traitance



		\subsubsection{L'autonomie de l'intervenant}

			Pour qu’il ait contrat de sous-traitance il faut nécessairement une absence de subordination, le sous-traitant doit donc être autonome.


			En pratique, il va recevoir des instructions de l’Entrepreneur Principal mais choisi librement les moyens pour les mettre en œuvre. Donc relation hiérarchique mais pas de subordination.


		\subsubsection{L'étendue de la sous-traitance}


			En droit privé la sous-traitance peut être totale. Il faut alors regarder le contrat pour vérifier si contrat de sous-traitance ou de cession de contrat.

			Il s'agit d'une appréciation souveraine du juge du fonds.

			Cependant, en droit positif français, la sous-traitance est majoritairement partielle.


			En droit public, seule sous-traitance partielle est possible.



		\subsubsection{La date de conclusion du contrat de sous-traitance}

			Il n’y a pas de condition de date pour le contrat de sous-traitance, il n’est pas obligatoire qu’il soit signé après, on peut le signer avant le contrat principal dès lors qu’il est assorti d’une condition suspensive de régularisation du contrat principal.

			Le \Mo peut passer un contrat << tout corps d’état >> avec un constructeur qui va sous-traiter les lots. Dans ce cas-là, il va signer des contrats de sous-traitance sous condition suspensive de signer le contrat principal, pour ensuite signer le contrat principal.
Cela permet au constructeur de proposer une offre juste.


			Attention, différence de la sous-traitance avec la co-traitance dans le groupement momentané d’entreprise, ici le \Mo va conclure avec le groupement qui n’a pas de personnalité morale donc contrat de louage d’ouvrage mais les entrepreneurs entre eux sont en co-traitance. Ici la loi de 1975 ne s’applique pas.


			On peut interdire le contrat de sous-traitance dans un contrat de droit privé.



	\subsection{Les conditions spécifiques à la sous-traitance en matière de contrat de \ccmi}

		Le contrat de \ccmi obéit à réglementation particulière, qui distingue deux types de contrats selon qu'il soit avec ou sans fourniture de plan :
		\begin{itemize}
			\item \articlesDuEtSuivants[L]{231-1}{\cch} ;
			\item  \articlesDuEtSuivants[L]{232-1}{\cch} ;
		\end{itemize}

		Le \ccmi est un contrat d’entreprise, un contrat de louage d’ouvrage.


		Les contrats passés par le constructeur de maisons individuelles avec les entreprises sont des contrats de sous-traitance. La loi du \printdate{31/12/1975} s’applique. Aux terme de l'\articleDu[L]{231-13}{\cch} le constructeur est tenu de conclure par écrit le contrat de sous-traitance et le contrat doit contenir les justificatifs de garantie de paiement. Il s'agit d'une condition de forme, donc un écrit avant tout travaux.


		La sanction est pénale : 2 ans de prison et \montant{18.000} d’amende.  Il en va de même si écrit mais pas de justificatif de garantie de paiement.


\section{Une loi d'ordre public}

	La loi du \printdate{31/12/1975} est d’ordre public, son article 15 dispose en effet que sont << {\itshape nuls et de nul effet, quelle qu'en soit la forme, les clauses, stipulations et arrangements qui auraient pour effet de faire échec aux dispositions de la présente loi}.
>>

	Le sous-traitant ne peut pas renoncer aux droits conférés par la loi de 1975\footnote{\jurisCourDeCas{\civTrois*}{9/7/2003}}.
De même, interdiction toute renonciation de la garantie\footnote{\jurisCourDeCas{\civTrois*}{14/09/2017}}.
	\subparagraph{Quid en cas de contrats internationaux ?}
	Deux grandes idées :

	\begin{itemize}
		\item les dispositions d'ordre public s’appliquent aux parties françaises alors même que la construction serait réalisée à l’étranger. Donc entrepreneur principal doit fournir la garantie de paiement à son sous-traitant\footnote{\jurisCourDeCas{\civTrois*}{14/10/1992}}.
		\item s'agissant d'une loi de police au sens de la convention de Rome, elle doit  être respectée par toute personne si la construction est réalisée en France\footnote{\jurisCourDeCas{ch. mixte}{30/11/2007} ;  \jurisCourDeCas{\civTrois*}{25/02/2009}}.
	\end{itemize}


\section{La protection du sous-traitant dans les marchés privés}

	La loi protège le sous-traitant contre la faillite de l’\ep.


	La loi de 1975 ne traite que des problèmes financiers et non de responsabilité.


	La loi le protège face au concours des créanciers en cas de liquidation judiciaire. Le sous-traitant est alors dans une situation à part, la loi lui offrant un autre débiteur que l’\ep.



	\subsection{La protection principale du sous-traitant}

		La loi impose aux partenaires du sous-traitant des obligations particulières.

		\subsubsection{Les obligations imposés à l'\ep}

			\paragraph{L'obligation de faire accepter le sous-traitant et faire agréer ses conditions de paiement\\}

				\begin{citationArticleLoi}[75-1334]{3}{31/12/1975}
					L'entrepreneur qui entend exécuter un contrat ou un marché en recourant à un ou plusieurs sous-traitants doit, au moment de la conclusion et pendant toute la durée du contrat ou du marché, faire accepter chaque sous-traitant et agréer les conditions de paiement de chaque contrat de sous-traitance par le maître de l'ouvrage ; l'entrepreneur principal est tenu de communiquer le ou les contrats de sous-traitance au maître de l'ouvrage lorsque celui-ci en fait la demande.

					Lorsque le sous-traitant n'aura pas été accepté ni les conditions de paiement agréées par le maître de l'ouvrage dans les conditions prévues à l'alinéa précédent, l'entrepreneur principal sera néanmoins tenu envers le sous-traitant mais ne pourra invoquer le contrat de sous-traitance à l'encontre du sous-traitant
				\end{citationArticleLoi}

				\subparagraph{Le contenu de l'obligation} L’\ep doit demander au \Mo d’exprimer son accord sur la présence du sous-traitant sur le chantier et ses conditions de paiement.
				Il s'agit d'une obligation de résultat, mais l’obligation ici est de présentation pour faire agréer, mais pas d’obtenir l'agrément.


				Si le contrat principal interdit la sous-traitance, alors l’agrément sera obligatoire. Mais le contrat principal est silencieux la \jp a considéré que le caractère discrétionnaire du refus est limité par un éventuel abus de droit\footnote{\jurisCourDeCas{\civTrois*}{2/02/2005} : caractère discrétionnaire jusqu’à l’abus de droit}.
				Le \Mo peut donc refuser un sous-traitant pour des motifs raisonnables.


				En cas de sous-traitance en chaine, la qualité de \Mo reste sur le \Mo mais la qualité d'\ep se décale. Chaque sous-traitant devant présenter et faire agréer son propre sous-traitant et lui accorder une garantie de paiement.
				Il n’y a pas de limite sauf disposition contractuelle contraire.

				En pratique les entrepreneurs principaux ne présentent pas leur sous-traitant pour ne pas révéler les marges, d'autant plus que les sanctions sont limitées.


				\subparagraph{La sanction de l'obligation : l'impossibilité pour l'entrepreneur principal d'invoquer le contrat de sous-traitance}

					Dans un arrêt du \printdate{13/04/1998} la 3\ieme{} chambre civile de la \CourDeCas a interprété et considérer que le sous-traitant non agréé ne peut pas à la fois se prévaloir du contrat de sous-traitance pour le paiement, et le rejeter quant à son obligation de résultat. Le sous-traitant demeure donc tenu à son obligation de résultat.

					L'hypothèse dans laquelle le sous-traitant peut valablement invoquer la sanction en cas d'absence d'agrément pour échapper à une obligation est celle dans laquelle le sous-traitant cherche à éviter des pénalités de retard.

			\paragraph{L'obligation de fournir au sous-traitant une garantie de paiement, soit avant, soit au plus tard au moment de la conclusion du contrat}

				L’obligation de l’\ep est de présenter et faire agréer, mais pas d’obtenir l’agrément, et de fournir une garantie de paiement.

				\subparagraph{Le contenu de l'obligation}

					L’entrepreneur doit fournir une garantie de paiement à son sous-traitant avant, ou au plus tard en même temps, que la conclusion du contrat de sous-traité.

					\bigbreak Deux garanties alternatives qui s’imposent à l’\ep :

					\begin{enumerate}
						\item \textbf{Caution personnelle et solidaire.}

						La caution doit être personnelle. La \CourDeCas fait une lecture stricte de l’article 14, la caution doit être accordée nominativement au sous-traitant. L’acte de caution doit indiquer le nom du sous-traitant et le montant de son marché.

						Cette décision semble condamner le système de caution flotte \cad une caution générale pour tous les contrats pour une année donné par un établissement bancaire et financier. cependant, dans un arrêt du \printdate{20/6/2012}\footnote{\jurisCourDeCas[11-18463]{\civTrois*}{20/6/2012}}, la troisième chambre civile a admis ce mécanisme à des conditions strictes : un accord cadre conclu entre une entreprise et un établissement bancaire, l’entreprise fait connaître sous forme d’un avis les contrats de sous-traitance sur lesquels elle entend user l’accord cadre en précisant le montant et l’opération, et la banque donne une attestation au nom du sous-traitant.

						La caution doit être solidaire donc pas de privilège de discussion \cad elle ne doit pas obliger le sous-traitant à agir en priorité contre le débiteur principal.

						En pratique, cette solution est peu utilisée car couteuse.

						\item 	\textbf{Délégation de paiement.}

						Ce mécanisme repose sur l'\articlesDu{1336}{\cciv} : l’\ep --- le déléguant --- peut déléguer au \Mo --- le délégué --- le paiement au sous-traitant --- le délégataire. En clair : l’\ep va demander au \Mo de payer le sous-traitant.

						Dans une délégation parfaite, le déléguant disparait, il y a novation, mais ce n’est pas notre cas, dans la délégation imparfaite, il y a un débiteur de plus. Il n’y a pas de novation, elle créé un second rapport. \aVerifier

						Il faut nécessairement l’accord du \Mo qui ne serait résulter de la simple acceptation du sous-traitant.

						La délégation de paiement peut se combiner avec la garantie de paiement de l'\articleDu{1799-1}{\cciv}

						\end{enumerate}
						Dans les faits la délégation de paiement est peu mise en œuvre malgré ses intérêts qui sont :
						\begin{itemize}
							\item Absence de concours avec les autres créanciers de l’\ep car le sous-traitant a un nouveau débiteur : le \Mo.

							\item Inopposabilité des exceptions du contrat principal et des rapports entre le déléguant et le délégataire.

							En application de l'\articleDu{1336}{\cciv}, le délégué --- le \Mo --- ne peut, sauf stipulation contraire, opposer au délégataire --- le sous-traitant --- aucune exception tirée de son rapport avec le déléguant ou des rapports entre ce dernier et le délégataire. Le \Mo ne peut donc pas refuser de payer le sous-traitant pour des raisons tirées de son contrat avec l’\ep, ni invoquer des exceptions provenant du contrat entre l'\ep et le sous-traitant.
							\aVerifier Il y a inopposabilité des exceptions du contrat principal et du sous-traité, et l'on peut uniquement opposer les exceptions de leur relation sur un fondement délictuel\footnote{\jurisCourDeCas{\civTrois*}{07/06/2018} concernant l’inopposabilité des exceptions en matière de délégation de paiement pour les contrats de sous-traitance}.
						\end{itemize}


						\bigbreak En matière de \CCMI, le législateur a créé une troisième branche de garantie : l'\articleDu[L]{331-13}{\cch} fourni au sous-traitant toute forme de crédit donnant l’assurance de paiement. \aVerifier


						\bigbreak La question de constitutionnalité de l'article 14 a été posée, et la \CourDeCas a jugé qu'il n'avait pas lieu de renvoyer au Conseil constitutionnel une \qpc qui ne présentait pas de caractère sérieux dès lors que le législateur a prévu une alternative gratuite par la délégation de paiement\footnote{\jurisCourDeCas{\civTrois*}{10/06/2014}}.



						\bigbreak {\bfseries Jusqu'à quand le sous-traitant peut-il réclamer à l'entrepreneur principal une garantie de paiement ?}
						Il faut d'abord souligner que le sous-traitant ne peut pas valablement renoncer à la garantie de paiement de l'article 14 de la Loi dès lors que les dispositions que cet article renferme sont d'ordre public.

						Le sous-traitant peut solliciter auprès de l'entrepreneur principal une garantie de paiement tant qu'il n'a pas été intégralement réglé des sommes qui lui sont dues en vertu du contrat de sous-traitance.

				\subparagraph{La sanction de l'obligation : la nullité du contrat de sous-traitance}

					L'article 14 prévoit que la sanction du défaut de garantie est la nullité du contrat de sous-traitance, le sous-traité, si aucune garantie n’est mise en œuvre.


					C’est une nullité relative prescrite sous 5 ans à compter de la conclusion du contrat\footnote{\jurisCourDeCas{\civTrois*}{20/02/2002}}.

					La nullité du contrat de \ST* peut être demandée par le \ST alors qu’il a intégralement été payé. Dans ce cas l’intérêt est d’échapper aux pénalités de retard ou obtenir une sortie du forfait\footnote{\jurisCourDeCas{Civ}{18/07/2001}}.

\aCompleter

					Dans le cas du \CCMI, l'\articleDu[L]{231-13}{\cch} prévoit que le constructeur doit fournir dans le contrat, le justificatif de la caution ou de la délégation. La sanction est pénale\footnote{\ArticleDu[L]{241-9}{\cch}}.


			\paragraph{L'interdiction de nantir ou céder à une banque une créance supérieure à celle qui résulte des travaux qu'il effectue personnellement pour le compte du \Mo}

				L'article 13-1 de la \loiST stipule que l’\ep ne peut céder ou nantir qu’à concurrence du contrat avec \Mo.

				\subparagraph{Le contenu de l'obligation}

					L'\ep peut céder ou nantir la totalité de sa créance s’il constitue une caution personnelle ou solidaire au profit de son \ST.

					\begin{exemple}
						dans le cadre d'un marché de \montant{1 000 000} mais avec \montant{600 000} de \ST*, l'\ep ne peut nantir ou céder que \montant{400 000}.
					\end{exemple}

					Cette disposition interdit au banquier d’accepter une cession ou créance supérieur au montant de son obligation



				\subparagraph{La sanction de l'obligation : l'inopposabilité de la cession de la créance excessive}

					Inopposabilité de la cession excessive, si la banque reçoit du \MO une somme supérieure de la part de l’\ep.\aVerifier

					\begin{exemple}
						dans le cadre d'un marché marché de \montant{1 000 000}, l'\ep soustraite pour \montant{600 000} et effectue \montant{400 000} de travaux personnellement, il peut donc céder ou nantir \montant{400 000}. S'il cède  \montant{500 000}, il y a une créance excessive de \montant{100 000},  et le \ST peut agir à l’encontre de la banque pour obtenir \montant{100 000} de celle-ci.
					\end{exemple}

					On s’est aperçu que les garanties de paiement n’étaient pas efficaces donc mise en place d’obligations à l’égard du MO



		\subsubsection{Les obligations imposées au \Mo}

			L'article 14-1 de la loi \printdate{6/1/1986} \aVerifier stipule que le \Mo doit vérifier que les obligations de l’entrepreneur ont été respectées, car il détient le pouvoir économique sur l’opération.

			\paragraph{Le champs d'application des obligations du \Mo}

				\subparagraph{La nature du contrat} La loi s'applique pour les marchés privés et publics.

				\subparagraph{Conditions d’application }
Il s’agit de conditions cumulatives :\index{SousTraitance@\ST!Conditions}
				\begin{enumerate}
					\item \textbf{Le type de prestation} : l'article 14-1 s’applique aux travaux de bâtiment, à cet égard la \CourDeCas a une vision extensive et les travaux de démolition peuvent avoir la notion juridique de bâtiment\footnote{\jurisCourDeCas{\civTrois*}{24/09/2014}\aVerifier}.


					Elle s’applique également à la sous-traitance industrielle


					\item 	\textbf{La qualité du \MO} : l'article s'applique aux \MO personnes morales et personnes physiques professionnels. Ainsi, le \MO personne physique qui fait construire pour lui-même et sa famille n’est pas tenue des obligations.


					\item 	\textbf{Le \MO doit avoir connaissance de l’existence du \ST sur le chantier}. La connaissance est un fait juridique qui se prouve par tout moyen. C’est le ST qui doit prouver que le MO savait qu’il intervenait sur le chantier. Pour cela il peut se fonder sur l’indication du ST sur le CR de chantier, la présence de véhicule avec logo sur le chantier.

					\item 	\textbf{Le \MO ne doit pas avoir payé intégralement l’\ep} au jour de la connaissance de la présence du \ST sur le chantier

				\end{enumerate}


				Le \ST n’a pas d’obligation de se déclarer auprès du \MO, il ne peut se voir reprocher de ne pas s’être rapproché du \MO. Néanmoins on peut recommander au \ST de se manifester au \MO pour bénéficier des garanties de la loi.
En effet,  aux termes de l'article 14-1, le \MO en est tenu, dès qu’il en a connaissance, alors même que le chantier est terminé ou que le \ST n’est pas ou plus sur le chantier, tant que le marché n’est pas soldé. Cet article s'applique en cas de \ST* en chaine, le \MO restant toujours le même.


			\paragraph{L'étendue des obligations}

				Il existe 2 types obligations sur le \MO, qui sont prévues aux
article 3 et 14.

				\subparagraph{Première obligation}
				Dans le cas d'un \ST qui n’a pas été présenté ni ses conditions de paiement agréées, celui-ci doit mettre en demeure l’\ep de respecter ses obligations.

				Le \MO est alors libre d’accepter ou non le \ST sauf abus de droit.

				\subparagraph{Deuxième obligation}

				Si le \MO a accepté ou agréé le \ST il doit exiger de l’\ep de fournir une garantie de paiement.

				Obligation de vérification de la caution bancaire, et vérification de la preuve de la transmission au \ST de l’identité de l’établissement financier.

				Obligation de résultat du \MO, pas juste \MED, il doit faire pression sur l’entrepreneur principal d’obtenir la preuve de la garantie de paiement, par la menace de résiliation du contrat, le \MO n’est pas fautif s’il résilie le contrat car c’est en raison de la faute de l’\ep.

			\paragraph{La sanction des obligations : la responsabilité pour faute du \Mo}

				\subparagraph{Responsabilité pour faute du \MO.} Le \MO engage sa responsabilité délictuelle à l’égard du \ST s’il n’obtient pas la preuve de garantie de paiement ou de délégation.

				Il appartient au \ST de caractériser la faute, le préjudice, le lien de causalité.

				\subparagraph{Faute}
				Caractérisée à défaut de \MED de l’\ep.


				La \CourDeCas a pu juger que le \MO engage sa responsabilité, après \med restée infructueuse, s’il ne met pas tout en œuvre pour obtenir la preuve des garanties nécessaires.


				\subparagraph{Préjudice}



				La question s’est posée de savoir si le dommage consistait en la perte de chance d'être payé ou le non-paiement.
 La \JP a considéré que le préjudice est l’impossibilité d’être payé, et donc que le dommage réside dans le fait que le \ST n’a pas pu être payé..

				Il y alors réparation intégrale de sa créance.

				Le \ST peut agir à l’encontre du \MO sans justifier l’impossibilité de recouvrir sa créance à l’égard de l’entrepreneur principal, c’est une action directe sur le \MO.

				En cas d’annulation du contrat de \ST*, le règlement de \DI peut être fixé à hauteur du montant des travaux réalisés\footnote{\jurisCourDeCas{\civTrois*}{18/02/2015}}

				\subparagraph{Lien de causalité}

				La faute du \MO qui a privé \ST du bénéfice de garantie qu’il lui aurait permis d’être intégralement payé.

				\bigbreak \textbf{Attention : limite à l'action du \ST} Le \ST ne peut plus agir s’il a été intégralement payé sauf si nullité et que le montant des travaux dépasse le marché à forfait\footnote{\jurisCourDeCas{\civTrois*}{1/1/1}\aVerifier}

				Elle tient à la fois à la date où le \MO a connaissance de l’existence du \ST sur le chantier \textbf{et} sur le montant des sommes qu’il doit encore à l’entrepreneur principal le jour de la connaissance.

				Si le \MO doit \montant{300 000} à l’entrepreneur au moment où il a connaissance du \ST sur le chantier, alors l’action du \ST sur le \MO repose sur cette assiette, même s’il le \ST avait une créance de \montant{500 000}.


				Attention si entre le jour où il a connaissance de la présence du \ST et l’action du \ST le \MO a payé l’entrepreneur il peut être condamné a payé 2 fois, la connaissance a pour effet de cristalliser les sommes entre ces mains.\aVerifier

				Il aura la possibilité de se retourner contre l’entrepreneur principal sur le fondement de la répétition de l’indu mais si en faillite alors il devra déclarer sa créance.


				Prescription de l’action du \ST sur le \MO est de 5 ans à compter de la connaissance de l’existence du \ST.

				Le \MO peut voir sa responsabilité engagée si en cas de condamnation, alors il a une action récursoire sur le maitre d’œuvre qu’il s’est abstenu d’alerter le \MO de la présence du \ST sur le chantier\footnote{\jurisCourDeCas[09-11562]{\civTrois*}{10/02/2010} : responsabilité du Maitre d’œuvre faute d’avoir alerté la présence de \ST ; \jurisCourDeCas[13-24892]{\civTrois*}{10/12/2014} : responsabilité du Maitre d’œuvre engagée si abstenu d’avertir le \MO de la présence du \ST \textbf{et} faute de lui avoir conseillé de respecter ses obligations \cad de se le faire présenter à l’acceptation et à l’agrément}.



	\subsection{La protection subsidiaire du sous-traitant : l'action directe}

		Les article 11 à 13
sont d'ordre public, il n'est pas possible de renoncer à l’action directe. Le sous-traitant y bénéficie dès lors que les conditions sont réunies.

		\subsubsection{Les conditions de l'action directe}

			\paragraph{Les marchés concernés}

				Ils sont visés à l'article 11 de la loi : à tous les contrats de sous-traitant qui ne rentrent pas dans le champ d’application du paiement direct.

				Le paiement direct est prévu par le titre 2 de la loi en son article 4 : aux marchés publics passés en application de l’ordonnance du 23 Juillet 2015 d’un montant supérieur à 600 € = champ d’application du paiement direct. \aVerifier

				\medbreak Ce qui permet de déterminer le champ d’application de l’action directe :

				\begin{itemize}
					\item Les marchés privés

					\item Les marchés publics d’un montant inférieur à \montant{600}
					\item Les marchés conclus par une personne publique d’un montant inférieur à \montant{600}

				\end{itemize}

			\paragraph{Les conditions relatives à l'auteur de l'action directe}

				L'\AD ne bénéficie qu’au sous-traitant accepté et dont les conditions de paiement ont été agréées par le \MO.

				dans l''article 3 de la loi de 1975, le législateur a précisé que l'acceptation et l'agrément sont une condition de l’action directe.

				Ces conditions sont en pratique très souvent non remplies, et cela prive le sous-traitant d’agir sur ce fondement. C’est donc première limite fondamentale.

				La \CourDeCas en est consciente et a développé une \JP favorable au \ST. Le \ST peut faire l’objet d’un agrément tacite et tardif
:
				\begin{itemize}
					\item \textbf{L’agrément tacite} est caractérisée lorsqu’il existe un acte de volonté non équivoque du \MO\footnote{\jurisCourDeCas[16-10.719]{\civTrois*}{18/05/2017} : une MED de fournir une garantie principale de paiement adressé par le MO à l’entrepreneur suivi d’effet accompagné \aVerifier vaut acceptation du \ST (arrêt d’espèce)}. Il faut donc un acte positif, une simple attitude passive étant insuffisante.

					Cet agrément tacite peut être aménagée contractuellement. Ainsi l'article 4.4.1 de la norme AFNOR stipule que si le \MO n’a pas répondu dans un délai de 15 jours après la \med de l'\ep par \LRAR, il y a alors acceptation tacite.

					\item \textbf{L'agrément tardif} a été accepté par la troisième chambre civile. L'agrément reste possible jusqu’au moment où l’action directe est exercée alors même que l’\ep est en faillite.

					En pratique si le \ST exerce l’\AD et que le \MO ne relève pas l’absence d’agrément, les juges n’ont pas à relever d’office cette absence. Donc ne pas soulever ce moyen de défense équivaut à une acceptation tardive.
				\end{itemize}

				L’action directe bénéficie au \ST direct et au \ST en chaîne, donc conditions cumulatives. Le \MO demeure toujours le même le \ST de second rang peut agir à l’égard du \MO sur le fondement de l’\AD s’il a été accepté et agréé selon les conditions déjà vues.


				L’établissement financier qui a fourni le cautionnement selon l’article 14 de la loi du \printdate{31/12/1975 } peut exercer l'\AD\footnote{\jurisCourDeCas[16-10719]{\civTrois*}{18/05/2017}}, avec toujours la condition d’acceptation et d’agrément, l’établissement étant subrogé dans les droits du \ST.



			\paragraph{Les conditions de mise en œuvre}

				L'article 12, premier alinéa, de la \loiST, l'\AD s'exerce après \MED de l’\ep par le \ST.

				il y a trois conditions cumulatives :

				\begin{enumerate}
					\item Le \ST doit mettre en demeure l’entrepreneur principal de payer. Même s’il est en liquidation judiciaire (Civ 3°)
\aVerifier

					En la forme d’une LRAR, un acte d’huissier. Une lettre simple ne suffit pas.

					\item Le \ST doit envoyer une copie de la \MED au \MO. Cela a pour effet de bloquer entre les mains du \MO l’argent qu’il doit encore à l’\ep.

					\item La \MED doit demeurer infructueuse à l’expiration d’un délai d’un mois.
				\end{enumerate}

				L’action directe subsiste même si l’entrepreneur principal est en liquidation ou redressement judiciaire\footnote{Article 12 alinéa 3}.


		\subsubsection{Les effets de l'action directe}

			Ils sont précisés à l’article 13 de la \loiST.

			\paragraph{Les créances garanties}

				Seules sont garanties les créances de travaux stipulés dans le contrat de \ST et qui bénéficient au \MO.

				Sur un chantier il y a des travaux supplémentaires qui sont par principe pas prévus au contrat initial. L'\AD st possible s’ils sont opposables au \MO au sens de l’\articleDu{793}{\cciv}.


			\paragraph{L'assiette de l'action directe}

				L’action est limitée au somme restant due par le MO à l’entrepreneur principal, moins les sommes dues par l’\ep au \MO


				\subparagraph{Les sommes dues par le \MO}
				Ce sont les sommes dues en vertu du marché au jour de la réception de la \MED, toute cause confondue au contrat de marché principal.

				Le \ST ne peut pas réclamer les sommes dues par le \MO à l’entrepreneur principal au titre d’un autre contrat, juste au titre du contrat principal qui a fait l’objet du contrat de \ST*\footnote{\jurisCourDeCas[16-10719]{\civTrois*}{18/05/2017}}.

				\subparagraph{Les sommes dues par l’entrepreneur au \MO}
				Si l’entrepreneur n’a pas respecté les délais donc possibilité d’opposer au \ST les pénalités de retard, donc opposabilité des exceptions tirées de la relation \MO - \ep.

				Cela peut être aussi le cas pour non-respect du marché (non façon ou malfaçon), le \MO peut aussi faire jouer l’exception de compensation.

				Si plusieurs \ST exercent en même temps une \AD à l'encontre du \MO pour un montant supérieur qu’il doit à l’entrepreneur principal alors répartition à part égale entre les \ST, en proportion de leur créance respective, à défaut c’est « le premier arrivé ».



			\paragraph{Le versement direct du crédit au sous-traitant}

				Le \ST peut se faire verser directement par la banque le montant du crédit consenti au \MO\footnote{Article 12, alinéa 4 de la \loiST}.

\section{La protection du sous-traitant dans les marchés de travaux publics}

	Il s'agit principalement du mécanisme du paiement direct.

	\subsection{Les sous-traitants directs}

		\aCompleter

		\subsubsection{Le paiement direct}

			\aCompleter

			\paragraph{Les conditions du paiement direct}

				\aCompleter

			\paragraph{Les modalités du paiement direct}

				\aCompleter

		\subsubsection{Les autres garanties}

			\aCompleter

	\subsection{Les sous-traitants en chaine}

		\aCompleter

\section{La fin du contrat de sous-traitance}

	Le contrat peut être annulé pour défaut de capacité, ou vice de consentement, ou contenu illicite.

	\paragraph{Nullité} Le défaut de garantie de paiement par l’entrepreneur principal est une cause de nullité spécifique.

	\paragraph{Résolution} Par application d’une clause de résolution de plein droit. Aussi résolution unilatérale ou judiciaire possible. Également en cas de force majeure ou de circonstances imprévisibles.


	Cf. section \ref{Resolution}

	\paragraph{Résiliation} Le contrat de \ST* peut être résilié par application d’une clause de résiliation de plein droit ou décès de l’une des parties

	Le sous-traité peut également être frappé de caducité en cas de disparition du contrat principal.


	Cf. section \ref{Resiliation}

	\paragraph{Caducité} La caducité est définie aux articles 1186 et 1187 du \cciv.

	\begin{citationArticle}{1186}{\cciv}
		Un contrat valablement formé devient caduc si l'un de ses éléments essentiels disparaît.

		Lorsque l'exécution de plusieurs contrats est nécessaire à la réalisation d'une même opération et que l'un d'eux disparaît, sont caducs les contrats dont l'exécution est rendue impossible par cette disparition et ceux pour lesquels l'exécution du contrat disparu était une condition déterminante du consentement d'une partie.

		La caducité n'intervient toutefois que si le contractant contre lequel elle est invoquée connaissait l'existence de l'opération d'ensemble lorsqu'il a donné son consentement.
	\end{citationArticle}

	\begin{citationArticle}{1187}{\cciv}
		La caducité met fin au contrat.

		Elle peut donner lieu à restitution dans les conditions prévues aux articles 1352 à 1352-9.
	\end{citationArticle}

	La caducité met fin au contrat et donne lieu à restitution dans les condition de l'\articleDu{1352-8}{\cciv} : en valeur à la date à laquelle la prestation a été fournie.


	Le quantum des restitutions consécutives à la caducité du contrat de \ST doit correspondre au coût de la prestation réalisée sans tenir compte du coût contractuellement convenu et sans tenir compte de la valeur de l’ouvrage réalisé.


	\paragraph{Réception} Le contrat de \ST* peut également prendre fin par le prononcé de la réception.



\section{Les règles de responsabilités liées à la sous-traitance}

	C’est la \JP qui a dégagé ce régime.

	\subsection{La responsabilité du sous-traitant à l'égard de l'\ep}

		\subsubsection{Nature et fondement de la responsabilité}

			Le \ST est un co-contractant de l’\ep donc responsabilité contractuelle.
Ce sont donc les règles du contrat d'entrepreneur qui s'applique, et en particulier l'\articleDu{1788}{\cciv} sur la perte de la chose.

			\begin{citationArticle}{1788}{\cciv}
				Si, dans le cas où l'ouvrier fournit la matière, la chose vient à périr, de quelque manière que ce soit, avant d'être livrée, la perte en est pour l'ouvrier, à moins que le maître ne fût en demeure de recevoir la chose.
			\end{citationArticle}

			\paragraph{Obligation de conseil} Le \ST est tenu d’une obligation de conseil à l’égard de l’\ep il ne doit pas se contenter de suivre les instructions de l’\ep s’il les estime contraires aux règle de l’art et le conseiller s’il estime devoir procéder différemment.


			\paragraph{Les \garSpec} Le \ST n’est pas débiteur des garanties spécifiques des constructeurs (\lesGarSpec), seuls les contractants avec le \MO y sont débiteurs.


			Mais il est possible de contractualiser les garanties spécifiques des constructeurs, et dans ce cas l’entrepreneur principal à la possibilité d'agir à l’encontre du \ST sur le fondement de ces garanties spécifiques.


			A défaut, le \ST n’est débiteur que de la responsabilité contractuelle, qui suppose en principe un lien de causalité : existence d’une faute, d'un préjudice et lien de causalité entre la faute et le préjudice\index{LienCausalité@Lien de causalité}.
Cependant la \JP a considéré que le \ST est soumis à une obligation de résultat. Il n'est donc pas nécessaire de démontrer une faute, un simple dommage suffit à engager sa responsabilité\footnote{\jurisCourDeCas{\civTrois*}{02/02/2017} ; \jurisCourDeCas{\civTrois*}{26/04/2006}}.

			L'obligation de résultat subsiste même si le \ST n'est pas accepté et agréé, et même si le \ST n’a pas été payé.

			\medbreak Le \ST est responsable en cas de dommage, de défaut de conseil, de retard de travaux, vice de construction ou de défaut de conformité.

			En cas de condamnation de l’entrepreneur principal sur les \TAV il peut se retourner contre le \ST s’il démontre la preuve d’une faute du \ST\footnote{\jurisCourDeCas{\civTrois*}{26/04/2006}}. %alors action récursoire.\aVerifier

			La responsabilité du fournisseur du \ST à l’égard de l’\ep est de nature contractuelle.



		\subsubsection{Le régime de la responsabilité}

			Il s'agit du délai de droit commun. Le recours d’un constructeur contre un autre constructeur ou un \ST
repose sur l'\articleDu{2224}{\cciv} : 5 ans à compter du jour où le premier a connu ou aurait dû connaitre les faits lui permettant de l’exercer.

			La \CourDeCas fixe le point de départ de la prescription le jour où l’entrepreneur principal a été actionné par le \MO. Soit la date de l’assignation en référé expertise, soit la date d’assignation au fond\footnote{Néanmoins, \jurisCourDeCas{\civTrois*}{13/09/2006} fixe le point de départ au jour où le dommage s’est manifesté à l’égard du \MO. Cet arrêt date toutefois d'avant la réforme du droit des obligations et de la réforme de la prescription et des décisions qui fixe le point de départ au jour de l’assignation}.


			L’action récursoire de l’entrepreneur principal sur le fondement des \TAV, alors 5 ans à compter de l’action\footnote{\jurisCourDeCas[18-25915]{\civTrois*}{16/01/2020} --- 3 arrêts attendus rendu le même jour concernant les \TAV et l'action récursoire}.


			\textbf{Pour les sinistres avant réception} on s’interroge entre l’application du délai de droit commun : \articleDu{2224}{\cciv}, et celui du \ccom : \articleDu[L]{110-4}{\ccom}. Dans tous les cas, le délai est de cinq ans.

			Les causes d’exonération sont les causes étrangères\index{CausesEtrangeres@Causes étrangères}\label{causesEtrangeres} : cas de force majeure, fait d’un tiers ou fait de la victime.



	\subsection{La responsabilité du sous-traitant à l'égard du \Mo}

		\subsubsection{Nature et fondement de la responsabilité}

			La \civUn a pu juger qu'il s'agit d'une responsabilité de nature contractuelle car présence d'une chaine homogène de contrat. Cependant, la \civTrois a jugé qu'il s'agit d'un responsabilité délictuelle. En \assPlen\footnote{\jurisCourDeCas{\assPlen}{12/07/1991}, arrêt \nom{Besse}}, la \CourDeCas a tranché en faveur de la nature délictuelle de la responsabilité du \ST.

			L’engagement de la responsabilité du \ST impose la démonstration : d’une faute, d’un préjudice et d’un lien de causalité entre les deux.

			Le \MO ne peut pas invoquer une obligation de résultat mais la faute peut trouver son origine dans le sous-traité

			La \CourDeCas considère qu’en dépit de l’effet relatif des contrat, le tiers peut invoquer un manquement contractuel dès lors que ce manquement lui a causé un préjudice, la faute contractuelle devient une faute délictuelle à l’égard d’un tiers AP 06/10/2006 AP 13/01/2020 qui est juste une confirmation.


			Le \MO peut alors se prévaloir au plan délictuel de l’inexécution du contrat entre l’entrepreneur principal et son exécutant ainsi que des fautes d’inexécution.

			La faute contractuelle du \ST à l’égard de l’entrepreneur principal est donc une faute délictuelle à l'égard du \Mo.

			Civ 3° 18/05/2017, le seul manquement à une obligation de résultat est impropre à caractériser l’existence d’une faute délictuelle

			En matière de \TAV le \MO dispose d’une action récursoire à l’encontre du ST, c’est une action fondée sur le fondement du TAV si le MO a indemnisé le voisin, alors il est subrogé dans les droits du voisin pour agir à l’encontre du ST.

			L’action est en revanche de nature délictuelle si le MO n’a pas indemnisé le voisin (le mécanisme de subrogation intervient qu’en cas de paiement) le MO doit alors apporter la preuve d’une faute.

			Le MO peut rechercher la responsabilité du ST sur le fondement de la responsabilité des produits défectueux.1245 et suivants du code civil

			La responsabilité du fournisseur du ST à l’égard du \MO est de nature délictuelle Civ 3° 21/11/2001



		\subsubsection{Le régime de la responsabilité}

			\paragraph{Prescription}
			Par 10 Ans à compter de la réception des travaux pour les dommages relevant de la garantie décennale

			2 Ans à compter de la réception des travaux 1792-3

			Sinon 1792-4-3 Code civil

			Ici on s’intéresse à la nature du désordre et pas sur le fondement juridique.

			Ici pas réception de l’ouvrage mais des travaux. La réception doit-elle intervenir par travaux ou sur l’ouvrage ?

			En pratique la question ne se pose pas toujours mais dans la rédaction du contrat « délai à compter de la réception de l’ouvrage ou des travaux dans la globalité »

			L’action récursoire du MO à l’encontre du ST dont l’activité a engendré un TAV, doit être engagée 5 ans à compter du jour où le MO a assigné le voisin.

			Pour les sinistres avant réception : 1224 et L210-4 du Code de commerce : 5 Ans à compter de la survenance du sinistre

			Les causes d’exonération sont les causes étrangères\footnote{Cf. \vref{causesEtrangeres}}.


	\subsection{La responsabilité de l'\ep à l'égard du \Mo}

		\subsubsection{Nature et fondement de la responsabilité}

			Ici responsabilité contractuelle

			L’entrepreneur principal sous-traite sous sa responsabilité et est donc responsable des fautes de son sous-traitant.

			La simple faute du ST à l’égard de l’entrepreneur principal suffit à engager la responsabilité de l’entrepreneur principal à l’égard du MO

			Civ 3° 11/05/2006

			L’entrepreneur principal est responsable des fautes de son ST à l’égard du MO et de ses acquéreurs successifs.

			La responsabilité de l’entrepreneur peut être recherché sur le fondement spécifiques des constructeurs.

			En matière de TAV le MO peut agir à l’encontre de l’entrepreneur principal sur le fondement délictuel du TAV s’il a payé la victime, et s’il n’a pas indemnisé le voisin alors que sur le fondement contractuel.



		\subsubsection{Le régime de la responsabilité}

			La prescription

			Le délai d’action varie en fonction de la nature du dommage subi par le \MO.


			Si les conditions des garanties spécifiques sont réunies alors il doit agir sur ce fondement et alors la forclusion est de 10 /2 / 1 ou 10 ans pour tous les autres 1792-4-3 du Code civil


			Pour les désordres avant réception, alors dans un délai de 5 ans à compter de la manifestation du dommage


			Les causes d’exonération sont les causes étrangères\footnote{Cf. \vref{causesEtrangeres}}.



	\subsection{La responsabilité de l'\ep à l'égard des tiers}

		\subsubsection{Nature et fondement de la responsabilité}

			L’\ep n’est pas délictuellement responsable des fautes de son \ST à l’égard des tiers\footnote{\jurisCourDeCas{\civTrois*}{8/09/2009} ; \jurisCourDeCas{\civTrois*}{22/09/2010} : l’entrepreneur principal n’est pas responsable des dommages au tiers par le ST dès lors qu’il n’est pas son commettant}. S’il sous-traite sous sa responsabilité c’est à l’égard du \MO mais pas à l’égard des tiers.


			Mais il est possible de mettre en cause la responsabilité de l’entrepreneur principal si le tiers démontre une faute personnelle de celui-ci, par exemple un défaut de surveillance. Alors le tiers peut invoquer la réalisation défectueuse de son contrat\footnote{\jurisCourDeCas{\civTrois*}{27/03/2008}}.


		\subsubsection{Le régime de la responsabilité}

			Prescription : 5 Ans à compter de la manifestation du dommage 2224 du Code civil

			Les causes d’exonération sont les causes étrangères\footnote{Cf. \ref{causesEtrangeres}}.

	\subsection{La responsabilité du sous-traitant à l'égard des tiers}

		Tiers : toute personne extérieure au chantier et tout intervenant sur le chantier extérieur à la relation de sous-traitance (différent du tiers voisin, ici tiers au chantier).

		\subsubsection{Nature et fondement de la responsabilité}

		Les voisins peuvent agir à l’encontre du \ST sur le fondement des \TAV, aux côtés de l’entrepreneur principal ou non, aux côtés du \MO ou non
.

		Il appartiendra au demandeur de démontrer un lien d’imputabilité entre l’activité du \ST et le dommage.


		\bigbreak Le tiers voisin peut aussi agir sur le fondement délictuel de l'\articleDu{1241-1}{\cciv}, mais en pratique il agira sur le fondement des  \TAV car plus simple car sinon il faut démontrer une faute, un lien de causalité, et un préjudice. Dans les \TAV juste un lien d’imputabilité


		\bigbreak Le voisin victime peut agir sur le fondement de l'\articleDu{1242}{\cciv} sur la responsabilité du fait des choses, TAV aussi

		Les colocataire d’ouvrage peuvent agir à son encontre sur le fondement de la responsabilité délictuelle en démontrant la faute


		\subsubsection{Le régime de la responsabilité}

		Prescription : 5 Ans à compter de la manifestation du dommage 2224 du Code civil

		Les causes d’exonération sont les causes étrangères
