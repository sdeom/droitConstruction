% !TEX root = ./droitConstruction.tex

\chapter{Les opérateurs de la construction}

	\section{Le Maître de l'Ouvrage}

		Le \Mo{} est toute personne disposant d'un droit l'habilitant à construire et qui commande des travaux.

		Le \Mo{} construit pour lui-même ou pour vendre.

		La \Mo{} délégué est un mandataire. Il est parfois considéré comme un constructeur, mais il relève surtout du droit des mandats.

	\section{Les Locateurs d'Ouvrage}

		Les locateurs d'ouvrages sont tous ceux qui participent à la construction et qui interviennent de manière directe ou indirecte.

		\subsection{L'architecte}

			L'\archi{} (ou \Moe) conçoit et le cas échéant conduit les travaux (mais ce n'est pas obligatoire).

			Il s'agit d'une profession libérale, l'\archi{} perçoit donc des honoraires.

			La profession est réglementée par la loi du \printdate{3/1/1971}. Elle est organisée en ordre professionnel et dispose de déontologie.

			Elle dispose d'un monopole sur le projet architectural dès que la superficie construite est supérieure à \surface{150}  (\articleDu[L]{431-3}{\cu})). Un arrêt du \CE{} donne qualité à l'Ordre pour attaquer un \PC{} qui ne respecterait pas cette disposition.

			L'architecte est soumis à l'obligation d'assurance\footnote{Attention au mécanisme de réduction proportionnel}, et son absence constitue une faute professionnelle.

			\begin{conseil}
				Demandez la preuve de déclaration à l'assureur et vérifier qu'elle porte sur l'intégralité du chantier. Il est possible de téléphoner à l'assurance si l'attestation est peu disserte.
			\end{conseil}

		\subsection{L'entrepreneur}

			L'\E{} est celui qui réalise matériellement les travaux, qu'il ai contracté directement avec le \Mo{}, qu'il soit cocontractant ou sous-traitant. Dans ces derniers cas, les règles de responsabilité diffèrent.

			Il est possible de créer un << \GME{} >> qui désigne alors un mandataire, dont la responsabilité pourra être conjointe ou solidaire. Le \GME{} n'a pas la personnalité morale, le \Mo{} agit donc contre l'ensemble de ses membres.

		\subsection{Les ingénieurs et les bureaux d'études}

			Un contrat avec un ingénieur peut également constituer un contrat de louage d'ouvrage (ex. : contrat de maîtrise d'œuvre d'exécution).

			L'ingénieur ou le bureau d'étude est alors également soumis à l'obligation d'assurance décennale.

		\subsection{Le contrôleur technique}

			La profession de contrôleur technique est apparue en 1929, suscitée par les assureurs. Elle a pour mission la prévention.

			Elle a été réglementée en 1978 par la loi \nom{Spinetta}. Elle est réglementée par les articles \articleCodifie{L}{111-23} à \articleDu[L]{111-26}{\cch}.

			Elle est incompatible avec l'exercice d'autres profession (conception et exécution).

			\subsubsection{La convention de contrôle}

				Le contrôleur technique donnant son avis sur la construction, il s'agit d'un contrat de louage d'ouvrage.

				\paragraph{Le recours au contrôle}

					Le recours au contrôle peut être :
					\begin{itemize}
						\item incité,
						\item spontané,
						\item ou obligatoire.
					\end{itemize}

					\subparagraph{Le recours obligatoire} est prévu par l'\articleDu[R]{111-38}{\cch}, et notamment pour :
					\begin{itemize}
						\item les \emph{Établissement Recevant du Public} (\ERP), en application de l'\articleDu[R]{123-2}{\cch} ;
						\item les \emph{Immeubles de Grande Hauteur} (\IGH), conformément à l'\articleDu[R]{122-2}{\cch} ;
						\item les bâtiments autres qu'industriels qui présentent certaines contraintes techniques\footnote{ex. : tribunes}.
					\end{itemize}

					\subparagraph{Le recours incité} l'est par les assureurs en établissant un calcul de prime décennale en fonction du contrôleur technique.

				\paragraph{L'objet du contrôle}

					\begin{citationArticle}[L]{111-23}{\cch}
						Le contrôleur technique a pour mission de contribuer à la prévention des différents aléas techniques susceptibles d'être rencontrés dans la réalisation des ouvrages.

						Il intervient à la demande du maître de l'ouvrage et donne son avis à ce dernier sur les problèmes d'ordre technique, dans le cadre du contrat qui le lie à celui-ci. Cet avis porte notamment sur les problèmes qui concernent la solidité de l'ouvrage et la sécurité des personnes.
					\end{citationArticle}

					Le contrôleur technique contrôle \textbf{à la conception}. Les documents de la construction lui sont soumis. Il ne se prononce que sur la conformité des plans avec la réglementation.

					Le contrôleur technique contrôle \textbf{lors de l'exécution}. Il se prononce sur les plans d'exécution, mais en pratique il est fréquent qu'il se déplace sur le chantier.

					Le contrôleur technique n'émet que des avis. Il ne peut être tenu responsable si ses avis ne sont pas suivis.

			\subsubsection{La responsabilité du contrôleur technique}

				\paragraph{À l'égard du maître d'ouvrage}

					avant 1978, le contrôleur technique échappait à la responsabilité décennale\footnote{Articles 1792 et suivants du \cciv*}. Depuis, il rentre dans son champs d'application.

					L'\articleDu[L]{111-24}{\cch} pose une présomption de responsabilité dans les limites de sa mission.

				\paragraph{À l'égard des autres constructeurs}

					le contrôleur technique peut être mis en cause dans le cas de recours \emph{in solidum}.

					Il n'est tenu qu'à concurrence des limites de ses missions. Il n'est donc pas tenu de l'insolvabilité des autres co-obligés.

		\subsection{Le coordinateur santé et sécurité}

			Le contrat de coordinateur santé et sécurité (\CSPS) est un contrat de louage au sens de code civil de 1804.

			\subsubsection{L'intervention du coordonnateur}

			\subsubsection{La responsabilité du coordonnateur}

	\section{Les Assureurs Construction}

		Avant 1978, les assurances de construction étaient purement volontaires, sauf pour les \archi{}. Depuis 1978, il y a obligation pour tous ceux qui participent à l'acte de construction d'un bâtiment.

		Depuis 2005, toute personne qui construit un \textbf{ouvrage} est soumis à la \do{}. L'\E{} est soumis à la \rcd{}. Il a obligation de s'assurer sous peine de sanction civile et pénale.
