% !TEX root = ./droitConstruction.tex

\chapter{Le contrat de l'entrepreneur}

	\section{La qualification du contrat d'entreprise}

		\subsection{Distinction avec le contrat de travail}

			\subsubsection{Le critère : l'absence de subordination}

				Indépendance du \lo{}. Il n'y a pas de lien de subordination. Il décide des moyens à mettre en œuvre pour réaliser sa mission.

				Le \lo{} ne touche pas un salaire, il est payé à la tâche.

			\subsubsection{Intérêt de la distinction}

				Le régime social est différent de celui d'un salarié.

				Le régime de responsabilité est également différent. Le \Mo{} n'est pas un commettant, il n'y a donc pas de relation commis-commettant et par conséquent pas de responsabilité du \Mo{} à l'égard des tiers sur cette base.

				\begin{citationArticleCciv}{1242}
					On est responsable non seulement du dommage que l'on cause par son propre fait, mais encore de celui qui est causé par le fait des personnes dont on doit répondre, ou des choses que l'on a sous sa garde.

					\lips

					Les maîtres et les commettants, du dommage causé par leurs domestiques et préposés dans les fonctions auxquelles ils les ont employés ;

					\lips
				\end{citationArticleCciv}

				L'\E a une obligation de résultat. Le simple constat de l'absence de résultat engage sa responsabilité.

		\subsection{Distinction avec le contrat de mandat}

			\subsubsection{Le critère : l'absence de représentation}

				Le contrat de louage d'ouvrage ne donne pas lieu à représentation.

				\begin{conseil}
					Il faut toutefois faire attention aux cas où un contrat de mandat se superpose.
				\end{conseil}

			\subsubsection{Intérêt de la distinction}

			L'intérêt est double :
			\begin{itemize}
				\item d'une part pour le paiement des travaux commandés.
				\item d'autre part à la réception d'un ouvrage.
			\end{itemize}

			Dans le contrat de louage d'ouvrage le \Mo{} n'est pas lié par les contrats du \lo{}\footnote{Attention toutefois à l'application de la théorie du mandat apparent}, alors que dans le contrat de mandat le mandataire << s'efface >>.

			Dans le cadre d'un contrat de mandat, le mandant est débiteur d'une obligation de moyen. Dans le contrat de louage d'ouvrage, le \lo{} est débiteur d'une obligation de résultat et il est soumis à la responsabilité décennale.

		\subsection{Distinction avec la vente}

			Il s'agit de la distinction la plus délicate.

			\subsubsection{Le critère}

				\paragraph{La construction d'un immeuble\\}

				 Dans cette configuration il faut s'interroger sur l'identité du propriétaire du terrain.

				Si le client est propriétaire, il s'agit d'un contrat de louage d'ouvrage.

				Si le terrain appartient à l'\E{}, il s'agit d'un contrat de vente d'immeuble\footnote{Attention à la \VEFA}.

				\paragraph{Les travaux exécutés sur un immeuble\\}

				Dans ce cas, le critère est celui de la conception des produits fournis.

				Si la conception est le fait du donneur d'ordre et porte sur un produit individualisé, non substituable, alors il s'agit d'un contrat de louage d'ouvrage.

				Si le produit est standard, il s'agit d'un contrat de vente.

				Si le contrat porte sur des travaux et de la fourniture --- adaptation de produits << catalogues >> par exemple --- il faut alors mesurer l'importance de chaque prestation. Si la part de travaux matériels, appréciée d'une manière globale et pas seulement financière, est la plus importante, alors il s'agit d'un contrat de louage d'ouvrage.

			\subsubsection{Intérêt de la distinction}

				Le régime de responsabilité est différent. Dans le cadre d'un contrat de vente, elle s'appuie sur le vice caché. Dans le cadre d'un contrat de louage d'ouvrage elle repose sur l'\articleCciv{1792} et sur les garanties spécifiques.

		\subsection{Distinction avec les contrats de construction}

			La loi de 1967 a créé plusieurs types de contrat de construction.

			\subsubsection{Les types de contrats}

				\paragraph{La vente d'immeuble à construire\\}

				\begin{citationArticleCciv}{1601-1}
					La vente d'immeubles à construire est celle par laquelle le vendeur s'oblige à édifier un immeuble dans un délai déterminé par le contrat.

					Elle peut être conclue à terme ou en l'état futur d'achèvement.
				\end{citationArticleCciv}

				Le critère est celui de la propriété du terrain. Si le terrain appartient au << client >>, il s'agit d'un contrat de louage d'ouvrage. Si le terrain appartient au constructeur, il s'agit d'un contrat de vente d'immeuble à construire.

				L'idée est de protéger l'acquéreur avec un régime d'ordre public favorable au \Mo.

				\paragraph{Le contrat de promotion immobilière\\}

				\begin{citationArticleCciv}{1831-1}
					Le contrat de promotion immobilière est un mandat d'intérêt commun par lequel une personne dite << promoteur immobilier >> s'oblige envers le maître d'un ouvrage à faire procéder, pour un prix convenu, au moyen de contrats de louage d'ouvrage, à la réalisation d'un programme de construction d'un ou de plusieurs édifices ainsi qu'à procéder elle-même ou à faire procéder, moyennant une rémunération convenue, à tout ou partie des opérations juridiques, administratives et financières concourant au même objet. Ce promoteur est garant de l'exécution des obligations mises à la charge des personnes avec lesquelles il a traité au nom du maître de l'ouvrage. Il est notamment tenu des obligations résultant des articles 1792, 1792-1, 1792-2 et 1792-3 du présent code.

					Si le promoteur s'engage à exécuter lui-même partie des opérations du programme, il est tenu, quant à ces opérations, des obligations d'un locateur d'ouvrage.
				\end{citationArticleCciv}

				Le propriétaire du terrain confie au prestataire (le << promoteur immobilier >>) la réalisation d'un programme de construction d'un ou de plusieurs édifices \textbf{pour son compte}. Le promoteur immobilier peut réaliser lui-même tout ou partie des travaux.

				Il s'agit d'un contrat hybride, en partie contrat de louage d'ouvrage, en partie contrat de mandat.

				Le promoteur immobilier est considéré comme un constructeur au sens de l'\articleCciv{1792-1}. Il est donc soumis, en plus de la \rcdc, aux garanties spécifiques : \lesGarSpec. Il est également soumis à l'obligation d'assurance.

				\paragraph{Le contrat de construction des maisons individuelles (\CCMI)}

					est un contrat de louage d'ouvrage. Il est défini par les articles \articleCodifie[L]{231-1} et \articleCodifie[L]{232-1} du \cch.

					On lui applique les règles de droit commun en matière de louage d'ouvrage, auxquelles se rajoute des règles spécifiques visant à favoriser la protection de l'acquéreur.

					Un \ccmi{} doit être conclut impérativement si :
					\begin{enumerate}
						\item le terrain appartient au \Mo{} ;
						\item le contrat porte sur une maison individuelle ne comportant pas plus de deux logements ;
						\item soit l'\E{} s'engage à réaliser le gros œuvre, soit il propose les plans.
					\end{enumerate}

					\subparagraph{Intérêt de la distinction}

					L'intérêt réside une fois de plus dans le régime de responsabilité.

					Le constructeur de maison individuelle est débiteur des \garSpec{}, mais en plus, s'agissant d'un contrat d'ordre public, son formalisme doit être respecté à peine de nullité du contrat et de sanction pénale.

	\section{La formation du contrat}

		Le \Mo{} est libre du nombre de contrat. On parle de << marché en corps d'état séparé >> --- marché qui présente \emph{a priori} un avantage sur le prix et un inconvénient en matière de coordination des travaux --- lorsque celui-ci contractualise avec plusieurs entrepreneurs, et de marché << tout corps d'état >> dans le cas contraire --- marché qui présente \emph{a priori} un avantage en matière de coordination des travaux et un inconvénient sur le prix.

		\subsection{Les conditions de validité du contrat}

		Ces conditions sont fonctions du contrat et de la responsabilité, mais consistent principalement :
		\begin{enumerate}
			\item dans le consentement des parties (dont les trois types de vice sont : l'erreur, la violence et le dol) ;
			\item dans la capacité des parties ;
			\item et dans le caractère licite et certain du contenu du contrat.
		\end{enumerate}

		\begin{conseil}
			On se reportera utilement à consulter les ouvrages sur le droit des obligations qui abondent depuis la réforme de 2016.
		\end{conseil}

		\subsection{Le mode de conclusion du contrat}

			Il s'agit d'un contrat consensuel, librement consenti sans règle de forme, même si en pratique l'écrit est fortement recommandé\footnote{uniquement pour la preuve} notamment en matière d'étendue de la mission. L'écrit est obligatoire si le montant est supérieur à \montant{1 500}. On parle << d'acte d'engagement >>.

			Les contrats sont conclus soit de gré à gré, soit après mise en concurrence.

			\subsubsection{Les marchés de gré à gré}

			Ils sont définis à l'\articleCciv{1110} :
			\begin{quote}
				\itshape Le contrat de gré à gré est celui dont les stipulations sont négociables entre les parties. \lips
			\end{quote}

			\subsubsection{Les contrats conclus après mise en concurrence}

			Ce sont ceux qui résultent d'un appel d'offre.

			Il y a toujours possibilité de poursuivre un appel d'offre par une négociation de gré à gré.

		\subsection{Le contenu du contrat}

			Le contrat doit comporter des précisions quant :
			\begin{enumerate}
				\item au prix ;
				\item aux caractéristiques de l'ouvrage ;
				\item et à ses conditions juridiques et administratives.
			\end{enumerate}
			Par ailleurs, depuis 2016, certaines clauses sont réputées non écrites.

			\subsubsection{Le prix}

				Le prix est un élément essentiel au contrat. A défaut de prix, le contrat peut être requalifié en contrat d'assistance bénévole qui n'ouvre pas le droit aux \garSpec.

				\paragraph{Les règles générales}

				\emph{Quid} en absence de stipulation du prix ?

				Avant la réforme des obligations, le juge pouvait fixer le prix. Depuis la réforme, le prix est fixé par le l'\E (le créancier) en application de l'\articleCciv{1165}. Une contestation est possible en cas d'abus.

				Les clauses de variation du prix sont licites, tant en matière d'indexation --- variation du prix entre son évaluation et son paiement --- qu'en matière d'actualisation --- variation du prix entre la signature du contrat et le début des travaux.

				\paragraph{Le prix dans le marché à forfait}\index{MarcheForfait@Marché à forfait}

					Il s'agit du type de prix le plus courant. Il s'agit d'un prix fixé à l'avance et globalement pour des travaux dont la nature et la consistance sont précisément définies. Il est défini aux articles 1793 et 1794 du \cciv.

					\begin{citationArticleCciv}{1793}
						Lorsqu'un architecte ou un entrepreneur s'est chargé de la construction à forfait d'un bâtiment, d'après un plan arrêté et convenu avec le propriétaire du sol, il ne peut demander aucune augmentation de prix, ni sous le prétexte de l'augmentation de la main-d'œuvre ou des matériaux, ni sous celui de changements ou d'augmentations faits sur ce plan, si ces changements ou augmentations n'ont pas été autorisés par écrit, et le prix convenu avec le propriétaire.
					\end{citationArticleCciv}

					\begin{citationArticleCciv}{1794}
						Le maître peut résilier, par sa seule volonté, le marché à forfait, quoique l'ouvrage soit déjà commencé, en dédommageant l'entrepreneur de toutes ses dépenses, de tous ses travaux, et de tout ce qu'il aurait pu gagner dans cette entreprise.
					\end{citationArticleCciv}

					\subparagraph{Les conditions du marché à forfait}

					Elles sont au nombre de trois.

					\begin{enumerate}\index{MarcheForfait@Marché à forfait!Conditions}
						\item \textbf{Le prix global et définitif}. S'agissant d'un contrat consensuel, il n'y a pas de condition de forme. Toutefois, s'il est précisé que le prix est forfaitaire ou si \lips.

						Le marché à forfait est compatible avec l'indexation ou l'actualisation du prix.

						La jurisprudence admet la mixité des prix (en partie forfaitaire, en partie autre chose $\dots$)

						\item \textbf{La construction d'un bâtiment}. Il faut noter que les cours d'appel sont souple sur l'interprétation de ce critère, alors que la \CourDeCas est rigide à ce propos.	Par exemple, cette dernière a refusé le caractère de bâtiment au terrassement d'un bassin ou à la construction d'une piscine. Mais elle l'a reconnu pour une rénovation importante\footnote{\jurisCourDeCas[16-14317]{civ 3\ieme{}}{27/4/2017}}.

						\item \textbf{L'existence d'un << plan arrêté et convenu >>}. On entend par plan tous les documents contractuels qui définissent le bâtiment à réaliser, y compris un devis ou tout autre document descriptif.

						Le << plan arrêté et convenu >> doit l'être avec le propriétaire du sol.
					\end{enumerate}

					L'\articleCciv{1793} n'est pas d'ordre public, les parties peuvent donc convenir de ne pas s'y soumettre même si les conditions sont réunies. De manière similaire, il est possible de convenir de s'y soumettre alors même que les conditions ne sont pas réunies. Il faut cependant que l'intention des parties soit clairement exprimé\footnote{A titre d'exemple : arrondir le prix n'a pas été jugé suffisant --- \jurisCourDeCas{civ. 3\ieme{}}{25/3/2014}.}.

					Il est également possible d'aménager la sortie du forfait. Par exemple, le contrat peut prévoir des travaux supplémentaire possible avec un mandat donné à l'\archi.

					\subparagraph{La stabilité du marché à forfait\\}

					Le prix d'un marché à forfait est stable quel qu'en soit le motif\footnote{\jurisCourDeCas[06-18876]{civ. 3\ieme{}}{18/12/2007}}. Il ne peut pas être unilatéralement modifié même au motif d'un changement ou d'une erreur dans la description des travaux.

					Toutefois, le \Mo a pu obtenir remboursement des trop perçus au motif d'une erreur de métrés et dans le cas d'un contrat qui prévoyait leur vérification, au titre de la bonne foi et de l'obligation de conseil de l'\E.

					La réforme du droit des contrats a introduit dans le \cciv un nouvel article 1195\footnote{Ordonnance \no 2016-131 du 10 février 2016 - art. 2} :
					\begin{quote}
						\itshape Si un changement de circonstances imprévisible lors de la conclusion du contrat rend l'exécution excessivement onéreuse pour une partie qui n'avait pas accepté d'en assumer le risque, celle-ci peut demander une renégociation du contrat à son cocontractant. Elle continue à exécuter ses obligations durant la renégociation.

						En cas de refus ou d'échec de la renégociation, les parties peuvent convenir de la résolution du contrat, à la date et aux conditions qu'elles déterminent, ou demander d'un commun accord au juge de procéder à son adaptation. A défaut d'accord dans un délai raisonnable, le juge peut, à la demande d'une partie, réviser le contrat ou y mettre fin, à la date et aux conditions qu'il fixe.
					\end{quote}
					Cet article s'applique-t-il aux marchés à forfait ? Les marchés à forfait y échappent-ils en vertu de l'adage \emph{specialia generalibus derogant} ? En tout état de cause, il est possible d'écarter contractuellement l'application de cet article car il n'est pas d'ordre public.

					\bigskip Le principe est donc que le prix est intangible. L'exception est qu'il est possible de sortir du marché à forfait. Il faut distinguer si les travaux sont nécessaires ou non à la perfection de l'immeuble.
					\begin{itemize}
						\item \textbf{Les travaux supplémentaires nécessaires} sont ceux indispensables à la réalisation de l'ouvrage selon les règles de l'art --- par exemple : des travaux de soutènement --- ou indispensable à la sécurité de l'immeuble --- par exemple : mise en place de garde-corps.

						Ils doivent par principe être compris dans le forfait. L'\E ne peut pas demander une augmentation du prix, sauf si le \Mo est d'accord pour payer\footnote{Le juge peut déduire des éléments qui lui sont présentés que le \Mo était implicitement d'accord de manière non équivoque (volonté non équivoque visible).}.

						\item \textbf{Les travaux supplémentaires non nécessaires} sont les autres travaux.

						Par principe ils ne donnent pas lieu à paiement, sauf accord du \Mo (intangibilité du prix). Néanmoins leur paiement est possible :
						\begin{itemize}
							\item \textbf{conformément à l'\articleCciv{1793}}, s'il y a accord écrit préalable du \Mo\footnote{Et non du \Moe ou du \Mo délégué en l'absence de mandat express} sur les travaux et sur leur prix --- il est à noté que l'inscription des travaux sur un compte rendu de chantier ne suffit pas\footnote{Il serait possible d'imaginer invoquer la théorie << l'enrichissement injustifié >>, mais il serait vain de l'invoquer pour des travaux non nécessaires.} ;

							\item par appréciation souveraine du juge \textbf{en cas d'acceptation express et non équivoque \emph{a posteriori} des travaux par le \Mo}\footnote{Il s'agit d'une création praetorienne, invoquable dès lors que le \Mo a accepté sans équivoque les travaux supplémentaires après leur réalisation.} --- il a été ainsi jugé que << le paiement \lips vaut acceptation des travaux >> ;

							\item par appréciation souveraine du juge \textbf{en cas de bouleversement de l'économie du contrat}\footnote{Il n'y a pas de règle, mais 25 à \pourcent{30} semble être une règle tacite.}, notamment si le \Mo apporte en cours de travaux des modifications substantielles au marché initial d'une telle ampleur que celui-ci les a implicitement accepté\footnote{\jurisCourDeCas{civ. 3\ieme{}}{19/01/2017}}.
						\end{itemize}
					\end{itemize}

				\paragraph{Le prix dans le marché au métré\\}

				Le règlement est effectué en appliquant des prix unitaires, soit des prix unitaires spécialement établis --- on parle alors de << bordereau de prix >> --- soit des tarifs préexistant --- on parle dans ce cas de << série de prix >> --- soit sur un devis.

				Le prix global n'est pas défini à la signature mais devient définitif à la fin du marché.

				Il faut néanmoins apporter kla preuve de l'accord du \Mo, et l'on parle dans ce cas de << commande >> lorsque cet accord est antérieur aux travaux, et << d'acceptation >> lorsqu'il est postérieur.

				\paragraph{Le prix dans les marchés sur dépenses contrôlées\\}

				Également appelé << cost + fit contract >>. Les travaux sont rémunérés sur les dépenses réelles majorées d'un certain pourcentage pour les frais généraux.

				Ce type de contrat est utilisé lorsqu'il n'y a aucune visibilité sur les coûts des travaux --- par exemple : le tunnel sous la Manche. Il s'agit d'un contrat très risqué pour le \Mo, car l'aléa repose sur lui.

				Généralement, on fixe un prix cible et on prévoit comment l'acrt entre le prix réel et ce prix cible sera réparti, que l'écart soit favorable ou défavorable.

			\subsubsection{Les caractéristiques de l'ouvrage}

				Le contrat précise la nature, la consistance et la qualité des prestations. Cette composante s'apprécie avec l'ensemble des documents annexés ou visés dans le contrat.

				\begin{citationArticleCciv}{1166}
					Lorsque la qualité de la prestation n'est pas déterminée ou déterminable en vertu du contrat, le débiteur doit offrir une prestation de qualité conforme aux attentes légitimes des parties en considération de sa nature, des usages et du montant de la contrepartie.
				\end{citationArticleCciv}

			Par ailleurs, cette aspect peut évoluer grâce à des avenants.

			\subsubsection{Les conditions juridiques et administratives du marché}

			Le contrat peut fixer les conditions de réalisation des travaux et les modalités d'exécution du marché. Il peut renvoyer pour cela à des documents externes comme la norme AFNOR.

			Le contrat peut imposer des délais, une date butoir, ou encore une période butoir. Il peut prévoir des causes légitimes de suspension --- intempéries, grèves, \etc
			Les critères de la force majeure\footnote{L’événement doit avoir un caractère extérieur, imprévisible et inévitable (anciennement irrésistible). Cf. l'\articleCciv{1218}.} n'entrent pas en ligne de compte dans les causes de suspension légitime, mais les cas de force majeure peuvent faire partie des causes de suspension légitime.

			Le juge doit appliquer les stipulations contractuelles.

			\subsubsection{Les clauses réputées non écrites}

			\begin{citationArticleCciv}{1170}
				Toute clause qui prive de sa substance l'obligation essentielle du débiteur est réputée non écrite.
			\end{citationArticleCciv}

			A ce titre, sera réputée non écrite une cluse limitant ou exonérant de réparation.

			\bigskip Par ailleurs, il y a possibilité de requalification d'un contrat en contrat d'adhésion.
			\begin{citationArticleCciv}{1110}
				\lips

				Le contrat d'adhésion est celui qui comporte un ensemble de clauses non négociables, déterminées à l'avance par l'une des parties.
			\end{citationArticleCciv}


			\begin{citationArticleCciv}{1171}
				Dans un contrat d'adhésion, toute clause non négociable, déterminée à l'avance par l'une des parties, qui crée un déséquilibre significatif entre les droits et obligations des parties au contrat est réputée non écrite.

				L'appréciation du déséquilibre significatif ne porte ni sur l'objet principal du contrat ni sur l'adéquation du prix à la prestation.
			\end{citationArticleCciv}

	\section{Les effets du contrat}

		\subsection{Les obligations de l'\E}

			\subsubsection{L'obligation d'information et de conseil}

				Cette obligation est fondée sur le qualités professionnelles du \lo. Sa compétence est présumée totale et absolue. Il doit connaitre les règles de l'art, même non codifiées.

				Le \lo doit informer, conseiller et se renseigner --- notamment sur l'attente du \Mo. Il doit se renseigner sur la finalité de l'ouvrage, quand bien même il y a un \Moe. Toutefois, le \lo n'est pas tenu d'informer de ce qui est évident.

				Il est soumis à une obligation de moyen. La preuve est à la charge du \lo, sauf dans certains cas :
				\begin{itemize}
					\item si le \Mo est plus compétent que le constructeur\footnote{Appréciation souveraine du juge du fonds.} --- par exemple si s'agit d'un ouvrage très spécifique comme une clinique vétérinaire ;

					\item s'il y a plusieurs intervenants.
				\end{itemize}

				Il appartient au \lo de refuser des travaux dangereux ou inefficace. Il doit attirer l'attention sur les erreurs du \Moe, mettre en garde contre les propres choix du \Mo --- en particulier sur les conséquences négatives.

				Si les manquements à ses obligations se traduisent par un désordre, il y a basculement de la \rcdc à celles spécifiques des constructeurs. L'obligation de moyen se mue alors en obligation de résultat. Elle est renforcée en l'absence d'\archi.

				\bigskip Par ailleurs, il est bien sûr tenu par l'obligation d'information pré-contractuelle.

				\begin{citationArticleCciv}{1112-1}
					Celle des parties qui connaît une information dont l'importance est déterminante pour le consentement de l'autre doit l'en informer dès lors que, légitimement, cette dernière ignore cette information ou fait confiance à son cocontractant.

					Néanmoins, ce devoir d'information ne porte pas sur l'estimation de la valeur de la prestation.

					Ont une importance déterminante les informations qui ont un lien direct et nécessaire avec le contenu du contrat ou la qualité des parties.

					Il incombe à celui qui prétend qu'une information lui était due de prouver que l'autre partie la lui devait, à charge pour cette autre partie de prouver qu'elle l'a fournie.

					Les parties ne peuvent ni limiter, ni exclure ce devoir.

					Outre la responsabilité de celui qui en était tenu, le manquement à ce devoir d'information peut entraîner l'annulation du contrat dans les conditions prévues aux articles 1130 et suivants.
				\end{citationArticleCciv}


			\subsubsection{L'obligation de conservation des existants}

				Le \lo doit vérifier au préalable que les travaux sont compatibles avec les existants.

				\paragraph{En cas de sinistre avant la réception} on applique la << théorie des risques >> des articles 1788 et 1789 du \cciv.

				\subparagraph{Le \lo fournit la matière} la perte est donc pour l'\E jusqu'à la réception, sauf s'il a mit en demeure le \Mo de recevoir la chose.

				\begin{citationArticleCciv}{1788}
					Si, dans le cas où l'ouvrier fournit la matière, la chose vient à périr, de quelque manière que ce soit, avant d'être livrée, la perte en est pour l'ouvrier, à moins que le maître ne fût en demeure de recevoir la chose.
				\end{citationArticleCciv}

				\subparagraph{Le \lo fournit le travail} il n'est donc responsable qu'en cas de faute. Néanmoins la jurisprudence considère qu'il y a présomption de faute, qui n'est toutefois pas irréfragable.

				\begin{citationArticleCciv}{1789}
					Dans le cas où l'ouvrier fournit seulement son travail ou son industrie, si la chose vient à périr, l'ouvrier n'est tenu que de sa faute.
				\end{citationArticleCciv}

				Cette théorie s'applique également si il y a perte de l'ouvrage avant la réception.

				\paragraph{En cas de sinistre après la réception des travaux} La jurisprudence a admis l'application de l'\articleCciv{1792} dans deux hypothèses :
				\begin{enumerate}
					\item si les désordres sont causés aux existants indissociables des travaux neufs ;

					\item s'il n'est pas possible de déterminer si les dommages proviennent des travaux neufs ou des existants\footnote{Cette hypothèse vise principalement les cas d'effondrement.}.
				\end{enumerate}

				Dans les autres cas, on applique le droit commun, l'\E n'est tenu que par sa faute.

			\subsubsection{L'obligation d'exécution conforme}

				L'\E doit exécuter de qui a été commandé et qui a été stipulé dans le contrat.

				\paragraph{Le respect des délais convenus} implique que la livraison de l'ouvrage doit se faire à la date déterminée par le contrat --- l'\E étant tenu à une obligation de résultat.

				A défaut de précision, le contrat doit être réalisé dans un délai raisonnable, sauf :
				\begin{itemize}
					\item si la cause est légitime,
					\item si force majeure,
					\item si faute du \Mo --- retard dans les autorisations, modification constante du projet \etc,
					\item si faute d'un tiers --- mais pas d'un co-\lo.
				\end{itemize}
				Dans le cas contraire, des pénalités sont possibles si prévues au contrat. S'il n'ya pas de pénalités prévues, il reste possible d'en toucher sur la base :
				\begin{itemize}
					\item soit des << illisible >> dommages,
					\item soit des troubles de jouissance.
				\end{itemize}

			Les pénalités sont des clauses pénales, donc réductibles par le juge en vertu de l'\articleCciv{1231-5}\footnote{\itshape Lorsque le contrat stipule que celui qui manquera de l'exécuter paiera une certaine somme à titre de dommages et intérêts, il ne peut être alloué à l'autre partie une somme plus forte ni moindre.

				Néanmoins, le juge peut, même d'office, modérer ou augmenter la pénalité ainsi convenue si elle est manifestement excessive ou dérisoire.

				Lorsque l'engagement a été exécuté en partie, la pénalité convenue peut être diminuée par le juge, même d'office, à proportion de l'intérêt que l'exécution partielle a procuré au créancier, sans préjudice de l'application de l'alinéa précédent.

				Toute stipulation contraire aux deux alinéas précédents est réputée non écrite.

				Sauf inexécution définitive, la pénalité n'est encourue que lorsque le débiteur est mis en demeure.}. On demande plutôt dans ce cas des dommages et intérêts. En pratique on trouve des clauses de tolérances.

				\paragraph{Le respect des prestations et de la qualité convenues}

		\subsection[L'obligation du \Mo]{L'obligation du \Mo : le paiement du prix}

			Les obligations du \Mo sont au nombre de quatre :
			\begin{itemize}
				\item il doit être de bonne foi ;
				\item il doit informer l'\E, notamment sur la nature du sol ;
				\item il ne doit pas gêner --- tant en s'immisçant dans les travaux qu'en changeant d'avis de manière intempestive ;
				\item mais surtout, \textbf{il doit payer le prix} des travaux.
			\end{itemize}

			\subsubsection{Les modalités du paiement du prix}

				Le \Mo doit payer le prix convenu aux époques convenues.

				L'\articleCodifie[L]{111-3-1} du \cch relatif au paiement entre professionnel pose le principe du versement d’acompte et rappelle les délais maximums de l'\articleCodifie[L]{441-6} du Code du commerce. Il intègre le délai de vérification du \Moe dans le délai de paiement.

				En cas de non paiement \textbf{la suspension des travaux est autorisée}.

				La norme AFNOR prévoit un échéancier, un état de situation et un \dgd.

				Des pénalités de retard sont possibles.

			\subsubsection{La retenue de garantie}\index{RetenueGarantie@Retenue légale de garantie}\label{retenueLegaleDeGarantie}

				La retenue de garantie est une technique développée par les Maîtres d'Ouvrage et consacrée par la loi \no 71-584 du \printdate{16/7/1971} tendant à réglementer les retenues de garantie en matière de marchés de travaux définis au 3\ieme{} alinéa de l'\articleCciv{1779}.

				Elle a pour objet de << \textit{satisfaire, le cas échéant, aux réserves faites à la réception par le maître de l'ouvrage} >>. La jurisprudence a étendu ses effets aux non façons, mais a refusé de la faire jouer pour les retards ou abandons de chantiers\footnote{Il est toujours possible de provoquer une réception, notamment pour la faire jouer.}.

				Elle est limitée à \pourcent{5} du montant des acomptes successifs, et ne s'applique que si elle a été contractualisée. Les disposition de la loi du \printdate{16/7/1971} sont d'ordre public, elles s'appliquent donc en bloc dès lors que la retenue de garantie a été contractualisée.

				Les sommes doivent être consignées entre les mains d'un tiers, et il est possible d'y substituer une caution équivalente.

				Sa mise en jeux requière trois conditions cumulatives :
				\begin{enumerate}
					\item l'existence d'une réception,
					\item assortie de réserves,
					\item non levées par l'\E.
				\end{enumerate}

				Délai d'un an à compter de la réception, sauf opposition motivée de la part du \Mo auprès du consignataire.

			\subsubsection{La garantie de paiement de l'entrepreneur}

				Il s'agit d'une création de la loi \no94-475 du \printdate{10/6/1994} qui a introduit l'\articleCciv{1799-1}.

				Elle garantie à l'\E le paiement du prix. Elle ne porte que sur les travaux exécutés. Suite à l'arrêt du \printdate{1/12/2004} de la 3\ieme{} chambre civile de la \CourDeCas, elle est d'ordre public et il n'est pas besoin qu'elle soit contractualisée pour qu'elle s'applique.

				\begin{citationArticleCciv}{1799-1}
					Le maître de l'ouvrage qui conclut un marché de travaux privé visé au 3\degres{} de l'article 1779 doit garantir à l'entrepreneur le paiement des sommes dues lorsque celles-ci dépassent un seuil fixé par décret en Conseil d'État.

					Lorsque le maître de l'ouvrage recourt à un crédit spécifique pour financer les travaux, l'établissement de crédit ne peut verser le montant du prêt à une personne autre que celles mentionnées au 3\degres{}  de l'article 1779 tant que celles-ci n'ont pas reçu le paiement de l'intégralité de la créance née du marché correspondant au prêt. Les versements se font sur l'ordre écrit et sous la responsabilité exclusive du maître de l'ouvrage entre les mains de la personne ou d'un mandataire désigné à cet effet.

					Lorsque le maître de l'ouvrage ne recourt pas à un crédit spécifique ou lorsqu'il y recourt partiellement, et à défaut de garantie résultant d'une stipulation particulière, le paiement est garanti par un cautionnement solidaire consenti par un établissement de crédit, une entreprise d'assurance ou un organisme de garantie collective, selon des modalités fixées par décret en Conseil d'État. Tant qu'aucune garantie n'a été fournie et que l'entrepreneur demeure impayé des travaux exécutés, celui-ci peut surseoir à l'exécution du contrat après mise en demeure restée sans effet à l'issue d'un délai de quinze jours.

					Les dispositions du présent article ne s'appliquent pas aux marchés conclus par un organisme visé à l'article L. 411-2 du code de la construction et de l'habitation, ou par une société d'économie mixte, pour des logements à usage locatif aidés par l'État et réalisés par cet organisme ou cette société.
				\end{citationArticleCciv}

				\paragraph{Le champs d'application de la garantie}

					Elle ne s'applique que si le montant dépasse un certain seuil, et elle ne bénéficie que à l'\E.

					\subparagraph{Les marchés concernés} Ce sont les marchés de travaux privés commandés par un \Mo professionnel ou profane, dont le prix, déduction faite des arrhes et des acomptes, est supérieur à \montant{12 000}\footnote{Décret du \printdate{30/7/1999}.}. Les travaux commandés par des sociétés d'\HLM ou des \SEM sont exclus.

					\subparagraph{Les débiteurs de la garantie} Le débiteur est le \Mo, qu'il soit professionnel ou profane. La loi du \printdate{1/2/95} a exclus certaines dispositions pour le \Mo profane. Il n'est tenu que d'une forme de garantie spécifique : le crédit spécifique.

					\subparagraph{Les bénéficiaires de la garantie} Le bénéficiaire est l'\E qui contracte, et donc ni l'\archi ni le bureau d'étude technique. Elle bénéfice également aux sous-traitants.

				\paragraph{Le mécanisme de la garantie}

					Il peut être de trois types :
					\begin{enumerate}
						\item si le \Mo sollicite un crédit spécifique, alors versement du crédit directement à l'\E ;
						\item si non, le \Mo peut fournir une garantie particulière --- une garantie à première demande par exemple ;
						\item si non, le \Mo peut mettre en place un cautionnement solidaire.
					\end{enumerate}

					\subparagraph{Le versement direct du crédit} Il n'est possible que si le crédit est spécifique, c'est-à-dire exclusivement et en totalité destiné aux travaux. Le versement se fait même si le crédit ne finance qu'en partie les travaux.

					Sa mise en œuvre se fait uniquement sur demande écrite du \Mo. c'est une délégation de paiement.

					\subparagraph{Les autres garanties}

					Le 3\ieme{} alinéa de l'\articleCciv{1799-1} ne s'impose qu'aux professionnels. Il peut consister en :
					\begin{enumerate}
						\item stipulation particulière --- consignation du prix, hypothèque, garantie à première demande, \etc ;
						\item un cautionnement solidaire --- sans privilège de discussion.
					\end{enumerate}

					Seule l'absence totale de garantie est sanctionnée.

				\paragraph{Le moment auquel la garantie doit être fournie} Il n'y a pas de moment imposé, elle peut être demandée à tout moment, même après la réalisation des travaux, même après le réception ou la résilitaion du marché.

				\paragraph{La sanction du défaut de garantie} L'\articleCciv{1799-1} prévoit dans son 3\ieme{} alinéa la possibilité pour l'\E de sursoir à l'exécution du contrat.

				Dans le cas où :\index{GarantiePaiement@Garantie de paiement!Conditions@Conditions de sursi à l'exécution du contrat}
				\begin{enumerate}
					\item le \Mo ne propose pas de garantie,
					\item l'\E demeure impayé,
					\item après mise en demeure ;
				\end{enumerate}
				l'\E peut saisir le juge pour ordonner de constituer un garantie --- en référé --- sous astreinte.

	\section{La fin du contrat}

		\subsection{L'interruption du contrat en cours d'exécution}

			\subsubsection{La nullité du contrat}

				Le contrat peut être interrompu par la nullité du contrat (en cas d’incapacité, du vice du consentement et contenu illicite)

				\textbf{Prescription en nullité} : 5 ans à compter de la conclusion du contrat


				En construction les demandes de nullité sont rares car souvent il y a eu des commencements d’exécution, les conséquences de la nullité sont donc trop importantes et risquées.

				Il existe quand même la nullité du contrat de sous-traitance qui est intéressant (cf ci-après) : l'article 14 de la loi \printdate{31 /12/1975}, le sous-traitant peut demander la nullité du contrat de louage d’ouvrage si l’entrepreneur principal n’a pas fourni de garantie de paiement --- différent de la garantie de paiement du << \Mo >> prévue à l'\articleCciv{1799-1}.
				Ici, il s’agit d’une cause de nullité propre au contrat de sous-traitance.



			\subsubsection{La résolution du contrat}\index{Resolution@Résolution du contrat}\label{resolutionContrat}

				\paragraph{Les cas de résolution du contrat}

					\subparagraph{La résolution par le jeu d'une clause résolutoire}

					\subparagraph{La résolution unilatérale}

							L'\articleCciv{1226} prévoit que le créancier peut à ses risques et périls, résoudre le contrat par voie de notification avec au préalable mise en demeure de satisfaire à ses engagements.


							Alors le créancier est en droit de résoudre le contrat qu’il notifie à son débiteur avec les raisons qui la motive.


							Le débiteur peut saisir le juge pour contester la résolution alors le créancier doit démontrer les raisons sérieuses.


					\subparagraph{La résolution judiciaire}

							La résolution peut être en toute hypothèse demandée en justice. Elle est prononcée par la juge.

					\subparagraph{La résolution en cas de force majeure}\index{CasForceMajeure@Cas de force majeure}

						L'\articleCciv{1218} prévoie dans son deuxième alinéa : la suspension du contrat en cas d'empêchement provisoire, et la résolution en cas d'empêchement définitif.

						\begin{citationArticleCciv}{1218}
							Il y a force majeure en matière contractuelle lorsqu'un événement échappant au contrôle du débiteur, qui ne pouvait être raisonnablement prévu lors de la conclusion du contrat et dont les effets ne peuvent être évités par des mesures appropriées, empêche l'exécution de son obligation par le débiteur.

							Si l'empêchement est temporaire, l'exécution de l'obligation est suspendue à moins que le retard qui en résulterait ne justifie la résolution du contrat. Si l'empêchement est définitif, le contrat est résolu de plein droit et les parties sont libérées de leurs obligations dans les conditions prévues aux articles 1351 et 1351-1.
						\end{citationArticleCciv}

						La force majeure est appréciée par les juges pour chaque cas, l’appréciation se faisant \emph{in concreto} donc selon les circonstances, chaque cas étant un cas particulier.

						Trois critères cumulatifs permettent d'identifier juridiquement la force majeure. ce sont, depuis la réforme\footnote{Avant la réforme du droit des contrats de 2016, il s'agissait de : imprévisibilité, irrésistibilité et extériorité} :
						\begin{itemize}
							\item \textbf{Extériorité} \emph{échappant au contrôle du débiteur}. Ce critère s’apprécie généralement par rapport à une personne. La personne concernée n’est en rien responsable de la survenance de l’événement. L’événement est totalement indépendant de ce qu’il souhaite, de sa volonté. L’événement ne doit en rien pouvoir être imputé à la personne.

							\item \textbf{Imprévisibilité} \emph{qui ne pouvait être raisonnablement prévu}. L’événement concerné ne doit, par aucun moyen, pouvoir être anticipé ou prévu.

							\item \textbf{Inévitabilité} \emph{dont les effets ne peuvent être évités par des mesures appropriées}. Le caractère inévitable d’un événement est primordial pour que la force majeure soit juridiquement reconnue. Cela signifie qu’il doit être impossible d’éviter l'événement. Ou plus précisément, il est impossible d’éviter ses conséquences. Elle est caractérisée dès lors que les conséquences de l’événement surviennent malgré le fait que tout ait été mis en œuvre pour les réduire ou les éviter. Malgré toutes les précautions, les conséquences sont inévitables.
						\end{itemize}

					\subparagraph{La résolution en cas de circonstance imprévisibles}

						La réforme des obligations de 2016 a consacré la théorie de l’imprévision, qui était tirée de la théorie des sujétions imprévues en droit public admis, en l'encadrant par
l'\articleCciv{1195}.

						\begin{citationArticleCciv}{1195}
							Si un changement de circonstances imprévisible lors de la conclusion du contrat rend l'exécution excessivement onéreuse pour une partie qui n'avait pas accepté d'en assumer le risque, celle-ci peut demander une renégociation du contrat à son cocontractant. Elle continue à exécuter ses obligations durant la renégociation.

							En cas de refus ou d'échec de la renégociation, les parties peuvent convenir de la résolution du contrat, à la date et aux conditions qu'elles déterminent, ou demander d'un commun accord au juge de procéder à son adaptation. A défaut d'accord dans un délai raisonnable, le juge peut, à la demande d'une partie, réviser le contrat ou y mettre fin, à la date et aux conditions qu'il fixe.
						\end{citationArticleCciv}

						La partie doit continuer à exécuter ses obligations pendant la négociation.


						En cas de refus ou échec de négociation, les parties peuvent convenir de la résolution du contrat avec possibilité de demander au juge d’un commun accord de procéder à son adaptation.


						Les dispositions de l'\articleCciv{1195} ne sont pas d’ordre public. Il faut donc vérifier ce qu’il y a dans le contrat, le risque d’imprévision peut peser sur chacune des parties et donc pas de nature à remettre en cause le contrat, mais il faut l’écarter conventionnellement.


				\paragraph{Les effets de la résolution du contrat}

					En application de l'\articleCciv{1229}, <<~\emph{la résolution met fin au contrat} >> :
					\begin{itemize}
						\item soit dans les conditions prévues par la clause résolutoire ;
						\item soit à la date de la réception par le débiteur de la notification faite par le créancier ;
						\item soit à la date prévue par le juge.
					\end{itemize}

					Si les prestations échangées ne trouvent leur utilité que dans l’exécution complète du contrat résolu, alors la résolution oblige les parties à des restitutions réciproques.


					Mais si l’utilité se trouve au fur et à mesure de l’exécution du contrat, alors pas de restitution pour la période antérieure à la résolution.

					Alors on parle de résiliation du contrat, avec effet sur le contrat que pour l’avenir.


					Donc pour les dommages d’ouvrage essentiellement résiliation, sauf si les travaux n’ont pas commencé alors résolution.


					Les restitutions sont encadrées par les articles 1352 à 1352-9 du \cciv. L'\articleCciv{1352-8} concerne notamment la restitution en cas de prestation de service.


			\subsubsection{La résiliation du contrat}\index{Resiliation@Résiliation du contrat}\label{resiliationContrat}

				C’est un anéantissement du contrat pour le futur. Plusieurs cas de résiliation prévues par le code ou conventionnellement.

				\paragraph{La résiliation par le décès de l'entrepreneur}

					L'\articleCciv{1795} dispose que le contrat de louage d’ouvrage est dissout de plein droit par la mort de l’ouvrier, de l’architecte ou de l’entrepreneur



					L'\articleCciv{1795} n’est pas d’ordre public et donc peut être écartée par les parties.
La norme AFNOR en fait une cause de résiliation de plein droit aux torts du défaillant.


					L'\articleCciv{1796} prévoit les conséquences de la résiliation pour le décès.
Si contractant est une personne morale, il ya alors renvoi à la liquidation judiciaire qui emporte résiliation du contrat de louage d’ouvrage.



				\paragraph{La résiliation par le jeu d'une clause de résiliation de plein droit}

					Les parties peuvent insérer dans le contrat de louage d’ouvrage des clauses de résiliation de plein droit. La clause doit identifier clairement les modalités de mise en jeu de cette clause, préciser le comportement sanctionné, les conditions de la résiliation --- avec ou pas \med infructueuse et pendant combien de temps --- et les conséquences de la résiliation --- avec ou sans pénalités, avec ou sans \di.


					La norme AFNOR érige comme cause de plein droit : l’abandon de chantier, l'interruption de plus de 6 mois imputable au \Mo, la tromperie grave sur la qualité ou l’exécution des travaux.



				\paragraph{Le cas particulier de la résiliation unilatérale du marché à forfait}

					L'\articleCciv{1794} stipule que si les conditions du marché à forfait sont remplies ou si les parties s’y soumettent, le \Mo peut résilier de sa seule volonté le marché à forfait, en dédommageant l’\E de toutes ses dépenses, de tous ses travaux et de tout ce qu’il aurait pu gagner pour son entreprise (manque à gagner).


					En pratique la résiliation unilatérale de l’\articleCciv{1794} est peu mise en œuvre.


					\bigskip La résiliation et résolution ne sont pas les seules sanctions envisageables en cas d’inexécution :
					\begin{itemize}
						\item Le co-contractant peut refuser ou suspendre ses obligations en application des articles 1219 et 1220 du Code civil : exception d’inexécution.

						\item L’exécution forcée peut être invoquée, mais l'\articleCciv{1221} ne la rends pas possible en cas de disproportion manifeste entre le coût pour le débiteur et l’intérêt pour le créancier.

						Le \Mo a la possibilité de faire exécuter l’obligation par un tiers au frais du co-contractant.


						C’est une possibilité expressément offerte en cas de garantie de parfait achèvement, mais sous contrôle du juge.


						\item Le \Mo peut également solliciter une réduction du prix prévue à l'\articleCciv{1223}. Le créancier peut après \med accepter une exécution imparfaite du contrat et solliciter une réduction du prix.
Le créancier peut demander réparation des préjudices résultant de l’inexécution du contrat, soit des \di, cela peut être cumulé avec l’acceptation de l’exécution imparfaite.

					\end{itemize}


		\subsection{La fin du contrat par le prononcé de réception}

			La réception marque la fin du contrat de louage d’ouvrage. C’est le terme de la relation contractuelle.


			\medskip Ensuite on bascule dans la responsabilité.


			\medskip La réception n’impose pas l’achèvement des travaux. Cette position de la \CourDeCas est bancale car considère que la réception met fin aux relations contractuelles mais la construction ne sera jamais parfaite. Il y a des tolérances (exemple de tolérance dans le DTU).

			\medskip La réception marque le point de départ des garanties spécifiques. Il peut donc y avoir réception d’un ouvrage inachevé.


			L’\articleCciv{1792-6}, sur la responsabilité des constructeurs, définit la réception : acceptation de l’ouvrage avec ou sans réserve, soit à l’amiable ou à défaut judiciairement. Elle doit être contradictoire.


			C’est un acte juridique unilatéral avec des conséquences considérables, le \Mo donne quitus à l’entrepreneur sur l’exécution de ses travaux. S’il y a des désordres, non-conformité, il faut les réserver, de manière exhaustive, tous les désordres, les non façons malfaçons, vices de construction apparents doivent être listés.


			Si trop de malfaçons, l’ouvrage ne peut être reçu.


			Il y a également la réception tacite qui est une création prétorienne.


			La réception rend le prix du marché exigible, à l’exception le cas échéant de la retenue légale de garantie. Le prix est celui initialement convenu + les pénalités de retard éventuelles.


			La réception a pour effet de transférer les risques de l’ouvrage, après la réception, le risque pèse sur le \Mo.


			La réception purge les désordres apparents à défaut de mention dans le \pv de réception, le \Mo ne pourra plus obtenir réparation de ces désordres sur une quelconque fondement à l’encontre de l’entrepreneur.


			Celui qui invoque l’apparence du désordre devra le prouver.


			La \jp a conscience du caractère sévère de la purge, on apprécie donc l’apparence aux yeux d’un \Mo profane, au regard des désordres apparents et ses conséquences.


			Tous les désordres cachés sont couverts par les garanties spécifiques et de droit commun.


			La réception marque le point départ des garanties spécifiques et des responsabilités contractuelles de droit commun comme pour les dommages intermédiaires.


			Elle marque également le point de départ des assurances obligatoires.


			Enfin, elle permet la mise en œuvre de la retenue légale de garantie. Ces sommes consignées ou cautionnées pour lever les désordres réservés, donc nécessairement une réception.
