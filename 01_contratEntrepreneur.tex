\chapter{Le contrat de l'entrepreneur}

	\section{La qualification du contrat d'entreprise}
	
		\subsection{Distinction avec le contrat de travail}
		
			\subsubsection{Le critère : l'absence de subordination}
			
			\subsubsection{Intérêt de la distinction}
		
		\subsection{Distinction avec le contrat de mandat}
		
			\subsubsection{Le critère : l'absence de représentation}
			
			\subsubsection{Intérêt de la distinction}
		
		\subsection{Distinction avec la vente}
		
			\subsubsection{Le critère}
			
				\paragraph{La construction d'un immeuble}
				
				\paragraph{Les travaux exécutés sur un immeuble}
			
			\subsubsection{Intérêt de la distinction}
		
		\subsection{Distinction avec les contrats de construction}
		
			\subsubsection{Les types de contrats}
			
				\paragraph{La vente d'immeuble à construire}
				
				\paragraph{Le contrat de promotion immobilière}
				
				\paragraph{Le contrat de construction des maisons individuelles (CCMI)}
			
			\subsubsection{Intérêt de la distinction}
	
	\section{La formation du contrat}
	
		\subsection{Les conditions de validité du contrat}
		
		\subsection{Le mode de conclusion du contrat}
		
			\subsubsection{Les marchés de gré à gré}
			
			\subsubsection{Les contrats conclus après mise en concurrence}
		
		\subsection{Le contenu du contrat}
		
			\subsubsection{Le prix}
			
				\paragraph{Les règles générales}
				
				\paragraph{Le prix dans le marché à forfait}
				
					\subparagraph{Les conditions du marché à forfait}
					
					\subparagraph{La stabilité du marché à forfait}
				
				\paragraph{Le prix dans le marché au métré}
				
				\paragraph{Le prix dans les marchés sur dépenses contrôlées}
			
			\subsubsection{Les caractéristiques de l'ouvrage}
			
			\subsubsection{Les conditions juridiques et administratives du marché}
			
			\subsubsection{Les clauses réputées non écrites}
	
	\section{Les effets du contrat}
	
		\subsection{Les obligations de l'entrepreneur}
		
			\subsubsection{L'obligation d'information et de conseil}
			
			\subsubsection{L'obligation de conservation des existants}
			
			\subsubsection{L'obligation d'exécution conforme}
			
				\paragraph{Le respect des délais convenus}
				
				\paragraph{Le respect des prestations et de la qualité convenues}
		
		\subsection[L'obligation du Maître de l'Ouvrage]{L'obligation du Maître de l'Ouvrage : le paiement du prix}
		
			\subsubsection{Les modalités du paiement du prix}
			
			\subsubsection{La retenue de garantie}
			
			\subsubsection{La garantie de paiement de l'entrepreneur}
			
				\paragraph{Le champs d'application de la garantie}
				
					\subparagraph{Les marchés concernés}
					
					\subparagraph{Les débiteurs de la garantie}
					
					\subparagraph{Les bénéficiaires de la garantie}
				
				\paragraph{Le mécanisme de la garantie}
				
					\subparagraph{Le versement direct du crédit}
					
					\subparagraph{Les autres garanties}
				
				\paragraph{Le moment auquel la garantie doit être fournie}
				
				\paragraph{La sanction du défaut de garantie}
	
	\section{La fin du contrat}
	
		\subsection{L'interruption du contrat en cours d'exécution}
		
			\subsubsection{La nullité du contrat}
			
			\subsubsection{La résolution du contrat}
			
				\paragraph{Les cas de résolution du contrat}
				
					\subparagraph{La résolution par le jeu d'une clause résolutoire}
					
					\subparagraph{La résolution unilatérale}
					
					\subparagraph{La résolution judiciaire}
					
					\subparagraph{La résolution en cas de force majeure}
					
					\subparagraph{La résolution en cas de circonstance imprévisibles}
				
				\paragraph{Les effets de la résolution du contrat}
			
			\subsubsection{La résiliation du contrat}
			
				\paragraph{La résiliation par le décès de l'entrepreneur}
				
				\paragraph{La résiliation par le jeu d'une clause de résiliation de plein droit}
				
				\paragraph{Le cas particulier de la résiliation unilatérale du marché à forfait}
		
		\subsection{La fin du contrat par le prononcé de réception}