% !TEX root = ./droitConstruction.tex

\chapter{Définition}

\section{Responsabilité civile}

  En droit français, une distinction doit être faite entre deux types de responsabilités civiles ne reposant pas sur les mêmes textes de loi : la responsabilité contractuelle et la responsabilité délictuelle. Celles-ci ne peuvent pas être cumulées pour un même dommage.

\section{Responsabilité contractuelle}\index{ResponsabiliteContractuelle@Responsabilité contractuelle}\label{responsabiliteContractuelle}

  La responsabilité contractuelle consiste en la réparation d'un dommage causé par l'inexécution ou le retard dans l'exécution d'un contrat. Les conditions de cette mise en jeu reposent sur les dispositions de l'\articleDu{1231-1}{\cciv}.

  \paragraph{Fondement}
  La responsabilité contractuelle est fixée par plusieurs articles du Code civil :
  \begin{itemize}
    \item L'\articleDu{1103}{\cciv} dispose que « \emph{les contrats légalement formés tiennent lieu de loi à ceux qui les ont faits} ».
    \item L'\articleDu{1104}{\cciv} prévoit que « \emph{les contrats doivent être négociés, formés et exécutés de bonne foi} ».
    \item L'\articleDu{1193}{\cciv} dispose que « \emph{les contrats ne peuvent être modifiés ou révoqués que du consentement mutuel des parties, ou pour les causes que la loi autorise} ».
  \end{itemize}

  Ainsi, une fois les termes du contrat fixé, celui-ci doit être respecté par chacune des parties. L'on notera également que « \emph{les contrats obligent non seulement à ce qui y est exprimé, mais encore à toutes les suites que leur donnent l'équité, l'usage ou la loi}\footnote{\ArticleDu{1194}{\cciv}}. »

  Enfin, les juges du fond ont le pouvoir de rechercher quelle a été « la commune intention des parties » afin de trancher un litige\footnote{\jurisCourDeCas{sect. réunies}{2/2/1808}}. S'agissant de la fin du contrat, elle intervient dès lors que les obligations ont été intégralement exécutées.

  Le contrat prend fin par un accord des parties : elles peuvent défaire ce qu'elles ont fait.

  Il existe également la résiliation pour inexécution qui peut donner lieu à contentieux.

  \paragraph{Mise en œuvre et sanction de la responsabilité contractuelle}

  \subparagraph{Sur la mise en œuvre de la responsabilité contractuelle}

  La mise en œuvre de la responsabilité contractuelle repose sur la démonstration d'une inexécution de la part d'un des deux cocontractants.

  Le contrat risque alors la résolution ou la résiliation\footnote{La résolution est l'anéantissement rétroactif du contrat. La résiliation opère rupture du contrat mais uniquement pour l'avenir.}.

  Une clause prévoyant la résiliation en cas d'inexécution est couramment insérée dans les contrats commerciaux.

  Il y est généralement prévu que la résiliation interviendra aux termes d'une lettre recommandée valant mise en demeure d'avoir à exécuter la ou les obligations contractuelles, demeurée sans effet, suivant un préavis fixé dans la convention.

  En tout état de cause et, dès lors que l'inexécution persiste, le contrat peut être résilié même en l'absence d'une telle clause.

  En effet, l'\articleDu{1224}{\cciv} distingue trois modes de résolution :
  \begin{itemize}
    \item la résolution conventionnelle (clause résolutoire) ;
    \item la résolution par notification (résolution unilatérale aux risques et périls de son auteur) ;
    \item la résolution judiciaire.
  \end{itemize}


  \subparagraph{Sur la réparation de l'inexécution}

  Pour faire valoir ses droits à réparation, celui qui s'estime lésé doit faire valoir :
  \begin{itemize}
    \item un fait fautif ou générateur de responsabilité (l'inexécution contractuelle) ;
    \item un lien de causalité ;
    \item un dommage (ou préjudice) subi.
  \end{itemize}

  La réparation peut intervenir par le biais de l'exécution forcée, de la résolution du contrat, de la diminution de prix, d'une demande de dommages et intérêts\footnote{\articleDu{1217}{\cciv}}. Les sanctions qui ne sont pas incompatibles peuvent être cumulées ; des dommages et intérêts peuvent toujours s'y ajouter.

  La réparation de l'inexécution n'empêche pas l'octroi de dommages et intérêts supplémentaires. Tel est le cas, en effet, selon l'\articleDu{1231-1}{\cciv} : « Le débiteur est condamné, s'il y a lieu, au paiement de dommages et intérêts, soit à raison de l'inexécution de l'obligation, soit à raison du retard dans l'exécution, s'il ne justifie pas que l'exécution a été empêchée par la force majeure. »

\section{Responsabilité délictuelle}

  La responsabilité délictuelle repose sur une obligation générale consistant à devoir réparer le dommage causé à autrui. Par rapport à la responsabilité contractuelle, la responsabilité délictuelle est envisageable « par défaut », c'est-à-dire lorsque le dommage causé ne résulte pas d'une inexécution d'un contrat conclu entre l'auteur et la victime du dommage. L'obligation de réparation est prévue au sein de l'\articleDu{1240}{\cciv}. Cette responsabilité recouvre notamment les dommages causés du fait des choses dont on a la garde ou du fait des personnes dont on répond\footnote{\ArticleDu{1242}{\cciv}}.
