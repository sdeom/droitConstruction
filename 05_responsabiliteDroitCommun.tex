% !TEX root = ./droitConstruction.tex

\chapter{La responsabilité de droit commun des constructeurs}

Il existe deux types de responsabilité de droit commun, la responsabilité contractuelle (s’il existe un contrat) et délictuelle (en l’absence de contrat).

\section{La \rcdc des constructeurs}

	Ce sont les hypothèses de responsabilités fondées sur les articles 1103 du Code civil : les conventions légalement formées tiennent lieu de loi entre les parties, et 1231-1 (ancien 1134 et 1147) du Code civil.

	\textbf{Précisions liminaires} :
	\paragraph{Principe de non cumul.} Dès lors que les conditions de la décennale ou biennale sont remplies, il faut nécessairement agir sur ces fondements. Donc au-delà du délai de 2 ans pour la biennale, il n’est pas possible d’agir sur le fondement de la contractuelle et sauver le dossier.

	En revanche, la responsabilité contractuelle peut être invoquée de manière subsidiaire

	La seule exception est le cumul avec la GPA
	\paragraph{Elle est transmissible.} L’action sur le fondement de la responsabilité contractuelle de droit commun est transmissible aux acquéreurs aux acquéreurs et sous acquéreurs de l’ouvrage, au même titre que les garanties spécifiques, la limite étant le délai de 10 ans à compter de la réception (parfois 5 ans) ici on est en prescription et pas en forclusion
	\paragraph{Causes exonératoires.} Les causes d’exonération de responsabilité de droit commun sont identiques à celles des garanties spécifiques, il faut une cause étrangère : le fait d’un tiers, la faute de la victime \MO (il existe plusieurs hypothèses à savoir le mauvais usage, immixtion du MO notoirement compétent, et acceptation délibérée du risque donc le MO a pleinement été informé donc en connaissance de cause, délibérément le MO a accepté le risque), la force majeure.
	cause étrangère : force majeure, fait d'un tiers (mais pas co locateur d'ouvrage), faute du Mo (mauvais usage = acceptation délibéré (= pleinement informé) du risque)
	La faute d’un co-locateur d’ouvrage n’est pas une cause d’exonération de responsabilité contractuelle de droit commun du constructeur, ici le tiers est extérieur à l’opération de construction
	\paragraph{Hors assurance obligatoire.}	Les assurances obligatoires ne couvrent pas les hypothèses de responsabilité contractuelle de droit commun, mais le constructeur peut souscrire une assurance facultative

	\subsection{Les désordres n'affectant pas un << ouvrage >>}

		Les peintures ne constitue jamais un ouvrage dans la jurisprudence.

		Les constructeurs sont tenus d'une obligation de résultat, de sorte qu'ils ne peuvent s'exonérer de leur responsabilité du fait d'une cause extérieure.

		délai 1792-4-3 s'applique car Mo

	\subsection{Les désordres affectant des travaux commandés par une personne autre que le propriétaire}

		Les désordres affectant des travaux autres que le Mo. Donc lorsque les travaux sont commandés par le locataire civ 3 1/7/09. Dans ce cas seule la rcdc peut etre actionnée.

		L'E est tenu d'une obligation de résultat, le simple constat des désordres suffit. Prescrit à compter de 5 ans 2224 du c. civ à compter de la manifestation des dommages. Civ 3 16/1/20 18-21895.

		Les titulaires de baux emphytheotique assimilé Mo (on prévoit dans le contrat).

	\subsection{Les désordres apparus avant la réception}

		Seule la rcdc est applicable avant la réception.

		Les locateurs d'ouvrages sont néanmoins redevable d'une obligation de résultat.  cas ci 3 eme ???. Le simple constat du désordre suffit. Le vendeur immeuble à construire. Les archi sont débiteurs d'une obligation de moyen, il faut donc apporter la preuve d'une faute.

		Avant réception, tous les dommages doivent être repris et les préjudices indemnisées doivent être indemnisé.

		Prescription 24/4/06 ccas considère que les desordres se prescrive à partir de la manifestation du dommage

	\subsection{Les désordres réservés à la réception}

		ca civ 3 13/12/1995 : cumul gpa et rcdc

		Dans un arrêt du 2/2/17 la 3 c civ a jugé que l'obligation de résultat persiste jusqu'à la levée des réserves. L'intérêt demander l'indemnisation des préjudices annexes en plus de la reprise des dommages

	\subsection{Les dommages intermédiaires}

		C'est une catégorie importante de la resp contra. Il s'agit d etous ls dommages cachés à la réception, qui ne . Hypo des garantie construceur réunie :
			...
		Mais sans garvité decennale. Par exemple : un décordre affectant le ravalement si il n'y a pas ou si ces désordes sont simplement esthétque. On y compte galement des désordres dans les tuiles, des défaus de carrelge.
		La gravité du désordre ets indifférent dès lors qu ela gravité décennale n'est pas atteinte.

		Il n'est pas possible ... en cas de mauvais fonctionnement d'un équipement dissociable. Il serait vain d'invoquer la théorie de dommages interméiaires pour échapper à la forclusion biennale.


		Mise en œuvre Les \lo ont une obligation de moyen. Les c civ 3eme 11 5 204 6 10 2010 il faut apporter la preuve de la faute du constructeur
			faute + préjudice + lien de causalité.
			Cette faute ne peut pas résulter simplement d el'obligation de résultat de livrer un ouvrage exempt de vice.

			L'\E sous-traite sous sa responsabilité. La simple faute du sous traitant suffit à engager la responsabilité de l4EP à l'egar d du Mo cas civ 3 1 5 2006

			La responsabilité de est engagé également à lagrd de sprop. succesif d el'ouvrage.
			Sont également débiteur :
				le viac - pour faute prouvée ...
				la personne qui vends après achèvement qu'elle a construit ou fait construire

			Prescription : 10 ans à compter de la réception 1792-4-3

	\subsection{Les défauts de conformité cachés à la réception}

		Cette hypo correspond au non respect de stipulation contractuelle dont il ne découle pas un dommage.

		Si d'un défaut de conforrmité dont il ne découla pas un dommage de gravité décennale.

		Ex : différence de superficie, non respect des plans

		Les constructeurs sont tenus d'une obligation de résultat. le simple constat de la non conformité suffit à engager leur responsabilité.

		presci : 1792-4-3 10 ans

		Les défaut de conformité apparents à la réception non réservé sont purgés.

	\subsection{La violation d'obligations contractuelles dont il ne découle pas de désordre à l'ouvrage lui-même}

		Hypo : le retard
		il s'agit d'une obligation de résultat, à la quelle il  ne peut échapper qu'en apportant la preuve d'une cause étrangère (force majeur, fait d'un tiers, faute de la victime)

		Il est recommandé de prévoir des clauses de suspension du délai d'exe., te donc de déterminer des cause légitimes de suspension -

		Sanctionné par des dommage set interets évalués sur la base du préjudice subi (pertes financières, etc.).

		des clauses de pénalité de retard sont possible. cen 'est pas parceque le contrat ne prevoit pas de pénalité de retard, que les d\&i ne sont pas dus. Si le contrat stipule des pénalités de retard ... il s'agit de clause pénale, le juge peut les réduires ou les augmenter. Le juge apprecie souverainement le quantum du préjudice.

		Hypo : manquement à l'obligation d'info et de conseil. le \lo est tenu d'une . Le contenu dépend de chaque intervenant et de sa compétence.

		c'est une obligation de moyen, de sorte qu'il revient au débiteur, le c/lo, qu'il a rempli.

		La sanction dépend de la nature du préjudice subit.

		S'il découle un dommage 1792 et suivant, alors dans ce cas la responsabilité doit être recherche sur le fondement de la responsabilité decennale. Ex; defaut d'info sur le choix des matérieux

		Si le préjudice n'est pas un dommage à l'ouvrage mais par exemple un surcout ou una llongement, alors le Mo pourra obtenir sur rcdc. Ex : Moe ne pas avoir vérifié la qualification ou la solvailité, s'il a commis une erreur sur l'estimation.
		le /lo s'il n'a pas averti le Mo sur le risque d etrouble sur les avoisinants

		Prescri 10 ans 1792-4-3

		Hypo : manquement à l'assitance à la reception du Moe ;

 		Il faut que le contrat le prévoit.

		S'il ne le liste pas les réserves. Obligation de moyen c'est au Mo

		Prescri : 10 ans 1792-4-3

		Hypo : Dommages causés aux ouvrages existants.

		A priori ils reeléevnt de la rcdc.

		Avant Obligation de cosnerver les existant et de . Sa faute est présum

		5 an à comptre de la manifetstaion 2224 cciv

		Après : ...

		Presci 10 1792-4-3, ou 5 ans car pas de réception. Me Pelon penche sur 10 ans.

		En pratique rare compte tenu de la juris favorable au Mo\footnote{non cumul} qui fait bénéficier de la décennale dans différents cas :
		 pas possible de détreiner si la cause provient des
		 provent des travaux net travaux neuf et existant sont techniquement indivisibles
		 trvx sur existant consttuen en eux meme un ouvrage (renovation lourde ravalement étanche)
		 élément équipement sur existant qui porte atteinte à la destination de l'ouvrage dans son ensemble (cheminée)

	\subsection{Le dol du constructeur}

		La juris le constr "sauf faute extérieure au contrat" contractuellement  tenu à l'agrd du Mo de sa faute dolosive. Il y a dol lorsque de propos délibéré, meme sans intention de nuire, le contructeut viole par dissimulation ou par fraude ses obligations contractueles 27 1 2001 27 3 2013

		Dans l'hypothèse du dol, le construc n'a pas necessaireemnt mais il a la volonté de le cacher. 8 9 2009 la cas a admis l'existence du dol en cas de travaux desastreux, i e contraire aux règles de l'art et aux précotion elementaire. En reveanche un défaut de surveillance des soustraitants ,ene suffit pas à caractèriser.

		Le dol est intersseant en ce qi permet le Mo au-dela des délais de prescription et de forclusion. Pres 5 à compter de la survenance. elle est transmissible avec l'ouvrage, de sorte un sous acquéreur est recevable à agir.

		Le dol n'est couvert par aucune assurance

	\subsection{Les désordres affectant ou provenant d'un élément d'équipement à vocation professionnelle}

		Aux termes 1792-7 les éléments ... exclusive ... ne bénéficie ni de la décennale ni de la biennale. Et ce même si leur défaillance rend l'ouvrage dans son ensemble impropre.

		Si dommage à l'éléments d'équipement lui même = vice caché

		Les dommages causé par l'éléments d'équipement = rcdc

\section{La responsabilité délictuelle du droit commun des constructeurs}

	Ce Les actions fondées sur les inconvénients anormaux de voisinage... 1240 et suivants du cciv

	\subsection{Les actions fondées sur les inconvénients anormaux de voisinage}

		Origine praterorinne Nul ne doit causer à autrui de trouble excedant les inconvenient normaus de voisinage.

		Responsabilité de plein droit qui ne requiert pas la preuve d'une faute.

		\subsubsection{Caractérisation du trouble anormal de voisinage}


			\paragraph{Les conditions de l'action}

				Elle suppose une relation de voisinage. Il n'existe pas de def legale de la notion de voisinage. Ni juris, de sorte que ce sont les tribunaux qui vont au cas par cas sur la base de la proximité géo.

				Deuxime condition : trouble anormal. Un simple trouble ne suffit pas à ouvrir droit à indenisation. Il papartiet au demandeur de caratériser l'anomraliré. Il doit demontrer que les nuisance subies, ressenties, excede les inconvenients normaux de voisinage. Ex : la tondeuse normal le dimanche, anorml

				il s'git d'une question defait soumise à l'appreciation souveraine des juges du fonds.

				L'anormalité du trouble peut etre caractérisée alors même qe des dispos legales, conventionnelel ou reglementaire sont réunies. Ainsi l'obtention d'un PC ne suffit pas car sous réserve des droits des tiers. De meme regelemnt de copro, permis de démolir, lotissement.

				3eme : trouble continu. Un trouble episodique, fugace n'est pas sufffisant. La preuve du caractere continu incombe à la victime

			\paragraph{La diversité des troubles}

				\subparagraph{Les troubles de jouissance dus au chantier}

					Il s'agi de tous les troubles causés bruit odeurs poussieres. Il faut qu'il revetent une certiane gravité, qu'il appartient à la victile de caratériser.

				\subparagraph{Les dommages matériels subis par l'immeuble voisin}

					Il s'agit de tous les dommages matériels causé par le chantier. Le plus souvent terrassement démolition fondation gros œuvre. référé préventif intérêt car on va pouvoir imputer les desordres au chantier et plus précisément à l'entre présente.

					Pratiquement : assigner l'ensemble des voisins, le juge nomme un expert. La constitution d'un avocat est obligatoire. Intervention volontaire possible.

				\subparagraph{Les troubles de jouissance dus à la construction}

					La construction. Grande diversité : dues aux dimensions à l'implantation de la construction, ouverture de vues obturation d'ouverture, diminution d el'ensoleillement 16 2 2019, 19 7 2017 attention il n'existe pas de droit acquis à l'ensoleillement, il appartient au juge \emph{in concreto} de la situation pour caractériser l'existence d'un trouble. civ 3 17 5 2018 -  21 10 2009 dans un lotissement
					diminution de la vue sur le paysage environnant et dommage d'ordre esthétique (perte totale de la vue sur la seine 26 5 2016)

		\subsubsection{Les spécificités du régime de l'action pour trouble anormal de voisinage en matière de construction}

			\paragraph{L'auteur de l'action} Peuvent agir toutes les personne s qui subissent directement et personnellement le trouble. il peut s'agir du proprietaire

			\paragraph{Les defendeurs} Peuvent voir le responsabilité engagées :
				le \Mo alors même qu'il n'est plus proprio du fonds --- 21 5 2008 : le Mo est responsable de plein droit. Il est considéré comme l'auteur intellectuel du dommage.
				les \E dès lors qu'ils interviennent matériellement sur le chantier, qu'ils soient ou non liés au Mo par un contrat. Ainsi, l'\E général, le titulaire d'un simple lot, les sous-traitants
				les Moe, les bureau xd'étude et les controleurs techniques

			Pour les constructeurs, il faut établi un lien d'imputabilité entre le trouble et les travaux litigieux 21 5 2008 c'est à dire une relation de cause directe (inverser la phrase ...). Le demandeur doit démontrer que l'activité de l\E est à l'origine matérielle du trouble, ou le cas échéant qu'il a commis une erreur dans le contrôle et a surveillance des travaux de son sous-traitant.

			pour les Moe, les lbureaux ... a responsabil d'une relation de cause directe entre els troubles et la réalisation des missions confiées civ 3 9 2 2011 ce qui revient à exiger la preuve d'une faute.

			\paragraph{Le délai d'action} L'action se prescrit par 5 ans à compter de la manifestation du dommage ou son aggravation. 16 1 2020 ... réduite à 5 ans à compter la

			en cas d'empiétement 30 ans 2272 cCiv

			\paragraph{La sanction du trouble} Le demandeur peut solliciter une réparation en nature jusqu'à la démolition (sauf disproportion) ou en équivalent (dommages et intérêts)

		\subsubsection{Les actions récursoires entre co-auteurs}

			\paragraph{Les recours contre le \lo}

				\subparagraph{Le fondement de l'action} Il faut envisager deux hypo selon que le Mo indemnise ou non les victimes.

					Si le Mo indemnise la victime il est subrogé dans les droits de la victime. ce qui le dispense d'apporter la preuve d'une faute c civ 3 21 5 2008

					Si le Mo n'a pas indemnisé la victime. Par exemple : . Dans ces cas il 'est pas subrogé dans les droits de la victime, il ne pourra agir que dans en fonction du lien de droit qui l'unit avec le responsable.

					Attention une exception : en cas de stipulation contractuelle entre le Mo et ses locateurss d'ouvrages --- garantie intégrale même sans faute par exemple, qui sera simplement appliquée 10102 du c civ.

				\subparagraph{Les cause exonératoires de responsabilité : la faute du \Mo}

					Les peuvent invoquer la faute du (acceptation délibéré des risques et immixtion fautive du \Mo notoirement compétent (pas de plein droit 19eme chambre  appel de paris 26 11 2008))

			\paragraph{Les recours entre locateurs d'ouvrage}

				Ce recours sera de nature contractuelle ou délictuelle selon qu'il existe ou non un contrat entre eux. La jurisprudence refuse de faire bénéficier du mécanisme de la subrogation au lo qui aurait indemnisé la victime.

				Dans les deux cas cela nécessite de la preuve du faute 26 4 2006. la charge dela dette se répartie en fonction de la faute de chacun c'est le mécanisme du recours \emph{in solidum}. En l'absence de faute, la répartition se fera par part virile.

	\subsection{Les autres actions en responsabilité délictuelle}

		Actions fondées sur 1240 et suivnats du cciv



		%\subsubsection{L'action du \Mo contre un sous-traitant} c'est une précision liminaire

		Ne pas oublier que l'action du \Mo à l'encontre du \ST sont délictuelle

		\subsubsection{Les actions menées par un tiers au chantier}

			Hypo 1 :1240 action fondée sur la faute d'un constructeur : faute + +

			LA faute peut etre prouvée par tout moyen (ex : absence de cloture et électrocution)

			Conseil : agir sur le trouble anormal de voisinage

			assemblée plénière 6/10/2006 confirmé 13/1/2020

			18/5/17 il faut prouver une faute, le simple manqueme

			Prescription 5 ans à compter de la manifestation du dommage.

			Hypo 2 : 1342 al 1 \& 2. action fondée sur la garde de l'immeuble

			Invoquée en cas de chute d'objet, de matériaux, d'un

			Conseil : traoble anmral de voisinage

			Prescription 5 ans à compter de la manifestation du dommage.

			Hypo 3 : action commis-comettant. 1242 al 5. Aller voir les ouvrages

			Hypo 3 : dommage caus par la ruine du bâtiment. 1244 Aller voir les ouvrages

		\subsubsection{Les actions délictuelles entre locateurs d'ouvrage}

			Ce sont les actions entre \lo non liés entre eux par un contrat. Ex : les \E{}s entre eux ST entre eux, un \E avec

			pas de présomption de responsablité de 1792 et suivant = nécessité : faute + préjudice + lien de causalité

			Quels hypothèses ?
			\begin{itemize}
				\item Action récursoire après condamnation \emph{in solidum}.
				\item si Entrepreneur dégrade les travaux d'un autre locateur d'ouvrage.
				\item Histoire du forfait top peu élevé à cause du \Moe...
			\end{itemize}

			Conseil : toujours mettre la petite phrase que l'\E doit vérifier les métrés.

			Prescrit par 5 ans à compter de la manifestation du dommage ou de son aggravation. Les tribunaux fixent communément le point de départ au jour de l'assignation du constructeur par le \Mo.

			Attention : Arret civ 3 13/9/06, point de départ au jour où le dommage s'est manifesté à l'égard du \Mo. elle a été remise en cause depuis 16/1/2020, probablement car de nature à remettre en cause ...

		\subsubsection{La responsabilité du fait des produits défectueux}

			Article 1245 Cciv et suivants.

			Le régime de responsabilité garantie les dommages causé par un produit défectueux, \cad un produit qui n'offre pas la sécurité à laquelle on peut légitiment s'attendre.

			1245-2 constitue un produi .. même ... immeuble

			ne couvre que les dommages portant à la personne ou a un bien, autre que le produit défectueux lui-même, dès lors qu'ils excedent \montant{500}.

			il peut s'agir d'un défaut de conception, de fabrication d'un matériaux de construction  d'une partie composante ou d'un élément d'équipement de la construction.

			elle s'applique au producteur du produit \cad du fabriquant ou ... importateur ... à l'encontre de tous ceux qui incorporent le produit à l'immeuble ... Ne s'applique pas aux personnes qui peuvent être recherché sur le fondement 1792 et suivant (garanties spécifiques), ni au vica sur la responsabilité 1646-1. Expressément prévu par 1245-5.

			Mise ne œuvre ne suppose la preuve d'une faute, << le producteur est responsable ... >>. Dommage + défaut + lien de causalité.

			Action encadré par un double délai 10 ans de la mise en circulation du produit / à l'intérieur de ce délai 3 ans à compte de la date. ???? 1245-10.
