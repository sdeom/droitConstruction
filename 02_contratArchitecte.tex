% !TEX root = ./droitConstruction.tex

\chapter{Le contrat de l'architecte (ou contrat de maîtrise d'œuvre)}

	Le \Mo va souvent s’adjoindre un architecte qui peut avoir une mission ponctuelle ou générale : une mission globale sur la conception de l’ouvrage, le dépôt du permis, le suivi du chantier, la réception.


	Dans le cas d'une mission globale on parlera de mission de maitre d’œuvre. Si la mission ne porte que sur la conception , on parlera alors de maitre d’œuvre de conception, et si c’est le maitre d’œuvre ne gère que l’exécution des travaux on parlera de maître d’œuvre d’exécution.
Dans tous les cas, il s'agit d'un contrat de louage d'ouvrage.

	Le code déontologie de l’architecte impose un contrat par écrit, sachant que l’ordre propose des modèles. La 3\ieme{} chambre civile considère que l’exigence de l’écrit n’est pas une condition de validité du contrat mais une obligation déontologique, une exigence de preuve.


	Le contrat d’architecte et sa forme son libre. En cas de litige il appartient aux parties de rapporter la preuve de l’étendue de la mission de l’architecte\footnote{\jurisCourDeCas[17-21329]{\civTrois*}{6/9/2018}}.

\section{Les obligations de l'architecte}

	les obligations dépendent du contenu du contrat.

	Classiquement, on compte quatre obligations :
	\begin{itemize}
		\item concevoir,
		\item diriger,
		\item assister,
		\item et dans tous les cas conseiller.
	\end{itemize}

	\subsection{Obligation de concevoir}

		\subsubsection{Les contraintes de la conception}

			Obligation de concevoir le projet architectural. Cette obligation est couverte par le monopole des architectes.

			\paragraph{L'appréciation de l'état du sol et des existants} L’architecte doit examiner le sol et les existants pour déterminer s’ils sont aptes à recevoir la construction envisagée.


			S’il y a un doute, il doit conseiller au \Mo de faire réaliser une étude de sol.

			Il est chargé d’établir les documents du PC, il est en charge du projet architectural, dans ces cas-là il doit proposer un projet réalisable tenant compte des contraintes du sol. A défaut il engage sa responsabilité\footnote{\jurisCourDeCas[16-23509]{\civTrois*}{21/11/2019}  : responsabilité décennale de l’architecte faute de vérifier la qualité des remblaies avant la construction}
.


			\paragraph{Le respect des règles de l'art} L’architecte est responsable de la qualité de son projet et doit respecter les règles de l’art.

			\paragraph{Les respect des autres règles s'appliquant à la construction} Il s’agit des règles d’urbanisme, il doit s’assurer de la constructibilité du terrain, au regard du PLU ou POS.


			Il doit à titre principal respecter les règles d’urbanisme, donc en cas de violation c’est sa responsabilité à titre principal qui sera recherchée.

			Néanmoins, en l’absence d’architecte, c’est l’entrepreneur qui doit vérifier les règles d’urbanisme\footnote{\jurisCourDeCas{\civTrois*}{2/12/2002} : responsabilité de l’architecte pour construction sur la base d’un PC annulé par la suite par le juge administratif}.

			L’architecte doit aussi respecter les règles du droit de l’environnement. Les règles contenues dans le CCH notamment aux IGH et aux ERP. Il doit également respecter les règles civiles, de servitude de vue, de passage, d’écoulement, les règles conventionnelles --- comme par exemple s’il y a des servitudes \emph{non aedificandi} ou \emph{non altius tollendi} dans les actes --- il doit se faire remettre les actes et vérifier qu’il n’y a pas de servitudes empêchant la construction.



		\subsubsection{Les étapes de la conception}

			Le \Mo est libre d’investir l’architecte de mission plus ou moins importante c’est la convention des parties qui s’applique. Il peut donc avoir une étude préliminaire qui aboutit sur la phase APS (avant-projet sommaire), ou APD (avant-projet définitif), lui demander de l’assistance pour le dépôt de la demande de PC. C’est souvent l’architecte qui fait interface entre l’urbanisme et le \Mo.


			Dossier de consultation générale, appel d’offre, éplucher la réponse aux appels d’offre, préparer les marchés, vérification de la compétence des entreprises et de leur assurance décennale.



	\subsection{L'obligation de diriger et surveiller les travaux}

		S’il a cette mission, l’architecte doit déterminer comment et dans quel ordre les entrepreneurs interviennent, établit les OS (ordre de service), anime les réunions de chantier et établit les CR, il doit surveiller le chantier \cad que les entreprises respectent les instructions reçues, les documents contractuels et les règles de l’art.


		Rôle financier et comptable à jouer : il reçoit les situations de travaux, les décomptes mensuels, les décomptes généraux et définitifs. Il doit vérifier les viser et les adresser au \Mo pour règlement.


		\paragraph{Quel est le degré de présence exigé de l’architecte sur le chantier ?}
		Pas d’exigence légale, mais l’architecte ne doit pas être toujours présent mais la JP exige une présence régulière souvent, au moins une fois par semaine et surtout une présence à l’occasion des étapes clés du chantier.

		Pour le reste, si le \Mo a engagé la responsabilité de l’architecte en raison d’une présence faible, le juge appréciera pour chaque espèce. \aVerifier


	\subsection{L'assistance aux opérations de réception}

		Ce n’est pas toujours le cas.
		Il appartient à l’architecte d’alerter le \Mo sur la présence de désordres, l’inviter à les lister dans le \pv de réception, de mettre les réserves sur tous les désordres apparents sinon sa responsabilité peut être engagée.

	\subsection{L'obligation d'information et de conseil}\index{ObligationConseil@Obligation de conseil!Archi@de l'\archi (\Moe)}

		L’architecte, comme tout contractant, est tenu d’une obligation générale d’information et de conseil durant toute sa prestation.


		Il est tenu de s’informer sur les souhaits du \Mo, la destination de l’ouvrage, il est obligé de se renseigner sur le mode d’occupation, sur les modalités d’exploitation éventuelles.


		Le contenu du devoir de conseil varie en fonction de l’étape de la construction.


		\paragraph{Avant le début des travaux}
		L'\archi doit s'informer  sur les contraintes du sol, sur la constructibilité du terrain, les contraintes d’urbanisme, les contraintes de droit privé. Il doit signaler au \Mo le danger ou inconvénient de certains programmes, refuser les lieux du client si aboutit à un risque sur la construction.

		Il peut être amené à proposer des entreprises, mais se bornera à conseiller et vérifier la police d’assurance.


		Dans certains cas la responsabilité de l’architecte est engagée quand il commet de grave faute dans son choix des entreprises.
		\begin{exemple}
			candidature d’une entreprise notoirement insolvable, ou qu’il a proposé 2 entreprises successivement défaillantes conduisant à l’abandon de chantier, ou lorsqu’il propose une entreprise dont il connaissait l’insuffisance : responsabilité possible de l’architecte
		\end{exemple}

		\paragraph{Pendant la réalisation des travaux}

		L’architecte a l’obligation d’aviser le client de tout incident notable survenu pendant l’accomplissement de sa mission. Pas simple mise en garde à l’égard du \Mo
		\begin{exemple}
			architecte responsable si juste indiqué que probable désordre alors qu’il aurait dû indiquer que les malfaçons allaient à coup sûr entrainant des désordres.
		\end{exemple}

		Pas simple mise en garde.


		Il peut voir sa responsabilité engagée pour ne pas avoir signalé les risques de la suppression d’un drainage, également s’il accepte aveuglement les modifications et suppression sollicitées par le \Mo. De même s’il ne tient pas suffisamment compte des capacités financières de son client, ou s’il sous-estime ou sous-évalue le montant des travaux\footnote{\jurisCourDeCas[18-16.643]{\civTrois*}{13/06/2019}}.

		Il engage sa responsabilité pour ne pas avoir conseillé au client de souscrire une \DO.

		Responsabilité pour ne pas avoir informé la présence de sous-traitant sur le chantier au \Mo, donc pas présenté ni agréé\footnote{Dans un arrêt de 2014, \CourDeCas le Maître d’œuvre doit informer la présence du sous-traitant au \Mo et de lui conseiller d’agréer}.


		\paragraph{Au moment de la réception}
Il doit guider et conseiller le \Mo, l’informer des conséquences sur l’absence de mention de réserve pour les désordres apparents.


		A défaut sa responsabilité peut être recherchée pour manquement à son devoir de conseil.


		Parfois quand le \Mo va rechercher la responsabilité de l’architecte, il peut avoir un partage de responsabilité avec l’entrepreneur, en raison du devoir de conseil du constructeur. Plus l’entrepreneur sera technique, plus sa part de responsabilité sera importante.


		La responsabilité de l’architecte peut également être partagée avec le \Mo lorsque celui-ci est à l’origine d’une aggravation des désordres


		\paragraph{Charge de la preuve} La preuve du manquement au devoir de conseil incombe à l’architecte qui doit démontrer qu’il a rempli son devoir de conseil.

		Lorsque l’opération est conduite sans architecte ni maitre d’œuvre c’est l’entrepreneur qui est tenu de l’obligation d’information renforcée
		\footnote{\jurisCourDeCas[15-11142]{\civTrois*}{10/12/2015} : en l’absence de maître d’œuvre l’entrepreneur est tenu des erreurs de conception}.



	\subsection{L'obligation de s'assurer}

		L’architecte doit s’assurer pour toutes les missions dont il est chargé, y compris pour une simple mission de conseil.


		Ce n’est pas une assurance qui couvre exclusivement décennale mais une assurance qui couvre tous les cas d’engagement de responsabilité.


		Aux termes l'article 16 alinéa 1 loi du \printdate{3/1/1077}, il s’engage sur ses actes ou de ses proposés.

		Lorsque l’architecte intervient en qualité d’associé de structure d’exercice, c’est la société qui doit être assurée, à raison de ses actes accomplis personnellement et de ceux accomplis par ses salariés.


		L’assurance souscrite doit garantir toutes les hypothèses de responsabilité. L’architecte doit produire chaque année au conseil régional des architectes son assurance pour l’année en cours, à défaut il peut être suspendu du tableau par le conseil régional.



\section{Les droits de l'architecte}

	L'architecte est un << artiste >>.

	\subsection{Les droits de l'architecte sur son œuvre}

		Il est concepteur inventeur, il réalise une œuvre, l'\articleDu[L]{112-2}{\cpi} protège les plans, croquis plastic relatifs à l’architecture, encore faut-il que l’œuvre présente un caractère original\footnote{\jurisCourDeCas{\civUn*}{6/03/1979}}

		\subsubsection{Le droit moral de l'architecte sur son œuvre}

			Le droit de paternité, ce droit impose à ceux qui souhaitent sur l’œuvre architectural indiquer le nom de l’architecte.


			Depuis loi du \printdate{7/6/2016}, indique sur la façade extérieure la date et le nom de l’architecte.


			Droit également au respect de l’intégrité de l’œuvre. Ce droit s’oppose à toute dénaturation de l’œuvre. Le problème est qu’il faut adapter l’ouvrage et y faire des travaux. Donc il faut trouver un équilibre entre ce droit et les besoins du \Mo. Ce droit ne s’oppose pas à toute modification de l’ouvrage mais elles sont encadrées.


			La \jp estime que le droit moral de l’architecte ne saurait imposer une intangibilité de l’œuvre, l’ouvrage a vocation à vivre, donc équilibre trouvé par les tribunaux entre ce droit et le droit de propriété.


			La formule de principe\jurisCourDeCas{\civUn*}{7/01/1992} relatif au théâtre des Champs Elysées, la vocation utilitaire interdit de prétendre imposer une intangibilité de son œuvre à laquelle son propriétaire est en droit d’apporter des modifications pour répondre aux besoins d’adaptation à des besoins nouveaux, il appartient à l’autorité judiciaire si l’adaptation est légitimée eu égard à la nature et l’importance des travaux et que les circonstances contraignant le propriétaire à y procéder. Donc contrôle de proportionnalité entre la nécessité de l’ouvrage et les altérations.


			\paragraph{Le droit moral s’oppose-t-il à la démolition ? }

			On a vu pour la dénaturalisation. Mais non le droit moral ne s’y oppose pas, hors le cas d’abus de droit ou que le public n’a pas eu l’occasion de l’admirer\footnote{\jurisCA{Versailles}{4/04/1996}}.

			Le droit moral est perpétuel donc se transmet à ses héritiers


		\subsubsection{Le patrimonial de l'architecte sur son œuvre}

			C’est le droit de reproduction, de publication. Ce droit patrimonial perdure, \articleDu[L]{111-1}{\cpi} précise que le louage d’ouvrage n’emporte aucune dérogation à la propriété intellectuelle de l’auteur, toujours si l’œuvre est original.


			La durée du droit patrimonial est de 70 ans qui suit le décès de l’auteur, au-delà cela tombe dans le domaine public, sous réserve de respecter le droit moral.


	\subsection{Le droit au paiement des honoraires}

		Le contrat peut être gratuit mais à défaut de mention il est présumé être onéreux et donc l’architecte a droit au paiement de ses prestations et perçoit des honoraires.


		Le mode de calcul est librement déterminé par les parties, au forfait, lau débours, ou, la plupart du temps, sous forme d'un pourcentage du montant des travaux


		Les honoraires doivent être payés aux époques et conditions convenues entre les parties, et les honoraires sur la conception de l’ouvrage sont dus même si aucune exécution sauf si l’inexécution est due par le fait de l’architecte.


		La \CourDeCas considère abusive la clause d’un contrat entre architecte et non professionnel impose le paiement intégral même si pas d’exécution, donc clause écartée. Le contrat peut néanmoins prévoir le versement de \di\footnote{\jurisCourDeCas[18-23259]{\civTrois*}{7/11/2019}\aVerifier}.
